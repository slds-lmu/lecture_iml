

\documentclass[11pt,compress,t,notes=noshow]{beamer}\usepackage[]{graphicx}\usepackage[]{color}
% maxwidth is the original width if it is less than linewidth
% otherwise use linewidth (to make sure the graphics do not exceed the margin)
\makeatletter
\def\maxwidth{ %
  \ifdim\Gin@nat@width>\linewidth
    \linewidth
  \else
    \Gin@nat@width
  \fi
}
\makeatother

\definecolor{fgcolor}{rgb}{0.345, 0.345, 0.345}
\newcommand{\hlnum}[1]{\textcolor[rgb]{0.686,0.059,0.569}{#1}}%
\newcommand{\hlstr}[1]{\textcolor[rgb]{0.192,0.494,0.8}{#1}}%
\newcommand{\hlcom}[1]{\textcolor[rgb]{0.678,0.584,0.686}{\textit{#1}}}%
\newcommand{\hlopt}[1]{\textcolor[rgb]{0,0,0}{#1}}%
\newcommand{\hlstd}[1]{\textcolor[rgb]{0.345,0.345,0.345}{#1}}%
\newcommand{\hlkwa}[1]{\textcolor[rgb]{0.161,0.373,0.58}{\textbf{#1}}}%
\newcommand{\hlkwb}[1]{\textcolor[rgb]{0.69,0.353,0.396}{#1}}%
\newcommand{\hlkwc}[1]{\textcolor[rgb]{0.333,0.667,0.333}{#1}}%
\newcommand{\hlkwd}[1]{\textcolor[rgb]{0.737,0.353,0.396}{\textbf{#1}}}%
\let\hlipl\hlkwb

\usepackage{framed}
\makeatletter
\newenvironment{kframe}{%
 \def\at@end@of@kframe{}%
 \ifinner\ifhmode%
  \def\at@end@of@kframe{\end{minipage}}%
  \begin{minipage}{\columnwidth}%
 \fi\fi%
 \def\FrameCommand##1{\hskip\@totalleftmargin \hskip-\fboxsep
 \colorbox{shadecolor}{##1}\hskip-\fboxsep
     % There is no \\@totalrightmargin, so:
     \hskip-\linewidth \hskip-\@totalleftmargin \hskip\columnwidth}%
 \MakeFramed {\advance\hsize-\width
   \@totalleftmargin\z@ \linewidth\hsize
   \@setminipage}}%
 {\par\unskip\endMakeFramed%
 \at@end@of@kframe}
\makeatother

\definecolor{shadecolor}{rgb}{.97, .97, .97}
\definecolor{messagecolor}{rgb}{0, 0, 0}
\definecolor{warningcolor}{rgb}{1, 0, 1}
\definecolor{errorcolor}{rgb}{1, 0, 0}
\definecolor{code}{rgb}{0.97, 0.96, 1.0}
\newenvironment{knitrout}{}{} % an empty environment to be redefined in TeX

\usepackage{alltt}

\usepackage[utf8]{inputenc}
\usepackage[ngerman]{babel}
\usepackage{dsfont}
\usepackage{verbatim}
\usepackage{amsmath}
\usepackage{amsfonts}
\usepackage{mathtools}
\usepackage{csquotes}
\usepackage{cmbright}
\usepackage{multirow}
\usepackage{longtable}
\usepackage{enumerate}
\usepackage[absolute,overlay]{textpos}
\usepackage{psfrag}
\usepackage{algorithm}
\usepackage{algpseudocode}
\usepackage{eqnarray}
\usepackage{bytefield}
\usepackage{animate}
\usepackage{tikz}
\usetikzlibrary{shapes,matrix,positioning,chains,arrows,shadows,decorations.pathmorphing,fit,backgrounds}
\usepackage{adjustbox}
\usepackage{colortbl}
\usepackage{tabularx} % for tables (incl. \hline)
\usepackage{arydshln} % Load after array, longtable, colortab and/or colortbl , otherwise problems with \hline in tabular env
\usepackage{etex} %increase registers for \dimenS to more than 256, otherwise we get "No room for a new \dimen"
\usepackage{graphicx}
\usepackage{booktabs} %used in epr lectures
\usepackage{bm} % bold greek letters
\usepackage{hyperref} % url citing
\usepackage{blkarray} % block arrays
\usepackage{listings} % block of code
\usepackage{xcolor} %colored math symbols
\usepackage{pgffor}
\usepackage{verbatimbox}
\usepackage{xcolor}
\usepackage{pifont}
\newcommand{\xmark}{\ding{55}}

%some colors
\definecolor{checkgreen}{HTML}{18A126}
\definecolor{errorred}{HTML}{FF0000}
\definecolor{blockbg}{HTML}{F7F7F7}
\definecolor{gray}{HTML}{A0A0A0}


% basic latex stuff
\newcommand{\col}{\par\colorbox{code}{\parbox{\textwidth}{\theverbbox}}\par}
\newcommand{\eg}{e.\,g.\xspace} %for example
\newcommand{\ie}{i.\,e.\xspace} %that is to say...
\newcommand{\pkg}[1]{{\fontseries{b}\selectfont #1}} %fontstyle for R packages
\newcommand{\lz}{\vspace{0.5cm}} %vertical space
\newcommand{\oneliner}[1] % Oneliner for important statements
{\begin{block}{}\begin{center}\begin{Large}#1\end{Large}\end{center}\end{block}}
\def\SpAr{\quad \Rightarrow \quad}



%new environments

\newenvironment{vbframe}  %frame with breaks and verbatim
{
 \begin{frame}[containsverbatim,allowframebreaks]
}
{
\end{frame}
}

\newenvironment{vframe}  %frame with verbatim without breaks (to avoid numbering one slided frames)
{
 \begin{frame}[containsverbatim]
}
{
\end{frame}
}

\newenvironment{blocki}[1]   % itemize block
{
 \begin{block}{#1}\begin{itemize}
}
{
\end{itemize}\end{block}
}

\newenvironment{fragileframe}[2]{  %fragile frame with framebreaks
\begin{frame}[allowframebreaks, fragile, environment = fragileframe]
\frametitle{#1}
#2}
{\end{frame}}


\newcommand{\myframe}[2]{  %short for frame with framebreaks
\begin{frame}[allowframebreaks]
\frametitle{#1}
#2
\end{frame}}



% ???? remove this
% \newcommand{\LS}{\mathfrak{L}}
% \newcommand{\TS}{\mathfrak{T}}
% \newcommand{\bmat}{\begin{pmatrix}}
% \newcommand{\emat}{\end{pmatrix}}
% \newcommand{\const}{\mathop{const}}
% \newcommand{\dist}{\operatorname{dist}}
% \newcommand{\D}{\displaystyle}
%\newcommand{\op}[1]{\operatorname{#1}}

%\usetheme{../style/lmu-lecture}
\usepackage{../../style/lmu-lecture}

\let\code=\texttt
\let\proglang=\textsf

\setkeys{Gin}{width=0.9\textwidth}








\usepackage{tikz}
\usetikzlibrary{shapes,arrows,snakes, calc}

% Define block styles
\tikzstyle{decision} = [diamond, draw, text width=6em, text badly centered, node distance=4cm, inner sep=0pt]
\tikzstyle{decision2} = [diamond, draw, fill=customgreen!35, text width=6em, text badly centered, node distance=4cm, inner sep=0pt]

\tikzstyle{block} = [rectangle, draw, text width=14em, text centered, rounded corners, node distance=3cm, minimum height=4em]
\tikzstyle{line} = [draw, -latex']
\tikzstyle{cloud} = [draw, ellipse, node distance=3cm, minimum height=2em]

\title{FCIM / Predictive Modeling}
\author{Bernd Bischl}

\institute{Department of Statistics - LMU Munich}
\date{Summer term 2020}

\setbeamertemplate{frametitle}{\expandafter\uppercase\expandafter\insertframetitle}

%% Get lecture number from currente directory
%% Add '\lecturechapter{lecture_nr}{[NAME LECTURE]}' at beginning of .Rnw files


\IfFileExists{upquote.sty}{\usepackage{upquote}}{}



\input{../../latex-math/basic-math}
\input{../../latex-math/basic-ml}

\begin{document}

\lecturechapter{13}{ICE Curve and Partial Dependence Plot}
\lecture{Fortgeschrittene Computerintensive Methoden}



\begin{vbframe}{Individual Conditional Expectation (ICE)}
The individual conditional expectation (ICE) plots visualize how the model prediction of individual observations $\xi$
change by varying the feature values in $\xv_S$ while keeping all other features in $\xi_C$ fixed (with $C = S^\complement$).
$$\fh_{S}^{(i)}(\xv_S) = \fh(\xv_S, \xi_C)$$

In practice, $\xv_S$ consists of one or two features.

\vspace{-0.2cm}
\begin{center}
\includegraphics[width=0.7\textwidth]{figure_man/ICE01.png}
\end{center}

\vspace{-0.3cm}
{\scriptsize{\textbf{Figure:} ICE Curves of the last 10 observations of the bike sharing dataset. Each line displays the change in prediction for a single observation due to varying temperature.}\par}

\vspace{0.4cm}
\tiny{Goldstein, A., Kapelner, A., Bleich, J., and Pitkin, E. (2013). Peeking Inside the Black Box: Visualizing Statistical Learning with Plots of Individual Conditional Expectation, 1-22. https://doi.org/10.1080/10618600.2014.907095 \par}
\normalsize


\framebreak

Steps to create an ICE plot for a single feature $x_S$ according to the SIPA framework:

\begin{enumerate}
\item \textbf{Sampling:} Choose grid points along $x_S$.
\item For each grid point:
  \begin{itemize}
    \item \textbf{Intervention:} Replace the feature value $x_S$ with the current grid value.
    \item \textbf{Prediction:} Get the model prediction with replaced feature value $x_S$.
  \end{itemize}
\item Draw a curve per observation with the grid points on the x-axis and the prediction on the y-axis.
\end{enumerate}

\framebreak

\begin{center}
\includegraphics[page=1, width=0.6\textwidth]{figure_man/ice_pd_plot_demo}
\end{center}

\begin{itemize}
\item[1.] \textbf{Sampling:} Choose grid points along $x_1  $ that will be used to intervene the data (here: all unique values $1, 2$ and $3$).
\end{itemize}

\framebreak

\begin{center}
\includegraphics[page=2, width=0.6\textwidth]{figure_man/ice_pd_plot_demo}
\end{center}

\begin{itemize}
\item \textbf{Intervention:} Replace all observed values in $x_1$ for each observation with the previously sampled grid points.
\end{itemize}

\framebreak

\begin{center}
\includegraphics[page=3, width=0.6\textwidth]{figure_man/ice_pd_plot_demo}
\end{center}
\vspace{-0.5cm}
\begin{itemize}
\item \textbf{Prediction:} Make predictions and plot $\fh_{1}^{(i)}(x_1)$ vs. $x_1$, where $$\fh_{1}^{(i)}(x_1) = \fh(x_1, \xi_{2, 3}).$$
\end{itemize}

\framebreak
\begin{onlyenv}<1>
  \vspace{1cm}
    \begin{columns}[T]
    \begin{column}{0.5\textwidth}
  \centering
  \includegraphics[page=5, width=\textwidth]{figure_man/ice_pd_plot_demo}
  \end{column}
 \begin{column}{0.5\textwidth}
  \vspace{0.3cm}

\begin{center}
\includegraphics[width=1\textwidth]{figure_man/ICE02.png}
\end{center}

\end{column}
\end{columns}
\end{onlyenv}

ICE curve for observation $i=1$ connects all predictions at the corresponding grid points associated to the $i$-th observation.

\framebreak
\begin{onlyenv}<1>
  \vspace{1cm}
    \begin{columns}[T]
\begin{column}{0.5\textwidth}
\centering
\includegraphics[page=6, width=\textwidth]{figure_man/ice_pd_plot_demo}
  \end{column}
 \begin{column}{0.5\textwidth}
\vspace{0.3cm}

\begin{center}
\includegraphics[width=1\textwidth]{figure_man/ICE03.png}
\end{center}

\end{column}
\end{columns}
\end{onlyenv}
ICE curve for observation $i=2$ connects all predictions at the corresponding grid points associated to the $i$-th observation.

\framebreak

\begin{onlyenv}<1>
  \vspace{1cm}
    \begin{columns}[T]
\begin{column}{0.5\textwidth}
\centering
\includegraphics[page=7, width=\textwidth]{figure_man/ice_pd_plot_demo}
\end{column}
\begin{column}{0.5\textwidth}
\vspace{0.3cm}

\begin{center}
\includegraphics[width=1\textwidth]{figure_man/ICE04.png}
\end{center}

\end{column}
\end{columns}
\end{onlyenv}

ICE curve for observation $i=3$ connects all predictions at the corresponding grid points associated to the $i$-th observation.
\end{vbframe}


\begin{vbframe}{Partial Dependence}

The partial dependence (PD) plot is the expectation of the ICE curves w.r.t. the marginal distribution of complementary features $\xv_C$,
$$\E_{\xv_C} \left( \fh(\xv_S, \xv_C) \right) = \int_{-\infty}^{\infty} \fh(\xv_S, \xv_C) \, d\mathcal{P}(\xv_C)$$

For a single $x_S$ it is estimated by the point-wise average of the ICE curves:
$$PD_S := \fh_{S}(x_S) = \frac{1}{n} \sum_{i=1}^n \fh(x_S, \xv_C^{(i)})$$

Within the SIPA framework the partial dependence builds the \textbf{Aggregation} step.

\end{vbframe}

\frame{
\frametitle{Partial Dependence}
\begin{onlyenv}
    \begin{columns}[T]
\begin{column}{0.5\textwidth}
\centering
\only<1>{
\includegraphics[page=8, width=\textwidth]{figure_man/ice_pd_plot_demo}
\end{column}
\begin{column}{0.5\textwidth}

\begin{center}
\includegraphics[width=1\textwidth]{figure_man/ICE05.png}
\end{center}

}


\only<2>{
\includegraphics[page=9, width=\textwidth]{figure_man/ice_pd_plot_demo}
\end{column}
\begin{column}{0.5\textwidth}

\begin{center}
\includegraphics[width=1\textwidth]{figure_man/ICE06.png}
\end{center}

}

\only<3>{
\includegraphics[page=10, width=\textwidth]{figure_man/ice_pd_plot_demo}
\end{column}
\begin{column}{0.5\textwidth}

\begin{center}
\includegraphics[width=1\textwidth]{figure_man/ICE07.png}
\end{center}

}


\end{column}
\end{columns}
\end{onlyenv}

\begin{itemize}
\item[2.] \textbf{Aggregation:} Estimate partial dependence by the point-wise average of the ICE curves:
$$PD_1 = \fh_{1}(x_1) = \frac{1}{n} \sum_{i=1}^n \fh(x_1, \xv_{2, 3}^{(i)})$$
\end{itemize}
}

\begin{vbframe}{Bike Sharing Dataset}

\begin{center}
\includegraphics[width=0.8\textwidth]{figure_man/bike-sharing-dataset01.png}
\end{center}

\begin{itemize}
  \item Higher temperature $\Rightarrow$ higher number of bike rentals.
  \item Temperatures of more than $27^\circ$C decreases bike rentals.
  \item The effect is very homogeneous across all observations.
  \item Averaging ICE curves yields a PD plot (yellow curve).
\end{itemize}

\framebreak

\begin{center}
\includegraphics[width=0.7\textwidth]{figure_man/bike-sharing-dataset02.png}
\end{center}

\begin{itemize}
 \item Humidity and temperature interact with each other.
 \item The number of bike rentals is especially high when humidity is below 75 percent and temperature is between 15 $^{\circ}$C and 27 $^{\circ}$C
\end{itemize}
\end{vbframe}

\begin{vbframe}{Extrapolation}

There are two sources of extrapolation:
\lz
\begin{enumerate}
  \item If the model predicts in regions where it was not trained. Predictions in such regions are a bad approximation to the real underlying relationship between input and target space.
  \lz
  \item Averaging ICE curves at specific grid points refers to Monte Carlo integration w.r.t. a   uniform distribution and ignores how likely the data points are.
  It might be better to integrate w.r.t. the (empirical) data distribution.
\end{enumerate}
\lz
$\Rightarrow$ Biased estimates, especially in case of correlated features.

\framebreak

\begin{center}
\includegraphics[width=0.8\textwidth]{figure_man/extrapolation01.png}
\end{center}

\begin{itemize}
\item The features $x_1$ and $x_2$ are strongly correlated.
\item \textcolor{red}{Red:} Observed points of the original data.
\item \textcolor{green}{Green:} Grid points used to calculate the ICE and PD curves.
\end{itemize}
$\Rightarrow$ unrealistic combination of feature values are used.

\framebreak


\begin{center}
\includegraphics[width=0.8\textwidth]{figure_man/extrapolation02.png}
\end{center}

\begin{itemize}
\item The features $x_1$ and $x_2$ are strongly correlated.
\item \textcolor{red}{Red:} Observed points of the original data.
\item \textcolor{green}{Green:} Grid points used to calculate the ICE and PD curves.
\item Example: PD plot at $x_1=1.9$ averages predictions over the whole marginal distribution of feature $x_2$.
\end{itemize}
\end{vbframe}

\begin{vbframe}{Comments}
\begin{itemize}
  \item For PD plots, the averaging of ICE curves might obfuscate heterogeneous effects and interactions. \\
  $\Rightarrow$ Ideally plot ICE curves and PD plots together.
\end{itemize}
 \vspace{-0.3cm}


\begin{center}
\includegraphics[width=0.8\textwidth]{figure_man/comments.png}
\end{center}

\vspace{-0.5cm}
\framebreak

\begin{itemize}
\lz
\item Extrapolation: interprete curves for highly correlated features and in feature regions with few observations with care.
\lz
\item Accumulated Local Effects (ALE) plots are a novel alternative to PD plots developed by Apley (2016) that do not suffer from extrapolation in case of correlated features.
\end{itemize}

\vspace{2cm}
\vfill\tiny
Apley (2016).
Visualizing the effects of predictor variables in black box supervised learning models.
arXiv preprint arXiv:1612.08468.

\end{vbframe}


\endlecture
\end{document}
