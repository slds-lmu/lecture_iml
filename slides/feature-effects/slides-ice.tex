\documentclass[11pt,compress,t,notes=noshow, xcolor=table]{beamer}
\usepackage[]{graphicx}\usepackage[]{color}
% maxwidth is the original width if it is less than linewidth
% otherwise use linewidth (to make sure the graphics do not exceed the margin)
\makeatletter
\def\maxwidth{ %
\ifdim\Gin@nat@width>\linewidth
\linewidth
\else
\Gin@nat@width
\fi
}
\makeatother

\definecolor{fgcolor}{rgb}{0.345, 0.345, 0.345}
\newcommand{\hlnum}[1]{\textcolor[rgb]{0.686,0.059,0.569}{#1}}%
\newcommand{\hlstr}[1]{\textcolor[rgb]{0.192,0.494,0.8}{#1}}%
\newcommand{\hlcom}[1]{\textcolor[rgb]{0.678,0.584,0.686}{\textit{#1}}}%
\newcommand{\hlopt}[1]{\textcolor[rgb]{0,0,0}{#1}}%
\newcommand{\hlstd}[1]{\textcolor[rgb]{0.345,0.345,0.345}{#1}}%
\newcommand{\hlkwa}[1]{\textcolor[rgb]{0.161,0.373,0.58}{\textbf{#1}}}%
\newcommand{\hlkwb}[1]{\textcolor[rgb]{0.69,0.353,0.396}{#1}}%
\newcommand{\hlkwc}[1]{\textcolor[rgb]{0.333,0.667,0.333}{#1}}%
\newcommand{\hlkwd}[1]{\textcolor[rgb]{0.737,0.353,0.396}{\textbf{#1}}}%
\let\hlipl\hlkwb

\usepackage{framed}
\makeatletter
\newenvironment{kframe}{%
\def\at@end@of@kframe{}%
\ifinner\ifhmode%
\def\at@end@of@kframe{\end{minipage}}%
\begin{minipage}{\columnwidth}%
\fi\fi%
\def\FrameCommand##1{\hskip\@totalleftmargin \hskip-\fboxsep
\colorbox{shadecolor}{##1}\hskip-\fboxsep
% There is no \\@totalrightmargin, so:
\hskip-\linewidth \hskip-\@totalleftmargin \hskip\columnwidth}%
\MakeFramed {\advance\hsize-\width
\@totalleftmargin\z@ \linewidth\hsize
\@setminipage}}%
{\par\unskip\endMakeFramed%
\at@end@of@kframe}
\makeatother

\definecolor{shadecolor}{rgb}{.97, .97, .97}
\definecolor{messagecolor}{rgb}{0, 0, 0}
\definecolor{warningcolor}{rgb}{1, 0, 1}
\definecolor{errorcolor}{rgb}{1, 0, 0}
\newenvironment{knitrout}{}{} % an empty environment to be redefined in TeX

\usepackage{alltt}
\newcommand{\SweaveOpts}[1]{}  % do not interfere with LaTeX
\newcommand{\SweaveInput}[1]{} % because they are not real TeX commands
\newcommand{\Sexpr}[1]{}       % will only be parsed by R

\usepackage[english]{babel}
\usepackage[utf8]{inputenc}

\usepackage{dsfont}
\usepackage{verbatim}
\usepackage{amsmath}
\usepackage{amsfonts}
\usepackage{bm}
\usepackage{csquotes}
\usepackage{multirow}
\usepackage{longtable}
\usepackage{booktabs}
\usepackage{enumerate}
\usepackage[absolute,overlay]{textpos}
\usepackage{psfrag}
\usepackage{algorithm}
\usepackage{algpseudocode}
\usepackage{eqnarray}
\usepackage{arydshln}
\usepackage{tabularx}
\usepackage{placeins}
\usepackage{tikz}
\usepackage{setspace}
\usepackage{colortbl}
\usepackage{mathtools}
\usepackage{wrapfig}
\usepackage{bm}

\usetikzlibrary{shapes,arrows,automata,positioning,calc,chains,trees, shadows}
\tikzset{
%Define standard arrow tip
>=stealth',
%Define style for boxes
punkt/.style={
rectangle,
rounded corners,
draw=black, very thick,
text width=6.5em,
minimum height=2em,
text centered},
% Define arrow style
pil/.style={
->,
thick,
shorten <=2pt,
shorten >=2pt,}
}

\usepackage{subfig}

% Defines macros and environments
\usepackage{bbm}
% basic latex stuff
\newcommand{\pkg}[1]{{\fontseries{b}\selectfont #1}} %fontstyle for R packages
\newcommand{\lz}{\vspace{0.5cm}} %vertical space
\newcommand{\dlz}{\vspace{1cm}} %double vertical space
\newcommand{\oneliner}[1] % Oneliner for important statements
{\begin{block}{}\begin{center}\begin{Large}#1\end{Large}\end{center}\end{block}}


%new environments
\newenvironment{vbframe}  %frame with breaks and verbatim
{
 \begin{frame}[containsverbatim,allowframebreaks]
}
{
\end{frame}
}

\newenvironment{vframe}  %frame with verbatim without breaks (to avoid numbering one slided frames)
{
 \begin{frame}[containsverbatim]
}
{
\end{frame}
}

\newenvironment{blocki}[1]   % itemize block
{
 \begin{block}{#1}\begin{itemize}
}
{
\end{itemize}\end{block}
}

\newenvironment{fragileframe}[2]{  %fragile frame with framebreaks
\begin{frame}[allowframebreaks, fragile, environment = fragileframe]
\frametitle{#1}
#2}
{\end{frame}}


\newcommand{\myframe}[2]{  %short for frame with framebreaks
\begin{frame}[allowframebreaks]
\frametitle{#1}
#2
\end{frame}}

\newcommand{\remark}[1]{
  \textbf{Remark:} #1
}


\newenvironment{deleteframe}
{
\begingroup
\usebackgroundtemplate{\includegraphics[width=\paperwidth,height=\paperheight]{../style/color/red.png}}
 \begin{frame}
}
{
\end{frame}
\endgroup
}
\newenvironment{simplifyframe}
{
\begingroup
\usebackgroundtemplate{\includegraphics[width=\paperwidth,height=\paperheight]{../style/color/yellow.png}}
 \begin{frame}
}
{
\end{frame}
\endgroup
}\newenvironment{draftframe}
{
\begingroup
\usebackgroundtemplate{\includegraphics[width=\paperwidth,height=\paperheight]{../style/color/green.jpg}}
 \begin{frame}
}
{
\end{frame}
\endgroup
}
% https://tex.stackexchange.com/a/261480: textcolor that works in mathmode
\makeatletter
\renewcommand*{\@textcolor}[3]{%
  \protect\leavevmode
  \begingroup
    \color#1{#2}#3%
  \endgroup
}
\makeatother


\providecommand{\tightlist}{%
  \setlength{\itemsep}{0pt}\setlength{\parskip}{0pt}}

%\setbeamerfont{footnote}{size=\tiny}
\usepackage[hang,flushmargin]{footmisc}
\renewcommand*{\footnotelayout}{\tiny}
\renewcommand*{\thefootnote}{} %\fnsymbol{footnote}

% https://tex.stackexchange.com/questions/30720/footnote-without-a-marker
% \makeatletter
% \def\blfootnote{\gdef\@thefnmark{}\@footnotetext}
% \makeatother

% https://tex.stackexchange.com/questions/357717/beamer-allowframebreaks-option-and-vertical-spacing-when-using-lists-itemize
% \setbeamertemplate{frametitle continuation}{%
%     (\insertcontinuationcount)%
%     \ifnum\insertcontinuationcount>1%
%     \vspace*{\topsep}%
%     \else%
%     %
%     \fi%
% }


%\usetheme{lmu-lecture}
\newcommand{\titlefigure}{figure_man/bike-sharing-dataset01.png}
\newcommand{\learninggoals}{
\item Motivation behind feature effects
\item Understand the steps involved in constructing ICE plots
%\item Understand how to interpret ICE curves and PD plots
}
\usepackage{../../style/lmu-lecture}

\let\code=\texttt
\let\proglang=\textsf

\setkeys{Gin}{width=0.9\textwidth}

\title{Interpretable Machine Learning}
% \author{Bernd Bischl, Christoph Molnar, Daniel Schalk, Fabian Scheipl}
\institute{\href{https://compstat-lmu.github.io/lecture_iml/}{compstat-lmu.github.io/lecture\_iml}}
\date{}

\setbeamertemplate{frametitle}{\expandafter\uppercase\expandafter\insertframetitle}

\begin{document}

\input{../../latex-math/basic-math}
\input{../../latex-math/basic-ml}

\lecturechapter{Individual Conditional Expectation (ICE) Plot}
\lecture{Interpretable Machine Learning}


\begin{vbframe}{Feature Effects}

\textbf{Feature Effects} visualize or quantify the (average) relationship between the features and the model predictions. %marginal contribution of a feature to the model \textbf{prediction}. % $\hat{y} = \fh(\xi[])$.
\begin{itemize}
%\tightlist
\item Methods: PD Plots, ICE curves, ALE plots
\item Similar to regression coefficients (LMs) or Splines (GAMs).
\end{itemize}

\centerline{\includegraphics[width=\textwidth]{figure_man/feature-effects.pdf}}

\hspace{8px} \small Individual (curves) \hspace{2px}
$\xrightarrow[]{\text{aggregate}}$ \hspace{2px} Global (curve) \hspace{2px}
$\xrightarrow[]{\text{aggregate}}$ \hspace{2px} Global (number)

\framebreak

Why visualizations of feature effects are meaningful in ML.

\lz

\end{vbframe}

\begin{vbframe}{Individual Conditional Expectation (ICE)}

%Consider an index set $S \subseteq \{1, \dots, p\}$ and its complement $C = S^\complement$.
Assume each observation $\xi$ can be partitioned into $\xi_S$ and $\xi_C$ containing only feature values addressed by the feature's index set $S \subseteq \{1, \dots, p\}$ and its complement $C = S^\complement$, respectively.

\lz

Individual conditional expectation (ICE) curves visualize how the model prediction of individual observations $\xi$
change by varying the feature values in $\xv_S$ while keeping all other features in $\xi_C$ fixed.
$$\fh_{S}^{(i)}(\xv_S) = \fh(\xv_S, \xi_C)$$

In practice, $\xv_S$ consists of one or two features.

\vspace{-0.2cm}
\begin{center}
\includegraphics[width=0.7\textwidth]{figure_man/ICE01.png}
\end{center}
\vspace{-0.3cm}
\scriptsize{\textbf{Figure:} ICE Curves of the last 10 observations of the bike sharing dataset. Each line displays the change in prediction for a single observation due to varying the feature temperature.\par}

\footnote[frame]{Goldstein, A., Kapelner, A., Bleich, J., and Pitkin, E. (2013). Peeking Inside the Black Box: Visualizing Statistical Learning with Plots of Individual Conditional Expectation, 1-22. https://doi.org/10.1080/10618600.2014.907095}
\end{vbframe}

\begin{vbframe}{Individual Conditional Expectation (ICE)}

Steps to create an ICE curve of an observation regarding a single feature $x_S$ according to the \textbf{SIPA} framework:

\begin{enumerate}
\item \textbf{Sampling:} Choose grid points along $x_S$.
\item For each grid point:
\begin{itemize}
\item \textbf{Intervention:} Replace the original feature value $x_S$ with the current grid value.
\item \textbf{Prediction:} Get the model prediction with replaced feature value $x_S$.
\end{itemize}
\item \textbf{Aggregation:} none.
\item \textbf{Visualization:} Draw a curve per observation with the grid points on the x-axis and the prediction on the y-axis.
\end{enumerate}

\footnote[frame]{Scholbeck, C. A., Molnar, C., Heumann, C., Bischl, B., and Casalicchio, G. (2019). Sampling, Intervention, Prediction, Aggregation: A Generalized Framework for Model Agnostic Interpretations. ECML PKDD 2019. (pp. 205-216).}
\end{vbframe}

\begin{vbframe}{Individual Conditional Expectation (ICE)}
%\vspace*{-\topsep}
\vspace*{-\lineskip}
%\vspace*{-\baselineskip}

%\begin{center}
\includegraphics[page=1, width=0.55\textwidth]{figure_man/ice_pd_plot_demo}
%\end{center}

\textbf{Sampling:} Choose grid points along $x_1  $ that will be used to intervene the data (here: all unique values $1, 2$ and $3$).

\framebreak

%\begin{center}
\includegraphics[page=2, width=0.55\textwidth]{figure_man/ice_pd_plot_demo}
%\end{center}

\textbf{Intervention:} Replace all observed values in $x_1$ for each observation with the previously sampled grid points.

\framebreak

%\begin{center}
\includegraphics[page=3, width=0.55\textwidth]{figure_man/ice_pd_plot_demo}
%\end{center}

\textbf{Prediction:} Make predictions and plot $\fh_{1}^{(i)}(x_1)$ vs. $x_1$, where $$\fh_{1}^{(i)}(x_1) = \fh(x_1, \xi_{2, 3}).$$

\framebreak

%\vspace*{-\baselineskip}
%\vspace{1cm}
\begin{columns}[T]
\begin{column}{0.55\textwidth}
\vspace*{-\topsep}
\vspace*{0.5\lineskip}
%\centering
\includegraphics[page=5, trim=-3.69cm 0cm 3.69cm 0cm, width=\textwidth]{figure_man/ice_pd_plot_demo}
\end{column}
\begin{column}{0.45\textwidth}
%\vspace{0.3cm}

\begin{center}
\includegraphics[page=1, width=\textwidth]{figure_man/ICE}
\end{center}

\end{column}
\end{columns}
\vspace*{\topsep}

\textbf{Visualization:} ICE curve for observation $i=1$ connects all predictions at the corresponding grid points associated to the $i$-th observation.

\framebreak

%\vspace{1cm}
\begin{columns}[T]
\begin{column}{0.55\textwidth}
\vspace*{-\topsep}
\vspace*{0.5\lineskip}
%\centering
\includegraphics[page=6, trim=-3.69cm 0cm 3.69cm 0cm, width=\textwidth]{figure_man/ice_pd_plot_demo}
\end{column}
\begin{column}{0.45\textwidth}
%\vspace{0.3cm}

\begin{center}
\includegraphics[page=2, width=\textwidth]{figure_man/ICE}
\end{center}

\end{column}
\end{columns}
\vspace*{\topsep}

\textbf{Visualization:} ICE curve for observation $i=2$ connects all predictions at the corresponding grid points associated to the $i$-th observation.

\framebreak

%\vspace{1cm}
\begin{columns}[T]
\begin{column}{0.55\textwidth}
%\centering
\vspace*{-\topsep}
\vspace*{0.5\lineskip}
\includegraphics[page=7, trim=-3.69cm 0cm 3.69cm 0cm, width=\textwidth]{figure_man/ice_pd_plot_demo}
\end{column}
\begin{column}{0.45\textwidth}
%\vspace{0.3cm}

\begin{center}
\includegraphics[page=3, width=\textwidth]{figure_man/ICE}
\end{center}

\end{column}
\end{columns}
\vspace*{\topsep}

\textbf{Visualization:} ICE curve for observation $i=3$ connects all predictions at the corresponding grid points associated to the $i$-th observation.
\end{vbframe}

\endlecture
\end{document}
