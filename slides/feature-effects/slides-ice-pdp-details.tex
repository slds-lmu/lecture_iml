\documentclass[11pt,compress,t,notes=noshow, xcolor=table]{beamer}
\usepackage[]{graphicx}\usepackage[]{color}
% maxwidth is the original width if it is less than linewidth
% otherwise use linewidth (to make sure the graphics do not exceed the margin)
\makeatletter
\def\maxwidth{ %
  \ifdim\Gin@nat@width>\linewidth
    \linewidth
  \else
    \Gin@nat@width
  \fi
}
\makeatother

\definecolor{fgcolor}{rgb}{0.345, 0.345, 0.345}
\newcommand{\hlnum}[1]{\textcolor[rgb]{0.686,0.059,0.569}{#1}}%
\newcommand{\hlstr}[1]{\textcolor[rgb]{0.192,0.494,0.8}{#1}}%
\newcommand{\hlcom}[1]{\textcolor[rgb]{0.678,0.584,0.686}{\textit{#1}}}%
\newcommand{\hlopt}[1]{\textcolor[rgb]{0,0,0}{#1}}%
\newcommand{\hlstd}[1]{\textcolor[rgb]{0.345,0.345,0.345}{#1}}%
\newcommand{\hlkwa}[1]{\textcolor[rgb]{0.161,0.373,0.58}{\textbf{#1}}}%
\newcommand{\hlkwb}[1]{\textcolor[rgb]{0.69,0.353,0.396}{#1}}%
\newcommand{\hlkwc}[1]{\textcolor[rgb]{0.333,0.667,0.333}{#1}}%
\newcommand{\hlkwd}[1]{\textcolor[rgb]{0.737,0.353,0.396}{\textbf{#1}}}%
\let\hlipl\hlkwb

\usepackage{framed}
\makeatletter
\newenvironment{kframe}{%
 \def\at@end@of@kframe{}%
 \ifinner\ifhmode%
  \def\at@end@of@kframe{\end{minipage}}%
  \begin{minipage}{\columnwidth}%
 \fi\fi%
 \def\FrameCommand##1{\hskip\@totalleftmargin \hskip-\fboxsep
 \colorbox{shadecolor}{##1}\hskip-\fboxsep
     % There is no \\@totalrightmargin, so:
     \hskip-\linewidth \hskip-\@totalleftmargin \hskip\columnwidth}%
 \MakeFramed {\advance\hsize-\width
   \@totalleftmargin\z@ \linewidth\hsize
   \@setminipage}}%
 {\par\unskip\endMakeFramed%
 \at@end@of@kframe}
\makeatother

\definecolor{shadecolor}{rgb}{.97, .97, .97}
\definecolor{messagecolor}{rgb}{0, 0, 0}
\definecolor{warningcolor}{rgb}{1, 0, 1}
\definecolor{errorcolor}{rgb}{1, 0, 0}
\newenvironment{knitrout}{}{} % an empty environment to be redefined in TeX

\usepackage{alltt}
\newcommand{\SweaveOpts}[1]{}  % do not interfere with LaTeX
\newcommand{\SweaveInput}[1]{} % because they are not real TeX commands
\newcommand{\Sexpr}[1]{}       % will only be parsed by R

\usepackage[english]{babel}
\usepackage[utf8]{inputenc}

\usepackage{dsfont}
\usepackage{verbatim}
\usepackage{amsmath}
\usepackage{amsfonts}
\usepackage{bm}
\usepackage{csquotes}
\usepackage{multirow}
\usepackage{longtable}
\usepackage{booktabs}
\usepackage{enumerate}
\usepackage[absolute,overlay]{textpos}
\usepackage{psfrag}
\usepackage{algorithm}
\usepackage{algpseudocode}
\usepackage{eqnarray}
\usepackage{arydshln}
\usepackage{tabularx}
\usepackage{placeins}
\usepackage{tikz}
\usepackage{setspace}
\usepackage{colortbl}
\usepackage{mathtools}
\usepackage{wrapfig}
\usepackage{bm}

\usetikzlibrary{shapes,arrows,automata,positioning,calc,chains,trees, shadows}
\tikzset{
  %Define standard arrow tip
  >=stealth',
  %Define style for boxes
  punkt/.style={
    rectangle,
    rounded corners,
    draw=black, very thick,
    text width=6.5em,
    minimum height=2em,
    text centered},
  % Define arrow style
  pil/.style={
    ->,
    thick,
    shorten <=2pt,
    shorten >=2pt,}
}

\usepackage{subfig}

% Defines macros and environments
\usepackage{bbm}
% basic latex stuff
\newcommand{\pkg}[1]{{\fontseries{b}\selectfont #1}} %fontstyle for R packages
\newcommand{\lz}{\vspace{0.5cm}} %vertical space
\newcommand{\dlz}{\vspace{1cm}} %double vertical space
\newcommand{\oneliner}[1] % Oneliner for important statements
{\begin{block}{}\begin{center}\begin{Large}#1\end{Large}\end{center}\end{block}}


%new environments
\newenvironment{vbframe}  %frame with breaks and verbatim
{
 \begin{frame}[containsverbatim,allowframebreaks]
}
{
\end{frame}
}

\newenvironment{vframe}  %frame with verbatim without breaks (to avoid numbering one slided frames)
{
 \begin{frame}[containsverbatim]
}
{
\end{frame}
}

\newenvironment{blocki}[1]   % itemize block
{
 \begin{block}{#1}\begin{itemize}
}
{
\end{itemize}\end{block}
}

\newenvironment{fragileframe}[2]{  %fragile frame with framebreaks
\begin{frame}[allowframebreaks, fragile, environment = fragileframe]
\frametitle{#1}
#2}
{\end{frame}}


\newcommand{\myframe}[2]{  %short for frame with framebreaks
\begin{frame}[allowframebreaks]
\frametitle{#1}
#2
\end{frame}}

\newcommand{\remark}[1]{
  \textbf{Remark:} #1
}


\newenvironment{deleteframe}
{
\begingroup
\usebackgroundtemplate{\includegraphics[width=\paperwidth,height=\paperheight]{../style/color/red.png}}
 \begin{frame}
}
{
\end{frame}
\endgroup
}
\newenvironment{simplifyframe}
{
\begingroup
\usebackgroundtemplate{\includegraphics[width=\paperwidth,height=\paperheight]{../style/color/yellow.png}}
 \begin{frame}
}
{
\end{frame}
\endgroup
}\newenvironment{draftframe}
{
\begingroup
\usebackgroundtemplate{\includegraphics[width=\paperwidth,height=\paperheight]{../style/color/green.jpg}}
 \begin{frame}
}
{
\end{frame}
\endgroup
}
% https://tex.stackexchange.com/a/261480: textcolor that works in mathmode
\makeatletter
\renewcommand*{\@textcolor}[3]{%
  \protect\leavevmode
  \begingroup
    \color#1{#2}#3%
  \endgroup
}
\makeatother


\providecommand{\tightlist}{%
  \setlength{\itemsep}{0pt}\setlength{\parskip}{0pt}}

%\setbeamerfont{footnote}{size=\tiny}
\usepackage[hang,flushmargin]{footmisc}
\renewcommand*{\footnotelayout}{\tiny}
\renewcommand*{\thefootnote}{} %\fnsymbol{footnote}

% https://tex.stackexchange.com/questions/30720/footnote-without-a-marker
% \makeatletter
% \def\blfootnote{\gdef\@thefnmark{}\@footnotetext}
% \makeatother

% https://tex.stackexchange.com/questions/357717/beamer-allowframebreaks-option-and-vertical-spacing-when-using-lists-itemize
% \setbeamertemplate{frametitle continuation}{%
%     (\insertcontinuationcount)%
%     \ifnum\insertcontinuationcount>1%
%     \vspace*{\topsep}%
%     \else%
%     %
%     \fi%
% }


%\usetheme{lmu-lecture}
\newcommand{\titlefigure}{figure_man/bike-sharing-dataset01.png}
\newcommand{\learninggoals}{
\item Understand c-ICE curves to identify the heterogeneity in the model
%\item Understand the usefullness of ICE curves and d-ICE curves in case of interaction effects
\item Understand the extrapolation issue}
\usepackage{../../style/lmu-lecture}

\let\code=\texttt
\let\proglang=\textsf

\setkeys{Gin}{width=0.9\textwidth}

\title{Interpretable Machine Learning}
% \author{Bernd Bischl, Christoph Molnar, Daniel Schalk, Fabian Scheipl}
\institute{\href{https://compstat-lmu.github.io/lecture_iml/}{compstat-lmu.github.io/lecture\_iml}}
\date{}

\setbeamertemplate{frametitle}{\expandafter\uppercase\expandafter\insertframetitle}

\begin{document}

\input{../../latex-math/basic-math}
\input{../../latex-math/basic-ml}

\lecturechapter{Issues and Extensions of ICE and PD Plots}
\lecture{Interpretable Machine Learning}


\begin{vbframe}{Centered ICE Plot (c-ICE)}

\textbf{Issue:} When ICE curves start at different intercepts and hence are stacked, it is difficult to identify heterogenous predictions.

\textbf{Solution:} Center all ICE curves at a fixed reference value $x^* \sim \P(\xv_S)$.


$$\begin{aligned}
\fh_{S, cICE}^{(i)}(\xv_S)
&= \fh(\xv_S, \xi_C) - \fh(x^*, \xi_C) \\
&= \fh_{S}^{(i)}(\xv_S) - \fh_{S}^{(i)}(x^*)
\end{aligned}$$


\begin{center}
\includegraphics[width=\textwidth]{figure_man/cICE}
\end{center}

\framebreak

\begin{center}
\includegraphics[width=0.8\textwidth]{figure_man/cICEcat}
\end{center}

\begin{itemize}
\item Centered ICE plots are also usefull for categorical features and can be interpreted as in linear regression models.
\item The reference category is $x^* =$ SPRING.
\item Golden crosses refer to the expected number of bike rentals if we move from season category SPRING to any other category.
\end{itemize}

\end{vbframe}


% \frame{
% \frametitle{Derivative ICE Plot (d-ICE)}
%
% Aim: Interactions
%
% }

\begin{vbframe}{Extrapolation}

There are two sources of extrapolation:
\lz
\begin{enumerate}
  \item If the model predicts in regions where it was not trained. Predictions in such regions are a bad approximation to the real underlying relationship between input and target space.
  \lz
  \item Averaging ICE curves at specific grid points refers to Monte Carlo integration w.r.t. a   uniform distribution and ignores how likely the data points are.
  It might be better to integrate w.r.t. the (empirical) data distribution.
\end{enumerate}
\lz
$\Rightarrow$ Biased estimates, especially in case of correlated features.

\framebreak

% \begin{center}
% \includegraphics[width=0.8\textwidth]{figure_man/extrapolation01.png}
% \end{center}

\begin{columns}[T]
\begin{column}{0.5\textwidth}
\centering
\includegraphics[width=\textwidth]{figure_man/ale_scatter_grid}
\end{column}
\begin{column}{0.5\textwidth}
\centering
\includegraphics[width=\textwidth]{figure_man/ale_pdplot}
\end{column}
\end{columns}

\begin{itemize}
\item The features $x_1$ and $x_2$ are strongly correlated.
\item \textbf{Black points:} Observed points of the original data.
\item \textbf{\textcolor{red}{Red:}} Grid points used to calculate the ICE and PD curves.
\end{itemize}
$\Rightarrow$ Unrealistic combination of feature values are used, e.g., the PD plot at $x_1=0$ averages predictions over the whole marginal distribution of feature $x_2$.
%
% \framebreak
%
%
% \begin{center}
% \includegraphics[width=0.8\textwidth]{figure_man/extrapolation02.png}
% \end{center}
%
% \begin{itemize}
% \item The features $x_1$ and $x_2$ are strongly correlated.
% \item \textcolor{red}{Red:} Observed points of the original data.
% \item \textcolor{green}{Green:} Grid points used to calculate the ICE and PD curves.
% \item Example: PD plot at $x_1=1.9$ averages predictions over the whole marginal distribution of feature $x_2$.
% \end{itemize}
\end{vbframe}




% \begin{vbframe}{Interactions}
%
% For PD plots, the averaging of ICE curves might \textbf{obfuscate} heterogeneous effects and interactions. \newline \(\Rightarrow\) Ideally plot ICE curves and PD plots together.
%
% \begin{center}\includegraphics[width=0.65\textwidth]{figure_man/pdp_xor.pdf} \end{center}
% %
% % \framebreak
% %
% % \begin{itemize}
% % \item
% %   For PD plots, the averaging of ICE curves might \textbf{obfuscate}
% %   heterogeneous effects and interactions. \newline \(\Rightarrow\)
% %   Ideally plot ICE curves and PD plots together.
% % \item
% %   \textbf{Extrapolation:} Interprete curves for highly correlated
% %   features and in feature regions with few observations with care.
% % \item
% %   Accumulated Local Effects (ALE) plots are a novel alternative to the PD plots developed by Apley (2020) that do not suffer from
% %   extrapolation in case of correlated features.
% % \end{itemize}
% %
% % \vspace{80pt}
% % \tiny{
% % Apley, D. W., \& Zhu, J. (2020). Visualizing the effects of predictor variables in black box supervised learning models. Journal of the Royal Statistical Society: Series B, 82(4), 1059-1086. \par}
% \end{vbframe}

\begin{vbframe}{Comments}
\begin{itemize}
\item Extrapolation: interprete curves for highly correlated features and in feature regions with few observations with care.
%\lz
%\item Accumulated Local Effects (ALE) plots are a novel alternative to PD plots developed by Apley (2016) that do not suffer from extrapolation in case of correlated features.

\item For PD plots, the averaging of ICE curves might \textbf{obfuscate} heterogeneous effects and interactions. \newline \(\Rightarrow\) Ideally plot ICE curves and PD plots together.

\begin{center}\includegraphics[width=0.65\textwidth]{figure_man/pdp_xor.pdf} \end{center}
\end{itemize}

%\footnote[frame]{Apley, Daniel W., and Jingyu Zhu (2020). Visualizing the Effects of Predictor Variables in Black Box Supervised Learning Models. Journal of the Royal Statistical Society: Series B (Statistical Methodology) 82.4: 1059-1086.}
\end{vbframe}


\endlecture
\end{document}
