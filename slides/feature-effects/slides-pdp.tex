\documentclass[11pt,compress,t,notes=noshow, xcolor=table]{beamer}
\usepackage[]{graphicx}\usepackage[]{color}
% maxwidth is the original width if it is less than linewidth
% otherwise use linewidth (to make sure the graphics do not exceed the margin)
\makeatletter
\def\maxwidth{ %
  \ifdim\Gin@nat@width>\linewidth
    \linewidth
  \else
    \Gin@nat@width
  \fi
}
\makeatother

\definecolor{fgcolor}{rgb}{0.345, 0.345, 0.345}
\newcommand{\hlnum}[1]{\textcolor[rgb]{0.686,0.059,0.569}{#1}}%
\newcommand{\hlstr}[1]{\textcolor[rgb]{0.192,0.494,0.8}{#1}}%
\newcommand{\hlcom}[1]{\textcolor[rgb]{0.678,0.584,0.686}{\textit{#1}}}%
\newcommand{\hlopt}[1]{\textcolor[rgb]{0,0,0}{#1}}%
\newcommand{\hlstd}[1]{\textcolor[rgb]{0.345,0.345,0.345}{#1}}%
\newcommand{\hlkwa}[1]{\textcolor[rgb]{0.161,0.373,0.58}{\textbf{#1}}}%
\newcommand{\hlkwb}[1]{\textcolor[rgb]{0.69,0.353,0.396}{#1}}%
\newcommand{\hlkwc}[1]{\textcolor[rgb]{0.333,0.667,0.333}{#1}}%
\newcommand{\hlkwd}[1]{\textcolor[rgb]{0.737,0.353,0.396}{\textbf{#1}}}%
\let\hlipl\hlkwb

\usepackage{framed}
\makeatletter
\newenvironment{kframe}{%
 \def\at@end@of@kframe{}%
 \ifinner\ifhmode%
  \def\at@end@of@kframe{\end{minipage}}%
  \begin{minipage}{\columnwidth}%
 \fi\fi%
 \def\FrameCommand##1{\hskip\@totalleftmargin \hskip-\fboxsep
 \colorbox{shadecolor}{##1}\hskip-\fboxsep
     % There is no \\@totalrightmargin, so:
     \hskip-\linewidth \hskip-\@totalleftmargin \hskip\columnwidth}%
 \MakeFramed {\advance\hsize-\width
   \@totalleftmargin\z@ \linewidth\hsize
   \@setminipage}}%
 {\par\unskip\endMakeFramed%
 \at@end@of@kframe}
\makeatother

\definecolor{shadecolor}{rgb}{.97, .97, .97}
\definecolor{messagecolor}{rgb}{0, 0, 0}
\definecolor{warningcolor}{rgb}{1, 0, 1}
\definecolor{errorcolor}{rgb}{1, 0, 0}
\newenvironment{knitrout}{}{} % an empty environment to be redefined in TeX

\usepackage{alltt}
\newcommand{\SweaveOpts}[1]{}  % do not interfere with LaTeX
\newcommand{\SweaveInput}[1]{} % because they are not real TeX commands
\newcommand{\Sexpr}[1]{}       % will only be parsed by R

\usepackage[english]{babel}
\usepackage[utf8]{inputenc}

\usepackage{dsfont}
\usepackage{verbatim}
\usepackage{amsmath}
\usepackage{amsfonts}
\usepackage{bm}
\usepackage{csquotes}
\usepackage{multirow}
\usepackage{longtable}
\usepackage{booktabs}
\usepackage{enumerate}
\usepackage[absolute,overlay]{textpos}
\usepackage{psfrag}
\usepackage{algorithm}
\usepackage{algpseudocode}
\usepackage{eqnarray}
\usepackage{arydshln}
\usepackage{tabularx}
\usepackage{placeins}
\usepackage{tikz}
\usepackage{setspace}
\usepackage{colortbl}
\usepackage{mathtools}
\usepackage{wrapfig}
\usepackage{bm}

\usetikzlibrary{shapes,arrows,automata,positioning,calc,chains,trees, shadows}
\tikzset{
  %Define standard arrow tip
  >=stealth',
  %Define style for boxes
  punkt/.style={
    rectangle,
    rounded corners,
    draw=black, very thick,
    text width=6.5em,
    minimum height=2em,
    text centered},
  % Define arrow style
  pil/.style={
    ->,
    thick,
    shorten <=2pt,
    shorten >=2pt,}
}

\usepackage{subfig}

% Defines macros and environments
\usepackage{bbm}
% basic latex stuff
\newcommand{\pkg}[1]{{\fontseries{b}\selectfont #1}} %fontstyle for R packages
\newcommand{\lz}{\vspace{0.5cm}} %vertical space
\newcommand{\dlz}{\vspace{1cm}} %double vertical space
\newcommand{\oneliner}[1] % Oneliner for important statements
{\begin{block}{}\begin{center}\begin{Large}#1\end{Large}\end{center}\end{block}}


%new environments
\newenvironment{vbframe}  %frame with breaks and verbatim
{
 \begin{frame}[containsverbatim,allowframebreaks]
}
{
\end{frame}
}

\newenvironment{vframe}  %frame with verbatim without breaks (to avoid numbering one slided frames)
{
 \begin{frame}[containsverbatim]
}
{
\end{frame}
}

\newenvironment{blocki}[1]   % itemize block
{
 \begin{block}{#1}\begin{itemize}
}
{
\end{itemize}\end{block}
}

\newenvironment{fragileframe}[2]{  %fragile frame with framebreaks
\begin{frame}[allowframebreaks, fragile, environment = fragileframe]
\frametitle{#1}
#2}
{\end{frame}}


\newcommand{\myframe}[2]{  %short for frame with framebreaks
\begin{frame}[allowframebreaks]
\frametitle{#1}
#2
\end{frame}}

\newcommand{\remark}[1]{
  \textbf{Remark:} #1
}


\newenvironment{deleteframe}
{
\begingroup
\usebackgroundtemplate{\includegraphics[width=\paperwidth,height=\paperheight]{../style/color/red.png}}
 \begin{frame}
}
{
\end{frame}
\endgroup
}
\newenvironment{simplifyframe}
{
\begingroup
\usebackgroundtemplate{\includegraphics[width=\paperwidth,height=\paperheight]{../style/color/yellow.png}}
 \begin{frame}
}
{
\end{frame}
\endgroup
}\newenvironment{draftframe}
{
\begingroup
\usebackgroundtemplate{\includegraphics[width=\paperwidth,height=\paperheight]{../style/color/green.jpg}}
 \begin{frame}
}
{
\end{frame}
\endgroup
}
% https://tex.stackexchange.com/a/261480: textcolor that works in mathmode
\makeatletter
\renewcommand*{\@textcolor}[3]{%
  \protect\leavevmode
  \begingroup
    \color#1{#2}#3%
  \endgroup
}
\makeatother


\providecommand{\tightlist}{%
  \setlength{\itemsep}{0pt}\setlength{\parskip}{0pt}}

%\setbeamerfont{footnote}{size=\tiny}
\usepackage[hang,flushmargin]{footmisc}
\renewcommand*{\footnotelayout}{\tiny}
\renewcommand*{\thefootnote}{} %\fnsymbol{footnote}

% https://tex.stackexchange.com/questions/30720/footnote-without-a-marker
% \makeatletter
% \def\blfootnote{\gdef\@thefnmark{}\@footnotetext}
% \makeatother

% https://tex.stackexchange.com/questions/357717/beamer-allowframebreaks-option-and-vertical-spacing-when-using-lists-itemize
% \setbeamertemplate{frametitle continuation}{%
%     (\insertcontinuationcount)%
%     \ifnum\insertcontinuationcount>1%
%     \vspace*{\topsep}%
%     \else%
%     %
%     \fi%
% }


%\usetheme{lmu-lecture}
\newcommand{\titlefigure}{figure_man/bike-sharing-dataset01.png}
\newcommand{\learninggoals}{
\item Understand PD plots and their relation to ICE plots
\item Understand how to interpret ICE curves and PD plots
}
\usepackage{../../style/lmu-lecture}

\let\code=\texttt
\let\proglang=\textsf

\setkeys{Gin}{width=0.9\textwidth}

\title{Interpretable Machine Learning}
% \author{Bernd Bischl, Christoph Molnar, Daniel Schalk, Fabian Scheipl}
\institute{\href{https://compstat-lmu.github.io/lecture_iml/}{compstat-lmu.github.io/lecture\_iml}}
\date{}

\setbeamertemplate{frametitle}{\expandafter\uppercase\expandafter\insertframetitle}

\begin{document}

\input{../../latex-math/basic-math}
\input{../../latex-math/basic-ml}

\lecturechapter{Partial Dependence (PD) plot}
\lecture{Interpretable Machine Learning}


\begin{vbframe}{Partial Dependence}

The partial dependence (PD) plot is the expectation of the ICE curves w.r.t. the marginal distribution of complementary features $\xv_C$,
$$\E_{\xv_C} \left( \fh(\xv_S, \xv_C) \right) = \int_{-\infty}^{\infty} \fh(\xv_S, \xv_C) \, d\mathcal{P}(\xv_C)$$

For a single $x_S$ it is estimated by the point-wise average of the ICE curves:
$$PD_S := \fh_{S}(x_S) = \frac{1}{n} \sum_{i=1}^n \fh(x_S, \xv_C^{(i)})$$

Within the SIPA framework, the partial dependence builds the \textbf{Aggregation} step.

\footnote[frame]{Friedman, Jerome H. (2001). Greedy Function Approximation: A Gradient Boosting Machine. Annals of Statistics: 1189-1232.}
\footnote[frame]{Scholbeck, C. A., Molnar, C., Heumann, C., Bischl, B., and Casalicchio, G. (2019). Sampling, Intervention, Prediction, Aggregation: A Generalized Framework for Model Agnostic Interpretations. ECML PKDD 2019. (pp. 205-216).}
\end{vbframe}

\frame{
\frametitle{Partial Dependence}
%\begin{onlyenv}
\begin{columns}[T]
\begin{column}{0.5\textwidth}
\centering
\only<1>{
\includegraphics[page=8, width=\textwidth]{figure_man/ice_pd_plot_demo}
\end{column}
\begin{column}{0.5\textwidth}

\begin{center}
\includegraphics[page=1, width=\textwidth]{figure_man/PD}
\end{center}
}

\only<2>{
\includegraphics[page=9, width=\textwidth]{figure_man/ice_pd_plot_demo}
\end{column}
\begin{column}{0.5\textwidth}

\begin{center}
\includegraphics[page=2, width=\textwidth]{figure_man/PD}
\end{center}
}

\only<3>{
\includegraphics[page=10, width=\textwidth]{figure_man/ice_pd_plot_demo}
\end{column}
\begin{column}{0.5\textwidth}

\begin{center}
\includegraphics[page=3, width=\textwidth]{figure_man/PD}
\end{center}
}
\end{column}
\end{columns}
%\end{onlyenv}

\only<1>{
\textbf{Aggregation:} Estimate partial dependence by the point-wise average of the ICE curves at \fcolorbox{red}{white}{$x_S = x_1 = 1$}:
$$PD_1 = \fh_{1}(x_1) = \frac{1}{n} \sum_{i=1}^n \fh(x_1, \xv_{2, 3}^{(i)})$$
}
\only<2>{
\textbf{Aggregation:} Estimate partial dependence by the point-wise average of the ICE curves at \fcolorbox{red}{white}{$x_S = x_1 = 2$}:
$$PD_1 = \fh_{1}(x_1) = \frac{1}{n} \sum_{i=1}^n \fh(x_1, \xv_{2, 3}^{(i)})$$
}
\only<3>{
\textbf{Aggregation:} Estimate partial dependence by the point-wise average of the ICE curves at \fcolorbox{red}{white}{$x_S = x_1 = 3$}:
$$PD_1 = \fh_{1}(x_1) = \frac{1}{n} \sum_{i=1}^n \fh(x_1, \xv_{2, 3}^{(i)})$$
}
}


\frame{
\frametitle{Interpretation: PD and ICE}
\begin{center}
\includegraphics[width=0.7\textwidth]{figure_man/bike-sharing-dataset01.png}
\end{center}

\begin{onlyenv}
\only<1>{
\begin{itemize}
\item
  \textbf{ICE curve:} Visualize how the prediction of an
  \textbf{individual observation} changes if the feature value is
  changed.\\
  \(\Rightarrow\) ICE is a local interpretation method (black curves).
\item
  \textbf{PD plot:} Visualizes the \textbf{average effect of a feature},
  i.e., how the expected model prediction changes if the feature value is changed.\\
  \(\Rightarrow\) PD plot is a global interpretation method (yellow curve).
\end{itemize}
}

\only<2>{
Insights from the Bike Sharing Dataset:
\begin{itemize}
  \item Higher temperature $\Rightarrow$ higher number of bike rentals.
  \item Temperatures of more than $27^\circ$C decreases bike rentals.
  \item The effect is very homogeneous across all observations.
  \item Averaging ICE curves yields the PD plot (yellow curve).
\end{itemize}
}
\end{onlyenv}
}


\frame{
\frametitle{Categorical Features}

Same concept, different way to convey the information (boxplots)
}


\begin{vbframe}{2D Partial Dependence}

Same concept, but two features are replaced

\framebreak

\begin{center}
\includegraphics[width=0.7\textwidth]{figure_man/bike-sharing-dataset02.png}
\end{center}

\begin{itemize}
 \item Humidity and temperature interact with each other.
 \item The number of bike rentals is especially high when humidity is below 75 percent and temperature is between 15 $^{\circ}$C and 27 $^{\circ}$C
\end{itemize}


\framebreak

add datapoints to previous figure and mention convex hull (see pdp package)

\end{vbframe}

\endlecture
\end{document}
