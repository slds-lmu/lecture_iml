\documentclass[10pt,compress,t,notes=noshow, xcolor=table]{beamer}

\input{../../style/preamble}
% Defines macros and environments
\usepackage{bbm}
% basic latex stuff
\newcommand{\pkg}[1]{{\fontseries{b}\selectfont #1}} %fontstyle for R packages
\newcommand{\lz}{\vspace{0.5cm}} %vertical space
\newcommand{\dlz}{\vspace{1cm}} %double vertical space
\newcommand{\oneliner}[1] % Oneliner for important statements
{\begin{block}{}\begin{center}\begin{Large}#1\end{Large}\end{center}\end{block}}


%new environments
\newenvironment{vbframe}  %frame with breaks and verbatim
{
 \begin{frame}[containsverbatim,allowframebreaks]
}
{
\end{frame}
}

\newenvironment{vframe}  %frame with verbatim without breaks (to avoid numbering one slided frames)
{
 \begin{frame}[containsverbatim]
}
{
\end{frame}
}

\newenvironment{blocki}[1]   % itemize block
{
 \begin{block}{#1}\begin{itemize}
}
{
\end{itemize}\end{block}
}

\newenvironment{fragileframe}[2]{  %fragile frame with framebreaks
\begin{frame}[allowframebreaks, fragile, environment = fragileframe]
\frametitle{#1}
#2}
{\end{frame}}


\newcommand{\myframe}[2]{  %short for frame with framebreaks
\begin{frame}[allowframebreaks]
\frametitle{#1}
#2
\end{frame}}

\newcommand{\remark}[1]{
  \textbf{Remark:} #1
}


\newenvironment{deleteframe}
{
\begingroup
\usebackgroundtemplate{\includegraphics[width=\paperwidth,height=\paperheight]{../style/color/red.png}}
 \begin{frame}
}
{
\end{frame}
\endgroup
}
\newenvironment{simplifyframe}
{
\begingroup
\usebackgroundtemplate{\includegraphics[width=\paperwidth,height=\paperheight]{../style/color/yellow.png}}
 \begin{frame}
}
{
\end{frame}
\endgroup
}\newenvironment{draftframe}
{
\begingroup
\usebackgroundtemplate{\includegraphics[width=\paperwidth,height=\paperheight]{../style/color/green.jpg}}
 \begin{frame}
}
{
\end{frame}
\endgroup
}
% https://tex.stackexchange.com/a/261480: textcolor that works in mathmode
\makeatletter
\renewcommand*{\@textcolor}[3]{%
  \protect\leavevmode
  \begingroup
    \color#1{#2}#3%
  \endgroup
}
\makeatother


\providecommand{\tightlist}{%
  \setlength{\itemsep}{0pt}\setlength{\parskip}{0pt}}

%\setbeamerfont{footnote}{size=\tiny}
\usepackage[hang,flushmargin]{footmisc}
\renewcommand*{\footnotelayout}{\tiny}
\renewcommand*{\thefootnote}{} %\fnsymbol{footnote}

% https://tex.stackexchange.com/questions/30720/footnote-without-a-marker
% \makeatletter
% \def\blfootnote{\gdef\@thefnmark{}\@footnotetext}
% \makeatother

% https://tex.stackexchange.com/questions/357717/beamer-allowframebreaks-option-and-vertical-spacing-when-using-lists-itemize
% \setbeamertemplate{frametitle continuation}{%
%     (\insertcontinuationcount)%
%     \ifnum\insertcontinuationcount>1%
%     \vspace*{\topsep}%
%     \else%
%     %
%     \fi%
% }


\title{Interpretable Machine Learning}
% \author{LMU}
%\institute{\href{https://compstat-lmu.github.io/lecture_iml/}{compstat-lmu.github.io/lecture\_iml}}
\date{}

\begin{document}

\titlemeta{
Feature Effects
}{
PDP - Comments and Extensions
}{
figure/pdp_bike
}{

\item Extrapolation and Interactions in PDPs
\item Centered ICE and PDP

}


\begin{frame}{Comments on Extrapolation}

% \begin{center}
% \includegraphics[width=0.8\textwidth]{figure_man/extrapolation01.png}
% \end{center}
 
\begin{columns}[T, totalwidth=\textwidth]
\begin{column}{0.5\textwidth}
\centering
\includegraphics[width=\textwidth]{figure/ale_scatter_grid}
\end{column}
\begin{column}{0.5\textwidth}
\centering
\includegraphics[width=\textwidth]{figure/ale_pdplot}
\end{column}
\end{columns}

Extrapolation occurs in regions with few obs. or if features are correlated
\begin{itemize}
\item \textbf{Example:} Features $x_1$ and $x_2$ are strongly correlated
\item \textbf{Black points:} Observed points of the original data
\item \textbf{\textcolor{red}{Red:}} Grid points to calculate ICE/PD (many unrealistic $x_1$, $x_2$ combinations)\\ %combination of feature values
$\Rightarrow$ %Unrealistic combination of feature values are used, e.g., the
\textbf{PD at $x_1=0$:} Averages predictions over \textit{full} marginal distribution of $x_2$\\
$\Rightarrow$ \textbf{Issue:} Model may behave strangely outside training distribution\\
$\Rightarrow$ Especially problematic for overfitted or interaction-heavy models 
% which amplify risk of erratic predictions at extrapolated grid point
%can bias ICE and PD curves Be careful with interpretations 
%\item For correlated features and in regions with few observations with care
%Extrapolation: interpret curves for highly correlated features and in feature regions with few observations with care
\end{itemize}

%
% \framebreak
%
%
% \begin{center}
% \includegraphics[width=0.8\textwidth]{figure_man/extrapolation02.png}
% \end{center}
%
% \begin{itemize}
% \item The features $x_1$ and $x_2$ are strongly correlated.
% \item \textcolor{red}{Red:} Observed points of the original data.
% \item \textcolor{green}{Green:} Grid points used to calculate the ICE and PD curves.
% \item Example: PD plot at $x_1=1.9$ averages predictions over the whole marginal distribution of feature $x_2$.
% \end{itemize}
\end{frame}




% \begin{frame}{Interactions}
%
% For PD plots, the averaging of ICE curves might \textbf{obfuscate} heterogeneous effects and interactions. \newline \(\Rightarrow\) Ideally plot ICE curves and PD plots together.
%
% \begin{center}\includegraphics[width=0.65\textwidth]{figure_man/pdp_xor.pdf} \end{center}
% %
% % \framebreak
% %
% % \begin{itemize}
% % \item
% %   For PD plots, the averaging of ICE curves might \textbf{obfuscate}
% %   heterogeneous effects and interactions. \newline \(\Rightarrow\)
% %   Ideally plot ICE curves and PD plots together.
% % \item
% %   \textbf{Extrapolation:} Interprete curves for highly correlated
% %   features and in feature regions with few observations with care.
% % \item
% %   Accumulated Local Effects (ALE) plots are a novel alternative to the PD plots developed by Apley (2020) that do not suffer from
% %   extrapolation in case of correlated features.
% % \end{itemize}
% %
% % \vspace{80pt}
% % \tiny{
% % Apley, D. W., \& Zhu, J. (2020). Visualizing the effects of predictor variables in black box supervised learning models. Journal of the Royal Statistical Society: Series B, 82(4), 1059-1086. \par}
% \end{frame}

\begin{frame}{Comments on Interactions}
% \begin{itemize}
% %\lz
% %\item Accumulated Local Effects (ALE) plots are a novel alternative to PD plots developed by Apley (2016) that do not suffer from extrapolation in case of correlated features.
% \item PD plots: averaging of ICE curves might \textbf{obfuscate} heterogeneous effects and interactions \\ 
% \(\Rightarrow\) Ideally plot ICE curves and PD plots together to uncover this fact\\
% \(\Rightarrow\) Different shapes of ICE curves suggest interaction (but do not tell with which  feature)

% \end{itemize}
PD plots average ICE curves \\$\leadsto$ May \textbf{obscure heterogeneous effects} (interactions)
    \begin{itemize}
      \item \textbf{Example:} Feature $x_1$ = treatment dosage; $x_2$ = gender \\
      \(\Rightarrow\)  Males ($\nearrow$) and females ($\searrow$) respond differently to dosage\\
      \(\Rightarrow\)  PD curve (yellow) hides this divergence
      \item Plotting ICE and PD together helps detect interaction
      \item Diverse ICE shapes suggest interaction  (but not with which feature)
    \end{itemize}
    
\begin{center}\includegraphics[width=0.65\textwidth]{figure/pdp_xor.pdf} \end{center}


%     \begin{columns}[T, totalwidth=\textwidth]
%   \begin{column}{0.45\textwidth}
%     \centering
%     \includegraphics[width=\textwidth]{figure/pdp_xor.pdf}
%   \end{column}
%   \begin{column}{0.55\textwidth}
%     \small
%     \textbf{Model:}
%     $
%     f(x) = 6 x_1 + \beta_2 x_2 - 12 x_1 x_2
%     $
%     \vspace{-0.3em}
%     \textbf{Effect by group ($x_2 \in \{0,1\}$):}
%     \begin{itemize}
%       \item $x_2 = 0$: $f(x) = \beta_0 + \beta_1 x_1$
%       \item $x_2 = 1$: $f(x) = (\beta_0 + \beta_2) + (\beta_1 + \beta_3)x_1$
%     \end{itemize}
%     \vspace{-0.3em}
%     \textbf{Example:} $\beta_1 = 2$, $\beta_3 = -4$
%     \begin{itemize}
%       \item Group 0: $x_1 \mapsto$ increasing
%       \item Group 1: $x_1 \mapsto$ decreasing
%     \end{itemize}
%     \(\Rightarrow\) ICE curves diverge, PD appears flat.
%   \end{column}
% \end{columns}
%\footnote[frame]{Apley, Daniel W., and Jingyu Zhu (2020). Visualizing the Effects of Predictor Variables in Black Box Supervised Learning Models. Journal of the Royal Statistical Society: Series B (Statistical Methodology) 82.4: 1059-1086.}
\end{frame}

\begin{frame}{Comments on Interactions - 2D PD plot}

\begin{center}
\includegraphics[width=\textwidth]{figure/pdp2d_bike}
\end{center}

\begin{itemize}
 \item Humidity and temperature interact at high values (see shape difference)\\
 $\leadsto$ ICE curve shape changes across different (higher) values of other feat.% %(esp. for high values)%at different horizontal and vertical slices varies 
  \begin{itemize}
    \item ICE (temp): At high humidity, temp effect flattens (pink line)
    %or reverses
    \item ICE (hum): At high temp., humidity effect falls steeper (blue/pink)
  \end{itemize}

 \item Most rentals occur at \textit{high temperature} and \textit{low to medium humidity}
 %Many rented bikesThe number of bike rentals is especially high when humidity is below 75 percent and temperature is between 15 $^{\circ}$C and 27 $^{\circ}$C
\end{itemize}


% \framebreak
%
% add datapoints to previous figure and mention convex hull (see pdp package)

\end{frame}

\begin{frame}{Centered ICE Plot (c-ICE) \furtherreading{Goldstein_2015}}

\textbf{Issue:} Varying-intercept (stacked) ICE curves obscure shape heterogeneity
%Difficult to identify heterogeneous ICE curves if curves have different intercepts (are stacked)
%When ICE curves start at different intercepts (are stacked), it is difficult to identify heterogenous predictions.

\textbf{Solution:} Center ICE curves at fixed reference value, often $x' = \min(\xv_S)$\\
$\Rightarrow$ Easier to identify heterogeneous shapes with c-ICE curves
$$\begin{aligned}
\fh_{S, cICE}^{(i)}(\xv_S) = \fh(\xv_S, \xi_{-S}) - \fh(x', \xi_{-S}) = \fh_{S}^{(i)}(\xv_S) - \fh_{S}^{(i)}(x')
\end{aligned}$$
\vspace{-0.2cm}
\begin{columns}[c, totalwidth=\textwidth]
\begin{column}{0.5\textwidth}
%\centering
% $$\begin{aligned}
% \fh_{S, cICE}^{(i)}(\xv_S)
% &= \fh(\xv_S, \xi_{-S}) - \fh(x', \xi_{-S}) \\
% &= \fh_{S}^{(i)}(\xv_S) - \fh_{S}^{(i)}(x')
% \end{aligned}$$
%$\Rightarrow$ Visualize $\fh_{S, cICE}^{(i)}(\xv_S^*)$ vs. $\xv_S^*$
%\lz
\pause
\textbf{Interpretation}% \\
\begin{itemize}
  \item Yellow: c-PDP (mean of c-ICE)
  \item \textbf{c-PDP:} At 97\% humidity, predicted rentals are 1000 fewer than at 0\% humidity (on average)
  \item \textbf{Opening of c-ICE curves:} suggests interaction or varying effect across instances
\end{itemize}
%(yellow curve: analog to PDP the average of c-ICE curves): \\
%On average, the number of bike rentals at $\sim 97$ \% humidity decreased by 1000 bikes compared to a humidity of 0 \%
\end{column}
\begin{column}{0.5\textwidth}
\begin{center}
%\vspace{-0.3cm}
\includegraphics[width=\textwidth]{figure/cICE}
\end{center}
\end{column}
\end{columns}

%\vspace{-0.4cm}

\end{frame}


\begin{frame}{Centered ICE Plot (c-ICE)}

Categorical features: c-ICE plots can be interpreted as in LMs due to reference value

\begin{columns}[c, totalwidth=\textwidth]
\begin{column}{0.4\textwidth}

\begin{center}
\includegraphics[width=\textwidth]{figure/cICEcat}
\end{center}

\end{column}
\begin{column}{0.6\textwidth}

\textbf{Interpretation}: \\
\begin{itemize}
%\item Centered ICE plots are useful for categorical features and can be interpreted as in LMs
\item The reference category is $x' =$ SPRING
\item Yellow crosses: Average rentals if we jump from SPRING to any other season\\
$\Rightarrow$ Number of bike rentals drops by $\sim 560$ in \code{WINTER} and is slightly higher in \code{SUMMER} and \code{FALL} compared to \code{SPRING}
\end{itemize}

\end{column}
\end{columns}

\end{frame}


\endlecture
\end{document}
