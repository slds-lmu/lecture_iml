\documentclass[11pt,compress,t,notes=noshow, aspectratio=169, xcolor=table]{beamer}

\usepackage{../../style/lmu-lecture}
% Defines macros and environments
% This file is included in slides and exercises

% Rarely used fontstyle for R packages, used only in 
% - forests/slides-forests-benchmark.tex
% - exercises/single-exercises/methods_l_1.Rnw
% - slides/cart/attic/slides_extra_trees.Rnw
\newcommand{\pkg}[1]{{\fontseries{b}\selectfont #1}}

% Spacing helpers, used often (mostly in exercises for \dlz)
\newcommand{\lz}{\vspace{0.5cm}} % vertical space (used often in slides)
\newcommand{\dlz}{\vspace{1cm}}  % double vertical space (used often in exercises, never in slides)
\newcommand{\oneliner}[1] % Oneliner for important statements, used e.g. in iml, algods
{\begin{block}{}\begin{center}\begin{Large}#1\end{Large}\end{center}\end{block}}

% Don't know if this is used or needed, remove?
% textcolor that works in mathmode
% https://tex.stackexchange.com/a/261480
% Used e.g. in forests/slides-forests-bagging.tex
% [...] \textcolor{blue}{\tfrac{1}{M}\sum^M_{m} [...]
% \makeatletter
% \renewcommand*{\@textcolor}[3]{%
%   \protect\leavevmode
%   \begingroup
%     \color#1{#2}#3%
%   \endgroup
% }
% \makeatother

\newcommand{\open}{}
\newcommand{\close}{}

\title{Interpretable Machine Learning}
% \author{LMU}
%\institute{\href{https://compstat-lmu.github.io/lecture_iml/}{compstat-lmu.github.io/lecture\_iml}}
\date{}

\begin{document}

\newcommand{\titlefigure}{figure/open_blackbox}
\newcommand{\learninggoals}{
\item Limitations of classical fANOVA
\item Overcoming these limitations with generalized fANOVA
\item How ALE Plots can be used as another, different approach to obtain functional decompositions
\item Sobol-Hoeffding decomposition ??}

\lecturechapter{Functional Decompositions: Further Methods}
\lecture{Interpretable Machine Learning}

\begin{frame}{Limitations of classical fANOVA}

    [Example going wrong with correlated features]

    \begin{example}

        Again consider

        \begin{equation*}
            \fh(x_1, x_2, x_3) = - 2x_1 - 2\sin(x_3) + |x_1|x_2 + 0.5 x_2 x_3 +1
        \end{equation*}

        and further assume that the following dependency always holds: \(2x_1^2 = x_2\). Then for example the following two decompositions would both "make sense":
        
    \end{example}

    [...]

    
    
\end{frame}

\begin{frame}{Generalized Functional ANOVA}

    \textbf{N.B.:} For dependent inputs, \citebutton{Hooker (2007)}{http://www.tandfonline.com/doi/abs/10.1198/106186007X237892} showed the existence of a unique solution for the components under a ``relaxed vanishing condition'' which leads to a ``hierarchical orthogonality''
    $$\mathbb{E}_{\Xv} (g_{V}(\xv_V) g_{S}(\xv_S)) = 0, \forall V \subset S$$
    $\leadsto$ Only components are orthogonal where features involved in $g_{V}(\xv_V)$ also appear in $g_{S}(\xv_S)$
    %Only components where all features involved in one component $g_{V}(\xv_V)$ also appear in the other component $g_{S}(\xv_S)$ are orthogonal

    \pause

    \textit{[Also talk about constraints corresponding to generalized fANOVA ?]}
    
\end{frame}

\begin{frame}{Generalized fANOVA: Example}

    Example from above ??
    
\end{frame}

\begin{frame}{Revisiting ALE Plots}

    Recap: ALE PLots [...]
    
\end{frame}

\begin{frame}{ALE Decomposition}
    
    ALE Plots also can lead to a full functional decomposition: [...]
    
\end{frame}

\begin{frame}{Sobol-Hoeffding decomposition}
    
\end{frame}

\begin{frame}{if enough time: Constraints (theory) for these other methods}

    NB: Even more possible constraints (leading to ever more different decompositions) have been developed in research papers.
    
\end{frame}

\begin{frame}{Conclusion: How useful are functional decompositions?}

    Obwohl fDecompositions eigtl (aus Interpretability-Sicht) die Lösung für alles wären / theoretisch die Endlösung sind, sind alle anderen Methoden trotzdem nötig, weil fDecompositions sehr schwierig \& kompliziert zu berechnen sind
      - Computation time skaliert exponentiell mit der Dimension / Anzahl features  =>  i.a. ist Erzwingen einer sparsen decomposition (s. GAMs / RPFs) die einzige realisitische Möglichkeit \\

    Nevertheless, functional decompositions are a very important interpretability concept, because they explain / are the idea behind many other methods and enable much better understanding of other methods.
    
\end{frame}










\endlecture
\end{document}
