\documentclass[11pt,compress,t,notes=noshow, aspectratio=169, xcolor=table]{beamer}

\usepackage{../../style/lmu-lecture}
% Defines macros and environments
\usepackage{bbm}
% basic latex stuff
\newcommand{\pkg}[1]{{\fontseries{b}\selectfont #1}} %fontstyle for R packages
\newcommand{\lz}{\vspace{0.5cm}} %vertical space
\newcommand{\dlz}{\vspace{1cm}} %double vertical space
\newcommand{\oneliner}[1] % Oneliner for important statements
{\begin{block}{}\begin{center}\begin{Large}#1\end{Large}\end{center}\end{block}}


%new environments
\newenvironment{vbframe}  %frame with breaks and verbatim
{
 \begin{frame}[containsverbatim,allowframebreaks]
}
{
\end{frame}
}

\newenvironment{vframe}  %frame with verbatim without breaks (to avoid numbering one slided frames)
{
 \begin{frame}[containsverbatim]
}
{
\end{frame}
}

\newenvironment{blocki}[1]   % itemize block
{
 \begin{block}{#1}\begin{itemize}
}
{
\end{itemize}\end{block}
}

\newenvironment{fragileframe}[2]{  %fragile frame with framebreaks
\begin{frame}[allowframebreaks, fragile, environment = fragileframe]
\frametitle{#1}
#2}
{\end{frame}}


\newcommand{\myframe}[2]{  %short for frame with framebreaks
\begin{frame}[allowframebreaks]
\frametitle{#1}
#2
\end{frame}}

\newcommand{\remark}[1]{
  \textbf{Remark:} #1
}


\newenvironment{deleteframe}
{
\begingroup
\usebackgroundtemplate{\includegraphics[width=\paperwidth,height=\paperheight]{../style/color/red.png}}
 \begin{frame}
}
{
\end{frame}
\endgroup
}
\newenvironment{simplifyframe}
{
\begingroup
\usebackgroundtemplate{\includegraphics[width=\paperwidth,height=\paperheight]{../style/color/yellow.png}}
 \begin{frame}
}
{
\end{frame}
\endgroup
}\newenvironment{draftframe}
{
\begingroup
\usebackgroundtemplate{\includegraphics[width=\paperwidth,height=\paperheight]{../style/color/green.jpg}}
 \begin{frame}
}
{
\end{frame}
\endgroup
}
% https://tex.stackexchange.com/a/261480: textcolor that works in mathmode
\makeatletter
\renewcommand*{\@textcolor}[3]{%
  \protect\leavevmode
  \begingroup
    \color#1{#2}#3%
  \endgroup
}
\makeatother


\providecommand{\tightlist}{%
  \setlength{\itemsep}{0pt}\setlength{\parskip}{0pt}}

%\setbeamerfont{footnote}{size=\tiny}
\usepackage[hang,flushmargin]{footmisc}
\renewcommand*{\footnotelayout}{\tiny}
\renewcommand*{\thefootnote}{} %\fnsymbol{footnote}

% https://tex.stackexchange.com/questions/30720/footnote-without-a-marker
% \makeatletter
% \def\blfootnote{\gdef\@thefnmark{}\@footnotetext}
% \makeatother

% https://tex.stackexchange.com/questions/357717/beamer-allowframebreaks-option-and-vertical-spacing-when-using-lists-itemize
% \setbeamertemplate{frametitle continuation}{%
%     (\insertcontinuationcount)%
%     \ifnum\insertcontinuationcount>1%
%     \vspace*{\topsep}%
%     \else%
%     %
%     \fi%
% }

\newcommand{\open}{}
\newcommand{\close}{}

\title{Interpretable Machine Learning}
% \author{LMU}
%\institute{\href{https://compstat-lmu.github.io/lecture_iml/}{compstat-lmu.github.io/lecture\_iml}}
\date{}

\begin{document}

\newcommand{\titlefigure}{figure/open_blackbox}
\newcommand{\learninggoals}{
\item Understanding why fANOVA theoretically works and what constraints it satisfies
\item Understand the important role further constraints play for any functional decomposition?
\item ??}

\lecturechapter{Theory of Functional Decompositions}
\lecture{Interpretable Machine Learning}

\begin{frame}{Example: fANOVA Algorithm}

    See before: The definition of functional decompositions is by far not unique \\

    BUT: The fANOVA Algo. yields a unique output / unique decomposition, seemingly "arbitrarily chosen".\\

    \(\rightarrow\) Idea: Uniquely (Eindeutig?) characterize the solution of the fANOVA by its properties. These properties are expressed as mathematical constraints that describe the calculated decomposition.

    \(\rightarrow\) Two equivalent ways of uniquely / completely defining this decomposition: Either by computation formula defining the single components, or by mathematical equations (the "constraints") defining what properties the components must fulfill
    
\end{frame}

\begin{frame}{Constraints for standard fANOVA Algorithm}

    One can prove that our definition of the components fulfills the following conditions, called \textit{vanishing condition}:
    \begin{equation*}
        \mathbb{E}_{X_j}\left[ g_{S}(\xv_S) \right] = \int g_{S}(\xv_S) d \mathbb{P}(x_j) = 0, \forall j \in S, \forall S \subseteq \{1, \ldots, p\}
    \end{equation*}

    % For independent inputs, the \textit{vanishing condition} is required to obtain a unique solution:
    % $$\mathbb{E}_{X_j} (g_{S}(\xv_S)) = \int g_{S}(\xv_S) d \mathbb{P}(x_j) = 0, \forall j \in S, \forall S \subseteq \{1, \ldots, p\}$$

    \textit{Proof that our definition fulfills these constraints? Not here}
    
    \pause 
    
    Vanishing conditions have the following implications:
    
    \begin{itemize}
        \item Marginalizing out $x_j, \forall j \in S$ for component $g_S(\xv_S)$ yields a constant 0\\
        %As we integrate out the marginal effect of $x_j, \forall j \in S$ on component $g_S(\xv_S)$ is zero (vanishes)\\
        $\leadsto$ Makes sure that component $g_S(\xv_S)$ does not contain effects of $x_j, \forall j \in S$
        %For $|S| = 1$ this is equivalent to mean-centering the component $g_S(\xv_S)$. For $|S| > 1$
        \item Components are orthogonal (i.e., mutually independent and uncorrelated):
        $$\mathbb{E}_{X} (g_{V}(\xv_V) g_{S}(\xv_S)) = 0, \forall V \neq S$$
        \item Variance can be decomposed:
    $ Var[\fh(\xv)] =  \textstyle\sum_{S \subseteq \{1,\ldots,p\}}  Var\left[ g_{S}(\xv_S)\right]$
    \end{itemize}
    
\end{frame}

\begin{frame}{...}

    [Proofs that these constraints hold / are fulfilled by our definition?? Only briefly]
    
\end{frame}

\begin{frame}{...}

    NB: One can also start the other way around, i.e. define fANOVA using the constraints it should fulfill, and then mathematically prove that the single components must have this form.
    
\end{frame}

\begin{frame}{Connection to PDP, H-statistics}

    It can be shown: A PDP for a group of variables is exactly equal to the sum of all fANOVA components which are part of this group.
    The other way round, a fANOVA component can be described as the difference between the respective PDP and all lower-degree PDPs. (plus / minus some other lower-degree PDPs ...)
    \\
    Vanishing condition implies: No lower-degree terms / interactions are included in any fANOVA component.
    \(\implies\)
    In particular, the function contains interactions of a specific type if and only if the corresponding fANOVA component is not 0.
    \\
    Together, we achieve an alternative definition / characterization of interactions:
    Interactions of certain type exist \(\iff\) corresponding fANOVA component \(\neq 0\) \(\iff\) the respective difference of PDPs, the so-called \textit{H-statistic}, is \(\neq 0\).

    More generally, the H-statistic is a measure of the strength of a specific interaction, which can in general be applied to any model or any function.
     
\end{frame}










\endlecture
\end{document}
