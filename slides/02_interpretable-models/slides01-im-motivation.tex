\documentclass[11pt,compress,t,notes=noshow, aspectratio=169, xcolor=table]{beamer}

\usepackage{../../style/lmu-lecture}
% Defines macros and environments
\usepackage{bbm}
% basic latex stuff
\newcommand{\pkg}[1]{{\fontseries{b}\selectfont #1}} %fontstyle for R packages
\newcommand{\lz}{\vspace{0.5cm}} %vertical space
\newcommand{\dlz}{\vspace{1cm}} %double vertical space
\newcommand{\oneliner}[1] % Oneliner for important statements
{\begin{block}{}\begin{center}\begin{Large}#1\end{Large}\end{center}\end{block}}


%new environments
\newenvironment{vbframe}  %frame with breaks and verbatim
{
 \begin{frame}[containsverbatim,allowframebreaks]
}
{
\end{frame}
}

\newenvironment{vframe}  %frame with verbatim without breaks (to avoid numbering one slided frames)
{
 \begin{frame}[containsverbatim]
}
{
\end{frame}
}

\newenvironment{blocki}[1]   % itemize block
{
 \begin{block}{#1}\begin{itemize}
}
{
\end{itemize}\end{block}
}

\newenvironment{fragileframe}[2]{  %fragile frame with framebreaks
\begin{frame}[allowframebreaks, fragile, environment = fragileframe]
\frametitle{#1}
#2}
{\end{frame}}


\newcommand{\myframe}[2]{  %short for frame with framebreaks
\begin{frame}[allowframebreaks]
\frametitle{#1}
#2
\end{frame}}

\newcommand{\remark}[1]{
  \textbf{Remark:} #1
}


\newenvironment{deleteframe}
{
\begingroup
\usebackgroundtemplate{\includegraphics[width=\paperwidth,height=\paperheight]{../style/color/red.png}}
 \begin{frame}
}
{
\end{frame}
\endgroup
}
\newenvironment{simplifyframe}
{
\begingroup
\usebackgroundtemplate{\includegraphics[width=\paperwidth,height=\paperheight]{../style/color/yellow.png}}
 \begin{frame}
}
{
\end{frame}
\endgroup
}\newenvironment{draftframe}
{
\begingroup
\usebackgroundtemplate{\includegraphics[width=\paperwidth,height=\paperheight]{../style/color/green.jpg}}
 \begin{frame}
}
{
\end{frame}
\endgroup
}
% https://tex.stackexchange.com/a/261480: textcolor that works in mathmode
\makeatletter
\renewcommand*{\@textcolor}[3]{%
  \protect\leavevmode
  \begingroup
    \color#1{#2}#3%
  \endgroup
}
\makeatother


\providecommand{\tightlist}{%
  \setlength{\itemsep}{0pt}\setlength{\parskip}{0pt}}

%\setbeamerfont{footnote}{size=\tiny}
\usepackage[hang,flushmargin]{footmisc}
\renewcommand*{\footnotelayout}{\tiny}
\renewcommand*{\thefootnote}{} %\fnsymbol{footnote}

% https://tex.stackexchange.com/questions/30720/footnote-without-a-marker
% \makeatletter
% \def\blfootnote{\gdef\@thefnmark{}\@footnotetext}
% \makeatother

% https://tex.stackexchange.com/questions/357717/beamer-allowframebreaks-option-and-vertical-spacing-when-using-lists-itemize
% \setbeamertemplate{frametitle continuation}{%
%     (\insertcontinuationcount)%
%     \ifnum\insertcontinuationcount>1%
%     \vspace*{\topsep}%
%     \else%
%     %
%     \fi%
% }


\title{Interpretable Machine Learning}
% \author{LMU}
%\institute{\href{https://compstat-lmu.github.io/lecture_iml/}{compstat-lmu.github.io/lecture\_iml}}
\date{}

\begin{document} 

\newcommand{\titlefigure}{figure/whitebox}
\newcommand{\learninggoals}{
%\item What characteristics does an interpretable model have?
\item Why should we use interpretable models?
\item Advantages and disadvantages of interpretable models
}

\lecturechapter{Inherently Interpretable Models - Motivation}
\lecture{Interpretable Machine Learning}

\begin{frame}{Motivation}
%Achieving interpretability by using interpretable models is the most straightforward approach
%\bigskip
\begin{columns}[T, totalwidth = \linewidth]
    \begin{column}{0.55\textwidth}
    \begin{itemize}
        %\item Obtaining interpretations by using interpretable models is the easiest and least error-prone approach
       \item Achieving interpretability by using interpretable models is the most straightforward approach
        \bigskip
        \item Classes of models deemed interpretable:
            %\item Linear regression models
            \item[] $\bullet$ (Generalized) linear models (LM, GLM)
            \item[] $\bullet$ Generalized additive models (GAM)
            \item[] $\bullet$ Decision trees
            \item[] $\bullet$ Rule-based learning
            \item[] $\bullet$ Model-based boosting / component-wise boosting
            \item[] $\bullet$ ...
    \end{itemize}
    \end{column}
    \begin{column}{0.44\textwidth}  %%<--- here
  \includegraphics[width = \textwidth]{figure/main_effect_lm_temp.pdf}
  \begin{center}
    $\leadsto$ LM provides straightforward interpretation
  \end{center}
    \end{column}
\end{columns}
\end{frame}

\begin{frame}{Advantages}
\begin{columns}[T, totalwidth=\textwidth]
\begin{column}{0.72\textwidth}
    \begin{itemize}[<+->]
    \itemsep1em
        \item Inherently interpretable - some additional model-agnostic interpretation methods not required \\
        $\leadsto$ Eliminates a source of error
        \item Interpretable models often simple \\
        $\leadsto$ training time is fairly small
        \item Some interpretable models fulfill the monotonicity constraint \\
        $\leadsto$ Larger feature values always lead to higher (or smaller) outcomes (e.g., GLMs)
        \item Many people are familiar with traditional interpretable models \\
        $\leadsto$ Increases trust, facilitates communication of results
        %\item Implementations are available in many programming languages \\
        %$\leadsto$ Simple models are easier to deploy in practice or implement from scratch
    \end{itemize}
\end{column}
\begin{column}{0.28\textwidth}
    \begin{center}
        \begin{tikzpicture}[scale=0.6, transform shape]
        \usetikzlibrary{arrows}
            \usetikzlibrary{shapes}
            \tikzset{treenode/.style={draw, circle, font=\small}}
            \tikzset{line/.style={draw, thick}}
            \node [treenode, draw=red] (a0) {$a_0$};
            \node [treenode, below=0.75cm of a0, xshift=-1cm]  (a1) {$a_1$};
            \node [treenode, draw=red, below=0.75cm of a0, xshift=1cm]  (a2) {$a_2$};
     
            \node [treenode, draw=red, below=0.75cm of a2, xshift=-1cm] (a3) {$a_3$};
            \node [treenode, below=0.75cm of a2, xshift=1cm]  (a4) {$a_4$};
     
            \node [treenode, below=0.75cm of a3, xshift=-1cm] (a5) {$a_5$};
            \node [treenode, below=0.75cm of a3, xshift=1cm]  (a6) {$a_6$};
     
            \path [line] (a0.south) -- + (0,-0.4cm) -| (a1.north) node [midway, above] {$x_1<0.3$};
            \path [line] (a0.south) -- +(0,-0.4cm) -|  (a2.north) node [midway, above] {$x_1\geq0.3$};
     
            \path [line] (a2.south) -- + (0,-0.4cm) -| (a3.north) node [midway, above] {$x_1<0.6$};;
            \path [line] (a2.south) -- +(0,-0.4cm) -|  (a4.north) node [midway, above] {$x_1\geq0.6$};
     
          
            \path [line] (a3.south) -- + (0,-0.4cm) -| (a5.north) node [midway, above] {$x_2<0.2$};;
            \path [line] (a3.south) -- +(0,-0.4cm) -|  (a6.north) node [midway, above] {$x_2\geq0.2$};
     
        \end{tikzpicture}
        
        \vspace{0.5cm}
        
        \includegraphics<3->[height = 2.5cm, width = \textwidth]{figure/intro_lm_bike.pdf}
    \end{center}
\end{column}
\end{columns}
\end{frame}

\begin{frame}{Disadvantages}
\begin{columns}
\begin{column}{0.7\textwidth}
    \begin{itemize}%[<+->]
    \itemsep1em
        \item<1-> Often certain assumptions about data / model structure required\\
        $\leadsto$ If assumptions are wrong, models may perform bad 
        %\item In high-dimensional settings, it may not be feasible to find and define all feature relationships
        %\item If the wrong assumptions are made, interpretable models may have a bad predictive performance
        %\pause
        \item<2-> Interpretable models may also be difficult to interpret, e.g.:
    \begin{itemize}
        \item Linear model with hundreds of features and interactions 
        \item Decision trees with huge tree depth
    \end{itemize}
    \end{itemize}
\end{column}
\begin{column}{0.3\textwidth}
    \begin{center}
        \includegraphics[width = \textwidth]{figure/lm_bad_fit.pdf} 
    \end{center}
\end{column}
\end{columns}
\quad \\

\begin{itemize}
\itemsep1em
        \item<3-> 
        %No automatic modeling of complex relationships due to limited model flexibility
        Often not able to automatically model complex relationships due to limited model flexibility
        \\
        e.g., high-order main or interaction effects need to be specified manually in a LM

        %\pause
     \item<4-> Inherently interpretable models do not provide all types of explanations
     %% one might be interested
     \\
     $\leadsto$ Methods providing other types of explanations still useful (e.g., counterfactual explanations)
     %Still requires application of model-agnostic interpretation techniques if certain types of explanations are of interest (e.g., counterfactual explanations)
\end{itemize}
\end{frame}

\begin{frame}{Further Comments}

    \begin{itemize}
    %\itemsep1em
        \item<1-> Some argue that one should always use interpretable models in the first place \citebutton{Rudin 2019}{https://www.nature.com/articles/s42256-019-0048-x}
        \begin{itemize}
            \item \ldots and not try to explain uninterpretable models post-hoc
            \item Can sometimes work out by spending enough time and energy on data pre-processing %feat. engineering and data cleaning
        \end{itemize}
        %\pause
        %\footnote[frame]{Rudin, C. Stop explaining black box machine learning models for high stakes decisions and use interpretable models instead. Nat Mach Intell 1, 206–215 (2019).}
        % {https://www.nature.com/articles/s42256-019-0048-x}
       %\item Interpretable models also have the potential for a high predictive performance, but require more knowledge and time spent on data preprocessing
       \item<2->[$\leadsto$] Drawback: Hard to achieve for data for which end-to-end learning is crucial\\ 
        (e.g., hard to extract good features for image / text data\\ $\leadsto$ information loss = bad performance)
       %(e.g., for image / text data extraction of good features is hard $\leadsto$ information loss = bad performance)
       %(e.g., image and text data would require good feature extraction steps $\leadsto$ information loss)
        %\pause
        %\item One can assume a trade-off between interpretability and model performance which generally (but not always) holds
        \item<3-> Often there is a trade-off between interpretability and model performance 
        %(but not always)
    \end{itemize}
    
    \smallskip
    
    \only<3>{\centering\includegraphics[width=0.65\textwidth]{figure/performance_vs_interpretability.pdf}}
    % Quelle: https://docs.google.com/presentation/d/12ZPrTjBKEUT-7drdyUJCQGK0oHVDtYIVd2_6byE62f0/edit?usp=sharing

\end{frame}


\begin{frame}{Recommendation}
    \begin{itemize}
        \item Start with most simple model that makes sense for application at hand
        \item Gradually increase complexity if performance is insufficient\\
        $\leadsto$ will usually lower interpretability and require additional interpretation methods
        \item Choose the most simple, sufficient model (Occam's razor)
    \end{itemize} 

    \bigskip

% \begin{columns}[T, totalwidth=\textwidth]
% \begin{column}{0.5\textwidth}
% \centering \textbf{Bike Data}
%     \begin{table}[ht]
%         \centering
%         \begin{tabular}{lrr}
%             \hline
%             Model & RMSE & $R^2$ \\ 
%             \hline
%             LM & 142.39 & 0.38 \\ 
%             Tree & 98.71 & 0.70 \\ 
%             Random Forest & 57.07 & 0.90 \\ 
%             Boosting & 197.42 & -0.18 \\  
%             \hline
%         \end{tabular}
%     \end{table}
% \end{column}
% \begin{column}{0.5\textwidth}
% \centering \textbf{Boston Housing Data}
%     \begin{table}[ht]
%         \centering
%         \begin{tabular}{lrr}
%             \hline
%             Model & RMSE & $R^2$ \\ 
%             \hline
%             LM & 0.43 & 1.00 \\ 
%             Tree & 1.81 & 0.96 \\ 
%             Random Forest & 1.77 & 0.96 \\ 
%             Boosting & 16.89 & -2.43 \\
%             \hline
%         \end{tabular}
%     \end{table}
% \end{column}
% \end{columns}
% code for tables in rsrc/tab_ml_comparison.R

\centering \textbf{Bike Data, 4-fold CV}
\begin{table}[ht]
\centering
\begin{tabular}{lrr}
  \hline
Model & RMSE & $R^2$ \\ 
  \hline
  LM & 800.15 & 0.83 \\ 
  Tree & 981.83 & 0.74 \\ 
  Random Forest & 653.25 & 0.88 \\ 
  Boosting (tuned) & 638.42 & 0.89 \\ 
   \hline
\end{tabular}
\end{table}
\end{frame}



\endlecture
\end{document}
