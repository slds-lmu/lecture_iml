\documentclass[11pt,compress,t,notes=noshow, aspectratio=169, xcolor=table]{beamer}

\usepackage{../../style/lmu-lecture}
% Defines macros and environments
\usepackage{bbm}
% basic latex stuff
\newcommand{\pkg}[1]{{\fontseries{b}\selectfont #1}} %fontstyle for R packages
\newcommand{\lz}{\vspace{0.5cm}} %vertical space
\newcommand{\dlz}{\vspace{1cm}} %double vertical space
\newcommand{\oneliner}[1] % Oneliner for important statements
{\begin{block}{}\begin{center}\begin{Large}#1\end{Large}\end{center}\end{block}}


%new environments
\newenvironment{vbframe}  %frame with breaks and verbatim
{
 \begin{frame}[containsverbatim,allowframebreaks]
}
{
\end{frame}
}

\newenvironment{vframe}  %frame with verbatim without breaks (to avoid numbering one slided frames)
{
 \begin{frame}[containsverbatim]
}
{
\end{frame}
}

\newenvironment{blocki}[1]   % itemize block
{
 \begin{block}{#1}\begin{itemize}
}
{
\end{itemize}\end{block}
}

\newenvironment{fragileframe}[2]{  %fragile frame with framebreaks
\begin{frame}[allowframebreaks, fragile, environment = fragileframe]
\frametitle{#1}
#2}
{\end{frame}}


\newcommand{\myframe}[2]{  %short for frame with framebreaks
\begin{frame}[allowframebreaks]
\frametitle{#1}
#2
\end{frame}}

\newcommand{\remark}[1]{
  \textbf{Remark:} #1
}


\newenvironment{deleteframe}
{
\begingroup
\usebackgroundtemplate{\includegraphics[width=\paperwidth,height=\paperheight]{../style/color/red.png}}
 \begin{frame}
}
{
\end{frame}
\endgroup
}
\newenvironment{simplifyframe}
{
\begingroup
\usebackgroundtemplate{\includegraphics[width=\paperwidth,height=\paperheight]{../style/color/yellow.png}}
 \begin{frame}
}
{
\end{frame}
\endgroup
}\newenvironment{draftframe}
{
\begingroup
\usebackgroundtemplate{\includegraphics[width=\paperwidth,height=\paperheight]{../style/color/green.jpg}}
 \begin{frame}
}
{
\end{frame}
\endgroup
}
% https://tex.stackexchange.com/a/261480: textcolor that works in mathmode
\makeatletter
\renewcommand*{\@textcolor}[3]{%
  \protect\leavevmode
  \begingroup
    \color#1{#2}#3%
  \endgroup
}
\makeatother


\providecommand{\tightlist}{%
  \setlength{\itemsep}{0pt}\setlength{\parskip}{0pt}}

%\setbeamerfont{footnote}{size=\tiny}
\usepackage[hang,flushmargin]{footmisc}
\renewcommand*{\footnotelayout}{\tiny}
\renewcommand*{\thefootnote}{} %\fnsymbol{footnote}

% https://tex.stackexchange.com/questions/30720/footnote-without-a-marker
% \makeatletter
% \def\blfootnote{\gdef\@thefnmark{}\@footnotetext}
% \makeatother

% https://tex.stackexchange.com/questions/357717/beamer-allowframebreaks-option-and-vertical-spacing-when-using-lists-itemize
% \setbeamertemplate{frametitle continuation}{%
%     (\insertcontinuationcount)%
%     \ifnum\insertcontinuationcount>1%
%     \vspace*{\topsep}%
%     \else%
%     %
%     \fi%
% }


\title{Interpretable Machine Learning}
% \author{LMU}
%\institute{\href{https://compstat-lmu.github.io/lecture_iml/}{compstat-lmu.github.io/lecture\_iml}}
\date{}

\begin{document}
	
	% Set style/preamble.Rnw as parent.
	
	% Load all R packages and set up knitr
	
	% This file loads R packages, configures knitr options and sets preamble.Rnw as 
	% parent file
	% IF YOU MODIFY THIS, PLZ ALSO MODIFY setup.Rmd ACCORDINGLY...
	
	% Defines macros and environments
	
	\newcommand{\titlefigure}{figure/lime5}
    \newcommand{\learninggoals}{
    	\item Learn why LIME should be used with caution
    	\item Possible pitfalls of LIME}
	
	\lecturechapter{LIME Pitfalls}
	\lecture{Interpretable Machine Learning}
	
	% Prerequisite: le-into, le-lime
	
	% ------------------------------------------------------------------------------


\begin{vbframe}[c]{LIME Pitfalls}
  \begin{itemize}
  	\item LIME is one of the best known interpretable machine learning methods\\ but several papers caution to be careful in their use. 
  	\item Problems can occur in several places. 
  	\item These are discussed in more detail in the following. 
  \end{itemize}
  
\end{vbframe}
  
\begin{vbframe}[c]{LIME Pitfalls: Sampling}
	\begin{itemize}
	  \item \textbf{Pitfall}: Common sampling strategies for $\zv \in Z$ do not take the correlation between features into account. 
      \item \textbf{Implication}:  Unlikely data points might be used to learn local explanation models.
      \item \textbf{Solution I}: Ideally, local sampler directly on $\Xspace$. Derivation is particularly difficult for high dimensional or mixed feature spaces. 
      \item \textbf{Solution II}: Use original training data to fit the surrogate model;\\ but this only works well with enough data near $\xv$.
    \end{itemize}
    
\end{vbframe}

\begin{vbframe}[c]{LIME Pitfall: Locality}

	\begin{itemize} 
     \item \textbf{Pitfall}: Difficult to define locality (= how samples are weighted locally). 
     \begin{itemize}
         \item[$\leadsto$] Huge influence on the local model, but there is no automatic procedure for choosing the neighborhood.
     \end{itemize}
     \item Originally, an exponential kernel as proximity measure between $\xv$ and $\zv$ was used:\\
     	$\neigh(\zv) = exp(-d(\xv, \zv)^2/\sigma^2)$ where $d$ is a distance measure and $\sigma$ is the kernel width. 
     	 \begin{center}
     		\includegraphics[width=0.6\textwidth]{figure/lime_locality}
     		\vspace{-0.5cm}
     		
     		\scriptsize{Linear surrogate models for two different data points based on the same model with one target and one feature. Each line displays one linear surrogate model with a different kernel width. For the data point in the right figure, larger kernel sizes are more severe.}
     		
     	\end{center}
     \end{itemize}
\end{vbframe}

\begin{vbframe}[c]{LIME Pitfall: Locality \citebutton{Kopper et al. 2019}{https://slds-lmu.github.io/iml_methods_limitations/}}
    \begin{itemize} 
         \item \textbf{Solution I}: Kernel width strongly interacts with locality:
         \begin{itemize}
             \item Large kernel width leads to interaction with points further away (unwanted)
             \item Small kernel width leads to small neighborhood\\
             $\leadsto$ Risk to few data points $\leadsto$ potentially fitting more noise
         \end{itemize}
    	\item \textbf{Solution II}: Gower proximity where no kernel width needs to be specified. 
    	\begin{itemize}
    	    \item \textbf{Problem}: data points far away receive a weight $ > 0$; their resulting explanations are rather global surrogates than local surrogates.   
    	\end{itemize}
    \end{itemize}
\vspace{0.3cm}

\end{vbframe}

\begin{vbframe}[c]{LIME Pitfalls: Local vs. Global Features \citebutton{Laugel et al. 2018}{http://webia.lip6.fr/~laugel/files/WHI_ICML_slides.pdf}}

\begin{columns}
	\begin{column}{0.78\textwidth}
\begin{itemize}
	\item \textbf{Problem}: Some features influence the \textbf{global} shape of the black-box model vs. \textbf{Local} features impact predictions for a small area of $\Xspace$. 
	\item \textbf{Example}: Decision trees; splitting variables close to the root have a more global influence than the ones close to the leaves.
	\item \textbf{Problem}: Instances for the surrogate model are sampled from the whole input space 
	\begin{itemize}
	    \item[$\leadsto$] Hide the features with a local influence for the benefit of features with a global influence even when the kernel width was reduced. 
	\end{itemize}
	\item \textbf{Solution}: Sample new instances $\zv$ around the decision boundary closest from point $\xv$ for higher local accuracy.
\end{itemize}
\end{column}
\begin{column}{0.2\textwidth}

		\includegraphics[width=1\textwidth]{figure/lime_bordersample2}
		
		\vspace{-0.3cm}
		
 	\end{column}
\end{columns}

\end{vbframe}


\begin{vbframe}{LIME Pitfalls: Local vs. Global Features: Visualization \citebutton{Laugel et al. 2018}{http://webia.lip6.fr/~laugel/files/WHI_ICML_slides.pdf}}

\begin{columns}
	\begin{column}{0.6\textwidth}
		\begin{itemize}
		\item Given in figure to the right:
		\begin{itemize}
		    \item Green dot: To be explained
		    \item Background color: Classification of random forest
		    \item Grey curve: Decision boundary
		    \item Red dot: Point closest to the border detected by this method. 
		\end{itemize}
		\item \textbf{Observation: }The decision boundary of LIME with different kernels (blue and green) do not match the direction of the steeper local boundary. 
		\item The method \citebutton{Laugel et al. 2018}{http://webia.lip6.fr/~laugel/files/WHI_ICML_slides.pdf} (red line) approximates the local border direction better. 
	\end{itemize}
\end{column}
\begin{column}{0.39\textwidth}
\vspace{0.3cm}

	\begin{center}
	\includegraphics[width=1\textwidth]{figure/lime-globallocal2}
	
	\vspace{-0.3cm}
	{Half-moons dataset.}
	
\end{center}

	\end{column}
\end{columns}
%\footnote[frame]{Laugel et al. (2018). Defining Locality for Surrogates in Post-hoc Interpretability. \url{https://arxiv.org/pdf/1806.07498.pdf}.}
\end{vbframe}

\begin{vbframe}[c]{LIME Pitfalls: Faithfulness}
\begin{itemize}
	\item \textbf{Problem}: Trade-off between local fidelity vs. sparsity
	\item \textbf{Observation I}: Low fidelity $\leadsto$ Unreliable explanations
	\item \textbf{Observation II}: High fidelity requires more complex models $\leadsto$ surrogate model cannot easily interpreted
\end{itemize}

\end{vbframe}

\begin{vbframe}{LIME Pitfalls: Hiding biases \citebutton{Slack et al. 2020}{https://arxiv.org/abs/1911.02508}}

\begin{itemize}
	\item \textbf{Problem}: Developer could manipulate their model to hide biases. 
	\item \textbf{Insight}: LIME can sample out-of-distribution points
	\item \textbf{Attack}:
	\begin{enumerate}
	    \item classifier to discriminate between in-distribution and out-of-distribution data points
	    \item for in-distribution points, use the original (biased) model
	    \item for out-of-distribution points, use an unbiased model 
	    \item[$\leadsto$] LIME will sample many points from the unbiased model and the local explanation will hide the bias of the true model
	\end{enumerate}
	\item \textbf{Example}: Credit dataset; a biased model trained on Feature `gender`; unbiased model could be trained on features uncorrelated with `gender`
\end{itemize}
\end{vbframe}

\begin{vbframe}[c]{LIME Pitfalls: Robustness \citebutton{Alvarez-Melis, D., \& Jaakkola, T. 2018}{https://arxiv.org/abs/1806.08049}}
\begin{itemize}
	\item \textbf{Problem}: Instability of explanations. 
	\item \textbf{Observation}: Explanations of two very close points could vary greatly. 
	\item \textbf{Rashomon Effect}: There could be more than one explanation for a prediction.
	\begin{itemize}
	    \item[$\leadsto$] For example can happen because of other models $\fh$ (trained on the same data) or other sampled data points $\zv$
	\end{itemize}
\end{itemize}
\vspace{-0.7cm}
\begin{columns}
	\begin{column}{0.48\textwidth}
		\begin{center}
		
		\includegraphics[width=0.55\textwidth]{figure/lime_robustness_1.png}
		
		{Linear prediction task (logistic regression). \\Linear surrogate returns similar coefficients for similar points.}
		
		\end{center}
	\end{column}
	\begin{column}{0.48\textwidth}
		\begin{center}
	\includegraphics[width=0.55\textwidth]{figure/lime_robustness_2.png}
	
	{Circular prediction task (random forest). \\Linear surrogate returns different coefficients for similar points.}
	
	\end{center}
\end{column}
\end{columns}
\end{vbframe}

\begin{vbframe}{LIME Pitfalls: Definition of Superpixels}

\begin{columns}
    
    \begin{column}{0.6\textwidth}
        
        \begin{itemize}
        	\item \textbf{Problem}: Instability because of specification of superpixels for image data. 
        	\item \textbf{Observation}: Multiple specification of superpixels exist, influencing both the shape and size. 
        	\item \textbf{Implication}: The specification of superpixel has a large influence on the explanations. 
        	\item \textbf{Attack}: Change superpixels as part of an adversarial attack $\leadsto$ changed explanation
        \end{itemize}
        
    \end{column}
    
    \begin{column}{0.4\textwidth}
    
        \centering
        \includegraphics[width=0.7\textwidth]{figure/superpixel_woman}
        
    \end{column}
    
\end{columns}

\end{vbframe}


\endlecture
\end{document}