\documentclass[11pt,compress,t,notes=noshow, aspectratio=169, xcolor=table]{beamer}

\usepackage{../../style/lmu-lecture}
% Defines macros and environments
\usepackage{bbm}
% basic latex stuff
\newcommand{\pkg}[1]{{\fontseries{b}\selectfont #1}} %fontstyle for R packages
\newcommand{\lz}{\vspace{0.5cm}} %vertical space
\newcommand{\dlz}{\vspace{1cm}} %double vertical space
\newcommand{\oneliner}[1] % Oneliner for important statements
{\begin{block}{}\begin{center}\begin{Large}#1\end{Large}\end{center}\end{block}}


%new environments
\newenvironment{vbframe}  %frame with breaks and verbatim
{
 \begin{frame}[containsverbatim,allowframebreaks]
}
{
\end{frame}
}

\newenvironment{vframe}  %frame with verbatim without breaks (to avoid numbering one slided frames)
{
 \begin{frame}[containsverbatim]
}
{
\end{frame}
}

\newenvironment{blocki}[1]   % itemize block
{
 \begin{block}{#1}\begin{itemize}
}
{
\end{itemize}\end{block}
}

\newenvironment{fragileframe}[2]{  %fragile frame with framebreaks
\begin{frame}[allowframebreaks, fragile, environment = fragileframe]
\frametitle{#1}
#2}
{\end{frame}}


\newcommand{\myframe}[2]{  %short for frame with framebreaks
\begin{frame}[allowframebreaks]
\frametitle{#1}
#2
\end{frame}}

\newcommand{\remark}[1]{
  \textbf{Remark:} #1
}


\newenvironment{deleteframe}
{
\begingroup
\usebackgroundtemplate{\includegraphics[width=\paperwidth,height=\paperheight]{../style/color/red.png}}
 \begin{frame}
}
{
\end{frame}
\endgroup
}
\newenvironment{simplifyframe}
{
\begingroup
\usebackgroundtemplate{\includegraphics[width=\paperwidth,height=\paperheight]{../style/color/yellow.png}}
 \begin{frame}
}
{
\end{frame}
\endgroup
}\newenvironment{draftframe}
{
\begingroup
\usebackgroundtemplate{\includegraphics[width=\paperwidth,height=\paperheight]{../style/color/green.jpg}}
 \begin{frame}
}
{
\end{frame}
\endgroup
}
% https://tex.stackexchange.com/a/261480: textcolor that works in mathmode
\makeatletter
\renewcommand*{\@textcolor}[3]{%
  \protect\leavevmode
  \begingroup
    \color#1{#2}#3%
  \endgroup
}
\makeatother


\providecommand{\tightlist}{%
  \setlength{\itemsep}{0pt}\setlength{\parskip}{0pt}}

%\setbeamerfont{footnote}{size=\tiny}
\usepackage[hang,flushmargin]{footmisc}
\renewcommand*{\footnotelayout}{\tiny}
\renewcommand*{\thefootnote}{} %\fnsymbol{footnote}

% https://tex.stackexchange.com/questions/30720/footnote-without-a-marker
% \makeatletter
% \def\blfootnote{\gdef\@thefnmark{}\@footnotetext}
% \makeatother

% https://tex.stackexchange.com/questions/357717/beamer-allowframebreaks-option-and-vertical-spacing-when-using-lists-itemize
% \setbeamertemplate{frametitle continuation}{%
%     (\insertcontinuationcount)%
%     \ifnum\insertcontinuationcount>1%
%     \vspace*{\topsep}%
%     \else%
%     %
%     \fi%
% }


\newcommand{\pih}{\fh}

\title{Interpretable Machine Learning}
% \author{LMU}
%\institute{\href{https://compstat-lmu.github.io/lecture_iml/}{compstat-lmu.github.io/lecture\_iml}}
\date{}

\begin{document}

	
% Set style/preamble.Rnw as parent.

% Load all R packages and set up knitr

% This file loads R packages, configures knitr options and sets preamble.Rnw as 
% parent file
% IF YOU MODIFY THIS, PLZ ALSO MODIFY setup.Rmd ACCORDINGLY...

% Defines macros and environments

\newcommand{\titlefigure}{figure/lime5}
\newcommand{\learninggoals}{
	\item Understand motivation for LIME
	\item Develop a mathematical intuition
	\item See various applications}

\lecturechapter{LIME}
\lecture{Interpretable Machine Learning}

% Prerequisite: le-intro

% ------------------------------------------------------------------------------
\begin{frame}[c]{LIME}
\begin{itemize}
		\item Local Interpretable Model-agnostic Explanations (LIME) assume that even if a machine learning model is very complex, the local prediction can be described with a simpler model.
		\smallskip\pause
		\item  Therefore, LIME explains \textbf{individual} predictions of \textbf{any} black-box model by approximating the model \textbf{locally} with an interpretable model.
		\smallskip\pause
		\item These approximations are called local surrogate models. \\
		Typically, they are linear models or trees as interpretable models by design.
		\smallskip\pause
		\item LIME should answer why a machine learning model predicted $y$ for input $\xv$.
		\smallskip\pause
		\item Since LIME can be applied to any black-box model LIME is model-agnostic.  
		\smallskip\pause
		\item LIME can handle tabular, image and text data. 
\end{itemize}
\end{frame}


\begin{frame}[c]{LIME: Characteristics}

    \textbf{Definition:}\\
	LIME provides a local explanation for a black-box model $\fh$ in form of a model $g \in \Gspace$ with $\Gspace$ as the class of potential (interpretable) models.\\[2em]
	
	
	Model $g$ should have two characteristics:
	\begin{enumerate}
		\item \textbf{Interpretable}: relation between the input variables and the response are easy to understand.  
		\item \textbf{Locally faithful / Fidelity}: similar behavior as $\fh$ in the vicinity of the instance being predicted.
	\end{enumerate}
	
	\vspace{2em}
	Formally, we want to receive a model $g$ with \textbf{minimal complexity and maximal local-fidelity}. 
\end{frame}


\begin{frame}[c]{Model Complexity}
    
    We can measure the complexity of a model $g$ using $J(g)$. \\[2em]

 	\textbf{Example: Linear model}\\
 	Let $\Gspace = \left\{g: \Xspace \to \R ~|~g(\xv) = s(\thetab^\top \xv)\right\}$ be the class of linear models with $s(\cdot)$ being either the identity function for linear regression or the logistic sigmoid function for logistic regression. \\
 	Then, $J(g) = \sum_{j = 1}^p \Ind_{\{ \theta_j \neq 0 \}}$ could be the L$_0$ loss, i.e. the number of non-zero coefficients. 
 	\vspace{0.5cm}
 	
 	\textbf{Example: Tree}\\
 	Let $\Gspace = \left\{g:\Xspace \to \R ~|~g(\xv) = \sum_{m=1}^M c_m \Ind_{\{ \xv \in Q_m \}}\right\}$ be the class of trees (i.e., class of additive model over the leaf-rectangles $Q_m$) then $J(g)$ could measure the number of terminal/leaf nodes.\\
 	
\end{frame}
 
\begin{frame}{Local model fidelity}
 	\begin{itemize}
 		\item A model $g$ is locally faithful to $\fh$ w.r.t. a point $\xv$\\ 
 		if for points $\zv \in \Zspace \subseteq \R^p$ close to $\xv$, the predictions of $g(\zv)$ are close to $\fh(\zv)$. 
 		\item In an optimization task: the closer $\zv$ is to $\xv$, the closer $g(\xv)$ should be to $\fh(\zv)$.  
 		\pause
 		\item Two required measures:
 		\begin{enumerate}
 			\item A proximity (similarity) measure $\neigh(\zv)$ between $\zv$ and $\xv$, e.g. the exponential kernel:
 			$$\neigh(\zv) = exp(-d(\xv, \zv)^2/\sigma^2)$$ 
 			with $\sigma$ as the kernel width and $d$ as the Euclidean distance (numeric features) or the Gower distance $d_G$ (mixed features). 
 			\pause
 			\item A distance measure or loss function $L(\fh(\zv), g(\zv))$, e.g. the L$_2$ loss/squared error
 			$$L(\fh(\zv), g(\zv)) = (g(\zv) - \fh(\zv))^2.$$ 
 		\end{enumerate}
 		\pause
 		\item Given points $\zv$, we can measure local fidelity of $g$ with respect to $\fh$ in terms of a weighted loss
 		$$ L(\fh, g, \neigh) = \sum_{\zv \in \Zspace} \neigh(\zv) L(\fh(\zv), g(\zv)) $$
 		%\item Note that identifying \textbf{locally} faithful explanations that are interpretable is less of a challenge than identifying \textbf{globally} faithful explanations. Yet, global fidelity implies local fidelity but not vice versa.
 	\end{itemize}
\end{frame}

\begin{frame}[c]{Minimization task}
	\begin{itemize}
		\item Optimization objective of LIME: 
		$$ \argmin_{g \in \Gspace} L(\fh, g, \neigh) + J(g)$$
		\item In practice:
		\begin{itemize}
		    \item LIME only optimizes $L(\fh, g, \neigh)$ (model-fidelity). 	
		    \item Users decide threshold on model complexity $J(g)$ beforehand
		\end{itemize}
		\item Goal: \textbf{model-agnostic} explainer
		\begin{itemize}
    		\item[$\leadsto$] optimize $L(\fh, g, \neigh)$ without making any assumptions about $\fh$. 
    		\item[$\leadsto$] learn $g$ only approximately.  
		\end{itemize}
		\end{itemize}
\end{frame} 

\begin{frame}[c]{LIME Algorithm: Outline}
		
		\textbf{Input}:
		\begin{itemize}
		    \item a pre-trained model $\fh$
		    \item an $\xv$ data points whose prediction $\fh(\xv)$ we want to explain
		    \item model class $\Gspace$ for local surrogate (to limit the complexity of the explanation)
		\end{itemize}
		
		\pause
		\medskip
		
		\textbf{Algorithm}:
		\begin{enumerate}
    		\item Independently sample new points $\zv \in \Zspace$. 
    		\item Retrieve predictions $\fh(\zv)$ for obtained points $\zv$. 
    		\item Weight $\zv \in \Zspace$ by their proximity $\neigh(\zv)$.
    		\item Train an interpretable surrogate model $g$ on weighted data points $\zv \in \Zspace$.\\ The obtained predictions $\fh(\zv)$ is the target of this model.
    		\item Return the interpretable model $g$ as the explainer.
		\end{enumerate}
		

	
\end{frame} 

\begin{frame}[c]{LIME Algorithm: Example}

    	\textbf{Illustration} of LIME based on a classification task:
		\begin{itemize}
			\item Light/dark gray background: prediction surface of a classifier.
			\item Yellow point: $\xv$. 
			\item $\Gspace$: class of logistic regression models. 
		\end{itemize}
		\begin{center}
			\includegraphics[width=0.35\textwidth]{figure/lime2}
		\end{center}

\end{frame} 


\begin{frame}[c]{LIME Algorithm: Example (Step 1+2: Sampling) \citebutton{Ribeiro. 2016}{https://github.com/marcotcr/lime}}
		
		Strategies for sampling: 
		\begin{itemize} 
			\item Uniformly sample new points from the feasible feature range. 
			\item Use the training data set with or without perturbations.
			\item Draw samples from the estimated univariate distribution of each feature.
			\item Create an equidistant grid over the supported feature range.  
		\end{itemize}
		\begin{center}
			\includegraphics[width=0.4\textwidth]{figure/lime3} \hspace{0.1cm}
			\includegraphics[width=0.4\textwidth]{figure/lime3a}
		\end{center}

\end{frame}
		
\begin{frame}{LIME Algorithm: Example (Step 3: Proximity) \citebutton{Ribeiro. 2016}{https://github.com/marcotcr/lime}}

	In this example, we use the exponential kernel defined on the Euclidean distance $d$
		 $$\neigh(\zv) = exp(-d(\xv, \zv)^2/\sigma^2).$$ 
		\begin{center}
			\includegraphics[width=0.4\textwidth]{figure/lime4}
		\end{center}
		
% MARIUS: Not relevant for the example; if we want to introduce it, we should do it somewhere else
% 	An alternative is the Gower proximity: 
% 	$\neigh(\zv) = 1 - \Gower(\zv, \xv) =  1 - \frac{1}{p}\sum_{j = 1}^{p} \delta_G(z_j, x_j) $ 
% 	$\textnormal{ with } \delta_G(z_j, x_j) = 
% 	\begin{cases}
% 	\frac{1}{\widehat{R}_j}|z_j- x_j| & \text{if $x_j$ and $z_j$ are numerical} \\
% 	\mathbb{I}_{z_j \neq x_j} & \text{if $x_j$ and $z_j$ are categorical}
% 	\end{cases}.$
		
\end{frame}
		
\begin{frame}[c]{LIME Algorithm: Example (Step 4: Surrogate) \citebutton{Ribeiro. 2016}{https://github.com/marcotcr/lime}}
		Popular interpretable models include linear models, LASSO, classification/regression trees and\\ decision rules.\\
		In our example, we fit a \textbf{logistic regression} model. (Consequently, $L(\fh(\zv), g(\zv))$ is the Bernoulli loss.)
		\begin{center}
			\includegraphics[width=0.4\textwidth]{figure/lime5}
		\end{center}
\end{frame}

\begin{frame}[c]{Example on Credit Dataset (Tabular)}
	\begin{itemize}
		\item Model: SVM with RBF kernel
		\item $\xv$: first data point of the dataset with $\fh_{bad}(\xv) = 0.658$
		\item $\zv$: training data. They are weighted by the Gower proximity. 
		\item Surrogate model $g$: L1-regularized linear model with 5 features. 
	\end{itemize}

    \bigskip

	\begin{table}[ht]
		\centering
		\scriptsize
		\begin{tabular}{rlrlllrrl}
			\hline
			age & sex & job & housing & saving & checking & credit.amount & duration & purpose \\ 
			\hline
			 22 & female &   2 & own & little & moderate & 5951 &  48 & radio/TV \\ 
			\hline
		\end{tabular}
	\end{table}

\end{frame}

\begin{frame}[c]{Example on Credit Dataset (cont'd)}


\begin{center}
	\includegraphics[width=0.45\textwidth]{figure/lime_credit.pdf}\\
	{Effects of surrogate model, i.e. $\thetah^T \xv$.}
\end{center}

\begin{itemize}
	\item The local model prediction for $\xv$ is $g(\xv) = 0.64$ vs. $\fh(\xv) = 0.658$. 
	\item $g$ has a local fidelity of $L(\pih, g, \neigh) = 4.82$ with $\neigh(\zv)$ as the Gower proximity and $L(\pih_{bad}(\zv), g(\zv))$ as the euclidean distance. 
\end{itemize}

\end{frame}
	
\begin{frame}[c]{Example on Credit Dataset (cont'd)}

\begin{itemize}	
	\item 2-dim ICE plots (aka. prediction surface plot) of credit amount and duration show how the surrogate model $g$ linearly approximates the previously nonlinear prediction surface of $\pih_{bad}$. 
\end{itemize}
\vspace{-0.4cm}
 \begin{columns}
	\begin{column}{0.47\textwidth}
		\begin{center}
		\includegraphics[width=1\textwidth]{figure/lime_credit_ice1.pdf}
		\end{center}		
	\end{column}
	\begin{column}{0.46\textwidth}  
		\begin{center}
				\includegraphics[width=1\textwidth]{figure/lime_credit_ice2.pdf}
		\end{center}
			
	\end{column}
\end{columns}
\vspace{-0.4cm}
\begin{center}
		{2-dim ICE plot of $\fh_{bad}$ (\textbf{left}) and surrogate $g$ (\textbf{right}) for features duration and credit amount. \\The white dot is $\xv$. The histograms display the marginal distribution of the training data $\Xmat$.}
\end{center}

\end{frame}

%\begin{frame}[containsverbatim,allowframebreaks]{Bike Sharing Dataset}
%\vspace{-.3cm}
%
%\begin{center}
%\includegraphics[width=0.7\textwidth]{figure/bike-figure.png}
%\end{center} 
%
%\footnotesize \textbf{Figure:} LIME for two example instances of the bike sharing dataset.
%
%\normalsize
%\vspace{0.2cm}
%The plots show the feature effect of the sparse linear model, i.e. the model coefficients times the feature value of the instance.
%Warmer temperature has a positive effect on the prediction, 
%while the year 2011 has a large negative effect as well as the springtime.
%\end{frame}

\begin{frame}{LIME for Text Data \citebutton{Shen, Ian, (2019)}{https://medium.com/just-another-data-scientist/explain-sentiment-prediction-with-lime-f90ae83da2da}}
    LIME can also be applied to text data: 
	\begin{itemize}
		\item Raw text representations: 
		\begin{itemize}
		    \item binary vector indicating the presence or absence of a word 
		    \item a vector of word counts
		\end{itemize}
		\item Examples for \textit{``This text is the first text."} and \textit{``Finally, this is the last one."}:
		\begin{center}
			\begin{tabular}{c|c|c|c|c|c|c|c} 
				this & text & is & the & first & finally & last & one \\ 
				\hline
				1 & 2 & 1 & 1 & 1 & 0 & 0 & 0 \\
				1 & 0 & 1 & 1 & 0 & 1 & 1 & 1 \\
			\end{tabular}
		\end{center} 
		\item \textbf{Sampling}: Randomly set the entry of individual words to $0$; equal to removing all occurrences of this word in the text. 
		\item \textbf{Proximity}: Exponential kernel with cosine distance. 
		\begin{itemize}
		    \item neglects words that do not occur in both texts 
		    \item measures the distance irrespective of the text size.
		\end{itemize}
	\end{itemize}
\end{frame}
	
\begin{frame}{LIME for Text Data (cont'd) \citebutton{Shen, Ian, (2019)}{https://medium.com/just-another-data-scientist/explain-sentiment-prediction-with-lime-f90ae83da2da}}	 
	\begin{itemize}
		\item Random forest classifier labeling movie reviews from IMDB. 
		\begin{itemize}
		    \item \textcolor{blue}{0}: negative
		    \item \textcolor{orange}{1}: positive
		\end{itemize}
		\item Surrogate model is a sparse linear model. 
	\end{itemize}
	
	\begin{figure}
		\begin{center}
			%\captionsetup{font = scriptsize, labelfont = {bf, scriptsize}}
			\includegraphics[width=0.9\textwidth]{figure/lime_movier}
		\end{center}
	\end{figure}
	
	{Words like ``worst`` or ``waste`` indicate a negative review while words like ``best`` or ``great`` indicate a positive review.}
	
	\end{frame}
	
\begin{frame}[c]{LIME for image data}
	\begin{columns}
		\begin{column}{0.67\textwidth}
			LIME also works for image data:  
			\begin{itemize}
				\item \textbf{Idea}: Each instance is represented as a binary vector indicating the presence or absence of superpixels. \citebutton{Achanta et al. 2012}{https://ieeexplore.ieee.org/document/6205760}
				\item Superpixels are interconnected pixels with similar colors. A single pixel might not change a prediction by much.
				\item \textbf{Warning}: The size of superpixels needs to be determined before the segmentation takes place.
				\item \textbf{Sampling}: Randomly switching some of the super pixels ``off", i.e., by coloring some superpixels uniformly
			\end{itemize}		
		\end{column}
		\begin{column}{0.26\textwidth}  
			\begin{center}
				\includegraphics[width=1\textwidth]{figure/superpixel_woman}
				{Example for superpixels of different sizes.}
			\end{center}
		\end{column}
	\end{columns}
    
\end{frame}
\begin{frame}{LIME for image data (cont'd) \citebutton{Ribeiro. 2016}{https://github.com/marcotcr/lime}}
	\begin{itemize}
		\item Explaining prediction of pre-trained Inception neural network classifier
		\item \textbf{Sampling}: Graying out all superpixels besides 10
		\item \textbf{Surrogate}: Locally weighted sparse linear models 
		\item \textbf{Proximity}: Exponential kernel with euclidean distance.
	\end{itemize}
% https://lime-ml.readthedocs.io/en/latest/lime.html#module-lime.lime_image
	\vspace{-0.3cm}
	\begin{center}
		\includegraphics[width=0.8\textwidth]{figure/lime-images}
		
		{Top 3 classes predicted.}
	\end{center}
	
\end{frame}
\endlecture
\end{document}