\documentclass[11pt,compress,t,notes=noshow, xcolor=table]{beamer}
\usepackage[]{graphicx}\usepackage[]{color}
% maxwidth is the original width if it is less than linewidth
% otherwise use linewidth (to make sure the graphics do not exceed the margin)
\makeatletter
\def\maxwidth{ %
\ifdim\Gin@nat@width>\linewidth
\linewidth
\else
\Gin@nat@width
\fi
}
\makeatother

\definecolor{fgcolor}{rgb}{0.345, 0.345, 0.345}
\newcommand{\hlnum}[1]{\textcolor[rgb]{0.686,0.059,0.569}{#1}}%
\newcommand{\hlstr}[1]{\textcolor[rgb]{0.192,0.494,0.8}{#1}}%
\newcommand{\hlcom}[1]{\textcolor[rgb]{0.678,0.584,0.686}{\textit{#1}}}%
\newcommand{\hlopt}[1]{\textcolor[rgb]{0,0,0}{#1}}%
\newcommand{\hlstd}[1]{\textcolor[rgb]{0.345,0.345,0.345}{#1}}%
\newcommand{\hlkwa}[1]{\textcolor[rgb]{0.161,0.373,0.58}{\textbf{#1}}}%
\newcommand{\hlkwb}[1]{\textcolor[rgb]{0.69,0.353,0.396}{#1}}%
\newcommand{\hlkwc}[1]{\textcolor[rgb]{0.333,0.667,0.333}{#1}}%
\newcommand{\hlkwd}[1]{\textcolor[rgb]{0.737,0.353,0.396}{\textbf{#1}}}%
\newcommand{\predvar}{Var\left[\hat{f}(x)\right]}
\let\hlipl\hlkwb

\usepackage{framed}
\makeatletter
\newenvironment{kframe}{%
\def\at@end@of@kframe{}%
\ifinner\ifhmode%
\def\at@end@of@kframe{\end{minipage}}%
\begin{minipage}{\columnwidth}%
\fi\fi%
\def\FrameCommand##1{\hskip\@totalleftmargin \hskip-\fboxsep
\colorbox{shadecolor}{##1}\hskip-\fboxsep
% There is no \\@totalrightmargin, so:
\hskip-\linewidth \hskip-\@totalleftmargin \hskip\columnwidth}%
\MakeFramed {\advance\hsize-\width
\@totalleftmargin\z@ \linewidth\hsize
\@setminipage}}%
{\par\unskip\endMakeFramed%
\at@end@of@kframe}
\makeatother

\definecolor{shadecolor}{rgb}{.97, .97, .97}
\definecolor{messagecolor}{rgb}{0, 0, 0}
\definecolor{warningcolor}{rgb}{1, 0, 1}
\definecolor{errorcolor}{rgb}{1, 0, 0}
\newenvironment{knitrout}{}{} % an empty environment to be redefined in TeX

\usepackage{alltt}
\newcommand{\SweaveOpts}[1]{}  % do not interfere with LaTeX
\newcommand{\SweaveInput}[1]{} % because they are not real TeX commands
\newcommand{\Sexpr}[1]{}       % will only be parsed by R

\usepackage[english]{babel}
\usepackage[utf8]{inputenc}

\usepackage{dsfont}
\usepackage{verbatim}
\usepackage{amsmath}
\usepackage{amsfonts}
\usepackage{bm}
\usepackage{csquotes}
\usepackage{multirow}
\usepackage{longtable}
\usepackage{booktabs}
\usepackage{enumerate}
\usepackage[absolute,overlay]{textpos}
\usepackage{psfrag}
\usepackage{algorithm}
\usepackage{algpseudocode}
\usepackage{eqnarray}
\usepackage{arydshln}
\usepackage{tabularx}
\usepackage{placeins}
\usepackage{tikz}
\usepackage{setspace}
\usepackage{colortbl}
\usepackage{mathtools}
\usepackage{wrapfig}
\usepackage{bm}
\usepackage{verbatim}

\usetikzlibrary{shapes,arrows,automata,positioning,calc,chains,trees, shadows}
\tikzset{
%Define standard arrow tip
>=stealth',
%Define style for boxes
punkt/.style={
rectangle,
rounded corners,
draw=black, very thick,
text width=6.5em,
minimum height=2em,
text centered},
% Define arrow style
pil/.style={
->,
thick,
shorten <=2pt,
shorten >=2pt,}
}

\usepackage{subfig}

% Defines macros and environments
\usepackage{bbm}
% basic latex stuff
\newcommand{\pkg}[1]{{\fontseries{b}\selectfont #1}} %fontstyle for R packages
\newcommand{\lz}{\vspace{0.5cm}} %vertical space
\newcommand{\dlz}{\vspace{1cm}} %double vertical space
\newcommand{\oneliner}[1] % Oneliner for important statements
{\begin{block}{}\begin{center}\begin{Large}#1\end{Large}\end{center}\end{block}}


%new environments
\newenvironment{vbframe}  %frame with breaks and verbatim
{
 \begin{frame}[containsverbatim,allowframebreaks]
}
{
\end{frame}
}

\newenvironment{vframe}  %frame with verbatim without breaks (to avoid numbering one slided frames)
{
 \begin{frame}[containsverbatim]
}
{
\end{frame}
}

\newenvironment{blocki}[1]   % itemize block
{
 \begin{block}{#1}\begin{itemize}
}
{
\end{itemize}\end{block}
}

\newenvironment{fragileframe}[2]{  %fragile frame with framebreaks
\begin{frame}[allowframebreaks, fragile, environment = fragileframe]
\frametitle{#1}
#2}
{\end{frame}}


\newcommand{\myframe}[2]{  %short for frame with framebreaks
\begin{frame}[allowframebreaks]
\frametitle{#1}
#2
\end{frame}}

\newcommand{\remark}[1]{
  \textbf{Remark:} #1
}


\newenvironment{deleteframe}
{
\begingroup
\usebackgroundtemplate{\includegraphics[width=\paperwidth,height=\paperheight]{../style/color/red.png}}
 \begin{frame}
}
{
\end{frame}
\endgroup
}
\newenvironment{simplifyframe}
{
\begingroup
\usebackgroundtemplate{\includegraphics[width=\paperwidth,height=\paperheight]{../style/color/yellow.png}}
 \begin{frame}
}
{
\end{frame}
\endgroup
}\newenvironment{draftframe}
{
\begingroup
\usebackgroundtemplate{\includegraphics[width=\paperwidth,height=\paperheight]{../style/color/green.jpg}}
 \begin{frame}
}
{
\end{frame}
\endgroup
}
% https://tex.stackexchange.com/a/261480: textcolor that works in mathmode
\makeatletter
\renewcommand*{\@textcolor}[3]{%
  \protect\leavevmode
  \begingroup
    \color#1{#2}#3%
  \endgroup
}
\makeatother


\providecommand{\tightlist}{%
  \setlength{\itemsep}{0pt}\setlength{\parskip}{0pt}}

%\setbeamerfont{footnote}{size=\tiny}
\usepackage[hang,flushmargin]{footmisc}
\renewcommand*{\footnotelayout}{\tiny}
\renewcommand*{\thefootnote}{} %\fnsymbol{footnote}

% https://tex.stackexchange.com/questions/30720/footnote-without-a-marker
% \makeatletter
% \def\blfootnote{\gdef\@thefnmark{}\@footnotetext}
% \makeatother

% https://tex.stackexchange.com/questions/357717/beamer-allowframebreaks-option-and-vertical-spacing-when-using-lists-itemize
% \setbeamertemplate{frametitle continuation}{%
%     (\insertcontinuationcount)%
%     \ifnum\insertcontinuationcount>1%
%     \vspace*{\topsep}%
%     \else%
%     %
%     \fi%
% }


%\usetheme{lmu-lecture}
\usepackage{../../style/lmu-lecture}

\let\code=\texttt
\let\proglang=\textsf

\setkeys{Gin}{width=0.9\textwidth}

\title{Interpretable Machine Learning}
% \author{Bernd Bischl, Christoph Molnar, Daniel Schalk, Fabian Scheipl}
\institute{\href{https://compstat-lmu.github.io/lecture_iml/}{compstat-lmu.github.io/lecture\_iml}}
\date{}

\setbeamertemplate{frametitle}{\expandafter\uppercase\expandafter\insertframetitle}

\begin{document}

\input{../../latex-math/basic-math}
\input{../../latex-math/basic-ml}
\newcommand{\titlefigure}{figure/open_blackbox}
\newcommand{\learninggoals}{
\item Examples for interpretable models: (generalized) linear models, generalized additive models, model-based boosting}

\lecturechapter{Interpretable Models}
\lecture{Interpretable Machine Learning}

\begin{vbframe}{Linear Regression}

\begin{itemize}
\item For linear regression models, we only estimate the model parameters. The model equation is manually specified and known in advance:
$$
y = \beta_0 + \beta_1 x_1 + \dots + \beta_p x_p + \dots + \epsilon
$$
\item The predictive power of LMs is determined by specifying the correct model structure.
\item The model equation is identical across the entire feature space, i.e., local interpretations are identical to global ones. By knowing the model equation, we can exactly determine feature effects (e.g., beta coefficients, effect plots) and importance scores (e.g., p-values, t-statistics).
\item Note that for inference-based metrics (p-values, t-statistics, confidence intervals) to be valid, the error term needs to be normally distributed with zero mean (i.e., $(Y \vert X) \sim N(x^T\beta, \sigma^2)$. This severely restricts the usage of LMs in practice.
\end{itemize}
\end{vbframe}

\begin{vbframe}{Linear Regression}
\tiny
\verbatiminput{figure/lm_output.txt}
\end{vbframe}

\begin{vbframe}{Linear Regression}
\begin{figure}
  \includegraphics[width = 0.6\textwidth]{figure/lm_effect_plot.png}
\end{figure}
\end{vbframe}


\begin{vbframe}{Generalized Linear Regression}

\begin{itemize}
\item Generalized linear models (GLMs) are able to model data more flexibly through the link function, but keep the pre-specified model equation: 
$$
g\left(\mathbb{E}_Y(y \vert x)\right) = \beta_0 + \beta_1 x_1 + \dots + \beta_p x_p
$$
\item This flexibility relates to the distribution of the target. The target conditional on the features can follow a distribution of the exponential family, e.g., Binomial, Poisson, Exponential, Gamma. We still need to specify the correct model equation in advance in order to receive a good model fit.
\item As the model equation is still known, interpretations are possible the same way as for linear models. However, even linear terms become non-linear through the link function!
\end{itemize}
\end{vbframe}


\begin{vbframe}{Generalized Linear Regression}
\tiny
\verbatiminput{figure/glm_output.txt}
\end{vbframe}

\begin{vbframe}{Generalized Linear Regression}
\begin{figure}
  \includegraphics[width = 0.6\textwidth]{figure/glm_effect_plot.png}
\end{figure}
\end{vbframe}

\begin{vbframe}{Generalized Additive Models}

\begin{itemize}
\item A generalized additive model (GAM) adds flexibility and predictive power by replacing pre-specified terms with smoothing functions:
$$
g\left(\mathbb{E}_Y(y \vert x)\right) = \beta_0 + \beta_1 h_1(x_1) + \dots + \beta_p h_p(x_p) + \dots
$$
\item For the component functions, we may either specify a parametric form (e.g., a regression splines), or a non-parametric one, e.g., locally estimated scatterplot smoothing (LOESS).
\item This makes GAMs much more flexible regarding the model structure, which is largely determined by the structure of the data instead of premade assumptions as in LMs and GLMs.
\item A GAM retains interpretability by keeping the additive model equation (as long as the component functions are interpretable). As the model equation is known, we can use similar interpretation methods as for LMs and GLMs, e.g., evaluating the estimated functions that depend on the features of interest.
\end{itemize}


\end{vbframe}


\begin{vbframe}{Interpretable Models}


\textbf{Model-based boosting:}

\begin{itemize}

\item 
Idea: Combine boosting with interpretable base learners (e.g., linear model with single parameter).
\item
Consider two linear base learners $b_j(x, \Theta)$ and $b_j(x, \Theta^{\star})$ with the same type, but distinct parameter vectors $\Theta$ and $\Theta^{\star}$. They can be combined in a base learner of the same type:

$$
b_j(x, \Theta) + b_j(x, \Theta^{\star}) = b_j(x, \Theta + \Theta^{\star})
$$
\item We create a selection of interpretable base learners. In each iteration, all base learners are trained on the so-called pseudo residuals, and the one with the best fit is added to the previously computed model.
\item The final model has an additive structure (equivalent to a GAM), where each component function is itself interpretable.
\end{itemize}



\end{vbframe}

\begin{vbframe}{Interpretable Models}


\textbf{Rule-based ML:}


Decision rules follow a general structure: IF the conditions are met THEN make a certain prediction. A single decision rule or a combination of several rules can be used to make predictions.

\vspace{0.5cm}
There are many ways to learn rules from data:
\begin{itemize}
\item OneR learns rules from a single feature. OneR is characterized by its simplicity, interpretability and its use as a benchmark.
\item Sequential covering is a general procedure that iteratively learns rules and removes the data points that are covered by the new rule. This procedure is used by many rule learning algorithms.
\item Bayesian Rule Lists combine pre-mined frequent patterns into a decision list using Bayesian statistics. Using pre-mined patterns is a common approach used by many rule learning algorithms.
\end{itemize}

\end{vbframe}


\endlecture
\end{document}
