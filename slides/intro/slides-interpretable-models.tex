\documentclass[11pt,compress,t,notes=noshow, xcolor=table]{beamer}
\usepackage[]{graphicx}\usepackage[]{color}
% maxwidth is the original width if it is less than linewidth
% otherwise use linewidth (to make sure the graphics do not exceed the margin)
\makeatletter
\def\maxwidth{ %
\ifdim\Gin@nat@width>\linewidth
\linewidth
\else
\Gin@nat@width
\fi
}
\makeatother

\definecolor{fgcolor}{rgb}{0.345, 0.345, 0.345}
\newcommand{\hlnum}[1]{\textcolor[rgb]{0.686,0.059,0.569}{#1}}%
\newcommand{\hlstr}[1]{\textcolor[rgb]{0.192,0.494,0.8}{#1}}%
\newcommand{\hlcom}[1]{\textcolor[rgb]{0.678,0.584,0.686}{\textit{#1}}}%
\newcommand{\hlopt}[1]{\textcolor[rgb]{0,0,0}{#1}}%
\newcommand{\hlstd}[1]{\textcolor[rgb]{0.345,0.345,0.345}{#1}}%
\newcommand{\hlkwa}[1]{\textcolor[rgb]{0.161,0.373,0.58}{\textbf{#1}}}%
\newcommand{\hlkwb}[1]{\textcolor[rgb]{0.69,0.353,0.396}{#1}}%
\newcommand{\hlkwc}[1]{\textcolor[rgb]{0.333,0.667,0.333}{#1}}%
\newcommand{\hlkwd}[1]{\textcolor[rgb]{0.737,0.353,0.396}{\textbf{#1}}}%
\newcommand{\predvar}{Var\left[\hat{f}(x)\right]}
\let\hlipl\hlkwb

\usepackage{framed}
\makeatletter
\newenvironment{kframe}{%
\def\at@end@of@kframe{}%
\ifinner\ifhmode%
\def\at@end@of@kframe{\end{minipage}}%
\begin{minipage}{\columnwidth}%
\fi\fi%
\def\FrameCommand##1{\hskip\@totalleftmargin \hskip-\fboxsep
\colorbox{shadecolor}{##1}\hskip-\fboxsep
% There is no \\@totalrightmargin, so:
\hskip-\linewidth \hskip-\@totalleftmargin \hskip\columnwidth}%
\MakeFramed {\advance\hsize-\width
\@totalleftmargin\z@ \linewidth\hsize
\@setminipage}}%
{\par\unskip\endMakeFramed%
\at@end@of@kframe}
\makeatother

\definecolor{shadecolor}{rgb}{.97, .97, .97}
\definecolor{messagecolor}{rgb}{0, 0, 0}
\definecolor{warningcolor}{rgb}{1, 0, 1}
\definecolor{errorcolor}{rgb}{1, 0, 0}
\newenvironment{knitrout}{}{} % an empty environment to be redefined in TeX

\usepackage{alltt}
\newcommand{\SweaveOpts}[1]{}  % do not interfere with LaTeX
\newcommand{\SweaveInput}[1]{} % because they are not real TeX commands
\newcommand{\Sexpr}[1]{}       % will only be parsed by R

\usepackage[english]{babel}
\usepackage[utf8]{inputenc}

\usepackage{dsfont}
\usepackage{verbatim}
\usepackage{amsmath}
\usepackage{amsfonts}
\usepackage{bm}
\usepackage{csquotes}
\usepackage{multirow}
\usepackage{longtable}
\usepackage{booktabs}
\usepackage{enumerate}
\usepackage[absolute,overlay]{textpos}
\usepackage{psfrag}
\usepackage{algorithm}
\usepackage{algpseudocode}
\usepackage{eqnarray}
\usepackage{arydshln}
\usepackage{tabularx}
\usepackage{placeins}
\usepackage{tikz}
\usepackage{setspace}
\usepackage{colortbl}
\usepackage{mathtools}
\usepackage{wrapfig}
\usepackage{bm}
\usepackage{verbatim}

\usetikzlibrary{shapes,arrows,automata,positioning,calc,chains,trees, shadows}
\tikzset{
%Define standard arrow tip
>=stealth',
%Define style for boxes
punkt/.style={
rectangle,
rounded corners,
draw=black, very thick,
text width=6.5em,
minimum height=2em,
text centered},
% Define arrow style
pil/.style={
->,
thick,
shorten <=2pt,
shorten >=2pt,}
}

\usepackage{subfig}

% Defines macros and environments
\usepackage{bbm}
% basic latex stuff
\newcommand{\pkg}[1]{{\fontseries{b}\selectfont #1}} %fontstyle for R packages
\newcommand{\lz}{\vspace{0.5cm}} %vertical space
\newcommand{\dlz}{\vspace{1cm}} %double vertical space
\newcommand{\oneliner}[1] % Oneliner for important statements
{\begin{block}{}\begin{center}\begin{Large}#1\end{Large}\end{center}\end{block}}


%new environments
\newenvironment{vbframe}  %frame with breaks and verbatim
{
 \begin{frame}[containsverbatim,allowframebreaks]
}
{
\end{frame}
}

\newenvironment{vframe}  %frame with verbatim without breaks (to avoid numbering one slided frames)
{
 \begin{frame}[containsverbatim]
}
{
\end{frame}
}

\newenvironment{blocki}[1]   % itemize block
{
 \begin{block}{#1}\begin{itemize}
}
{
\end{itemize}\end{block}
}

\newenvironment{fragileframe}[2]{  %fragile frame with framebreaks
\begin{frame}[allowframebreaks, fragile, environment = fragileframe]
\frametitle{#1}
#2}
{\end{frame}}


\newcommand{\myframe}[2]{  %short for frame with framebreaks
\begin{frame}[allowframebreaks]
\frametitle{#1}
#2
\end{frame}}

\newcommand{\remark}[1]{
  \textbf{Remark:} #1
}


\newenvironment{deleteframe}
{
\begingroup
\usebackgroundtemplate{\includegraphics[width=\paperwidth,height=\paperheight]{../style/color/red.png}}
 \begin{frame}
}
{
\end{frame}
\endgroup
}
\newenvironment{simplifyframe}
{
\begingroup
\usebackgroundtemplate{\includegraphics[width=\paperwidth,height=\paperheight]{../style/color/yellow.png}}
 \begin{frame}
}
{
\end{frame}
\endgroup
}\newenvironment{draftframe}
{
\begingroup
\usebackgroundtemplate{\includegraphics[width=\paperwidth,height=\paperheight]{../style/color/green.jpg}}
 \begin{frame}
}
{
\end{frame}
\endgroup
}
% https://tex.stackexchange.com/a/261480: textcolor that works in mathmode
\makeatletter
\renewcommand*{\@textcolor}[3]{%
  \protect\leavevmode
  \begingroup
    \color#1{#2}#3%
  \endgroup
}
\makeatother


\providecommand{\tightlist}{%
  \setlength{\itemsep}{0pt}\setlength{\parskip}{0pt}}

%\setbeamerfont{footnote}{size=\tiny}
\usepackage[hang,flushmargin]{footmisc}
\renewcommand*{\footnotelayout}{\tiny}
\renewcommand*{\thefootnote}{} %\fnsymbol{footnote}

% https://tex.stackexchange.com/questions/30720/footnote-without-a-marker
% \makeatletter
% \def\blfootnote{\gdef\@thefnmark{}\@footnotetext}
% \makeatother

% https://tex.stackexchange.com/questions/357717/beamer-allowframebreaks-option-and-vertical-spacing-when-using-lists-itemize
% \setbeamertemplate{frametitle continuation}{%
%     (\insertcontinuationcount)%
%     \ifnum\insertcontinuationcount>1%
%     \vspace*{\topsep}%
%     \else%
%     %
%     \fi%
% }


%\usetheme{lmu-lecture}
\usepackage{../../style/lmu-lecture}

\let\code=\texttt
\let\proglang=\textsf

\setkeys{Gin}{width=0.9\textwidth}

\title{Interpretable Machine Learning}
% \author{Bernd Bischl, Christoph Molnar, Daniel Schalk, Fabian Scheipl}
\institute{\href{https://compstat-lmu.github.io/lecture_iml/}{compstat-lmu.github.io/lecture\_iml}}
\date{}

\setbeamertemplate{frametitle}{\expandafter\uppercase\expandafter\insertframetitle}

\begin{document}

\input{../../latex-math/basic-math}
\input{../../latex-math/basic-ml}
\newcommand{\titlefigure}{figure/open_blackbox}
\newcommand{\learninggoals}{
\item Examples for interpretable models: linear and polynomial regression models, generalized linear models, generalized additive models, model-based boosting, rule-based learning
\item Methodological summary, motivation behind the model, model interpretations}

\lecturechapter{Interpretable Models}
\lecture{Interpretable Machine Learning}

\begin{vbframe}{Linear and Polynomial Regression}

\begin{itemize}
\setlength\itemsep{2em}
\item For linear regression models, we only estimate the model parameters. The model equation is manually specified and known in advance:
\begin{align*}
y &= \beta_0 + \beta_1 x_1 + \dots + \beta_p x_p + \dots + \epsilon \\
Y &= X^T\beta + \mathcal{E}
\end{align*}
\item The model equation is identical across the entire feature space.
\item The predictive power of LMs is determined by specifying the correct model structure. A polynomial regression model is an extension of the LM that includes higher order terms or interactions. This way we can additively model non-linear data and make use of the entire LM toolbox.
\item
By knowing the model equation, we can exactly determine feature effects (e.g., beta coefficients, effect plots) and importance scores (e.g., p-values, t-statistics). For higher order effects or interactions, beta coefficients cannot be interpreted in isolation, i.e., we need to use marginal effects or effect plots with simultaenous changes in feature values.
\item Note that for inference-based metrics (p-values, t-statistics, confidence intervals) to be valid, the error term needs to be normally distributed with zero mean, i.e., $\epsilon \sim N(0, \sigma^2)$. It follows that $(y \vert x) \sim N(x^T \beta, \sigma^2)$. This restricts the usage of LMs in practice, as the distribution of the error term is a prior assumption about the data.
\end{itemize}
\end{vbframe}

\begin{vbframe}{Linear and Polynomial Regression Models}
\tiny
\verbatiminput{figure/lm_output.txt}
\end{vbframe}

\begin{vbframe}{Linear and Polynomial Regression Models}
\begin{figure}
  \includegraphics[width = 0.6\textwidth]{figure/lm_effect_plot.png}
\end{figure}
\end{vbframe}


\begin{vbframe}{Generalized Linear Regression Models}

\begin{itemize}
\setlength\itemsep{2em}
\item Generalized linear models (GLMs) are more flexible regarding the target distribution. They keep the linear predictor $X^T\beta$ which now explains the transformed, expected conditional target through a link function $g$: 
\begin{align*}
g\left(\mathbb{E}_Y(Y \vert X)\right) &= X^T\beta \\
\mathbb{E}_Y \left(Y \vert X\right) &= g^{-1}(X^T\beta) 
\end{align*}
\item GLMs are a framework for target distributions of the exponential family, e.g., Gaussian, Binomial, Poisson, Exponential, Gamma. A Gaussian target distribution with identity link corresponds to a linear / polynomial regression model.
\item The link function describes how the linear predictor $X^T\beta$ relates to the expected, conditional target $\mathbb{E}_Y(Y \vert X)$, e.g., if the target is distributed binomially, a natural link function is the logit link $log\left(\frac{\mathbb{E}_Y(Y \vert X)}{1 - \mathbb{E}_Y(Y \vert X)}\right)$.
\item We need to specify the correct model equation, target distribution, and link function in order to receive a good model fit.
\item As the model equation is still known, interpretations are possible the same way as for polynomial regression models. However, even linear terms become non-linear through the link function (if the identity link is not used)!
\end{itemize}
\end{vbframe}


\begin{vbframe}{Generalized Linear Regression Models}
\tiny
\verbatiminput{figure/glm_output.txt}
\end{vbframe}

\begin{vbframe}{Generalized Linear Regression Models}
\begin{figure}
  \includegraphics[width = 0.6\textwidth]{figure/glm_effect_plot.png}
\end{figure}
\end{vbframe}

\begin{vbframe}{Generalized Additive Models}
\begin{itemize}
\setlength\itemsep{2em}
\item A generalized additive model (GAM) adds flexibility and predictive power to the GLM framework by replacing pre-specified terms with smoothing functions:
$$
g\left(\mathbb{E}_Y(y \vert x)\right) = \beta_0 + \beta_1 h_1(x_1) + \dots + \beta_p h_p(x_p) + \dots
$$
\item For the component functions, we may either specify a parametric form (e.g., a regression splines), or a non-parametric one, e.g., locally estimated scatterplot smoothing (LOESS).
\item This makes GAMs much more adaptive. The estimated model is largely determined by the structure of the data instead of premade assumptions as in LMs and GLMs. However, we still need to specify the order and type of the component functions to be estimated, e.g., which interactions to include.
\item Furthermore, the smoothing degree of each component function can be tuned to avoid overfitting. Regularization increases interpretability, as the interpretations drawn from the model can be transferred more easily to new data.
\item A GAM retains interpretability by keeping the additive model equation (as long as the component functions are interpretable). As the model equation is known, we can use similar interpretation methods as for LMs and GLMs, e.g., evaluating the estimated components that depend on the features of interest.
\end{itemize}
\end{vbframe}


\begin{vbframe}{Generalized Additive Models}
\tiny
\verbatiminput{figure/gam_output.txt}
\end{vbframe}


\begin{vbframe}{Generalized Additive Models}
\begin{figure}
  \includegraphics[width = 0.6\textwidth]{figure/gam_effect_plot.png}
\end{figure}
\end{vbframe}

\begin{vbframe}{Model-Based Boosting}

\begin{itemize}
\setlength\itemsep{2em}
\item 
Boosting iteratively joins weak base learners to create a powerful ensemble model.
\item 
Idea: Combine boosting with interpretable base learners (e.g., single feature LM). The resulting ensemble is also interpretable.
\item
Consider two linear base learners $b_j(x, \Theta)$ and $b_j(x, \Theta^{\star})$ with the same type, but distinct parameter vectors $\Theta$ and $\Theta^{\star}$. They can be combined in a base learner of the same type:
$$
b_j(x, \Theta) + b_j(x, \Theta^{\star}) = b_j(x, \Theta + \Theta^{\star})
$$
\item We create a selection of interpretable base learners. In each iteration, all base learners are trained on the so-called pseudo residuals, and the one with the best fit is added to the previously computed model:
\begin{align*}
\widehat{f}^{[1]}(x) &= f_0 + \beta b_3(x_3, \theta^{[1]}) \\
\widehat{f}^{[2]}(x) &= f_0 + \beta b_3(x_3, \theta^{[1]}) + \beta b_3(x_3, \theta^{[2]})\\
\widehat{f}^{[3]}(x) &= f_0 + \beta b_3(x_3, \theta^{[1]}) + \beta b_3(x_3, \theta^{[2]}) + \beta b_1(x_1, \theta^{[3]}) \\
\\
\widehat{f}^{[3]}(x) &= f_0 + \beta \left(b_3(x_3, \theta^{[1]} + \theta^{[2]}) + b_1(x_1, \theta^{[3]})\right)
\end{align*}

\item The final model has an additive structure (equivalent to a GAM), where each component function is interpretable itself.

\end{itemize}
\end{vbframe}

\begin{vbframe}{Model-Based Boosting}
\tiny
\verbatiminput{figure/mboost_output.txt}
\end{vbframe}


\begin{vbframe}{Model-Based Boosting}
\begin{figure}
  \includegraphics[width = 0.6\textwidth]{figure/mboost_effect_plot.png}
\end{figure}
\end{vbframe}


\begin{vbframe}{Rule-Based Learning}

\begin{itemize}
\setlength\itemsep{2em}
\item 
Decision rules follow a general structure: IF the conditions are met, THEN make a certain prediction. Rule-based learning aims to capture a set of decision rules that accurately describe the data. 
\item Although a single decision rule is very intelligibile, a large collection of decision rules might not be. An interpretable rule-based model should therefore be based on as few rules as possible.
\item Descripte Rule Discovery (DRD) is based on mining databases for frequent item sets, and converting these into a set of decision rules, e.g., a rule \{mobile phone\} $\Rightarrow$ \{mobile phone case\} might stem from mobile phones and phone cases being bought together frequently. DRD is not suited to create a model that makes predictions on unseen data.
\item Instead, we are interested in Predictive Rule Learning (PRL), which aims to discover a set of rules that cover the entire feature space, thereby being able to predict for any data instance\footnote[frame]{Fürnkranz J., Kliegr T. (2015) A Brief Overview of Rule Learning. In: Bassiliades N., Gottlob G., Sadri F., Paschke A., Roman D. (eds) Rule Technologies: Foundations, Tools, and Applications. RuleML 2015. Lecture Notes in Computer Science, vol 9202. Springer, Cham.}. There are innumerable rule-based learning algorithms, e.g., decision tree learning, OneR, Sequential Covering, or Bayesian Rule Lists.
\item Decision tree learning, e.g, classification and regression trees (CART), is the most popular group of rule-based models, as every tree can be represented by a collection of decision rules. A decision tree recursively partitions the data by optimizing an objective function, e.g., minimizing the data impurity. A succession of tree splits corresponds to a decision rule with the succeeding split conditions as a conjunction.
\item The RuleFit algorithm combines polynomial regression with rule ensembles for automatic interaction detection. In the first stage, a tree ensemble is trained to produce decision rules (where the predictions are discarded and only the splits are utilized), e.g., $x_1 < 0 \land x_2 > 1$. In the second stage, a linear model with main effects is augmented with each decision rule being represented by a single dummy feature.
\end{itemize}
%\item OneR learns rules from a single feature. OneR is characterized by its simplicity, interpretability and its use as a benchmark.
%\item Sequential covering is a general procedure that iteratively learns rules and removes the data points that are covered by the new rule. This procedure is used by many rule learning algorithms.
%\item Bayesian Rule Lists combine pre-mined frequent patterns into a decision list using Bayesian statistics. Using pre-mined patterns is a common approach used by many rule learning algorithms.
%\end{itemize}

\end{vbframe}

\begin{vbframe}{Rule-Based Learning}

\begin{columns}
  \begin{column}{0.5\textwidth}
      \begin{center}
        \includegraphics[width = \textwidth]{figure/rpart_plot.png}     
      \end{center}
  \end{column}
  \begin{column}{0.5\textwidth}
     \tiny
     \verbatiminput{figure/rpart_output.txt}
  \end{column}
\end{columns}

\end{vbframe}

\endlecture
\end{document}
