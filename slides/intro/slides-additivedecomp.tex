\documentclass[11pt,compress,t,notes=noshow, xcolor=table]{beamer}
\usepackage[]{graphicx}\usepackage[]{color}
% maxwidth is the original width if it is less than linewidth
% otherwise use linewidth (to make sure the graphics do not exceed the margin)
\makeatletter
\def\maxwidth{ %
\ifdim\Gin@nat@width>\linewidth
\linewidth
\else
\Gin@nat@width
\fi
}
\makeatother

\definecolor{fgcolor}{rgb}{0.345, 0.345, 0.345}
\newcommand{\hlnum}[1]{\textcolor[rgb]{0.686,0.059,0.569}{#1}}%
\newcommand{\hlstr}[1]{\textcolor[rgb]{0.192,0.494,0.8}{#1}}%
\newcommand{\hlcom}[1]{\textcolor[rgb]{0.678,0.584,0.686}{\textit{#1}}}%
\newcommand{\hlopt}[1]{\textcolor[rgb]{0,0,0}{#1}}%
\newcommand{\hlstd}[1]{\textcolor[rgb]{0.345,0.345,0.345}{#1}}%
\newcommand{\hlkwa}[1]{\textcolor[rgb]{0.161,0.373,0.58}{\textbf{#1}}}%
\newcommand{\hlkwb}[1]{\textcolor[rgb]{0.69,0.353,0.396}{#1}}%
\newcommand{\hlkwc}[1]{\textcolor[rgb]{0.333,0.667,0.333}{#1}}%
\newcommand{\hlkwd}[1]{\textcolor[rgb]{0.737,0.353,0.396}{\textbf{#1}}}%
\newcommand{\predvar}{Var\left[\hat{f}(x)\right]}
\let\hlipl\hlkwb

\usepackage{framed}
\makeatletter
\newenvironment{kframe}{%
\def\at@end@of@kframe{}%
\ifinner\ifhmode%
\def\at@end@of@kframe{\end{minipage}}%
\begin{minipage}{\columnwidth}%
\fi\fi%
\def\FrameCommand##1{\hskip\@totalleftmargin \hskip-\fboxsep
\colorbox{shadecolor}{##1}\hskip-\fboxsep
% There is no \\@totalrightmargin, so:
\hskip-\linewidth \hskip-\@totalleftmargin \hskip\columnwidth}%
\MakeFramed {\advance\hsize-\width
\@totalleftmargin\z@ \linewidth\hsize
\@setminipage}}%
{\par\unskip\endMakeFramed%
\at@end@of@kframe}
\makeatother

\definecolor{shadecolor}{rgb}{.97, .97, .97}
\definecolor{messagecolor}{rgb}{0, 0, 0}
\definecolor{warningcolor}{rgb}{1, 0, 1}
\definecolor{errorcolor}{rgb}{1, 0, 0}
\newenvironment{knitrout}{}{} % an empty environment to be redefined in TeX

\usepackage{alltt}
\newcommand{\SweaveOpts}[1]{}  % do not interfere with LaTeX
\newcommand{\SweaveInput}[1]{} % because they are not real TeX commands
\newcommand{\Sexpr}[1]{}       % will only be parsed by R

\usepackage[english]{babel}
\usepackage[utf8]{inputenc}

\usepackage{dsfont}
\usepackage{verbatim}
\usepackage{amsmath}
\usepackage{amsfonts}
\usepackage{bm}
\usepackage{csquotes}
\usepackage{multirow}
\usepackage{longtable}
\usepackage{booktabs}
\usepackage{enumerate}
\usepackage[absolute,overlay]{textpos}
\usepackage{psfrag}
\usepackage{algorithm}
\usepackage{algpseudocode}
\usepackage{eqnarray}
\usepackage{arydshln}
\usepackage{tabularx}
\usepackage{placeins}
\usepackage{tikz}
\usepackage{setspace}
\usepackage{colortbl}
\usepackage{mathtools}
\usepackage{wrapfig}
\usepackage{bm}

\usetikzlibrary{shapes,arrows,automata,positioning,calc,chains,trees, shadows}
\tikzset{
%Define standard arrow tip
>=stealth',
%Define style for boxes
punkt/.style={
rectangle,
rounded corners,
draw=black, very thick,
text width=6.5em,
minimum height=2em,
text centered},
% Define arrow style
pil/.style={
->,
thick,
shorten <=2pt,
shorten >=2pt,}
}

\usepackage{subfig}

% Defines macros and environments
\usepackage{bbm}
% basic latex stuff
\newcommand{\pkg}[1]{{\fontseries{b}\selectfont #1}} %fontstyle for R packages
\newcommand{\lz}{\vspace{0.5cm}} %vertical space
\newcommand{\dlz}{\vspace{1cm}} %double vertical space
\newcommand{\oneliner}[1] % Oneliner for important statements
{\begin{block}{}\begin{center}\begin{Large}#1\end{Large}\end{center}\end{block}}


%new environments
\newenvironment{vbframe}  %frame with breaks and verbatim
{
 \begin{frame}[containsverbatim,allowframebreaks]
}
{
\end{frame}
}

\newenvironment{vframe}  %frame with verbatim without breaks (to avoid numbering one slided frames)
{
 \begin{frame}[containsverbatim]
}
{
\end{frame}
}

\newenvironment{blocki}[1]   % itemize block
{
 \begin{block}{#1}\begin{itemize}
}
{
\end{itemize}\end{block}
}

\newenvironment{fragileframe}[2]{  %fragile frame with framebreaks
\begin{frame}[allowframebreaks, fragile, environment = fragileframe]
\frametitle{#1}
#2}
{\end{frame}}


\newcommand{\myframe}[2]{  %short for frame with framebreaks
\begin{frame}[allowframebreaks]
\frametitle{#1}
#2
\end{frame}}

\newcommand{\remark}[1]{
  \textbf{Remark:} #1
}


\newenvironment{deleteframe}
{
\begingroup
\usebackgroundtemplate{\includegraphics[width=\paperwidth,height=\paperheight]{../style/color/red.png}}
 \begin{frame}
}
{
\end{frame}
\endgroup
}
\newenvironment{simplifyframe}
{
\begingroup
\usebackgroundtemplate{\includegraphics[width=\paperwidth,height=\paperheight]{../style/color/yellow.png}}
 \begin{frame}
}
{
\end{frame}
\endgroup
}\newenvironment{draftframe}
{
\begingroup
\usebackgroundtemplate{\includegraphics[width=\paperwidth,height=\paperheight]{../style/color/green.jpg}}
 \begin{frame}
}
{
\end{frame}
\endgroup
}
% https://tex.stackexchange.com/a/261480: textcolor that works in mathmode
\makeatletter
\renewcommand*{\@textcolor}[3]{%
  \protect\leavevmode
  \begingroup
    \color#1{#2}#3%
  \endgroup
}
\makeatother


\providecommand{\tightlist}{%
  \setlength{\itemsep}{0pt}\setlength{\parskip}{0pt}}

%\setbeamerfont{footnote}{size=\tiny}
\usepackage[hang,flushmargin]{footmisc}
\renewcommand*{\footnotelayout}{\tiny}
\renewcommand*{\thefootnote}{} %\fnsymbol{footnote}

% https://tex.stackexchange.com/questions/30720/footnote-without-a-marker
% \makeatletter
% \def\blfootnote{\gdef\@thefnmark{}\@footnotetext}
% \makeatother

% https://tex.stackexchange.com/questions/357717/beamer-allowframebreaks-option-and-vertical-spacing-when-using-lists-itemize
% \setbeamertemplate{frametitle continuation}{%
%     (\insertcontinuationcount)%
%     \ifnum\insertcontinuationcount>1%
%     \vspace*{\topsep}%
%     \else%
%     %
%     \fi%
% }


%\usetheme{lmu-lecture}
\usepackage{../../style/lmu-lecture}

\let\code=\texttt
\let\proglang=\textsf

\setkeys{Gin}{width=0.9\textwidth}

\title{Interpretable Machine Learning}
% \author{Bernd Bischl, Christoph Molnar, Daniel Schalk, Fabian Scheipl}
\institute{\href{https://compstat-lmu.github.io/lecture_iml/}{compstat-lmu.github.io/lecture\_iml}}
\date{}

\setbeamertemplate{frametitle}{\expandafter\uppercase\expandafter\insertframetitle}

\begin{document}

\input{../../latex-math/basic-math}
\input{../../latex-math/basic-ml}
\newcommand{\titlefigure}{figure/open_blackbox}
\newcommand{\learninggoals}{
\item Interpretable Models}

\lecturechapter{Introduction and Background}
\lecture{Interpretable Machine Learning}
 
\begin{vbframe}{High-Dimensional Model Representation}

\begin{itemize}
\item
A high-dimensional model representation (HDMR) decomposes the model into a sum of effect terms of increasing order:
\begin{align*}
\hat{f}(x) &= g_{\{0\}} + g_{\{1\}}(x_1) + g_{\{2\}}(x_2) + \;\dots\; + g_{\{1, 2\}}(x_1, x_2) \\
&\phantom{{}={}} + \;\dots\; + g_{\{1,\ldots,p\}}(x_1, \ldots,x_p)
\end{align*}
\item The features need to be independent to make the HDMR unique.
\item Different techniques to estimate an additive decomposition exist, e.g., repeated expectations (partial dependence / PD) or accumulated local effects (ALE).

\end{itemize}
\end{vbframe}

\begin{vbframe}{Additive Decomposition of a Prediction Function}

Consider the estimation via iterative expectations:
\begin{align*}
 g_{\{0\}} &= \mathbb{E}_X\left[\widehat{f}(x)\right] \\
 g_{\{1\}}(x_1) &= \mathbb{E}_{X_{-1}}\left[\widehat{f}(x) \; \vert  \; X_1 \right] - g_{\{0\}} \\
 g_{\{2\}}(x_2) &= \mathbb{E}_{X_{-2}}\left[\widehat{f}(x) \; \vert  \; X_2 \right] - g_{\{0\}} \\
 g_{\{1, 2\}}(x_1, x_2) &= \mathbb{E}_{X_{-\{1,2\}}}\left[\widehat{f}(x) \; \vert \; X_1, X_2 \right] - g_{\{2\}}(x_2) - g_{\{1\}}(x_1) - g_{\{0\}}\\
 &\vdots \\
 g_{\{1, \dots, p\}}(x) &= \widehat{f}(x) - \dots - g_{\{1, 2\}}(x_1, x_2) \\
 &\phantom{{}={}} - g_{\{2\}}(x_2) - g_{\{1\}}(x_1) - g_{\{0\}}\\
\end{align*}

\end{vbframe}

\begin{vbframe}{Functional ANOVA}

After $\hat{f}$ has been decomposed, we can conduct a functional analysis of variance (functional ANOVA / FANOVA):
\begin{itemize}
\item
\begin{align*}
Var\left[\hat{f}(x)\right] &= Var\left[g_{\{0\}} + g_{\{1\}}(x_1) + g_{\{2\}}(x_2) + \;\dots\; + g_{\{1, 2\}}(x_1, x_2) \right. \\
&\phantom{{}={}} \left. + \;\dots\; + g_{\{1,\ldots,p\}}(x) \right]
\end{align*}
\item If the features are independent, the variance can be additively decomposed without covariances:
\begin{align*}
Var\left[\hat{f}(x)\right] &= Var\left[g_{\{0\}}\right] + Var\left[g_{\{1\}}(x_1)\right] + Var\left[g_{\{2\}}(x_2)\right] \\
&\phantom{{}={}} + Var\left[g_{\{1, 2\}}(x_1, x_2)\right] + \;\dots\; + Var\left[g_{\{1,\ldots,p\}}(x)\right]
\end{align*}
\item Dividing by the prediction variance results in the fraction of variance explained by each term:
\begin{align*}
1 &= \frac{Var\left[g_{\{0\}}\right]}{\predvar} + \frac{Var\left[g_{\{1\}}(x_1)\right]}{\predvar} + \frac{Var\left[g_{\{2\}}(x_2)\right]}{\predvar} \\
&\phantom{{}={}} + \frac{Var\left[g_{\{1, 2\}}(x_1, x_2)\right]}{\predvar} + \;\dots\; + \frac{Var\left[g_{\{1,\ldots,p\}}(x)\right]}{\predvar}
\end{align*}

\item The fraction of variance explained by a term is referred to as the Sobol index:
$$
S_j = \frac{Var\left[g_{\{j\}}(x_j)\right]}{Var\left[\hat{f}(x)\right]}
$$
\end{itemize}

\end{vbframe}


\endlecture
\end{document}
