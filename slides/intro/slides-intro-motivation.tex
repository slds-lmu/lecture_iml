\documentclass[11pt,compress,t,notes=noshow, aspectratio=169, xcolor=table]{beamer}

\usepackage{../../style/lmu-lecture}
% Defines macros and environments
\usepackage{bbm}
% basic latex stuff
\newcommand{\pkg}[1]{{\fontseries{b}\selectfont #1}} %fontstyle for R packages
\newcommand{\lz}{\vspace{0.5cm}} %vertical space
\newcommand{\dlz}{\vspace{1cm}} %double vertical space
\newcommand{\oneliner}[1] % Oneliner for important statements
{\begin{block}{}\begin{center}\begin{Large}#1\end{Large}\end{center}\end{block}}


%new environments
\newenvironment{vbframe}  %frame with breaks and verbatim
{
 \begin{frame}[containsverbatim,allowframebreaks]
}
{
\end{frame}
}

\newenvironment{vframe}  %frame with verbatim without breaks (to avoid numbering one slided frames)
{
 \begin{frame}[containsverbatim]
}
{
\end{frame}
}

\newenvironment{blocki}[1]   % itemize block
{
 \begin{block}{#1}\begin{itemize}
}
{
\end{itemize}\end{block}
}

\newenvironment{fragileframe}[2]{  %fragile frame with framebreaks
\begin{frame}[allowframebreaks, fragile, environment = fragileframe]
\frametitle{#1}
#2}
{\end{frame}}


\newcommand{\myframe}[2]{  %short for frame with framebreaks
\begin{frame}[allowframebreaks]
\frametitle{#1}
#2
\end{frame}}

\newcommand{\remark}[1]{
  \textbf{Remark:} #1
}


\newenvironment{deleteframe}
{
\begingroup
\usebackgroundtemplate{\includegraphics[width=\paperwidth,height=\paperheight]{../style/color/red.png}}
 \begin{frame}
}
{
\end{frame}
\endgroup
}
\newenvironment{simplifyframe}
{
\begingroup
\usebackgroundtemplate{\includegraphics[width=\paperwidth,height=\paperheight]{../style/color/yellow.png}}
 \begin{frame}
}
{
\end{frame}
\endgroup
}\newenvironment{draftframe}
{
\begingroup
\usebackgroundtemplate{\includegraphics[width=\paperwidth,height=\paperheight]{../style/color/green.jpg}}
 \begin{frame}
}
{
\end{frame}
\endgroup
}
% https://tex.stackexchange.com/a/261480: textcolor that works in mathmode
\makeatletter
\renewcommand*{\@textcolor}[3]{%
  \protect\leavevmode
  \begingroup
    \color#1{#2}#3%
  \endgroup
}
\makeatother


\providecommand{\tightlist}{%
  \setlength{\itemsep}{0pt}\setlength{\parskip}{0pt}}

%\setbeamerfont{footnote}{size=\tiny}
\usepackage[hang,flushmargin]{footmisc}
\renewcommand*{\footnotelayout}{\tiny}
\renewcommand*{\thefootnote}{} %\fnsymbol{footnote}

% https://tex.stackexchange.com/questions/30720/footnote-without-a-marker
% \makeatletter
% \def\blfootnote{\gdef\@thefnmark{}\@footnotetext}
% \makeatother

% https://tex.stackexchange.com/questions/357717/beamer-allowframebreaks-option-and-vertical-spacing-when-using-lists-itemize
% \setbeamertemplate{frametitle continuation}{%
%     (\insertcontinuationcount)%
%     \ifnum\insertcontinuationcount>1%
%     \vspace*{\topsep}%
%     \else%
%     %
%     \fi%
% }


\title{Interpretable Machine Learning}
% \author{LMU}
%\institute{\href{https://compstat-lmu.github.io/lecture_iml/}{compstat-lmu.github.io/lecture\_iml}}
\date{}

\begin{document}

\newcommand{\titlefigure}{figure/open_blackbox}
\newcommand{\learninggoals}{
\item Why interpretability?
\item Developments until now?
\item Use cases for interpretability}

\lecturechapter{Introduction, Motivation, and History}
\lecture{Interpretable Machine Learning}

\begin{frame}{Why Interpretability?}

% 		\begin{itemize}
% 			\item Machine learning (ML) has a huge potential to aid the decision-making process in various  applications.
% 			\pause
% 			\smallskip
% 			\item ML models often are intransparent black boxes, e.g., XGBoost, RBF SVM, or NNs.
% 			\begin{itemize}
% 				\item[$\rightarrow$] too complex to be understood by humans.
% 			\end{itemize}
% 			\smallskip
% 			\item A lack in explanations diminishes trust in the model and creates barriers for adoption, especially in areas with critical decision-making consequences, e.g., medicine.
% 			\smallskip
% 			\item Such disciplines often rely on traditional models,\\ e.g., linear models, with less predictive performance.
% 			\smallskip
% 			\item Interpretable machine learning (IML) aims to bridge the gap between interpretability and predictive performance.
% 		\end{itemize}

		\begin{itemize}
			\item ML: huge potential to aid decision-making process due to its predictive performance
			\pause
			%\smallskip
			\item ML models are often black boxes, e.g., XGBoost, RBF SVM or DNNs
			\begin{itemize}
				\item[$\leadsto$] too complex to be understood by humans
			\end{itemize}
			\pause
			%\smallskip
			\item Lack of explanation
			\begin{enumerate}
				\item hurts trust
				\item creates barriers
			\end{enumerate}
			\pause
			%\smallskip
		    \item[$\leadsto$] Harder to adapt for critical areas with decisions affecting human life,\\
		    e.g., medicine or credit scoring
			\pause
			%\smallskip
			\item[$\leadsto$] Many disciplines with required trust rely on traditional models,\\ e.g., linear models, with less predictive performance
		\end{itemize}
	\end{frame}
	
	
	\begin{frame}{Need for Interpretability in High-stakes Decisions}

Examples of critical areas where decisions based on ML models can affect human life 
    \begin{itemize}
        %\item Crime predictions (e.g., COMPAS)
        
        \item Credit scoring and insurance applications
        \citebutton{Society of Actuaries}{https://www.soa.org/resources/research-reports/2021/interpretable-machine-learning}
        \begin{itemize}
            \item Providing reasons for not granting a loan
            \item Fraud detection in insurance claims
        \end{itemize}
\pause
        \item Medical applications %\citebutton{Society of Actuaries}{https://www.medrxiv.org/content/10.1101/2020.05.18.20105841v1}
        \begin{itemize}
            \item Identification of diseases
            \item Chance of recovering
            \item Recommendations of treatments
        \end{itemize}
\pause
        %\item Rating job applications

        \item \ldots
    \end{itemize}
    
    \pause
    The need for interpretability is also becoming increasingly important from a legal perspective
    
    \begin{itemize}
    \item General Data Protection Regulation (GDPR) requires for some applications that models have to be explainable \citebutton{Goodman \& Flaxman (2017)}{https://doi.org/10.1609/aimag.v38i3.2741}\\
    $\leadsto$ \textit{EU Regulations on Algorithmic Decision-Making and a ``Right to Explanation''} 
    
    \item \textit{Ethics guidelines for trustworthy AI}
    \citebutton{European Commission (2019)}{https://doi.org/10.2759/346720}

    \end{itemize}
\end{frame}

	%-----------------------------------------------------------------------------------------------------------------------------

	\begin{frame}{Brief History of Interpretability}
		\begin{itemize}
			\item 18th and 19th century: linear regression models (Gauss, Legendre, Quetelet)
			\medskip
			\item 1940s: emergence of sensitivity analysis (SA)
			\medskip
			\item Middle of 20th century: Rule-based ML, incl. decision rules and decision trees
			\medskip
			\item 2001: built-in feature importance measure of random forests
			\medskip
			\item >2010: Explainable AI (XAI) for deep learning
			\medskip
			\item >2015: IML as an independent field of research
		\end{itemize}
	\end{frame}

	%-----------------------------------------------------------------------------------------------------------------------------
\begin{frame}{When do we need interpretability?}
\begin{columns}[T]
\begin{column}{0.6\textwidth}
%  \begin{itemize}
%   \item Debugging machine learning models
%   \item Increasing trust in models
%   \item Analyzing newly developed systems with unknown consequences
%   \item Decisions about humans
%   \item Models using proxies instead of causal inputs, e.g. predicting flu outbreaks from google searches.
%   \item When loss function does not cover constraints like fairness (e.g. credit score) or need for insights (e.g. science).
% \end{itemize}
\begin{itemize}
  \item To \textbf{Discover}: Gain insights about data, distribution and model
  \pause 
  \item To \textbf{Debug, audit and improve}: Insights help to identify flaws (in data or model), which can be corrected (debug and audit)\\
  $\leadsto$ Global perspective
  \pause
  \item To \textbf{explain individual decisions}: Explaining individual decisions can prevent unwanted actions based on the model\\
  $\leadsto$ Local perspective
  \pause 
  \item To \textbf{Justify}: Investigate if and why biased, unexpected or discriminatory predictions were made, or improve/reject the model\\
  $\leadsto$ Fairness perspective

\end{itemize}
\end{column}
\begin{column}{0.4\textwidth}  %%<--- here
 %\vspace{0.5cm}
%  \begin{center}
%  \begin{figure}
  \includegraphics[width=0.9\textwidth]{figure/explain-to}
   \centering \citebutton{Adadi and Berrada 2018}{https://doi.org/10.1109/ACCESS.2018.2870052}
%  \end{figure}
%  \end{center}
\end{column}
\end{columns}
     %\lz
    %\footnote[frame]{Doshi-Velez, F., \& Kim, B. (2017). Towards A Rigorous Science of Interpretable Machine Learning. arXiv: 1702.08608.}
    %\footnote[frame]{A. Adadi and M. Berrada, "Peeking Inside the Black-Box: A Survey on Explainable Artificial Intelligence (XAI)," in IEEE Access, vol. 6, pp. 52138-52160, 2018.}
\end{frame}

\begin{frame}{Alternative: When do we need interpretability}
\begin{columns}[T]
\begin{column}{0.55\textwidth}
\begin{itemize}
  \item To \textbf{Discover}: Gain insights about data, distribution and model
  \pause 
  \item To \textbf{Improve}: Insights help to identify flaws (in data or model), which can be corrected (debug and audit)\\
  $\leadsto$ Global perspective
  \pause
  \item To \textbf{Control}: Explaining individual decisions can prevent unwanted actions based on the model\\
  $\leadsto$ Local perspective
  \pause 
  \item To \textbf{Justify}: Investigate if and why biased, unexpected or discriminatory predictions were made, or improve/reject the model\\
  $\leadsto$ Fairness perspective

\end{itemize}
\end{column}
\begin{column}{0.4\textwidth}
    \vspace{0.5cm}
    \def\firstcircle{(90:1cm) circle (2.5cm)}
    \def\secondcircle{(210:2.5cm) circle (2.5cm)}
    \def\thirdcircle{(330:2.5cm) circle (2.5cm)}
    \def\fourthcircle{(90:-3.5cm) circle (2.5cm)}
    \resizebox{6cm}{5.5cm}{
    \begin{tikzpicture}
        \begin{scope}[ fill opacity=0.5]
            \fill[red] \firstcircle;
            \fill[green] \secondcircle;
            \fill[blue] \thirdcircle;
            \fill[cyan] \fourthcircle;
        \end{scope}
        \draw \firstcircle node[text=black,above, text width=2cm, align=center] {Explain to discover};
        \draw \secondcircle node [text=black, left, text width=2cm, align=center] {Explain to improve};
        \draw \thirdcircle node [text=black,right, text width=2cm, align=center] {Explain to control};
        \draw \fourthcircle node [text=black,below, text width=2cm, align=center]{Explain to justify};
    \end{tikzpicture}}
\end{column}
\end{columns}
\end{frame}



% \begin{frame}{Why is Interpretability Important?}

% 	\begin{itemize}
% 	    \item Machine learning is (mostly) about discovering patterns in data.
% 	    \medskip
% 	    \item Unfortunately, it is not guaranteed that ML will identify the correct patterns.

% 	    \medskip
% 	    \item We humans might not be able to discover patterns ML models discovered.
% 	    \begin{itemize}
% 	        \item Good for science or to get new insights.
% 	        \item Bad in applications where unexpected behavior is not desired.
% 	    \end{itemize}
% 	    \medskip

% 	    \item \alert{How can you check whether the model is correct in its inference?}
% 	\end{itemize}

% \end{frame}

\begin{frame}{Example (Explain to discover): Bike Sharing Dataset }

Bike Sharing Dataset: Prediction of the number of bike rentals per day
\begin{center}
\includegraphics[width=0.6\textwidth]{figure/bike-sharing02.png}
\end{center}

\textbf{Interpretation:}

\begin{itemize}
 \item Year (\code{yr}) and Temperature (temp) are most important features
 \item $5 \%$ and $95 \%$ quantile of repetitions due multiple permutations are shown as error bars
\end{itemize}

\end{frame}

\begin{frame}{Example (Explain to improve): Clever Hans \citebutton{Lapuschkin et al. 2019}{https://www.nature.com/articles/s41467-019-08987-4} }

	\centering
	\begin{columns}[T]
	\begin{column}{0.6\textwidth}
	\includegraphics<1>[width=\textwidth]{figure/horse_without_label.PNG}
	\includegraphics<2>[width=\textwidth]{figure/horse_with_label.PNG}
	\includegraphics<3>[width=\textwidth]{figure/horse_map_with_label.PNG}
	\includegraphics<4>[width=\textwidth]{figure/horse_map_without_label.PNG}
	\end{column}
	\begin{column}{0.4\textwidth}

	\begin{itemize}
	    \item Horse with claimed math skills
	    \item Answered questions correctly by hoof tapping or head shaking
	    \item Correct answers were traced to involuntary cues from human's body language $\Rightarrow$ no math skills %asking person
	    %(e.g., tense attitude)
	    \item<2-> Image classification: \\
	    source tag present \\
	    \onslide<3->{$\Rightarrow$ classified as horse}
	    \item<4-> no source tag \\ $\Rightarrow$ not classified as horse
	\end{itemize}

	\end{column}
	\end{columns}
\end{frame}

\begin{frame}{Example (Explain to improve): Neural Net Tank \citebutton{gwern.net}{https://www.gwern.net/Tanks} }

	\centering
	\begin{columns}[T]
	\begin{column}{0.6\textwidth}
	\includegraphics[width=\textwidth]{figure/tank.jpg}
	\end{column}
	\begin{column}{0.4\textwidth}

	\begin{itemize}
	    \item Researchers created a neural network to detect camouflaged tanks
        \item Model showed good predictive performance in training data set
        \item Application outside this data set proved to be a failure
        \item Reasons vary depending on the source, in general the NN based its decision on irrelevant points in the image
	\end{itemize}

	\end{column}
	\end{columns}
\end{frame}


% \begin{frame}{Clever Hans \citebutton{Lapuschkin et al. 2019}{https://www.nature.com/articles/s41467-019-08987-4}}

% 	\centering
% 	\includegraphics[width=0.6\textwidth]{figure/boats_maps.PNG}

% \end{frame}

\begin{frame}{Example (Explain to control): tbd }

\end{frame}

%%%

\begin{frame}{Example (Explain to justify): COMPAS}

    \begin{itemize}
        \item Correctional Offender Management Profiling for Alternative Sanctions (COMPAS)
        \item Commercial algorithm used by judges to assess defendant’s likelihood of re-offending
        \pause
        \item Predict recidivism risk
        \begin{itemize}
            \item i.e., criminal re-offense after previous crime, resulting in jail booking
            \item different risk levels: high risk, medium risk or low risk
        \end{itemize}
        \pause
        \item Evaluation of recidivism risk based on a questionnaire the defendant has to answer
    \end{itemize}

\end{frame}

\begin{frame}{Example (Explain to justify): COMPAS~\citebutton{Larson et al. 2016}{https://www.propublica.org/article/how-we-analyzed-the-compas-recidivism-algorithm}}
    Discrete Analysis: 
    \vspace{0.5cm}
    
    \centering
    \includegraphics[width=0.7\textwidth]{figure/compass_black_white.PNG}
    % source https://www.propublica.org/article/how-we-analyzed-the-compas-recidivism-algorithm

    Decile score: 1 (low risk) to 10 (high risk)

	$\leadsto$ Model skewed towards low risk for white defendants
	
	$\leadsto$ Strong indication that the model is discriminating black defendants

\end{frame}

\begin{frame}{Example (Explain to justify): COMPAS~\citebutton{Alvarez-Melis and Jaakkola 2018}{https://arxiv.org/pdf/1806.08049.pdf}}
    Feature effects analysis for two examples, using different methods (SHAP and LIME): 
    \vspace{0.5cm}
    
    \centering
    \includegraphics[width=0.95\textwidth]{figure/COMPAS_shap_lime_example.png}
    
    \vspace{0.5cm}
    $\leadsto$ In both cases the race (african american) has a noticeable positive impact on violent score
\end{frame}





% \begin{frame}{Motivation - Adversarial Examples \citebutton{Goodfellow et al. 2016}{https://arxiv.org/pdf/1412.6572.pdf}}

%     \begin{center}
%     \includegraphics[width=0.7\textwidth]{figure/panda-airplane.pdf}
%     \end{center}
% 	\bigskip

% 	$\rightarrow$ ML Models might not capture human-like understanding
% \end{frame}


% \begin{frame}{Adversarial Noise \citebutton{Goodfellow et al. 2016}{https://arxiv.org/pdf/1412.6572.pdf}}
%     \begin{center}
%     \includegraphics[width=0.65\textwidth]{figure/adv-noise.pdf}
%     % https://arxiv.org/pdf/1412.6572.pdf
% 	\end{center}
% 	\normalsize
% 	%\bigskip
% 	$\rightarrow$\textbf{Adversarial Noise:} Noise not visible to \textbf{humans} but results in incorrect classification results
% \end{frame}

% \begin{frame}[c]{Adversarial Examples~\citebutton{Goodfellow et al. 2016}{https://arxiv.org/pdf/1412.6572.pdf}}
    
%     \centering
%     \includegraphics[width=0.7\textwidth]{figure/adv-noise-2.pdf}
	
% \end{frame}

\endlecture
\end{document}
