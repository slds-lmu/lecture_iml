\documentclass[11pt,compress,t,notes=noshow, aspectratio=169, xcolor=table]{beamer}

\usepackage{../../style/lmu-lecture}
% Defines macros and environments
\usepackage{bbm}
% basic latex stuff
\newcommand{\pkg}[1]{{\fontseries{b}\selectfont #1}} %fontstyle for R packages
\newcommand{\lz}{\vspace{0.5cm}} %vertical space
\newcommand{\dlz}{\vspace{1cm}} %double vertical space
\newcommand{\oneliner}[1] % Oneliner for important statements
{\begin{block}{}\begin{center}\begin{Large}#1\end{Large}\end{center}\end{block}}


%new environments
\newenvironment{vbframe}  %frame with breaks and verbatim
{
 \begin{frame}[containsverbatim,allowframebreaks]
}
{
\end{frame}
}

\newenvironment{vframe}  %frame with verbatim without breaks (to avoid numbering one slided frames)
{
 \begin{frame}[containsverbatim]
}
{
\end{frame}
}

\newenvironment{blocki}[1]   % itemize block
{
 \begin{block}{#1}\begin{itemize}
}
{
\end{itemize}\end{block}
}

\newenvironment{fragileframe}[2]{  %fragile frame with framebreaks
\begin{frame}[allowframebreaks, fragile, environment = fragileframe]
\frametitle{#1}
#2}
{\end{frame}}


\newcommand{\myframe}[2]{  %short for frame with framebreaks
\begin{frame}[allowframebreaks]
\frametitle{#1}
#2
\end{frame}}

\newcommand{\remark}[1]{
  \textbf{Remark:} #1
}


\newenvironment{deleteframe}
{
\begingroup
\usebackgroundtemplate{\includegraphics[width=\paperwidth,height=\paperheight]{../style/color/red.png}}
 \begin{frame}
}
{
\end{frame}
\endgroup
}
\newenvironment{simplifyframe}
{
\begingroup
\usebackgroundtemplate{\includegraphics[width=\paperwidth,height=\paperheight]{../style/color/yellow.png}}
 \begin{frame}
}
{
\end{frame}
\endgroup
}\newenvironment{draftframe}
{
\begingroup
\usebackgroundtemplate{\includegraphics[width=\paperwidth,height=\paperheight]{../style/color/green.jpg}}
 \begin{frame}
}
{
\end{frame}
\endgroup
}
% https://tex.stackexchange.com/a/261480: textcolor that works in mathmode
\makeatletter
\renewcommand*{\@textcolor}[3]{%
  \protect\leavevmode
  \begingroup
    \color#1{#2}#3%
  \endgroup
}
\makeatother


\providecommand{\tightlist}{%
  \setlength{\itemsep}{0pt}\setlength{\parskip}{0pt}}

%\setbeamerfont{footnote}{size=\tiny}
\usepackage[hang,flushmargin]{footmisc}
\renewcommand*{\footnotelayout}{\tiny}
\renewcommand*{\thefootnote}{} %\fnsymbol{footnote}

% https://tex.stackexchange.com/questions/30720/footnote-without-a-marker
% \makeatletter
% \def\blfootnote{\gdef\@thefnmark{}\@footnotetext}
% \makeatother

% https://tex.stackexchange.com/questions/357717/beamer-allowframebreaks-option-and-vertical-spacing-when-using-lists-itemize
% \setbeamertemplate{frametitle continuation}{%
%     (\insertcontinuationcount)%
%     \ifnum\insertcontinuationcount>1%
%     \vspace*{\topsep}%
%     \else%
%     %
%     \fi%
% }

\newcommand{\open}{\{}
\newcommand{\close}{\}}

\title{Interpretable Machine Learning}
% \author{LMU}
%\institute{\href{https://compstat-lmu.github.io/lecture_iml/}{compstat-lmu.github.io/lecture\_iml}}
\date{}

\begin{document}

\newcommand{\titlefigure}{figure/open_blackbox}
\newcommand{\learninggoals}{
\item What are additive decomposition of prediction functions?
\item Why are they useful?
\item How do we obtain them?}

\lecturechapter{Additive Decomposition}
\lecture{Interpretable Machine Learning}
 
\begin{frame}{High-Dimensional Model Representation  \citebutton{Sobol (1993)}{https://doi.org/10.1016/S0951-8320(02)00229-6}
\citebutton{Sobol (2003)}{http://www.andreasaltelli.eu/file/repository/sobol1993.pdf}}
%\citebutton{Li et al. (2001)}{https://doi.org/10.1021/jp010450t}

\begin{itemize}
%\itemsep1em
%\item Prediction function $\fh: \mathbb{R}^p \mapsto \mathbb{R}$ produces output from $p$-dimensional input space
%decomposed into a hierarchical 
\item \textbf{High-Dimensional Model Representation (HDMR):} Expansion of prediction function $\fh: \mathbb{R}^p \mapsto \mathbb{R}$ into independent summands of different dimensions w.r.t. input features:
%\item \textbf{High-Dimensional Model Representation (HDMR)} of prediction function $\fh: \mathbb{R}^p \mapsto \mathbb{R}$ as function expansion of low dimensional independent terms of increasing order:
%\vspace{5pt}
% \centerline{$\fh(\xv) = g_{\open  0  \close} + g_{\open 1 \close}(x_1) + g_{\open 2 \close}(x_2) + \dots + g_{\open 1, 2 \close}(x_1, x_2) + \dots + g_{\open 1,\ldots,p \close}(x_1, \ldots, x_p)$}
%\vspace{5pt}
\begin{equation*}
\begin{split}
\fh(\xv) =  %g_{\open  0  \close} +
\textstyle\sum_{S \subseteq \{1,\ldots,p\}} g_{S}(x_S) = \; & g_{\open 0 \close} + g_{\open 1 \close}(x_1) + g_{\open 2 \close}(x_2) + \dots + g_{\open p \close}(x_p) + \\
& g_{\open 1, 2 \close}(x_1, x_2) + \dots + g_{\open p-1, p \close}(x_{p-1}, x_p) + \dots + \\
& g_{\open 1,\ldots,p \close}(x_1, \ldots, x_p)
\end{split}
\end{equation*}
\vspace{-5pt}\pause
\begin{itemize}
\item $g_{\open  0  \close} \hat = $ Constant mean (intercept) %$\mathbb{E}_X (\fh(x)) $
\item Univariate terms $g_{\open j \close} \hat = $ first-order or main effect of $j$-th feature w.r.t. $\fh(\xv)$
%\item Bivariate terms $g_{\open j, k \close} \hat = $ second-order effect of features $j$ and $k$ w.r.t. $\fh(\xv)$%, etc.
\item General $|S|$-order effects $g_{S}(x_S)$ depend only on features included in $S$ %$x_j$ for all $j \in S$
\end{itemize}
%effect terms of increasing order 
%HDMR decomposes model into sum of effect terms of increasing order:
% \begin{align*}
% \fh(\xv) &= g_{\open  0  \close} + g_{\open 1 \close}(x_1) + g_{\open 2 \close}(x_2) + \;\dots\; + g_{\open 1, 2 \close}(x_1, x_2) \\
% &\phantom{{}={}} + \;\dots\; + g_{\open 1,\ldots,p \close}(x_1, \ldots,x_p)
% \end{align*}
%\item Features need to be independent to make HDMR unique
\pause
\item Different techniques to estimate the above additive decomposition exist, e.g.:
		\begin{itemize}
			\item recursive expectations (via partial dependence) $\leadsto$ functional ANOVA decomposition
			\item accumulated local effect (ALE) decomposition
		\end{itemize}
\end{itemize}
\end{frame}

\begin{frame}{High-Dimensional Model Representation}

Estimation of independent components via recursive expectations (where $-S = \{1, \ldots, p \} \setminus S$):

$$g_{\open S \close}(x_S) = \mathbb{E}_{X_{-S}}\left[\fh(x) \; \vert  \; x_S \right] - \sum_{V \subset S} g_V(x_V)$$

\begin{itemize}
    \item Expectation integrates $\fh(x)$ over all features except $x_S$
    \item Subtract all components $g_V$ with $V \subset S$ to remove all lower-order effects and center the effect
\end{itemize}
\pause
Example:
\begin{align*}
 g_{\open  0  \close} &= \mathbb{E}_X\left[\fh(x)\right] \\
 g_{\open 1 \close}(x_1) &= \mathbb{E}_{X_{-1}}\left[\fh(x) \; \vert  \; x_1 \right] - g_{\open  0  \close} \text{ and } g_{\open 2 \close}(x_2) = \mathbb{E}_{X_{-2}}\left[\fh(x) \; \vert  \; x_2 \right] - g_{\open  0  \close}  \\
 %g_{\open 2 \close}(x_2) &= \mathbb{E}_{X_{-2}}\left[\fh(x) \; \vert  \; x_2 \right] - g_{\open  0  \close} \\
 g_{\open 1, 2 \close}(x_1, x_2) &= \mathbb{E}_{X_{-\open 1,2 \close}}\left[\fh(x) \; \vert \; x_1, x_2 \right] - g_{\open 2 \close}(x_2) - g_{\open 1 \close}(x_1) - g_{\open  0  \close}\\
 &\vdots \\
 g_{\open 1, \dots, p \close}(x) &= \fh(x) - \dots - g_{\open 1, 2 \close}(x_1, x_2) %\\
 %&\phantom{{}={}} 
 - g_{\open 2 \close}(x_2) - g_{\open 1 \close}(x_1) - g_{\open  0  \close}\\
\end{align*}

\end{frame}

\begin{frame}{High-Dimensional Model Representation}
Example:

\begin{itemize}
    \item $\fh(x) = 2 + x_1^2 - x_2^2 + x_1 \cdot x_2$, with $x_1 = 5$ and $x_2 = 10$ $\Rightarrow$ $\fh(x) = -23$
    \item HDMR: $g_{\open 0 \close} = 2$, $g_{\open 1 \close} = -9.67$, $g_{\open 2 \close} = -65.33$, $g_{\open 1,2 \close} = 50$ $\Rightarrow$ $\fh(x) = -23$
\end{itemize} 

\begin{figure}
\includegraphics[width = 0.7 \textwidth]{figure/interaction2}
\end{figure}
\end{frame}

% \begin{frame}{High-Dimensional Model Representation}

% \begin{itemize}
% \itemsep1em
%     \item Even if your goal is not to decompose $\hat{f}(x)$, it is useful to keep in mind that a decomposition exists.
%     \item When interpreting a model, one is often interested in feature effects of various orders, e.g., first-order and second-order effects.
%     \item One may assume that the relevance of effects decreases with increasing order, e.g., that the model can be approximated via first-order and second-order effects.
%     \item Do not confuse the outputs of interpretation methods such as the ICE or PD with the effect terms of the additive decomposition!
%     \\$\rightarrow$ A second-order effect is the sole interaction effect on the target after a constant effect or main effects have been removed.
% \end{itemize}
% \end{frame}


\begin{frame}{Functional ANOVA}

\begin{itemize}
\item After $\hat{f}(x)$ has been decomposed, we can conduct a functional analysis of variance (fANOVA):
\begin{align*}
Var\left[\hat{f}(x)\right] &= Var\left[g_{\open  0  \close} + g_{\open 1 \close}(x_1) + g_{\open 2 \close}(x_2) + \;\dots\; + g_{\open 1, 2 \close}(x_1, x_2) \right. \\
&\phantom{{}={}} \left. + \;\dots\; + g_{\open 1,\ldots,p \close}(x) \right]
\end{align*}
\item If features are independent $\Rightarrow$ Variance can be additively decomposed without covariances:
\begin{align*}
Var\left[\hat{f}(x)\right] &= Var\left[g_{\open  0  \close}\right] + Var\left[g_{\open 1 \close}(x_1)\right] + Var\left[g_{\open 2 \close}(x_2)\right] \\
&\phantom{{}={}} + Var\left[g_{\open 1, 2 \close}(x_1, x_2)\right] + \;\dots\; + Var\left[g_{\open 1,\ldots,p \close}(x)\right]
\end{align*}
\end{itemize}
\end{frame}

\begin{frame}{Functional ANOVA}

\begin{itemize}
\item Dividing by the prediction variance, yields fraction of variance explained by each term:
\begin{align*}
1 &= \frac{Var\left[g_{\open  0  \close}\right]}{\predvar} + \frac{Var\left[g_{\open 1 \close}(x_1)\right]}{\predvar} + \frac{Var\left[g_{\open 2 \close}(x_2)\right]}{\predvar} \\
&\phantom{{}={}} + \frac{Var\left[g_{\open 1, 2 \close}(x_1, x_2)\right]}{\predvar} + \;\dots\; + \frac{Var\left[g_{\open 1,\ldots,p \close}(x)\right]}{\predvar}
\end{align*}

\item Fraction of variance explained by a term is the Sobol index:
$$
S_j = \frac{Var\left[g_{\open j \close}(x_j)\right]}{Var\left[\hat{f}(x)\right]}
$$
\end{itemize}

\end{frame}


\endlecture
\end{document}
