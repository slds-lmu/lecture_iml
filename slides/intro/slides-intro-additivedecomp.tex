\documentclass[11pt,compress,t,notes=noshow, aspectratio=169, xcolor=table]{beamer}

\usepackage{../../style/lmu-lecture}
% Defines macros and environments
\usepackage{bbm}
% basic latex stuff
\newcommand{\pkg}[1]{{\fontseries{b}\selectfont #1}} %fontstyle for R packages
\newcommand{\lz}{\vspace{0.5cm}} %vertical space
\newcommand{\dlz}{\vspace{1cm}} %double vertical space
\newcommand{\oneliner}[1] % Oneliner for important statements
{\begin{block}{}\begin{center}\begin{Large}#1\end{Large}\end{center}\end{block}}


%new environments
\newenvironment{vbframe}  %frame with breaks and verbatim
{
 \begin{frame}[containsverbatim,allowframebreaks]
}
{
\end{frame}
}

\newenvironment{vframe}  %frame with verbatim without breaks (to avoid numbering one slided frames)
{
 \begin{frame}[containsverbatim]
}
{
\end{frame}
}

\newenvironment{blocki}[1]   % itemize block
{
 \begin{block}{#1}\begin{itemize}
}
{
\end{itemize}\end{block}
}

\newenvironment{fragileframe}[2]{  %fragile frame with framebreaks
\begin{frame}[allowframebreaks, fragile, environment = fragileframe]
\frametitle{#1}
#2}
{\end{frame}}


\newcommand{\myframe}[2]{  %short for frame with framebreaks
\begin{frame}[allowframebreaks]
\frametitle{#1}
#2
\end{frame}}

\newcommand{\remark}[1]{
  \textbf{Remark:} #1
}


\newenvironment{deleteframe}
{
\begingroup
\usebackgroundtemplate{\includegraphics[width=\paperwidth,height=\paperheight]{../style/color/red.png}}
 \begin{frame}
}
{
\end{frame}
\endgroup
}
\newenvironment{simplifyframe}
{
\begingroup
\usebackgroundtemplate{\includegraphics[width=\paperwidth,height=\paperheight]{../style/color/yellow.png}}
 \begin{frame}
}
{
\end{frame}
\endgroup
}\newenvironment{draftframe}
{
\begingroup
\usebackgroundtemplate{\includegraphics[width=\paperwidth,height=\paperheight]{../style/color/green.jpg}}
 \begin{frame}
}
{
\end{frame}
\endgroup
}
% https://tex.stackexchange.com/a/261480: textcolor that works in mathmode
\makeatletter
\renewcommand*{\@textcolor}[3]{%
  \protect\leavevmode
  \begingroup
    \color#1{#2}#3%
  \endgroup
}
\makeatother


\providecommand{\tightlist}{%
  \setlength{\itemsep}{0pt}\setlength{\parskip}{0pt}}

%\setbeamerfont{footnote}{size=\tiny}
\usepackage[hang,flushmargin]{footmisc}
\renewcommand*{\footnotelayout}{\tiny}
\renewcommand*{\thefootnote}{} %\fnsymbol{footnote}

% https://tex.stackexchange.com/questions/30720/footnote-without-a-marker
% \makeatletter
% \def\blfootnote{\gdef\@thefnmark{}\@footnotetext}
% \makeatother

% https://tex.stackexchange.com/questions/357717/beamer-allowframebreaks-option-and-vertical-spacing-when-using-lists-itemize
% \setbeamertemplate{frametitle continuation}{%
%     (\insertcontinuationcount)%
%     \ifnum\insertcontinuationcount>1%
%     \vspace*{\topsep}%
%     \else%
%     %
%     \fi%
% }


\title{Interpretable Machine Learning}
% \author{LMU}
%\institute{\href{https://compstat-lmu.github.io/lecture_iml/}{compstat-lmu.github.io/lecture\_iml}}
\date{}

\begin{document}

\newcommand{\titlefigure}{figure/open_blackbox}
\newcommand{\learninggoals}{
\item What are additive decomposition of prediction functions?
\item Why are they useful?
\item How do we obtain them?}

\lecturechapter{Additive Decomposition}
\lecture{Interpretable Machine Learning}
 
\begin{vbframe}{High-Dimensional Model Representation}

\begin{itemize}
\itemsep2em
\item
A high-dimensional model representation (HDMR) decomposes the model into a sum of effect terms of increasing order:
\begin{align*}
\hat{f}(x) &= g_{\{0\}} + g_{\{1\}}(x_1) + g_{\{2\}}(x_2) + \;\dots\; + g_{\{1, 2\}}(x_1, x_2) \\
&\phantom{{}={}} + \;\dots\; + g_{\{1,\ldots,p\}}(x_1, \ldots,x_p)
\end{align*}
\item Univariate terms are referred to as first-order or main effects, bivariate terms as second-order effects, etc.
\item The features need to be independent to make the HDMR unique.
\item Different techniques to estimate an additive decomposition exist, e.g., repeated expectations (partial dependence / PD) or accumulated local effects (ALE).

\end{itemize}
\end{vbframe}

\begin{vbframe}{High-Dimensional Model Representation}

Consider the estimation via iterative expectations (in the later chapters we will learn how to compute these expectations):
\begin{align*}
 g_{\{0\}} &= \mathbb{E}_X\left[\widehat{f}(x)\right] \\
 g_{\{1\}}(x_1) &= \mathbb{E}_{X_{-1}}\left[\widehat{f}(x) \; \vert  \; X_1 \right] - g_{\{0\}} \\
 g_{\{2\}}(x_2) &= \mathbb{E}_{X_{-2}}\left[\widehat{f}(x) \; \vert  \; X_2 \right] - g_{\{0\}} \\
 g_{\{1, 2\}}(x_1, x_2) &= \mathbb{E}_{X_{-\{1,2\}}}\left[\widehat{f}(x) \; \vert \; X_1, X_2 \right] - g_{\{2\}}(x_2) - g_{\{1\}}(x_1) - g_{\{0\}}\\
 &\vdots \\
 g_{\{1, \dots, p\}}(x) &= \widehat{f}(x) - \dots - g_{\{1, 2\}}(x_1, x_2) \\
 &\phantom{{}={}} - g_{\{2\}}(x_2) - g_{\{1\}}(x_1) - g_{\{0\}}\\
\end{align*}

\end{vbframe}

\begin{vbframe}{High-Dimensional Model Representation}

\begin{itemize}
\itemsep2em
    \item Even if your goal is not to decompose the entire prediction function, it is helpful to keep in mind that a decomposition exists.
    \item When interpreting a model, one is often interested in feature effects of various orders, e.g., first-order and second-order effects.
    \item One may assume that the relevance of effects decreases with increasing order, e.g., that the model can be approximated via first-order and second-order effects.
    \item Do not confuse the outputs of interpretation methods such as the ICE or PD with the effect terms of the additive decomposition!
    \\$\rightarrow$ A second-order effect is the sole interaction effect on the target after a constant effect or main effects have been removed.
\end{itemize}
\end{vbframe}


\begin{vbframe}{Functional ANOVA}

\begin{itemize}
\item After the model has been decomposed one can analyze the variance of its constituent terms:
\begin{align*}
Var\left[\hat{f}(x)\right] &= Var\left[g_{\{0\}} + g_{\{1\}}(x_1) + g_{\{2\}}(x_2) + \;\dots\; + g_{\{1, 2\}}(x_1, x_2) \right. \\
&\phantom{{}={}} \left. + \;\dots\; + g_{\{1,\ldots,p\}}(x) \right]
\end{align*}
\item If the features are independent, the variance can be additively decomposed without covariances:
\begin{align*}
Var\left[\hat{f}(x)\right] &= Var\left[g_{\{0\}}\right] + Var\left[g_{\{1\}}(x_1)\right] + Var\left[g_{\{2\}}(x_2)\right] \\
&\phantom{{}={}} + Var\left[g_{\{1, 2\}}(x_1, x_2)\right] + \;\dots\; + Var\left[g_{\{1,\ldots,p\}}(x)\right]
\end{align*}
\item Dividing by the prediction variance results in the fraction of variance explained by each term:
\begin{align*}
1 &= \frac{Var\left[g_{\{0\}}\right]}{\predvar} + \frac{Var\left[g_{\{1\}}(x_1)\right]}{\predvar} + \frac{Var\left[g_{\{2\}}(x_2)\right]}{\predvar} \\
&\phantom{{}={}} + \frac{Var\left[g_{\{1, 2\}}(x_1, x_2)\right]}{\predvar} + \;\dots\; + \frac{Var\left[g_{\{1,\ldots,p\}}(x)\right]}{\predvar}
\end{align*}

\item The fraction of variance explained by a term is referred to as the Sobol index:
$$
S_j = \frac{Var\left[g_{\{j\}}(x_j)\right]}{Var\left[\hat{f}(x)\right]}
$$
\end{itemize}

\end{vbframe}


\endlecture
\end{document}
