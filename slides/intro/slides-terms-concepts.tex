\documentclass[11pt,compress,t,notes=noshow, xcolor=table]{beamer}
\usepackage[]{graphicx}\usepackage[]{color}
% maxwidth is the original width if it is less than linewidth
% otherwise use linewidth (to make sure the graphics do not exceed the margin)
\makeatletter
\def\maxwidth{ %
\ifdim\Gin@nat@width>\linewidth
\linewidth
\else
\Gin@nat@width
\fi
}
\makeatother

\definecolor{fgcolor}{rgb}{0.345, 0.345, 0.345}
\newcommand{\hlnum}[1]{\textcolor[rgb]{0.686,0.059,0.569}{#1}}%
\newcommand{\hlstr}[1]{\textcolor[rgb]{0.192,0.494,0.8}{#1}}%
\newcommand{\hlcom}[1]{\textcolor[rgb]{0.678,0.584,0.686}{\textit{#1}}}%
\newcommand{\hlopt}[1]{\textcolor[rgb]{0,0,0}{#1}}%
\newcommand{\hlstd}[1]{\textcolor[rgb]{0.345,0.345,0.345}{#1}}%
\newcommand{\hlkwa}[1]{\textcolor[rgb]{0.161,0.373,0.58}{\textbf{#1}}}%
\newcommand{\hlkwb}[1]{\textcolor[rgb]{0.69,0.353,0.396}{#1}}%
\newcommand{\hlkwc}[1]{\textcolor[rgb]{0.333,0.667,0.333}{#1}}%
\newcommand{\hlkwd}[1]{\textcolor[rgb]{0.737,0.353,0.396}{\textbf{#1}}}%
\newcommand{\predvar}{Var\left[\hat{f}(x)\right]}
\let\hlipl\hlkwb

\usepackage{framed}
\makeatletter
\newenvironment{kframe}{%
\def\at@end@of@kframe{}%
\ifinner\ifhmode%
\def\at@end@of@kframe{\end{minipage}}%
\begin{minipage}{\columnwidth}%
\fi\fi%
\def\FrameCommand##1{\hskip\@totalleftmargin \hskip-\fboxsep
\colorbox{shadecolor}{##1}\hskip-\fboxsep
% There is no \\@totalrightmargin, so:
\hskip-\linewidth \hskip-\@totalleftmargin \hskip\columnwidth}%
\MakeFramed {\advance\hsize-\width
\@totalleftmargin\z@ \linewidth\hsize
\@setminipage}}%
{\par\unskip\endMakeFramed%
\at@end@of@kframe}
\makeatother

\definecolor{shadecolor}{rgb}{.97, .97, .97}
\definecolor{messagecolor}{rgb}{0, 0, 0}
\definecolor{warningcolor}{rgb}{1, 0, 1}
\definecolor{errorcolor}{rgb}{1, 0, 0}
\newenvironment{knitrout}{}{} % an empty environment to be redefined in TeX

\usepackage{alltt}
\newcommand{\SweaveOpts}[1]{}  % do not interfere with LaTeX
\newcommand{\SweaveInput}[1]{} % because they are not real TeX commands
\newcommand{\Sexpr}[1]{}       % will only be parsed by R

\usepackage[english]{babel}
\usepackage[utf8]{inputenc}

\usepackage{dsfont}
\usepackage{verbatim}
\usepackage{amsmath}
\usepackage{amsfonts}
\usepackage{bm}
\usepackage{csquotes}
\usepackage{multirow}
\usepackage{longtable}
\usepackage{booktabs}
\usepackage{enumerate}
\usepackage[absolute,overlay]{textpos}
\usepackage{psfrag}
\usepackage{algorithm}
\usepackage{algpseudocode}
\usepackage{eqnarray}
\usepackage{arydshln}
\usepackage{tabularx}
\usepackage{placeins}
\usepackage{tikz}
\usepackage{setspace}
\usepackage{colortbl}
\usepackage{mathtools}
\usepackage{wrapfig}
\usepackage{bm}

\usetikzlibrary{shapes,arrows,automata,positioning,calc,chains,trees, shadows}
\tikzset{
%Define standard arrow tip
>=stealth',
%Define style for boxes
punkt/.style={
rectangle,
rounded corners,
draw=black, very thick,
text width=6.5em,
minimum height=2em,
text centered},
% Define arrow style
pil/.style={
->,
thick,
shorten <=2pt,
shorten >=2pt,}
}

\usepackage{subfig}

% Defines macros and environments
\usepackage{bbm}
% basic latex stuff
\newcommand{\pkg}[1]{{\fontseries{b}\selectfont #1}} %fontstyle for R packages
\newcommand{\lz}{\vspace{0.5cm}} %vertical space
\newcommand{\dlz}{\vspace{1cm}} %double vertical space
\newcommand{\oneliner}[1] % Oneliner for important statements
{\begin{block}{}\begin{center}\begin{Large}#1\end{Large}\end{center}\end{block}}


%new environments
\newenvironment{vbframe}  %frame with breaks and verbatim
{
 \begin{frame}[containsverbatim,allowframebreaks]
}
{
\end{frame}
}

\newenvironment{vframe}  %frame with verbatim without breaks (to avoid numbering one slided frames)
{
 \begin{frame}[containsverbatim]
}
{
\end{frame}
}

\newenvironment{blocki}[1]   % itemize block
{
 \begin{block}{#1}\begin{itemize}
}
{
\end{itemize}\end{block}
}

\newenvironment{fragileframe}[2]{  %fragile frame with framebreaks
\begin{frame}[allowframebreaks, fragile, environment = fragileframe]
\frametitle{#1}
#2}
{\end{frame}}


\newcommand{\myframe}[2]{  %short for frame with framebreaks
\begin{frame}[allowframebreaks]
\frametitle{#1}
#2
\end{frame}}

\newcommand{\remark}[1]{
  \textbf{Remark:} #1
}


\newenvironment{deleteframe}
{
\begingroup
\usebackgroundtemplate{\includegraphics[width=\paperwidth,height=\paperheight]{../style/color/red.png}}
 \begin{frame}
}
{
\end{frame}
\endgroup
}
\newenvironment{simplifyframe}
{
\begingroup
\usebackgroundtemplate{\includegraphics[width=\paperwidth,height=\paperheight]{../style/color/yellow.png}}
 \begin{frame}
}
{
\end{frame}
\endgroup
}\newenvironment{draftframe}
{
\begingroup
\usebackgroundtemplate{\includegraphics[width=\paperwidth,height=\paperheight]{../style/color/green.jpg}}
 \begin{frame}
}
{
\end{frame}
\endgroup
}
% https://tex.stackexchange.com/a/261480: textcolor that works in mathmode
\makeatletter
\renewcommand*{\@textcolor}[3]{%
  \protect\leavevmode
  \begingroup
    \color#1{#2}#3%
  \endgroup
}
\makeatother


\providecommand{\tightlist}{%
  \setlength{\itemsep}{0pt}\setlength{\parskip}{0pt}}

%\setbeamerfont{footnote}{size=\tiny}
\usepackage[hang,flushmargin]{footmisc}
\renewcommand*{\footnotelayout}{\tiny}
\renewcommand*{\thefootnote}{} %\fnsymbol{footnote}

% https://tex.stackexchange.com/questions/30720/footnote-without-a-marker
% \makeatletter
% \def\blfootnote{\gdef\@thefnmark{}\@footnotetext}
% \makeatother

% https://tex.stackexchange.com/questions/357717/beamer-allowframebreaks-option-and-vertical-spacing-when-using-lists-itemize
% \setbeamertemplate{frametitle continuation}{%
%     (\insertcontinuationcount)%
%     \ifnum\insertcontinuationcount>1%
%     \vspace*{\topsep}%
%     \else%
%     %
%     \fi%
% }


%\usetheme{lmu-lecture}
\usepackage{../../style/lmu-lecture}

\let\code=\texttt
\let\proglang=\textsf

\setkeys{Gin}{width=0.9\textwidth}

\title{Interpretable Machine Learning}
% \author{Bernd Bischl, Christoph Molnar, Daniel Schalk, Fabian Scheipl}
\institute{\href{https://compstat-lmu.github.io/lecture_iml/}{compstat-lmu.github.io/lecture\_iml}}
\date{}

\setbeamertemplate{frametitle}{\expandafter\uppercase\expandafter\insertframetitle}

\begin{document}

\input{../../latex-math/basic-math}
\input{../../latex-math/basic-ml}
\newcommand{\titlefigure}{figure/open_blackbox}
\newcommand{\learninggoals}{
\item What is interpretable machine learning (IML) and Explainable Artificial Intelligence (XAI)?
\item What is interpretability?
\item What are the fundamental terms and concepts of IML?}

\lecturechapter{Fundamental Terms and Concepts}
\lecture{Interpretable Machine Learning}

\begin{vbframe}{Interpretable Machine Learning}
\begin{itemize}
\itemsep2em
\item Machine learning (ML) algorithms algorithmically train predictive models with no or little pre-specifications and assumptions about the data.
\item Several algorithms such as decision tree learning create interpretable models. However, most algorithms create models which can be considered a black box.
\item We use the term black box, although the internal workings of the model are in fact accessible, but too complex for the human mind to comprehend.
\end{itemize}
\end{vbframe}

\begin{vbframe}{Explainable AI}
\begin{itemize}
\itemsep1em
\item IML is often used synonymously with Explainable AI (XAI). 
\item There is no unified standard for these terminologies. We find that XAI often is specifically concerned with the interpretation of neural networks, whereas IML is used as an encompassing term for everything related to model interpretability.
\item The nature of (deep) neural networks allows for powerful model-specific interpretation techniques, e.g., layer-wise relevance propagation (LRP) and saliency maps.
\item Also covering model-specific NN methods would exceed the timeframe of this lecture. This lecture will concentrate on model-agnostic techniques, as they are both versatile, and receive a lot of attention in industry and academia.
\end{itemize}
\end{vbframe}

\begin{vbframe}{XAI - Saliency Maps}

A saliency map is a heatmap indicating pixel influence on the prediction (e.g., a classification of an image): \footnote[frame]{Mundhenk, T., Chen, B.Y., Friedland, G. (2019). Efficient Saliency Maps for Explainable AI. ArXiv, abs/1911.11293.
}
\medskip
\begin{figure}
\includegraphics[width = 0.8 \textwidth]{figure/saliencymap}
\end{figure}
\end{vbframe}

\begin{vbframe}{What is Interpretability?}
\begin{itemize}
\itemsep1em
\item There is no scientific consensus on the definition of interpretability.
\item We need to differentiate between interpretations of a model or reality. The latter is distorted by all modeling fallacies involved in predictive modeling, e.g., data quality, under- and overfitting, or model extrapolations. 
\item We use a practical definition of interpretability.
Think back to the foundations of statistical modeling:  the linear regression model (LM). The LM, with its known equation of beta coefficients, represents a paradigm for statistical interpretability.
\item It follows that it would be beneficial to create techniques that give us an interpretation similar to the one of an LM.

\end{itemize}
\end{vbframe}




\begin{vbframe}{Terms and Concepts}

Methods from the field of IML can be organized in different dimensions, e.g.:

	\begin{itemize}
		\itemsep2em
		\item Intrinsic versus model-agnostic interpretability.
        \item Global versus local interpretations.
		\item Feature effects versus feature importance.
		\item Explanation types: feature attribution, data attribution, counterfactual explanations
		\item Fixed model analysis versus model refits.
	\end{itemize}
\end{vbframe}


\begin{vbframe}{Intrinsic versus Model-Agnostic}
	\begin{center}
		\includegraphics[width=\textwidth]{figure/overview}
	\end{center}
\end{vbframe}


\begin{vbframe}{Intrinsic versus Model-Agnostic}
	\begin{itemize}
		\item Intrinsically interpretable models:
		\begin{itemize}
			\item Examples are linear models and decision trees.
			\item Interpretable because of their simple structures, e.g., weighted combination of feature values or tree structure. 
			\item Difficult to interpret with many features or complex interaction terms.
		\end{itemize}
	\bigskip
	
		\item Model-agnostic interpretation methods:
		\begin{itemize}
			\item Applied after training (post-hoc).
			\item Also work for more complex black box models.
			\item Can also be applied to intrinsically interpretable models, e.g., feature importance for decision trees. 
		\end{itemize}
	\end{itemize}
\end{vbframe}

\begin{vbframe}{Model-Agnostic Interpretability}
	\begin{itemize}
		\itemsep2em
		\item Model-agnostic interpretation methods work for \textbf{any} kind of machine learning model.
		\item Explanation type is not tied to the underlying model type.
		\item Only access to data and trained model is required.\\
		 No further knowledge about the model itself is necessary.
		\item There are multiple types of explanations: feature attribution, data attribution, or counterfactual explanations.
	\end{itemize}
\end{vbframe}


\begin{vbframe}{Types of Explanations}
	\begin{center}
		\includegraphics[width=\textwidth]{figure/1-attributions.png}
    \end{center}
\end{vbframe}



\begin{vbframe}{Types of Explanations}

	The output of an interpretation technique is an \textbf{explanation.} We can differentiate different explanation types:
    
    \bigskip
	\begin{itemize}

    \itemsep2em
	\item 
		\textbf{Feature attribution:} Feature effect or importance.
		\\
		\smallskip
		%Input: feature $\rightarrow$ Output: target 
		Vary feature, inspect how target changes
	
	\item 
		\textbf{Data attribution:} Data attribution identifies  training instances that are most responsible for a decision.

	\item 
	   \textbf{Counterfactual Explanation:} 
	   Counterfactual explanations are identify the smallest necessary change in an input to change the prediction to the desired output.
	   \\
		%\smallskip
		%Input: target $\rightarrow$ Output: feature
		%Vary target, identify how feature needs to be changed
	  
	\end{itemize}
	
\end{vbframe}

% \begin{vbframe}{Explanation using training instances~\footnote{Koh et al. 2017}{https://arxiv.org/pdf/1703.04730.pdf}}
	
% 	\textbf{Data attribution:} Which training instances results in the decision for the instance $x$ of the model ?
% 	\begin{center}
% 		\includegraphics[page=1, width=0.7\textwidth]{figure/fish-attribution.pdf}
% 	\end{center}
	
% \end{vbframe}

% \begin{vbframe}{Explanation using training instances~\footnote{Koh et al. 2017}{https://arxiv.org/pdf/1703.04730.pdf}}
% 	\textbf{Data attribution:} Which training instances results in the decision for the instance $x$ of the model ?
% 	\begin{center}
% 		\includegraphics[page=1, width=0.7\textwidth]{figure/prototypes-fish.pdf}
% 	\end{center}
% 	\begin{itemize}
% 		\item Methods: 
% 		Influence functions, prototype generation.
% 	\end{itemize}
% \end{vbframe}

% \begin{vbframe}{Explanation using Counterfactuals}
%     A counterfactual is small ``imperceptible'' change in $x$. What if a small difference $ |x - x'| \leq \epsilon$ to $x$ causes a large change in the model output ?
% 	\begin{center}
% 		\includegraphics[page=1, width=0.7\textwidth]{figure/counterfactual.pdf}
% 	\end{center}
	
% \end{vbframe}


\begin{vbframe}{Global versus Local}
Global interpretation methods explain the model behavior for the entire input space by considering all available observations:
	\begin{itemize}
		\item Permutation Feature Importance (PFI)
		\item Partial Dependence (PD)
		\item Functional Anova (FANOVA)
		\item Accumulated Local Effects (ALE)
		\item ...
	\end{itemize}
\bigskip
Local interpretation methods explain the model behavior for single data instances:
	\begin{itemize}
		\item Individual Conditional Expectation (ICE)
		\item Local Interpretable Model-Agnostic Explanations (LIME)
		\item Shapley values
		\item ...
	\end{itemize}
\end{vbframe}

\begin{vbframe}{Feature Effects vs. Feature Importance}
	
	\textbf{Feature Effects} indicate the change in prediction due to changes in feature values.
	\medskip
	\begin{center}
		\includegraphics[page=1, width=\textwidth]{figure/feature-effects}
	\end{center}
	\begin{itemize}
		\item Methods: ICE, PD, ALE
		\item Pendant in linear models: Regression coefficient $\beta_j$
	\end{itemize}
\end{vbframe}

\begin{vbframe}{Feature Effects vs. Feature Importance}
	
	\textbf{Feature importance} methods rank features by how much they contribute to the predictive performance or prediction variance of the model.
	\begin{center}
		\includegraphics[page=1, width=0.4\textwidth]{figure/feature-importance}
	\end{center}
	\begin{itemize}
		\itemsep1em
		\item Methods: PFI, FANOVA
		\item Pendant in linear models: t-statistic
	\end{itemize}

\end{vbframe}



\begin{vbframe}{Fixed Model vs. Refits}
	\begin{itemize}
		\itemsep2em
		\item Most methods we will discuss analyze a fixed, trained model, e.g., PFI.
		\item Some methods require refitting the model (e.g., PIMP).
		\item Trained model $\Rightarrow$ Model is the object of analysis.
		\item Refitting $\Rightarrow$ Learning process is the object of analysis.
		\item The advantage of refitting is that it includes information about the variability in the learning process.
	\end{itemize}
\end{vbframe}



\begin{vbframe}{Correlation versus Interaction}
\begin{itemize}
\itemsep2em
\item A lot of problems in IML arise due to correlated features and feature interactions. One needs to be careful not to confuse them.
\item Correlated features imply that the joint probability density function (PDF) of a set of $p$ features is not a product of their marginal PDFs, i.e., $\text{PDF}(x_1, \dots, x_p) \neq \text{PDF}(x_1) \cdot \ldots \cdot \text{PDF}(x_p)$.
\item
Many interpretation methods rely on model evaluations with recombined feature values. If the training data is correlated, these recombinations are located in low-training-density areas of the feature space.
\item This results in model extrapolations.
% \item An interaction is a product term between features inside the prediction function. As such, interactions are detached from the constitution of the data, i.e., regardloss of the degree of correlation, the effect of a feature on the target will depend on the values of one or multiple other features.
\end{itemize}
\end{vbframe}

\begin{vbframe}{Correlation}

1000 randomly sampled observations with positive correlation between $x_1$ and $x_2$:
\medskip
\begin{figure}
\includegraphics[width = 0.7\textwidth]{figure/correlation}
\end{figure}
\end{vbframe}

\begin{vbframe}{Interaction}
\begin{itemize}
\itemsep2em
\item An interaction concerns the structure of the model itself.
\item With interactions present, the effect of a feature on the prediction depends on the values of others, e.g., $\widehat{f}(x) = x_1 x_2$.
\item Although correlation concerns the data and interactions the model, they are often connected as correlations in the training data are identified by the learning algorithm.
\item Interaction terms are especially difficult to identify and analyze, as the number of potential interactions increases exponentially with the number of features.
\end{itemize}
\end{vbframe}

\begin{vbframe}{Interaction}

Shape of the prediction function for a two-way interaction:

\begin{figure}
\includegraphics[width = 0.6\textwidth]{figure/interaction}
\end{figure}
\end{vbframe}


\begin{vbframe}{Extrapolation}
\begin{itemize}
\itemsep1em
\item Predictions of an ML model in regions with a low density of training points are subject to a high variance.
\item Essentially we are interested in the prediction uncertainty, i.e., the less the training data on a feature subspace, the higher the prediction uncertainty. Unfortunately, virtually no ML models are capable of quantifying prediction uncertainty.
\item There is no consensus definition of when a model extrapolates and to what degree. Furthermore, the severity of the problem depends on the model itself. Some models might extrapolate more reliably than others.
\item Theoretically, we could use the training density as a proxy. However, density estimation in many dimensions is often infeasible.
\end{itemize}

\end{vbframe}

\begin{vbframe}{Pitfalls of Interpretation Methods}
\begin{itemize}
\itemsep1em
\item Many methods in IML are theoretically defined for uncorrelated features, e.g., the PD or the PFI.
\item In practice, the features are usually correlated, but methods are applied regardless of potential misinterpretations.
\item Many people call for a more careful approach to conduct model interpretations, either by using intrinsically interpretable models (Rudin, 2019), or by avoiding feature permutations (Hooker, 2021).
\item There are many potential pitfalls to consider when interpreting ML models (Molnar, 2021).
\\
$\rightarrow$ Know the theory and be careful!

\footnote[frame]{Molnar, Christoph \& König, Gunnar \& Herbinger, Julia \& Freiesleben, Timo \& Dandl, Susanne \& Scholbeck, Christian \& Casalicchio, Giuseppe \& Grosse-Wentrup, Moritz \& Bischl, Bernd. (2020). Pitfalls to Avoid when Interpreting Machine Learning Models.}
\footnote[frame]{Rudin, C. Stop explaining black box machine learning models for high stakes decisions and use interpretable models instead. Nat Mach Intell 1, 206–215 (2019).}
\footnote[frame]{Hooker, Giles & Mentch, Lucas & Zhou, Siyu. (2021). Unrestricted permutation forces extrapolation: variable importance requires at least one more model, or there is no free variable importance. Statistics and Computing. 31. 10.1007/s11222-021-10057-z.}
\end{itemize}

\end{vbframe}

% \begin{vbframe}{Intrinsic and Model-Agnostic Interpretation}
% \begin{itemize}
%   \item Intrinsically interpretable models:
%   \begin{itemize}
%   \item Examples are linear models and decision trees.
%   \item They are interpretable because of their simple structures, e.g.,
%   a weighted combination of feature values or a tree structure. 
%   \item However, they are difficult to interpret with many features or complex interaction terms.
%   \end{itemize}
%   \lz
%   \item Model-agnostic interpretation methods:
%   \begin{itemize}
%   \item They are applied after training (post-hoc).
%   \item They also work for more complex black box models.
%   \item They can also be applied to intrinsically interpretable models, e.g.
%     feature importance for decision trees. 
%   \end{itemize}
% \end{itemize}
% \end{vbframe}
% 
% \begin{vbframe}{Model-Agnostic Interpretability}
%  \begin{itemize}
%   \itemsep2em
%   \item Model-agnostic interpretability methods work for \textbf{any} kind of machine learning model.
%   \item Explanation type is not tied to the underlying model type.
%   \item Often, only access to data and fitted predictor is required. No further knowledge about the model itself is necessary.
%   \item We usually distinguish between \textbf{feature effect} and \textbf{feature importance} methods.
%  \end{itemize}
% \end{vbframe}
% 
% 
% \begin{vbframe}{Feature Effects vs. Feature Importance}
% \textbf{Feature effects} indicate the direction and magnitude of a change in predicted outcome due to changes in feature values.
% \begin{center}
% \includegraphics[page=1, width=\textwidth]{figure/feature-effects}
% \end{center}
%   \begin{itemize}
%     \item Methods include: Partial Dependence Plots, Individual Conditional Expectation, Accumulated Local Effects (ALE)
%     \item Pendant in linear models: Regression coefficient $\hat{\theta}_j$
%   \end{itemize}
% \framebreak
% 
% \textbf{Feature importance} methods rank features by how much they contribute to the predictive performance or prediction variance of the model.
% \begin{columns}
% \begin{column}{0.6\textwidth}
% \begin{itemize}
%     \itemsep1em
%     \item Methods include: Permutation Feature Importance, Functional Anova
%     \item Analog in linear models: Absolute t-statistic $\left|\frac{\hat{\theta}_j}{SE(\hat{\theta}_j)}\right|$
% \end{itemize}
% \end{column}
% \begin{column}{0.4\textwidth}
% \begin{center}
% \includegraphics[page=1, width=\textwidth]{figure/feature-importance}
% \end{center}
% \end{column}
% \end{columns}
% \end{vbframe}
% 
% 
% \begin{vbframe}{Global and Local Interpretability}
% Global interpretability methods explain the expected model behavior for the entire feature space by considering all available observations (or representative subsets). For example:
%   \begin{itemize}
%     \item Permutation Feature Importance
%     \item Partial Dependence Plot
%     \item Functional Anova
%     \item ...
%   \end{itemize}
% \lz
% Local interpretability methods explain single predictions or a group of similar observations. For example:
%  \begin{itemize}
%   \item Individual Conditional Expectation (ICE) Plots
%   \item Local Interpretable Model-Agnostic Explanations (LIME)
%   \item Shapley Values
%   \item ...
%  \end{itemize}
% \end{vbframe}
% 
% 
% \begin{vbframe}{Fixed model vs. refits}
%   \begin{itemize}
%      \itemsep2em
%      \item Most methods presented in this lecture analyze a fixed, trained model 
%      (e.g., permutation feature importance).
%      \item Some methods require refitting the model (e.g., PIMP).
%      \item Trained model $\Rightarrow$ Model is the object of analysis.
%      \item Refitting $\Rightarrow$ Learning process is the object of analysis.
%      \item The advantage of refitting is that it includes information about the variability in the learning process.
%   \end{itemize}
% \end{vbframe}


\endlecture
\end{document}
