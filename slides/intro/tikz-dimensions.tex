\documentclass[11pt,compress,t,notes=noshow, aspectratio=169, xcolor=table]{beamer}

\usepackage{../../style/lmu-lecture}
% Defines macros and environments
\usepackage{bbm}
% basic latex stuff
\newcommand{\pkg}[1]{{\fontseries{b}\selectfont #1}} %fontstyle for R packages
\newcommand{\lz}{\vspace{0.5cm}} %vertical space
\newcommand{\dlz}{\vspace{1cm}} %double vertical space
\newcommand{\oneliner}[1] % Oneliner for important statements
{\begin{block}{}\begin{center}\begin{Large}#1\end{Large}\end{center}\end{block}}


%new environments
\newenvironment{vbframe}  %frame with breaks and verbatim
{
 \begin{frame}[containsverbatim,allowframebreaks]
}
{
\end{frame}
}

\newenvironment{vframe}  %frame with verbatim without breaks (to avoid numbering one slided frames)
{
 \begin{frame}[containsverbatim]
}
{
\end{frame}
}

\newenvironment{blocki}[1]   % itemize block
{
 \begin{block}{#1}\begin{itemize}
}
{
\end{itemize}\end{block}
}

\newenvironment{fragileframe}[2]{  %fragile frame with framebreaks
\begin{frame}[allowframebreaks, fragile, environment = fragileframe]
\frametitle{#1}
#2}
{\end{frame}}


\newcommand{\myframe}[2]{  %short for frame with framebreaks
\begin{frame}[allowframebreaks]
\frametitle{#1}
#2
\end{frame}}

\newcommand{\remark}[1]{
  \textbf{Remark:} #1
}


\newenvironment{deleteframe}
{
\begingroup
\usebackgroundtemplate{\includegraphics[width=\paperwidth,height=\paperheight]{../style/color/red.png}}
 \begin{frame}
}
{
\end{frame}
\endgroup
}
\newenvironment{simplifyframe}
{
\begingroup
\usebackgroundtemplate{\includegraphics[width=\paperwidth,height=\paperheight]{../style/color/yellow.png}}
 \begin{frame}
}
{
\end{frame}
\endgroup
}\newenvironment{draftframe}
{
\begingroup
\usebackgroundtemplate{\includegraphics[width=\paperwidth,height=\paperheight]{../style/color/green.jpg}}
 \begin{frame}
}
{
\end{frame}
\endgroup
}
% https://tex.stackexchange.com/a/261480: textcolor that works in mathmode
\makeatletter
\renewcommand*{\@textcolor}[3]{%
  \protect\leavevmode
  \begingroup
    \color#1{#2}#3%
  \endgroup
}
\makeatother


\providecommand{\tightlist}{%
  \setlength{\itemsep}{0pt}\setlength{\parskip}{0pt}}

%\setbeamerfont{footnote}{size=\tiny}
\usepackage[hang,flushmargin]{footmisc}
\renewcommand*{\footnotelayout}{\tiny}
\renewcommand*{\thefootnote}{} %\fnsymbol{footnote}

% https://tex.stackexchange.com/questions/30720/footnote-without-a-marker
% \makeatletter
% \def\blfootnote{\gdef\@thefnmark{}\@footnotetext}
% \makeatother

% https://tex.stackexchange.com/questions/357717/beamer-allowframebreaks-option-and-vertical-spacing-when-using-lists-itemize
% \setbeamertemplate{frametitle continuation}{%
%     (\insertcontinuationcount)%
%     \ifnum\insertcontinuationcount>1%
%     \vspace*{\topsep}%
%     \else%
%     %
%     \fi%
% }

\newcommand\tab[1][1cm]{\hspace*{#1}}

\tikzset{main node/.style={rectangle,draw,minimum size=1cm,inner sep=4pt},}

\title{Interpretable Machine Learning}
\date{}

\begin{document}
\newcommand{\titlefigure}{figure/open_blackbox}
\newcommand{\learninggoals}{
\item What is interpretable machine learning (IML) and Explainable Artificial Intelligence (XAI)?
\item What is interpretability?
\item What are the fundamental terms and concepts of IML?}

\lecturechapter{Fundamental Terms and Concepts}
\lecture{Interpretable Machine Learning}

\begin{vbframe}{Intrinsic vs. Model-Agnostic}
	\begin{tikzpicture}[every path/.style={->,line width=0.35mm,thick},
                        every label/.append style={align=left, font=\footnotesize, text width=3.2cm}]
        \node[main node] (1) { Model Interpretation };
        \node[main node,
            label={below:Decision trees \\ Decision rules \\ Generalized regression models}
            ] (2) [below left = 1cm and 1cm of 1]  { Interpretable Models };
        \node[main node] (3) [below right = 1cm and -0.5cm of 1] { Black Box Models };
        \node[main node,
            label={below:Random Forest Explainer \\ Visualizing activations of neural networks}
            ] (4) [below left = 1cm and 0cm of 3] { Model-specific Methods };
        \node[main node, fill=gray,
            label={below:\textbf{Advantage: Flexibility} \\ Underlying models can be exchanged\\ Several IML Visualizations can be used}
            ] (5) [below right = 1cm and 0cm of 3] { Model-agnostic Methods };
        \draw (1) -- (2);
        \draw (1) -- (3);
        \draw (3) -- (4);
        \draw (3) -- (5);

    \end{tikzpicture}
\end{vbframe}

\begin{vbframe}{Types of Explanations}
    \begin{center}
        \begin{tikzpicture}[every path/.style={->,line width=0.35mm,thick},
                            every label/.append style={align=left, font=\footnotesize, text width=4cm}]
            \node[main node] (1) { Model Interpretation };
            \node[main node,
                label={below:Saliency Maps\\ Hard Masking \\ Model-agnostic methods \\ \tab SHAP\\ \tab LIME}
                ] (2) [below left = 2.3cm and 1.5cm of 1]  { Feature Attribution };
            \node[main node,
                label={below:Influence Functions\\ ... }
                ] (3) [below = 2.3cm of 1] { Data Attribution };
            \node[main node,
                label={below:Contrastive explanations\\ Diverse counterfactuals \\ Feasible \& actionable explanations}
                ] (4) [below right = 2.3cm and 1.5cm of 1] { Counterfactual Explanations };
            \draw (1) -- (2);
            \draw (1) -- (3);
            \draw (1) -- (4);
    
        \end{tikzpicture}  
    \end{center}
\end{vbframe}


\begin{vbframe}{Global vs. Local}

\begin{center}
    \begin{tikzpicture}[every path/.style={->,line width=0.35mm,thick},
                        every label/.append style={align=left, font=\footnotesize, text width=4cm}]
        \node[main node] (1) { Model Interpretation };
        \node[main node,
            label={below:Instance wise feature selections\\ ALE \\ Model-agnostic methods \\ \tab SHAP\\ \tab LIME}
            ] (2) [below left = 2.3cm and 1.5cm of 1]  { Local Explanations };
        \node[main node,
            label={below:Rule-based Explanations\\ Permutation-based Feature importance \\ Functional Anova \\ PDP}
            ] (3) [below right = 2.3cm and 1.5cm of 1] { Global Explanations };
        \draw (1) -- (2);
        \draw (1) -- (3);

    \end{tikzpicture}  
\end{center}

\end{vbframe}



\endlecture
\end{document}
