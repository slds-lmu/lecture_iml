\documentclass[11pt,compress,t,notes=noshow, xcolor=table]{beamer}
\usepackage[]{graphicx}\usepackage[]{color}
% maxwidth is the original width if it is less than linewidth
% otherwise use linewidth (to make sure the graphics do not exceed the margin)
\makeatletter
\def\maxwidth{ %
  \ifdim\Gin@nat@width>\linewidth
    \linewidth
  \else
    \Gin@nat@width
  \fi
}
\makeatother

\definecolor{fgcolor}{rgb}{0.345, 0.345, 0.345}
\newcommand{\hlnum}[1]{\textcolor[rgb]{0.686,0.059,0.569}{#1}}%
\newcommand{\hlstr}[1]{\textcolor[rgb]{0.192,0.494,0.8}{#1}}%
\newcommand{\hlcom}[1]{\textcolor[rgb]{0.678,0.584,0.686}{\textit{#1}}}%
\newcommand{\hlopt}[1]{\textcolor[rgb]{0,0,0}{#1}}%
\newcommand{\hlstd}[1]{\textcolor[rgb]{0.345,0.345,0.345}{#1}}%
\newcommand{\hlkwa}[1]{\textcolor[rgb]{0.161,0.373,0.58}{\textbf{#1}}}%
\newcommand{\hlkwb}[1]{\textcolor[rgb]{0.69,0.353,0.396}{#1}}%
\newcommand{\hlkwc}[1]{\textcolor[rgb]{0.333,0.667,0.333}{#1}}%
\newcommand{\hlkwd}[1]{\textcolor[rgb]{0.737,0.353,0.396}{\textbf{#1}}}%
\let\hlipl\hlkwb

\usepackage{framed}
\makeatletter
\newenvironment{kframe}{%
 \def\at@end@of@kframe{}%
 \ifinner\ifhmode%
  \def\at@end@of@kframe{\end{minipage}}%
  \begin{minipage}{\columnwidth}%
 \fi\fi%
 \def\FrameCommand##1{\hskip\@totalleftmargin \hskip-\fboxsep
 \colorbox{shadecolor}{##1}\hskip-\fboxsep
     % There is no \\@totalrightmargin, so:
     \hskip-\linewidth \hskip-\@totalleftmargin \hskip\columnwidth}%
 \MakeFramed {\advance\hsize-\width
   \@totalleftmargin\z@ \linewidth\hsize
   \@setminipage}}%
 {\par\unskip\endMakeFramed%
 \at@end@of@kframe}
\makeatother

\definecolor{shadecolor}{rgb}{.97, .97, .97}
\definecolor{messagecolor}{rgb}{0, 0, 0}
\definecolor{warningcolor}{rgb}{1, 0, 1}
\definecolor{errorcolor}{rgb}{1, 0, 0}
\newenvironment{knitrout}{}{} % an empty environment to be redefined in TeX

\usepackage{alltt}
\newcommand{\SweaveOpts}[1]{}  % do not interfere with LaTeX
\newcommand{\SweaveInput}[1]{} % because they are not real TeX commands
\newcommand{\Sexpr}[1]{}       % will only be parsed by R

\usepackage[english]{babel}
\usepackage[utf8]{inputenc}

\usepackage{dsfont}
\usepackage{verbatim}
\usepackage{amsmath}
\usepackage{amsfonts}
\usepackage{bm}
\usepackage{csquotes}
\usepackage{multirow}
\usepackage{longtable}
\usepackage{booktabs}
\usepackage{enumerate}
\usepackage[absolute,overlay]{textpos}
\usepackage{psfrag}
\usepackage{algorithm}
\usepackage{algpseudocode}
\usepackage{eqnarray}
\usepackage{arydshln}
\usepackage{tabularx}
\usepackage{placeins}
\usepackage{tikz}
\usepackage{setspace}
\usepackage{colortbl}
\usepackage{mathtools}
\usepackage{wrapfig}
\usepackage{bm}

\usetikzlibrary{shapes,arrows,automata,positioning,calc,chains,trees, shadows}
\tikzset{
  %Define standard arrow tip
  >=stealth',
  %Define style for boxes
  punkt/.style={
    rectangle,
    rounded corners,
    draw=black, very thick,
    text width=6.5em,
    minimum height=2em,
    text centered},
  % Define arrow style
  pil/.style={
    ->,
    thick,
    shorten <=2pt,
    shorten >=2pt,}
}

\usepackage{subfig}

% Defines macros and environments
\usepackage{bbm}
% basic latex stuff
\newcommand{\pkg}[1]{{\fontseries{b}\selectfont #1}} %fontstyle for R packages
\newcommand{\lz}{\vspace{0.5cm}} %vertical space
\newcommand{\dlz}{\vspace{1cm}} %double vertical space
\newcommand{\oneliner}[1] % Oneliner for important statements
{\begin{block}{}\begin{center}\begin{Large}#1\end{Large}\end{center}\end{block}}


%new environments
\newenvironment{vbframe}  %frame with breaks and verbatim
{
 \begin{frame}[containsverbatim,allowframebreaks]
}
{
\end{frame}
}

\newenvironment{vframe}  %frame with verbatim without breaks (to avoid numbering one slided frames)
{
 \begin{frame}[containsverbatim]
}
{
\end{frame}
}

\newenvironment{blocki}[1]   % itemize block
{
 \begin{block}{#1}\begin{itemize}
}
{
\end{itemize}\end{block}
}

\newenvironment{fragileframe}[2]{  %fragile frame with framebreaks
\begin{frame}[allowframebreaks, fragile, environment = fragileframe]
\frametitle{#1}
#2}
{\end{frame}}


\newcommand{\myframe}[2]{  %short for frame with framebreaks
\begin{frame}[allowframebreaks]
\frametitle{#1}
#2
\end{frame}}

\newcommand{\remark}[1]{
  \textbf{Remark:} #1
}


\newenvironment{deleteframe}
{
\begingroup
\usebackgroundtemplate{\includegraphics[width=\paperwidth,height=\paperheight]{../style/color/red.png}}
 \begin{frame}
}
{
\end{frame}
\endgroup
}
\newenvironment{simplifyframe}
{
\begingroup
\usebackgroundtemplate{\includegraphics[width=\paperwidth,height=\paperheight]{../style/color/yellow.png}}
 \begin{frame}
}
{
\end{frame}
\endgroup
}\newenvironment{draftframe}
{
\begingroup
\usebackgroundtemplate{\includegraphics[width=\paperwidth,height=\paperheight]{../style/color/green.jpg}}
 \begin{frame}
}
{
\end{frame}
\endgroup
}
% https://tex.stackexchange.com/a/261480: textcolor that works in mathmode
\makeatletter
\renewcommand*{\@textcolor}[3]{%
  \protect\leavevmode
  \begingroup
    \color#1{#2}#3%
  \endgroup
}
\makeatother


\providecommand{\tightlist}{%
  \setlength{\itemsep}{0pt}\setlength{\parskip}{0pt}}

%\setbeamerfont{footnote}{size=\tiny}
\usepackage[hang,flushmargin]{footmisc}
\renewcommand*{\footnotelayout}{\tiny}
\renewcommand*{\thefootnote}{} %\fnsymbol{footnote}

% https://tex.stackexchange.com/questions/30720/footnote-without-a-marker
% \makeatletter
% \def\blfootnote{\gdef\@thefnmark{}\@footnotetext}
% \makeatother

% https://tex.stackexchange.com/questions/357717/beamer-allowframebreaks-option-and-vertical-spacing-when-using-lists-itemize
% \setbeamertemplate{frametitle continuation}{%
%     (\insertcontinuationcount)%
%     \ifnum\insertcontinuationcount>1%
%     \vspace*{\topsep}%
%     \else%
%     %
%     \fi%
% }


\input{../../latex-math/basic-ml.tex}
\input{../../latex-math/basic-math.tex}

%\usetheme{lmu-lecture}
\newcommand{\titlefigure}{figure_man/earnings.png}
% License of titlefigure: free pixabay license
% https://pixabay.com/de/vectors/ergebnis-geld-gesch%C3%A4ft-gehalt-5567652/
\newcommand{\learninggoals}{%
\item Understand attribution in cooperative games
\item Understand the concept of Shapley values
\item Learn about applications of Shapley values
  }
\usepackage{../../style/lmu-lecture}

\let\code=\texttt
\let\proglang=\textsf

\setkeys{Gin}{width=0.9\textwidth}

\title{Interpretable Machine Learning}
% \author{Bernd Bischl, Christoph Molnar, Daniel Schalk, Fabian Scheipl}
\institute{\href{https://compstat-lmu.github.io/lecture_i2ml/}{compstat-lmu.github.io/lecture\_i2ml}}
\date{}

\setbeamertemplate{frametitle}{\expandafter\uppercase\expandafter\insertframetitle}

\begin{document}

\lecturechapter{Shapley Values}
\lecture{Interpretable Machine Learning}

\begin{vbframe}{Cooperative Games}
\begin{itemize}
  \item In cooperative games, a set of players $P$ with $P = \{1, \hdots, p\}$ forms a coalition $S \subseteq P$. 
  \item A value function $v(S): \mathcal{P}(P)\mapsto \R$ describes the payout (or gain) achieved by the coalition $S$. Here, $\mathcal{P}$ is the powerset which contains all possible subbsets of $P$, including the empty set $\empty$ and $P$ itself.
  \item As some players contribute more than others, we are interested in the distribution of the payout among the players.
  \item We call the payout per player $\phi_j(v)$, $j \in P$.
  \item Problem: How to fairly distribute values among players?
\end{itemize}
\end{vbframe}


\begin{vbframe}{What Would Be a Fair Attribution?}
  We can posit these axioms for what a fair distribution would look like:
  \begin{itemize}
    \item \textbf{Efficiency}: Player contributions add up to total payout (with all players).
      $\sum\nolimits_{j=1}^p\phi_j = v(P)$
    \item \textbf{Symmetry}: Two players who contribute the same get the same payout: \\
      $v(S\cup\{j\}) = v(S\cup\{k\})$ for all $S \subseteq P\setminus\{j,k\}$ then $\phi_{j}=\phi_{k}$
    \item \textbf{Dummy / Null Player}: The payout is zero for players who don't contribute to the value of any coalition: \\
      $val(S\cup\{j\})=val(S)$ for all $S \subseteq P$ then $\phi_j=0$
    \item \textbf{Additivity}: For a game with combined payouts ($v1$ and $v2$), the payout is the sum of payouts: $\phi_j(v1) + \phi_{j}(v2)$
  \end{itemize}

  Is there a formula for payouts which adheres to all these axioms?

\end{vbframe}


\begin{vbframe}{Shapley Values}
 
  Shapley Values are a way to solve the attribution problem and pose a unique solution given the axioms of efficiency, symmetry, dummy and additivity.
\begin{itemize}
  \item The \textbf{Shapley value} assigns a value to each player according to the marginal contribution of each player in all possible coalitions.
  \item $\phi_j(v) = \sum_{S \subseteq P \setminus \{j\}} \frac{|S|!(|P| - |S| - 1)!}{|P|!}(v(S \cup \{j\}) - v(S))$
  \item Shapley values are the \textit{only} solution for the attribution with the specified axioms.
  \item Proposed by Lloyd Shapley in 1951 for cooperative game theory.
\end{itemize}

\tiny{ Shapley, Lloyd S. (August 21, 1951). "Notes on the n-Person Game -- II: The Value of an n-Person Game" (PDF). Santa Monica, Calif.: RAND Corporation.}

\end{vbframe}

  \frame{
\frametitle{Shapley Values - Illustration}
The Shapley value of the feature $x_2$ is the marginal contribution (prediction change) when $x_2$ enters an abitrary coalition. \\
\only<1>{Here, $x_2$ enters the coalition secondly, resulting in a prediction change of $24-12 = 12$. Overall, the coalition increases the prediction by 36.}
\only<2>{We produce all possible orders of feature coalitions and measure the prediction change if feature $x_2$ enters the coalition.}
\begin{center}
  \only<1>{
    \includegraphics[page=1, width=0.8\textwidth]{figure_man/shapley_feature_effect}
  }
  \only<2>{
    \includegraphics[page=2, width=0.8\textwidth]{figure_man/shapley_feature_effect}
  }
\end{center}

}
      

\begin{vbframe}{Applications of Shapley Value}

  \begin{itemize}
      \item Game theory
      \item Economics (e.g., cost allocation) % https://www.jstor.org/stable/2951524
      \item Marketing (e.g., channel attribution) % http://datafeedtoolbox.com/attribution-theory-the-two-best-models-for-algorithmic-marketing-attribution-implemented-in-apache-spark-and-r/
      \item Machine Learning:
       \begin{itemize}
         \item Feature selection: Attribution of loss to individual features
         \item Quantify value of data: Loss decomposition among data instances % http://proceedings.mlr.press/v97/ghorbani19c/ghorbani19c.pdf
         \item \textbf{Explain individual predictions}
       \end{itemize}
  \end{itemize}

\end{vbframe}

\endlecture
\end{document}
