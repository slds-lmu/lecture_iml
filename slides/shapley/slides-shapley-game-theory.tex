\documentclass[11pt,compress,t,notes=noshow, aspectratio=169, xcolor=table]{beamer}

\usepackage{../../style/lmu-lecture}
% Defines macros and environments
\usepackage{bbm}
% basic latex stuff
\newcommand{\pkg}[1]{{\fontseries{b}\selectfont #1}} %fontstyle for R packages
\newcommand{\lz}{\vspace{0.5cm}} %vertical space
\newcommand{\dlz}{\vspace{1cm}} %double vertical space
\newcommand{\oneliner}[1] % Oneliner for important statements
{\begin{block}{}\begin{center}\begin{Large}#1\end{Large}\end{center}\end{block}}


%new environments
\newenvironment{vbframe}  %frame with breaks and verbatim
{
 \begin{frame}[containsverbatim,allowframebreaks]
}
{
\end{frame}
}

\newenvironment{vframe}  %frame with verbatim without breaks (to avoid numbering one slided frames)
{
 \begin{frame}[containsverbatim]
}
{
\end{frame}
}

\newenvironment{blocki}[1]   % itemize block
{
 \begin{block}{#1}\begin{itemize}
}
{
\end{itemize}\end{block}
}

\newenvironment{fragileframe}[2]{  %fragile frame with framebreaks
\begin{frame}[allowframebreaks, fragile, environment = fragileframe]
\frametitle{#1}
#2}
{\end{frame}}


\newcommand{\myframe}[2]{  %short for frame with framebreaks
\begin{frame}[allowframebreaks]
\frametitle{#1}
#2
\end{frame}}

\newcommand{\remark}[1]{
  \textbf{Remark:} #1
}


\newenvironment{deleteframe}
{
\begingroup
\usebackgroundtemplate{\includegraphics[width=\paperwidth,height=\paperheight]{../style/color/red.png}}
 \begin{frame}
}
{
\end{frame}
\endgroup
}
\newenvironment{simplifyframe}
{
\begingroup
\usebackgroundtemplate{\includegraphics[width=\paperwidth,height=\paperheight]{../style/color/yellow.png}}
 \begin{frame}
}
{
\end{frame}
\endgroup
}\newenvironment{draftframe}
{
\begingroup
\usebackgroundtemplate{\includegraphics[width=\paperwidth,height=\paperheight]{../style/color/green.jpg}}
 \begin{frame}
}
{
\end{frame}
\endgroup
}
% https://tex.stackexchange.com/a/261480: textcolor that works in mathmode
\makeatletter
\renewcommand*{\@textcolor}[3]{%
  \protect\leavevmode
  \begingroup
    \color#1{#2}#3%
  \endgroup
}
\makeatother


\providecommand{\tightlist}{%
  \setlength{\itemsep}{0pt}\setlength{\parskip}{0pt}}

%\setbeamerfont{footnote}{size=\tiny}
\usepackage[hang,flushmargin]{footmisc}
\renewcommand*{\footnotelayout}{\tiny}
\renewcommand*{\thefootnote}{} %\fnsymbol{footnote}

% https://tex.stackexchange.com/questions/30720/footnote-without-a-marker
% \makeatletter
% \def\blfootnote{\gdef\@thefnmark{}\@footnotetext}
% \makeatother

% https://tex.stackexchange.com/questions/357717/beamer-allowframebreaks-option-and-vertical-spacing-when-using-lists-itemize
% \setbeamertemplate{frametitle continuation}{%
%     (\insertcontinuationcount)%
%     \ifnum\insertcontinuationcount>1%
%     \vspace*{\topsep}%
%     \else%
%     %
%     \fi%
% }


\title{Interpretable Machine Learning}
% \author{LMU}
%\institute{\href{https://compstat-lmu.github.io/lecture_iml/}{compstat-lmu.github.io/lecture\_iml}}
\date{}

\begin{document}

\newcommand{\titlefigure}{figure/Shapley_1.png}
\newcommand{\learninggoals}{%
\item Learn what game theory is
\item Understand the concept behind cooperative games
\item Understand the Shapley value in game theory
}

\lecturechapter{Shapley Values}
\lecture{Interpretable Machine Learning}

% License of titlefigure: free pixabay license
% https://pixabay.com/de/vectors/ergebnis-geld-gesch%C3%A4ft-gehalt-5567652/


% \begin{vbframe}{Game Theory}
% \begin{itemize}
% \itemsep1em
%   \item Game theory is the study of strategic games between players
%   \item Term \enquote{game} not restricted to actual games (e.g., chess or poker) but to any series of interactions between actors or agents with gains and losses of quantifiable utility value
%   \item Often used in social context where players correspond to people or organizations, e.g., warfare, provision of public goods, auctions and bargaining, formation of cartels, interrogation practices.
%   \item Example of prisoner's dilemma: Two prisoners A and B are interrogated in two separate rooms with no means of communication between each other.
%     \begin{itemize}
%         \item If they betray each other, each person serves 2 years in prison.
%         \item If one person remains silent and one betrays their partner, person staying silent receives 3 years in prison while betrayer is set free.
%         \item If both remain silent, both serve 1 year in prison.
%     \end{itemize}
%     Conditional on strategy of remaining partner, best personal outcome is achieved by betrayal.
%     \\
%     $\Rightarrow$ Best aggregate outcome not achieved (both staying silent). 
% \end{itemize}
% \end{vbframe}


\begin{vbframe}{Cooperative Games in Game Theory}
\begin{itemize}[<+->]
%\itemsep1em
  \item Game theory is the study of strategic games between players, \enquote{game} refers to any series of interactions between actors / agents with gains and losses of quantifiable utility value
  \item Cooperative games: For all possible players $P = \{1, \hdots, p\}$, each subset of players $\SsubP$ forms a coalition -- each coalition $S$ achieves a certain payout
  %\item A value function $v(S): 2^{|P|}\mapsto \R$ describes the payout (or gain) achieved by any coalition $S \subseteq P$
  \item A value function $v: 2^{|P|}\mapsto \R$ maps all $2^{|P|}$ possible coalitions to their payout (or gain)
  \item $v(S)$ is the payout of coalition $S \subseteq P$ (payout of empty coalition must be zero: $v(\emptyset) = 0$)
  %\\
  %\textit{Note:} The payout or gain not necessarily has a positive meaning. In the example on the prisoner's dilemma, the payout is the total number of years spend in prison.
  \item As some players contribute more than others, we want to fairly divide the total achievable payout $v(P)$ among the players according to a player's individual contribution
  \item We call the individual payout per player $\phi_j$, $j \in P$ (later: Shapley value)
  %\item What would be properties of a fair distribution of the payout?
\end{itemize}
\end{vbframe}

\begin{vbframe}{Cooperative Games without Interactions}

% https://docs.google.com/presentation/d/1-bK90Gv1vIDr61s1PfgC51Kvic2Avnb7v8PgoO9iN0I/edit?usp=sharing

\begin{figure}
    \centering
    \includegraphics{figure/Shapley_1.png}
\end{figure}

\end{vbframe}

\begin{vbframe}{Fair Payouts are Trivial Without Interactions}

\begin{figure}
    \centering
    \includegraphics{figure/Shapley_2.png}
\end{figure}

\end{vbframe}
\begin{vbframe}{Cooperative Games with Interactions}

\begin{figure}
    \centering
    \includegraphics{figure/Shapley_3.png}
\end{figure}

\end{vbframe}
\begin{vbframe}{What is a fair payout for player \enquote{yellow}?}

\begin{figure}
    \centering
    \includegraphics{figure/Shapley_4.png}
\end{figure}

\end{vbframe}

\begin{vbframe}{How to Weight Differences in Payout?}

\begin{itemize}
    \itemsep2em
    \item  Form a coalition one actor at a time -- each agent receives their contribution to the total payout, i.e., the increase in total payout when the agent joins the coalition
    \item The Shapley value is the agent's average contribution over all possible formations of coalitions $\leadsto$ order of how the agents joined the coalition matters
    \item Each contribution is weighted proportionally to the number of possible orders of its coalition -- the more agents in a coalition, the more possibilities of ordering the agents inside the coalition
    \item Set definition: $\phi_j = \sum_{\SsubPnoj} \frac{|S|!(|P| - |S| - 1)!}{|P|!}(v(\Scupj) - v(S))$
    \item Order definition: $\phi_j = \frac{1}{|P|!} \sum_{\tau \in \Pi} (v(Pre(\tau,j) \cup \{j\}) - v(Pre(\tau,j)))$
    \\
    with $\Pi$ being all possible orders of players and $Pre(\tau,j)$ being the set of players before player $j$ in order $\tau$ 
\end{itemize}

\end{vbframe}

% \begin{vbframe}{How to Weight Differences in Payout?}
%
% \begin{figure}
%     \centering
%     \includegraphics{figure/Shapley_5.png}
% \end{figure}
%
%\end{vbframe}

\frame{
\frametitle{How to Weight Differences in Payout?}
  \begin{center}
  \only<1>{\includegraphics{figure/Shapley_7.png}}%
  \only<2>{ \includegraphics{figure/Shapley_8.png}}%
  \only<3>{ \includegraphics{figure/Shapley_9.png}}%
  \end{center}
}


% \begin{vbframe}{From Game Theory To Machine Learning}

% \begin{figure}
%     \centering
%     \includegraphics{figure/Shapley_5.png}
% \end{figure}

% \end{vbframe}

% \begin{vbframe}{Shapley Values}

%   Shapley values provide a unique solution to the attribution problem while satisfying all axioms:
%     \vspace{0.25cm}
% \begin{itemize}
%   \itemsep1em
%   \item Shapley values were proposed by Lloyd Shapley in 1951.
%   \item The Shapley value assigns a value to each player according to the marginal contribution of each player in all possible coalitions.
%   \item $\phi_j = \sum_{\SsubPnoj} \frac{|S|!(|P| - |S| - 1)!}{|P|!}(v(\Scupj) - v(S))$
%   \item $v(\Scupj) - v(S)$ is the marginal contribution of player $j$ to coalition $S$.
%   \item To compute the Shapley payout for a player, we average, for all possible coalitions, how much the player would increase the value of the coalition (=marginal contribution).
%   \item Shapley values are the \textit{only} solution for the attribution with the specified axioms.
% \end{itemize}

% \footnote{Shapley, Lloyd S. (August 21, 1951). "Notes on the n-Person Game -- II: The Value of an n-Person Game" (PDF). Santa Monica, Calif.: RAND Corporation.}

% \end{vbframe}



\begin{vbframe}{Definition via orders}
The Shapley value was introduced as summation over sets $S$, but it's also possible to define it as a summation of all orders of players.
This also explains where the factor $\frac{|S|!(|P| - |S| - 1)!}{|P|!}$ comes from.
\begin{itemize}
  \item Let $\Pi$ be all possible orders of players ($|P|!$ in total)
  \item Then: $\phi_j = \frac{1}{|P|!} \sum_{\tau \in \Pi} (v(Pre(\tau,j) \cup \{j\}) - v(Pre(\tau,j)))$
  \item $Pre(\tau,j)$ is the set of players before player $j$ in order $\tau$ 
  \item For example players a,b,c: $\Pi = \{(a,b,c), (a,c,b), (b,a,c), (b,c,a), (c,a,b), (c,b,a)\}$. If $\tau = (b,a,c)$ and $j=c$, then $Pre(\tau,j) = \{b, a\}$
  \item For Shapley value computation via order definition, we sum the marginal contribution twice for orders that yield set $S = \{a,b\}$, which in the set definition has the weight $2! (3 - 2 - 1)! = 2 \cdot 0! = 2$
\end{itemize}

\end{vbframe}


\begin{vbframe}{Definition via orders}

\begin{itemize}
  \item The Shapley value definition via orders is equivalent to the definition via sets, since the number of orders which yield the same coalition $S$ is  $|S|!(|P| - |S| - 1)!$: There are $|S|!$ possible orders of players within coalition $S$ and $(|P| - |S| - 1)!$ possible orders of players without $S$ and $j$
  \item Relevance of the order definition: The Shapley value can be approximated by sampling permutations $\leadsto$ instead of producing all $|P|!$ permutations, a fixed number of $M$ can be sampled and averaged to approximate the Shapley values
\end{itemize}
\centering
  \begin{tabular}{|c|c|c|c|c|c|c|}
    \multicolumn{3}{c}{\enspace\raisebox{-3.3ex}[0pt][2.6ex]{$ \overbrace{\vphantom{-}\hspace{9em}}^{|S|! \text{ permutations}}$}} &
    \multicolumn{1}{c}{} &
    \multicolumn{3}{c}{\enspace\raisebox{-3.3ex}[0pt][2.6ex]{$ \overbrace{\vphantom{-}\hspace{9em}}^{(|P| - |S| - 1)! \text{ permutations}}$}}\\
    \hline
    $\tau(1)$ & \ldots & $\tau(|S|)$ & $\tau(|S| + 1)$ & $\tau(|S| + 2)$ & \ldots & $\tau(P)$ \\
    \hline
    \multicolumn{3}{c}{\enspace\raisebox{1.3ex}[0pt][2.6ex]{$ \underbrace{\vphantom{-}\hspace{9em}}^{}$}} &
    \multicolumn{1}{c}{\enspace\raisebox{1.3ex}[0pt][2.6ex]{$ \underbrace{\vphantom{-}\hspace{4em}}^{}$}} &
    \multicolumn{3}{c}{\enspace\raisebox{1.3ex}[0pt][2.6ex]{$ \underbrace{\vphantom{-}\hspace{9em}}^{}$}}\\
    \multicolumn{3}{c}{Players before player $j$} & \multicolumn{1}{c}{player $j$} & \multicolumn{3}{c}{Players after player $j$} \\
  \end{tabular}


\end{vbframe}


\begin{vbframe}{Shapley Values - Illustration}
\begin{itemize}
    \item The Shapley value of a player $j=2$ is the marginal contribution to the value function when player $2$ enters an abitrary coalition
\only<1>{\item Here, player $2$ enters the coalition after player $1$, resulting in a value change of $v(\{1,2\}) - v(\{1\}) = 24-12 = 12$ with a overall coalition value of $v(\{1,2,3\}) = 36$}
\only<2>{\item We produce all possible orders of player coalitions and measure the value change if player $2$ enters the coalition}
\end{itemize}

\begin{center}
  \only<1>{
    \includegraphics[page=1, width=0.6\textwidth]{figure_man/shapley_feature_effect}
  }
  \only<2>{
    \includegraphics[page=2, width=0.6\textwidth]{figure_man/shapley_feature_effect}
  }
\end{center}
\end{vbframe}


\begin{vbframe}{Axioms of Fair Payouts}
 Why is this a fair payout solution?
 \\
 One possibility to define fair payouts are the following axioms for a given value function $v$:
  \vspace{0.25cm}
  \begin{itemize}
  \itemsep1em
    \item \textbf{Efficiency}: Player contributions add up to the total payout of the game:
      $\sum\nolimits_{j=1}^p\phi_j = v(P)$
    \item \textbf{Symmetry}: Players $j,k \in P$ who contribute the same to any coalition get the same payout: \\
      If $v(\Scupj) = v(\Scupk)$ for all $\SsubP \setminus\{j,k\}$, then $\phi_j=\phi_k$
    \item \textbf{Dummy / Null Player}: The payout is zero for players who don't contribute to the value of any coalition: \\
      If $v(\Scupj)=v(S)\quad  \forall \quad \SsubP \setminus j$, then $\phi_j=0$
    \item \textbf{Additivity}: For a game $v$ with combined payouts $v(S) = v_1(S) + v_2(S)$, the payout is the sum of payouts: $\phi_{j,v} = \phi_{j,v_1} + \phi_{j, v_2}$
  \end{itemize}
  \vspace{0.5cm}
  

\end{vbframe}

% \begin{vbframe}{Proof of Axioms}
%   The Shapley values fulfills all the 4 axioms.
%   Symmetry, Dummy and Additivity are relatively easy to proof:
% \begin{itemize}
%     % See also: https://math.stackexchange.com/questions/2747088/shapley-value-is-efficient
%   \item \textbf{Symmetry}: Let's assume coalition $\SsubPnojk$ and $v(\Scupj) = v(\Scupk)$.
%     \begin{itemize}
%         \item Then all marginal contributions are equal: $v(\Scupj) -  v(S) = v(\Scupk) -  v(S), \forall \SsubPnojk$
%         \item Consequently, $\phi_j = \phi_k$.
%     \end{itemize}
%   \item \textbf{Dummy}: Let's assume we have player $j$ such that for all $\SsubP$, we have $v(S) = v(\Scupj)$. Then, each marginal contribution of player $j$ is zero, and therefore $\phi_j = 0$.
%   \item \textbf{Additivity}:  Assume two games $v_1$ and $v_2$ and a third game which is the sum of both $v(S) = v_1(S) + v_2(S)$. The marginal contribution for all $\SsubPnoj$ for game $v$ can be expressed as $v(\Scupj) - v(S) = v_1(\Scupj) - v_1(S) + v_2(\Scupj) - v_2(S)$. Since the Shapley value is additive in the marginal contributions, we can split the sum into two sums so that $\phi_{j,v} = \phi_{j, v_1} + \phi_{j,v_2}$.
% \end{itemize}
% Efficiency requires a bit more effort, see proof sketch:
%   \begin{itemize}
%   \item \textbf{Efficiency}: $v(P)$ exactly appears once per player ($=p$ times) for coalition $S = P \setminus \{j\}$ with the weight $\frac{|P - 1|!(|P| - |P - 1| - 1)!}{|P|!} = \frac{1}{p}$ each. The values for all other coalitions $v(S), \SsubP \{j,k\}$ appear with both minus and plus signs that cancel each other out.
% \end{itemize}
% \end{vbframe}


% \begin{vbframe}{Linearity Axiom Corollary}
%   \begin{itemize}
%   \item The Shapley values are also linear:
%   \item For a game with payout $v(S) = \alpha v_1(S) + v_2(S)$, the Shapley values are $\alpha \phi_{j,v_1} + \phi_{j,v_2}$.
%   \item The multiplication with $\alpha$ works since we can pull out $\alpha$ from the sum of marginal contributions, so that for $v(S) = \alpha v_1(S)$ the Shapley value is $\alpha \phi_{j,v_1}$.
%   \item We already know that Shapley values are additive which proofs the linearity.
%   \end{itemize}
% \end{vbframe}



% \begin{vbframe}{Applications of Shapley Value}

%   \begin{itemize}
%       \item Game theory
%       \item Economics (e.g., cost allocation)
%       \item Marketing (e.g., social network analysis to discover influencers)
%       \item ...
%       \item Machine learning
%       \begin{itemize}
%          \item Feature selection: Attribute loss reduction to features.
%          \item Quantify data value: Attribute loss reduction to data points.
%          \item \textbf{Explain individual predictions}.
%       \end{itemize}
%   \end{itemize}

%   % Economics
%   \tiny{Moulin, Hervé. "An application of the Shapley value to fair division with money." Econometrica: Journal of the Econometric Society (1992): 1331-1349.}
% % Network analysis
%   \tiny{Narayanam, Ramasuri, and Yadati Narahari. "A shapley value-based approach to discover influential nodes in social networks." IEEE Transactions on Automation Science and Engineering 8.1 (2010): 130-147.}
%   % Feature Selection
%   \tiny{Cohen, Shay B., Eytan Ruppin, and Gideon Dror. "Feature Selection Based on the Shapley Value." IJCAI. Vol. 5. 2005.}
%   % Value of data
%   \tiny{Ghorbani, Amirata, and James Zou. "Data shapley: Equitable valuation of data for machine learning." International Conference on Machine Learning. PMLR, 2019.}
% \end{vbframe}

\endlecture
\end{document}
