\documentclass[11pt,compress,t,notes=noshow, xcolor=table]{beamer}
\usepackage[]{graphicx}\usepackage[]{color}
% maxwidth is the original width if it is less than linewidth
% otherwise use linewidth (to make sure the graphics do not exceed the margin)
\makeatletter
\def\maxwidth{ %
	\ifdim\Gin@nat@width>\linewidth
	\linewidth
	\else
	\Gin@nat@width
	\fi
}
\makeatother

\definecolor{fgcolor}{rgb}{0.345, 0.345, 0.345}
\newcommand{\hlnum}[1]{\textcolor[rgb]{0.686,0.059,0.569}{#1}}%
\newcommand{\hlstr}[1]{\textcolor[rgb]{0.192,0.494,0.8}{#1}}%
\newcommand{\hlcom}[1]{\textcolor[rgb]{0.678,0.584,0.686}{\textit{#1}}}%
\newcommand{\hlopt}[1]{\textcolor[rgb]{0,0,0}{#1}}%
\newcommand{\hlstd}[1]{\textcolor[rgb]{0.345,0.345,0.345}{#1}}%
\newcommand{\hlkwa}[1]{\textcolor[rgb]{0.161,0.373,0.58}{\textbf{#1}}}%
\newcommand{\hlkwb}[1]{\textcolor[rgb]{0.69,0.353,0.396}{#1}}%
\newcommand{\hlkwc}[1]{\textcolor[rgb]{0.333,0.667,0.333}{#1}}%
\newcommand{\hlkwd}[1]{\textcolor[rgb]{0.737,0.353,0.396}{\textbf{#1}}}%
\let\hlipl\hlkwb

\usepackage{framed}
\makeatletter
\newenvironment{kframe}{%
	\def\at@end@of@kframe{}%
	\ifinner\ifhmode%
	\def\at@end@of@kframe{\end{minipage}}%
\begin{minipage}{\columnwidth}%
	\fi\fi%
	\def\FrameCommand##1{\hskip\@totalleftmargin \hskip-\fboxsep
		\colorbox{shadecolor}{##1}\hskip-\fboxsep
		% There is no \\@totalrightmargin, so:
		\hskip-\linewidth \hskip-\@totalleftmargin \hskip\columnwidth}%
	\MakeFramed {\advance\hsize-\width
		\@totalleftmargin\z@ \linewidth\hsize
		\@setminipage}}%
{\par\unskip\endMakeFramed%
	\at@end@of@kframe}
\makeatother

\definecolor{shadecolor}{rgb}{.97, .97, .97}
\definecolor{messagecolor}{rgb}{0, 0, 0}
\definecolor{warningcolor}{rgb}{1, 0, 1}
\definecolor{errorcolor}{rgb}{1, 0, 0}
\newenvironment{knitrout}{}{} % an empty environment to be redefined in TeX

\usepackage{alltt}
\newcommand{\SweaveOpts}[1]{}  % do not interfere with LaTeX
\newcommand{\SweaveInput}[1]{} % because they are not real TeX commands
\newcommand{\Sexpr}[1]{}       % will only be parsed by R

\usepackage[english]{babel}
\usepackage[utf8]{inputenc}

\usepackage{dsfont}
\usepackage{verbatim}
\usepackage{amsmath}
\usepackage{amsfonts}
\usepackage{bm}
\usepackage{csquotes}
\usepackage{multirow}
\usepackage{longtable}
\usepackage{booktabs}
\usepackage{enumerate}
\usepackage[absolute,overlay]{textpos}
\usepackage{psfrag}
\usepackage{algorithm}
\usepackage{algpseudocode}
\usepackage{eqnarray}
\usepackage{arydshln}
\usepackage{tabularx}
\usepackage{placeins}
\usepackage{tikz}
\usepackage{setspace}
\usepackage{colortbl}
\usepackage{mathtools}
\usepackage{wrapfig}
\usepackage{bm}
\usepackage[backend=biber]{biblatex}

\usetikzlibrary{shapes,arrows,automata,positioning,calc,chains,trees, shadows}
\tikzset{
	%Define standard arrow tip
	>=stealth',
	%Define style for boxes
	punkt/.style={
		rectangle,
		rounded corners,
		draw=black, very thick,
		text width=6.5em,
		minimum height=2em,
		text centered},
	% Define arrow style
	pil/.style={
		->,
		thick,
		shorten <=2pt,
		shorten >=2pt,}
}

\usepackage{subfig}

% Defines macros and environments
\usepackage{bbm}
% basic latex stuff
\newcommand{\pkg}[1]{{\fontseries{b}\selectfont #1}} %fontstyle for R packages
\newcommand{\lz}{\vspace{0.5cm}} %vertical space
\newcommand{\dlz}{\vspace{1cm}} %double vertical space
\newcommand{\oneliner}[1] % Oneliner for important statements
{\begin{block}{}\begin{center}\begin{Large}#1\end{Large}\end{center}\end{block}}


%new environments
\newenvironment{vbframe}  %frame with breaks and verbatim
{
 \begin{frame}[containsverbatim,allowframebreaks]
}
{
\end{frame}
}

\newenvironment{vframe}  %frame with verbatim without breaks (to avoid numbering one slided frames)
{
 \begin{frame}[containsverbatim]
}
{
\end{frame}
}

\newenvironment{blocki}[1]   % itemize block
{
 \begin{block}{#1}\begin{itemize}
}
{
\end{itemize}\end{block}
}

\newenvironment{fragileframe}[2]{  %fragile frame with framebreaks
\begin{frame}[allowframebreaks, fragile, environment = fragileframe]
\frametitle{#1}
#2}
{\end{frame}}


\newcommand{\myframe}[2]{  %short for frame with framebreaks
\begin{frame}[allowframebreaks]
\frametitle{#1}
#2
\end{frame}}

\newcommand{\remark}[1]{
  \textbf{Remark:} #1
}


\newenvironment{deleteframe}
{
\begingroup
\usebackgroundtemplate{\includegraphics[width=\paperwidth,height=\paperheight]{../style/color/red.png}}
 \begin{frame}
}
{
\end{frame}
\endgroup
}
\newenvironment{simplifyframe}
{
\begingroup
\usebackgroundtemplate{\includegraphics[width=\paperwidth,height=\paperheight]{../style/color/yellow.png}}
 \begin{frame}
}
{
\end{frame}
\endgroup
}\newenvironment{draftframe}
{
\begingroup
\usebackgroundtemplate{\includegraphics[width=\paperwidth,height=\paperheight]{../style/color/green.jpg}}
 \begin{frame}
}
{
\end{frame}
\endgroup
}
% https://tex.stackexchange.com/a/261480: textcolor that works in mathmode
\makeatletter
\renewcommand*{\@textcolor}[3]{%
  \protect\leavevmode
  \begingroup
    \color#1{#2}#3%
  \endgroup
}
\makeatother


\providecommand{\tightlist}{%
  \setlength{\itemsep}{0pt}\setlength{\parskip}{0pt}}

%\setbeamerfont{footnote}{size=\tiny}
\usepackage[hang,flushmargin]{footmisc}
\renewcommand*{\footnotelayout}{\tiny}
\renewcommand*{\thefootnote}{} %\fnsymbol{footnote}

% https://tex.stackexchange.com/questions/30720/footnote-without-a-marker
% \makeatletter
% \def\blfootnote{\gdef\@thefnmark{}\@footnotetext}
% \makeatother

% https://tex.stackexchange.com/questions/357717/beamer-allowframebreaks-option-and-vertical-spacing-when-using-lists-itemize
% \setbeamertemplate{frametitle continuation}{%
%     (\insertcontinuationcount)%
%     \ifnum\insertcontinuationcount>1%
%     \vspace*{\topsep}%
%     \else%
%     %
%     \fi%
% }


%\usetheme{lmu-lecture}
\newcommand{\titlefigure}{figure_man/feature-importance.png}
\newcommand{\learninggoals}{
	\item Extrapolation and Conditional Sampling
	\item Conditional Feature Importance (CFI)
	\item Interpretation of CFI}
\usepackage{../../style/lmu-lecture}

\let\code=\texttt
\let\proglang=\textsf

\setkeys{Gin}{width=0.9\textwidth}

\title{Feature Importance}
% \author{Bernd Bischl, Christoph Molnar, Daniel Schalk, Fabian Scheipl}
\institute{\href{https://compstat-lmu.github.io/lecture_i2ml/}{compstat-lmu.github.io/lecture\_i2ml}}
\date{}

\setbeamertemplate{frametitle}{\expandafter\uppercase\expandafter\insertframetitle}

\bibliography{feature-importance}

%\usepackage{Sweave}
\begin{document}
	
	% Set style/preamble.Rnw as parent.
	
	% Load all R packages and set up knitr
	
	% This file loads R packages, configures knitr options and sets preamble.Rnw as 
	% parent file
	% IF YOU MODIFY THIS, PLZ ALSO MODIFY setup.Rmd ACCORDINGLY...
	
	% Defines macros and environments
	
	\input{../../latex-math/basic-math.tex}
	\input{../../latex-math/basic-ml.tex}
	\input{../../latex-math/ml-interpretable.tex}
	
	\lecturechapter{Conditional Feature Importance}
	\lecture{Interpretable Machine Learning}
	
	% ------------------------------------------------------------------------------



% CFI IDEA
\begin{vbframe}{Conditional Feature Importance Idea}

\textbf{Permutation Feature Importance Idea:} Replace the feature of interest $x_j$ with an independent sample from the marginal distribution $\P(x_j)$, e.g. by randomly permuting observations in $x_j$.\\
\lz
\textbf{Problem:} Under dependent features, permutation leads to extrapolation.\\
\lz
\textbf{Conditional Feature Importance Idea:} Resample $x_j$ from the conditional distribution $\P(x_j|x_{-j})$, such that the covariate joint distribution is preserved. (I.e., $\P(x_j|x_{-j}) \P(x_{-j}) = \P(x_j, x_{-j})$).

\end{vbframe}



% RECAP EXTRAPOLATION
\begin{vbframe}{Recap: Extrapolation in PFI}
 
 \textbf{Example:} Let $y = x_3 + \epsilon_y$ with $\epsilon_Y \sim N(0, 0.1)$ where $x_1 :=  \epsilon_1$, $x_2 := x_1 + \epsilon_2$ are highly correlated ($\epsilon_1 \sim N(0,1), \epsilon_2 \sim N(0, 0.01)$) and $x_3 := \epsilon_3$, $x_4 := \epsilon_4$,  with $\epsilon_3, \epsilon_4 \sim N(0,1)$. All noise terms are independent.
A \texttt{lm} is fit, yielding $\fh(x) \approx 0.3 x_1 - 0.3 x_2 + x_3$.\\
 %
\begin{figure}
\hfill
  \includegraphics[width=0.25\linewidth]{figure_man/pfi_hexbin_pre.pdf}\hfill
  \includegraphics[width=0.25\linewidth]{figure_man/pfi_hexbin_post.pdf} \hfill
  \includegraphics[width=0.4\linewidth]{figure_man/pfi_extrapolation.pdf} \hfill
  \caption{Density plot for $x_1, x_2$ before permuting $x_1$ (left) and after permuting $x_1$ (center). Right: PFI including $.05$ to $.95$ confidence band.}
\end{figure}
% 
$\Rightarrow$ $x_1$ and $x_2$ are irrelevant for the prediction for $x: \P(x) > 0$ \\
$\Rightarrow$ Since PFI evaluates the model on unrealistic observations, the features $x_1$ and $x_2$ are nevertheless considered relevant

 \end{vbframe}


 
 % CONDITIONAL SAMPLING PRESERVES THE JOINT
 \begin{vbframe}{Conditional Sampling preserves the Joint}
 
 Let $\pert{x}{S}{-S}$ be the feature vector where features $x_S$ were replaced with an independent sample from $\P(x_S|x_{-S})$. Then
%
\begin{align*}
  \P(\pert{x}{S}{-S}) = \P(\pert{x}{S}{-S}_S, x_{-S}) &= \P(\pert{x}{S}{-S}_S|x_{-S}) P(x_{-S})\\
   &= \P(x_S|x_{-S}) \P(x_{-S}) = \P(x)
\end{align*}
 %
 meaning that the joint distribution is preserved in the perturbed dataset.\\
 $\Rightarrow$ CFI only evaluates the model within the observational covariate distribution\\
 \lz
 \textbf{Note:} That does not imply $\P(x,y) = \P(\pert{x}{S}{-S}, y)$! In the perturbed data, the relationship between $x_j$ and $y$ may still be destroyed.\\
 \end{vbframe}



% CFI DEFINITION
\begin{vbframe}{Conditional Feature Importance}
\normalsize

Conditional feature importance (CFI) for features $x_S$ using test data $\D$:
\begin{itemize}
  \item Measure the error \color{red}\textbf{with perturbed feature values} \color{black} $\pert{x}{S}{-S}$, where $\pert{x}{S}{-S}_S \sim \P(x_S|x_{-S})$
  \item Measure the error \color{blue}\textbf{with unperturbed features}\color{black}.
  \item Repeat permuting the feature (e.g., $m$ times) and average the difference of both errors: 
$$\widehat{PFI}_S = \tfrac{1}{m} \textstyle\sum\nolimits_{k = 1}^{m} \riske (\fh, {\color{red}\pert{\D}{S}{-S}_{(k)}}) - \riske (\fh, {\color{blue}\D})$$
\end{itemize}

Here, $\pert{\D}{S}{-S}$ denotes the dataset where features $x_S$ where sampled conditional on the remaining features $x_{-S}$.

\footnote[frame]{\fullcite{Strobl2008}}

\end{vbframe}



% INTERPRETATION OF CFI
\begin{vbframe}{Implications of CFI}

\textbf{Interpretation:} CFI quantifies the unique contribution of the feature to the performance of the model.\\
\lz
\textbf{Entanglement with the data:} If the feature does not contribute unique information about $y$, i.e. $x_S \ind y | x_{-S}$, CFI is zero.\\
\begin{itemize}
  \item Why? Under the conditional independence $\P(\pert{x}{S}{-S}, y) = \P(x,y)$ and therefore no prediction-relevant information is destroyed.
\end{itemize}
\lz
\textbf{Entanglement with the model:} If the model does not rely of a feature, CFI is zero.\\
\begin{itemize}
  \item Why? Then the prediction is not affected by any perturbation of the feature (and consequently the performance is invariant).
\end{itemize}

\footnote[frame]{\fullcite{konig_relative_2021}}
\end{vbframe}


% IMPLICATIONS OF CFI
\begin{vbframe}{Implications of CFI}

Can we gain insight into whether..

\begin{enumerate}
    \item the feature $x_j$ is causal for the prediction?
    \begin{itemize}
      \item Nonzero CFI implies that the model relies on $x_j$.
      \item The contraposition does not hold.
    \end{itemize}
    \item the variable $x_j$ contains prediction-relevant information?
    \begin{itemize}
      \item If $x_S \not \ind y$ but $x_S \ind y | x_{-S}$ CFI is zero.
      \item If a feature is not exploited by the model, CFI is zero, irrespective of whether the feature is useful or not.
    \end{itemize}
    \item Does the model require access to $x_j$ to achieve it's prediction performance?
\begin{itemize}
      \item Nonzero CFI implies that the feature contributes unique information (meaning $x_S \not \ind y | x_{-S}$).
      \item Only uncovers the relationships that were exploited by the model.
    \end{itemize}
\end{enumerate}
\end{vbframe}

\begin{vbframe}
  \printbibliography
\end{vbframe}

\endlecture
\end{document}
