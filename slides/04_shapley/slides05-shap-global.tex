\documentclass[11pt,compress,t,notes=noshow, aspectratio=169, xcolor=table]{beamer}
\newcommand{\btVFill}{\vskip0pt plus 1filll}
\newcommand\hmmax{0}
\newcommand\bmmax{0}
\usepackage{../../style/lmu-lecture}
\usepackage{siunitx}
% Defines macros and environments
\usepackage{bbm}
% basic latex stuff
\newcommand{\pkg}[1]{{\fontseries{b}\selectfont #1}} %fontstyle for R packages
\newcommand{\lz}{\vspace{0.5cm}} %vertical space
\newcommand{\dlz}{\vspace{1cm}} %double vertical space
\newcommand{\oneliner}[1] % Oneliner for important statements
{\begin{block}{}\begin{center}\begin{Large}#1\end{Large}\end{center}\end{block}}


%new environments
\newenvironment{vbframe}  %frame with breaks and verbatim
{
 \begin{frame}[containsverbatim,allowframebreaks]
}
{
\end{frame}
}

\newenvironment{vframe}  %frame with verbatim without breaks (to avoid numbering one slided frames)
{
 \begin{frame}[containsverbatim]
}
{
\end{frame}
}

\newenvironment{blocki}[1]   % itemize block
{
 \begin{block}{#1}\begin{itemize}
}
{
\end{itemize}\end{block}
}

\newenvironment{fragileframe}[2]{  %fragile frame with framebreaks
\begin{frame}[allowframebreaks, fragile, environment = fragileframe]
\frametitle{#1}
#2}
{\end{frame}}


\newcommand{\myframe}[2]{  %short for frame with framebreaks
\begin{frame}[allowframebreaks]
\frametitle{#1}
#2
\end{frame}}

\newcommand{\remark}[1]{
  \textbf{Remark:} #1
}


\newenvironment{deleteframe}
{
\begingroup
\usebackgroundtemplate{\includegraphics[width=\paperwidth,height=\paperheight]{../style/color/red.png}}
 \begin{frame}
}
{
\end{frame}
\endgroup
}
\newenvironment{simplifyframe}
{
\begingroup
\usebackgroundtemplate{\includegraphics[width=\paperwidth,height=\paperheight]{../style/color/yellow.png}}
 \begin{frame}
}
{
\end{frame}
\endgroup
}\newenvironment{draftframe}
{
\begingroup
\usebackgroundtemplate{\includegraphics[width=\paperwidth,height=\paperheight]{../style/color/green.jpg}}
 \begin{frame}
}
{
\end{frame}
\endgroup
}
% https://tex.stackexchange.com/a/261480: textcolor that works in mathmode
\makeatletter
\renewcommand*{\@textcolor}[3]{%
  \protect\leavevmode
  \begingroup
    \color#1{#2}#3%
  \endgroup
}
\makeatother


\providecommand{\tightlist}{%
  \setlength{\itemsep}{0pt}\setlength{\parskip}{0pt}}

%\setbeamerfont{footnote}{size=\tiny}
\usepackage[hang,flushmargin]{footmisc}
\renewcommand*{\footnotelayout}{\tiny}
\renewcommand*{\thefootnote}{} %\fnsymbol{footnote}

% https://tex.stackexchange.com/questions/30720/footnote-without-a-marker
% \makeatletter
% \def\blfootnote{\gdef\@thefnmark{}\@footnotetext}
% \makeatother

% https://tex.stackexchange.com/questions/357717/beamer-allowframebreaks-option-and-vertical-spacing-when-using-lists-itemize
% \setbeamertemplate{frametitle continuation}{%
%     (\insertcontinuationcount)%
%     \ifnum\insertcontinuationcount>1%
%     \vspace*{\topsep}%
%     \else%
%     %
%     \fi%
% }


\title{Interpretable Machine Learning}
% \author{LMU}
%\institute{\href{https://compstat-lmu.github.io/lecture_iml/}{compstat-lmu.github.io/lecture\_iml}}
\date{}

\begin{document}

% TODO
\newcommand{\titlefigure}{figure_man/global_shap_depend_season.pdf}
\newcommand{\learninggoals}{
\item Get an intuition of additive feature attributions
\item Understand the concept of Kernel SHAP
\item Ability to interpret SHAP plots
\item Global SHAP methods
}

\lecturechapter{Global SHAP}
\lecture{Interpretable Machine Learning}

\begin{frame}{Global SHAP \citebutton{Lundberg et al. 2018}{https://doi.org/10.48550/arXiv.1802.03888}}
\textbf{Idea: }
\begin{itemize}
    \item Run SHAP for every observation and thereby get a matrix of Shapley values
    \item The matrix has one row per data observation and one column per feature
    \item We can interpret the model globally by analyzing the Shapley values in this matrix
\end{itemize}
\vspace{2cm}
$$
\Phi =
\begin{bmatrix}
    \phi_{11} & \phi_{12} & \phi_{13} & \dots  & \phi_{1p} \\
    \phi_{21} & \phi_{22} & \phi_{23} & \dots  & \phi_{2p} \\
    \vdots & \vdots & \vdots & \ddots & \vdots \\
    \phi_{n1} & \phi_{n2} & \phi_{n3} & \dots  & \phi_{np} \\
\end{bmatrix}
$$

 \end{frame}

 \begin{frame}{Feature Importance}

\begin{onlyenv}<1>
\textbf{Idea:} Average the absolute Shapley values of each feature over all observations. This corresponds to calculating averages column by column in $\Phi$
$$
I_{j}=\frac{1}{n} \sum_{i=1}^{n}\left|\phi_{j}^{(i)}\right|
$$
\end{onlyenv}

\begin{onlyenv}<2>
\textbf{Interpretation:}
\begin{itemize}
    \item The features \enquote{temperature} and \enquote{year} have by far the highest influence on the model's prediction
    \item Compared to Shapley values, no effect direction is provided, but instead a feature ranking similar to PFI
    \item However, Shapley FI is based only on the model's predictions while PFI is based on the model's performance (loss)
\end{itemize}
 
\end{onlyenv}


\btVFill

\begin{figure}
    \centering
    \includegraphics[width=0.5\columnwidth]{figure_man/global_shap_fi.pdf}
\end{figure}

\end{frame}
 
\begin{frame}{Summary Plot}
\begin{onlyenv}<1>
Combines feature importance with feature effects
\begin{itemize}
    \item Each point is a Shapley value for a feature and an observation
    \item The color represents the value of the feature from low to high
    \item Overlapping points are jittered in y-axis direction
\end{itemize}
\end{onlyenv}

\begin{onlyenv}<2>
\textbf{Interpretation:}\\
\begin{itemize}
    \item Low temperatures have a negative impact while high temperatures lead to more bike rentals
    \item Year: two point clouds for 2011 and 2012 (other categorical features are gray)
    \item A high humidity has a huge, negative impact on the bike rental, while low humidity has a rather minor positive impact on bike rentals
\end{itemize}
 
\end{onlyenv}

\btVFill

\begin{figure}
    \centering
    \includegraphics[width=0.5\columnwidth]{figure_man/global_shap_jitter.pdf}
    
\end{figure}
\end{frame} 

\begin{frame}{Dependence Plot}

\begin{onlyenv}<1>
\begin{itemize}
    \item Visualize the marginal contribution of a feature similar to the PDP 
    \item Plot a point with the feature value on the x-axis and the corresponding Shapley value on the y-axis
\end{itemize}

\end{onlyenv}

\begin{onlyenv}<2>
\textbf{Interpretation:}\\
\begin{itemize}
    \item Increasing temperatures induce increasing bike rentals until $25^\circ\text{C}$
    \item If it gets too hot, the bike rentals decrease
\end{itemize}
\end{onlyenv}

\begin{onlyenv}<3>
\textbf{Interpretation:}\\
\begin{itemize}
    \item We can colour the observations by a second feature to detect interactions
    \item Visibly the temperatures interaction with the season is very strong
    \item In summer, when temperatures rise, we see an increase in the model's SHAP values, indicating more bike rentals. By contrast, winter's lower temperatures translate into negative SHAP contributions, suggesting fewer rentals.
\end{itemize}
\end{onlyenv}


\btVFill
\begin{onlyenv}<1-2>
\begin{figure}
    \centering
    \includegraphics[width=0.5\columnwidth]{figure_man/global_shap_depend.pdf}
\end{figure}
\end{onlyenv}

\begin{onlyenv}<3>
\begin{figure}
    \centering
    \includegraphics[width=0.5\columnwidth]{figure_man/global_shap_depend_season.pdf}
\end{figure}
\end{onlyenv}
\end{frame}

\begin{frame}{Discussion}

\textbf{Advantages}

\begin{itemize}
    \item All the advantages of Shapley values
    \item Unify the field of interpretable machine learning in the class of additive feature attribution methods
    \item Has a fast implementation for tree-based models
    \item Various global interpretation methods
    \item Can be used for images \citebutton{SHAP image examples}{https://github.com/shap/shap/tree/master/notebooks/image_examples}  and text \citebutton{SHAP text examples}{https://github.com/shap/shap/tree/master/notebooks/text_examples}
\end{itemize}

\medskip

\textbf{Disadvantages}

\begin{itemize}
\item Disadvantages of Shapley values also apply to SHAP
    % \begin{itemize}
    %     \item Shapley values can be misinterpreted and access to data is needed to compute them for new data (not if TreeSHAP is used)
    % \end{itemize}
    \item KernelSHAP is slow (TreeSHAP can be used as a faster alternative for tree-based models \citebutton{Lundberg et al 2018}{https://doi.org/10.48550/arXiv.1802.03888} -- and for an intuitive explanation  \citebutton{see Sukumar: TreeSHAP}{https://medium.com/analytics-vidhya/shap-part-3-tree-shap-3af9bcd7cd9b})
    \item KernelSHAP ignores feature dependence
    %\item TreeSHAP can produce unintuitive feature attributions
    %\item It is possible to create intentionally misleading interpretations with SHAP, which can hide biases \citebutton{Slack et al. 2020}{https://doi.org/10.1145/3375627.3375830}
\end{itemize}


\end{frame}

\endlecture
\end{document}
