\documentclass[11pt,compress,t,notes=noshow, aspectratio=169, xcolor=table]{beamer}

\usepackage{../../style/lmu-lecture}
% Defines macros and environments
% This file is included in slides and exercises

% Rarely used fontstyle for R packages, used only in 
% - forests/slides-forests-benchmark.tex
% - exercises/single-exercises/methods_l_1.Rnw
% - slides/cart/attic/slides_extra_trees.Rnw
\newcommand{\pkg}[1]{{\fontseries{b}\selectfont #1}}

% Spacing helpers, used often (mostly in exercises for \dlz)
\newcommand{\lz}{\vspace{0.5cm}} % vertical space (used often in slides)
\newcommand{\dlz}{\vspace{1cm}}  % double vertical space (used often in exercises, never in slides)
\newcommand{\oneliner}[1] % Oneliner for important statements, used e.g. in iml, algods
{\begin{block}{}\begin{center}\begin{Large}#1\end{Large}\end{center}\end{block}}

% Don't know if this is used or needed, remove?
% textcolor that works in mathmode
% https://tex.stackexchange.com/a/261480
% Used e.g. in forests/slides-forests-bagging.tex
% [...] \textcolor{blue}{\tfrac{1}{M}\sum^M_{m} [...]
% \makeatletter
% \renewcommand*{\@textcolor}[3]{%
%   \protect\leavevmode
%   \begingroup
%     \color#1{#2}#3%
%   \endgroup
% }
% \makeatother

\newcommand{\open}{}
\newcommand{\close}{}

\title{Interpretable Machine Learning}
% \author{LMU}
%\institute{\href{https://compstat-lmu.github.io/lecture_iml/}{compstat-lmu.github.io/lecture\_iml}}
\date{}

\begin{document}

\newcommand{\titlefigure}{figure/open_blackbox}
\newcommand{\learninggoals}{
\item Understand one classical kind of functional decomposition: functional ANOVA (fANOVA).
\item [...]
\item Algorithm for calculating the components in a fANOVA}

\lecturechapter{Functional ANOVA}
\lecture{Interpretable Machine Learning}

\begin{frame}{??? History of fANOVA ?????}
    
\end{frame}

\begin{frame}{Frame Title}

    Assumption: Independent features \\
    Today, we look at one method to obtain a functional decomposition; Funcational ANOVA (fANOVA).
    
\end{frame}

\begin{frame}{Idea}

    Consider example from before:

    \begin{equation}
        \fh(x_1, x_2) = 4 - 2x_1 + 0.3 e^{x_2} + |x_1|x_2
    \end{equation}

    Goal: Interaction term \(g_{1,2}\) should only contain the pure interaction, in this case only \(|x_1|x_2\), all the rest are actually main effects. \\
    General idea: Try to move as much as possible into lower-order effects, and only keep the rest in the higher-order term. \\
    \(\Rightarrow\) First compute the lower-order terms, then the higher order terms.
    
\end{frame}

\begin{frame}{Idea}

    \begin{equation}
        \fh(x_1, x_2) = 4 - 2x_1 + 0.3 e^{x_2} + |x_1|x_2
    \end{equation}

    Question: How to compute lower-order terms? \\
    Starting with the main effects: % One can use one of the global feature effect plots, since they exactly provide a function only depending on one feature, which models the effect of this feature on \(\fh\).
    \textit{Link to PDP??} \\
    \(\Rightarrow\) fANOVA uses the same idea as PDPs, but also for higher-order terms: the resp. feature is fixed and all other features are simply averaged out. \\

    Total formula for calculating the components \(g_S\) in the fANOVA algorithm:

    \begin{equation*}
        g_{S} (\xv_S) = (\text{average out all features not contained in } S) - (\text{All lower-order components})
    \end{equation*}
    
\end{frame}

\begin{frame}{Formal Definition and Computation
\citebutton{Hooker (2004)}{https://dl.acm.org/doi/10.1145/1014052.1014122}
}

Components are recursively defined using marginals (here $-S = \{1, \ldots, p \} \setminus S$ denotes the set of all indizes not contained in \(S\)):

$$g_{S}(\xv_S) = \mathbb{E}_{X_{-S}}\left[\fh(\xv) \; \vert  \; x_S \right] - \sum_{V \subset S} g_V(x_V)$$

\begin{itemize}
    \item Expectation integrates $\fh(\xv)$ over all input features except $\xv_S$
    \item Subtract all components $g_V$ with $V \subset S$ to remove all lower-order effects and center the effect
\pause
\item Recursive computation:
\begin{align*}
 g_{\open \emptyset \close} &= \mathbb{E}_X\left[\fh(\xv)\right] \\
 g_{\open j \close}(x_j) &= \mathbb{E}_{X_{-j}}\left[\fh(\xv) \; \vert  \; x_j \right] - g_{\open \emptyset \close}, \; \forall j \in \{1, \ldots, p\} \\% \text{ and } g_{\open 2 \close}(x_2) = \mathbb{E}_{X_{-2}}\left[\fh(\xv) \; \vert  \; x_2 \right] - g_{\open \emptyset \close}  \\
 %g_{\open 2 \close}(x_2) &= \mathbb{E}_{X_{-2}}\left[\fh(\xv) \; \vert  \; x_2 \right] - g_{\open \emptyset \close} \\
 g_{\open j, k \close}(x_j, x_k) &= \mathbb{E}_{X_{-\{ j,k \}}}\left[\fh(\xv) \; \vert \; x_j, x_k \right] - g_{\open k \close}(x_k) - g_{\open j \close}(x_j) - g_{\open \emptyset \close}, \; \forall j < k\\%,  \text{ etc.}\\
 &\vdots \\
 g_{\open 1, \dots, p \close}(\xv) &= \fh(\xv) -
 %\textstyle\sum_{S \subsetneqq \{1,\ldots,p\}} g_{S}(\xv_S) 
 g_{\open 1, \dots, p-1 \close}(x_{1}, \dots x_{p-1}) - \dots - g_{\open 1, 2 \close}(x_1, x_2)
 %&\phantom{{}={}} 
 - g_{\open p \close}(x_p)  - \dots - g_{\open 2 \close}(x_2) - g_{\open 1 \close}(x_1) - g_{\open \emptyset \close}\\
 &= \fh(\xv) -
 \textstyle\sum_{S \subsetneqq \{1,\ldots,p\}} g_{S}(\xv_S) 
\end{align*}

\end{itemize}
        
\end{frame}

\begin{frame}{Example}

    [calculate a full complete example] \\
    
    [problem with examples from beginning: If using fANOVA, we do not get the original components out of it again, right?]
    
\end{frame}

\begin{frame}{How to compute fANOVA in practice $\rightarrow$ Estimating Expectations}

    Estimating expectations over uniform distribution using e.g. Monte-Carlo sampling
    
\end{frame}

\begin{frame}{Example: More realistic calculation ?}
    
\end{frame}

\begin{frame}{Variance decomposition - Why is this method even called functional ANOVA?}

\begin{itemize}[<+->]
\item Decomposition of $\hat{f}(\xv)$ allows to conduct functional analysis of variance (fANOVA)
%$$ Var\left[\hat{f}(\xv)\right] = Var\left[g_{\open \emptyset \close} + g_{\open 1 \close}(x_1) + \;\dots\; + g_{\open 1, 2 \close}(x_1, x_2) + \;\dots\; + g_{\open 1,\ldots,p \close}(\xv) \right]$$
% \begin{align*}
% Var\left[\hat{f}(\xv)\right] &= Var\left[g_{\open \emptyset \close} + g_{\open 1 \close}(x_1) + g_{\open 2 \close}(x_2) + \;\dots\; + g_{\open 1, 2 \close}(x_1, x_2) \right. \\
% &\phantom{{}={}} \left. + \;\dots\; + g_{\open 1,\ldots,p \close}(\xv) \right]
% \end{align*}
\item If features are independent, one can prove that with the components defined in the fANOVA, the variance can be additively decomposed without covariances:
 \begin{align*}Var\left[\hat{f}(\xv)\right] &=  Var\left[g_{\open \emptyset \close} + g_{\open 1 \close}(x_1) + \;\dots\; + g_{\open 1, 2 \close}(x_1, x_2) + \;\dots\; + g_{\open 1,\ldots,p \close}(\xv) \right] \\
&= Var\left[g_{\open \emptyset \close}\right] + Var\left[g_{\open 1 \close}(x_1)\right] + \;\dots\; + Var\left[g_{\open 1, 2 \close}(x_1, x_2)\right] + \;\dots\; + Var\left[g_{\open 1,\ldots,p \close}(\xv)\right]\end{align*}
% \begin{align*}
% Var\left[\hat{f}(\xv)\right] &= Var\left[g_{\open \emptyset \close}\right] + Var\left[g_{\open 1 \close}(x_1)\right] + Var\left[g_{\open 2 \close}(x_2)\right] \\
% &\phantom{{}={}} + Var\left[g_{\open 1, 2 \close}(x_1, x_2)\right] + \;\dots\; + Var\left[g_{\open 1,\ldots,p \close}(\xv)\right]
% \end{align*}
% \end{itemize}
% \end{frame}

In other words: The single components are uncorrelated.

% \begin{frame}{Variance decomposition}

% \begin{itemize}
\item Dividing by the prediction variance, yields fraction of variance explained by each term:
\begin{align*}
1 &= \frac{Var\left[g_{\open \emptyset \close}\right]}{\predvar} + \frac{Var\left[g_{\open 1 \close}(x_1)\right]}{\predvar} + %\frac{Var\left[g_{\open 2 \close}(x_2)\right]}{\predvar} \\
%&\phantom{{}={}} 
\;\dots\;
+ \frac{Var\left[g_{\open 1, 2 \close}(x_1, x_2)\right]}{\predvar} + \;\dots\; + \frac{Var\left[g_{\open 1,\ldots,p \close}(\xv)\right]}{\predvar}
\end{align*}

\item Fraction of variance explained by a component $g_{\open V \close}(\xv_V)$ is the Sobol index:
$
S_V = \frac{Var\left[g_{\open V \close}(\xv_V)\right]}{Var\left[\hat{f}(\xv)\right]}
$\\
$\leadsto$ Importance measure of component $g_{\open V \close}(\xv_V)$\\
$\leadsto$ Small $S_V$ values $\Rightarrow$ Component $g_{\open V \close}$ does not explain much of total variance of $\fh$
\end{itemize}

\end{frame}











\endlecture
\end{document}
