\documentclass[11pt,compress,t,notes=noshow, aspectratio=169, xcolor=table]{beamer}

\usepackage{../../style/lmu-lecture}
% Defines macros and environments
\usepackage{bbm}
% basic latex stuff
\newcommand{\pkg}[1]{{\fontseries{b}\selectfont #1}} %fontstyle for R packages
\newcommand{\lz}{\vspace{0.5cm}} %vertical space
\newcommand{\dlz}{\vspace{1cm}} %double vertical space
\newcommand{\oneliner}[1] % Oneliner for important statements
{\begin{block}{}\begin{center}\begin{Large}#1\end{Large}\end{center}\end{block}}


%new environments
\newenvironment{vbframe}  %frame with breaks and verbatim
{
 \begin{frame}[containsverbatim,allowframebreaks]
}
{
\end{frame}
}

\newenvironment{vframe}  %frame with verbatim without breaks (to avoid numbering one slided frames)
{
 \begin{frame}[containsverbatim]
}
{
\end{frame}
}

\newenvironment{blocki}[1]   % itemize block
{
 \begin{block}{#1}\begin{itemize}
}
{
\end{itemize}\end{block}
}

\newenvironment{fragileframe}[2]{  %fragile frame with framebreaks
\begin{frame}[allowframebreaks, fragile, environment = fragileframe]
\frametitle{#1}
#2}
{\end{frame}}


\newcommand{\myframe}[2]{  %short for frame with framebreaks
\begin{frame}[allowframebreaks]
\frametitle{#1}
#2
\end{frame}}

\newcommand{\remark}[1]{
  \textbf{Remark:} #1
}


\newenvironment{deleteframe}
{
\begingroup
\usebackgroundtemplate{\includegraphics[width=\paperwidth,height=\paperheight]{../style/color/red.png}}
 \begin{frame}
}
{
\end{frame}
\endgroup
}
\newenvironment{simplifyframe}
{
\begingroup
\usebackgroundtemplate{\includegraphics[width=\paperwidth,height=\paperheight]{../style/color/yellow.png}}
 \begin{frame}
}
{
\end{frame}
\endgroup
}\newenvironment{draftframe}
{
\begingroup
\usebackgroundtemplate{\includegraphics[width=\paperwidth,height=\paperheight]{../style/color/green.jpg}}
 \begin{frame}
}
{
\end{frame}
\endgroup
}
% https://tex.stackexchange.com/a/261480: textcolor that works in mathmode
\makeatletter
\renewcommand*{\@textcolor}[3]{%
  \protect\leavevmode
  \begingroup
    \color#1{#2}#3%
  \endgroup
}
\makeatother


\providecommand{\tightlist}{%
  \setlength{\itemsep}{0pt}\setlength{\parskip}{0pt}}

%\setbeamerfont{footnote}{size=\tiny}
\usepackage[hang,flushmargin]{footmisc}
\renewcommand*{\footnotelayout}{\tiny}
\renewcommand*{\thefootnote}{} %\fnsymbol{footnote}

% https://tex.stackexchange.com/questions/30720/footnote-without-a-marker
% \makeatletter
% \def\blfootnote{\gdef\@thefnmark{}\@footnotetext}
% \makeatother

% https://tex.stackexchange.com/questions/357717/beamer-allowframebreaks-option-and-vertical-spacing-when-using-lists-itemize
% \setbeamertemplate{frametitle continuation}{%
%     (\insertcontinuationcount)%
%     \ifnum\insertcontinuationcount>1%
%     \vspace*{\topsep}%
%     \else%
%     %
%     \fi%
% }


\title{Interpretable Machine Learning}
% \author{LMU}
%\institute{\href{https://compstat-lmu.github.io/lecture_iml/}{compstat-lmu.github.io/lecture\_iml}}
\date{}

\begin{document}

\newcommand{\titlefigure}{figure/whitebox}
\newcommand{\learninggoals}{
%\item What characteristics does an interpretable model have?
\item Why should we use interpretable models at all?
%\item Examples for interpretable models.
}

\lecturechapter{Interpretable Models}
\lecture{Interpretable Machine Learning}

\begin{frame}{Motivation}
Achieving interpretability by using interpretable models is the most straightforward approach.
\bigskip
\begin{columns}[T]
    \begin{column}{0.45\textwidth}
    \begin{itemize}
        %\item Obtaining interpretations by using interpretable models is the easiest and least error-prone approach.
       % \item Achieving interpretability by using interpretable models is the most straightforward approach.
        \bigskip
        \item Classes of models deemed interpretable:
        \begin{itemize}
            %\item Linear regression models
            \item (Generalized) linear models
            \item Generalized additive models
            \item Decision trees
            \item Rule-based learning
            \item Model-based boosting / component-wise boosting
            \item ...
        \end{itemize}
    \end{itemize}
    \end{column}
    \begin{column}{0.45\textwidth}  %%<--- here
  \includegraphics[width = \textwidth]{slides/interpretable-models/figure/main_effect_lm_temp.pdf}
  \begin{center}
    $\leadsto$ Linear model provides straightforward interpretation
  \end{center}
    \end{column}
\end{columns}
\end{frame}

\begin{frame}{Advantages}

    \begin{itemize}[<+->]
    \itemsep1em
        \item Inherently interpretable - additional model-agnostic interpretation techniques not required.\\
        $\Rightarrow$ Eliminates a source of error.
        \item Since interpretable models are often simple, training time is fairly small.
        \item Some interpretable models fulfill the monotonicity constraint.\\
        $\leadsto$ Larger feature values always lead to higher (or smaller) outcomes (e.g., GLMs).
        \item Many people are familiar with traditional interpretable models.\\
        $\Rightarrow$ Increases trust in the model and facilitates communication about its results.
        \item Implementations are available in many programming languages. \\
        $\Rightarrow$ Simple models are easier to deploy in practice or implement from scratch.
    \end{itemize}

\end{frame}

\begin{frame}{Disadvantages}

    \begin{itemize}
    \itemsep1em
        \item Certain assumptions about data and / or model structure are often reqiured.\\
        $\Rightarrow$ If assumptions are wrong, models may have bad predictive performance.
        \item In high-dimensional settings it may not be feasible to find and define all feature relationships.
        %\item If the wrong assumptions are made, interpretable models may have a bad predictive performance.
        \pause
        \item Often not able to model highly complex relationships due to limited model flexibility (e.g., higher-order main effects in a standard linear regression).
        \pause
        \item A complex structure of an interpretable model may also be difficult to interpret, e.g.:
        \begin{itemize}
            \item linear model with hundreds of features and interactions 
            \item decision trees with huge tree depth
        \end{itemize}
        \pause
        \item Still requires application of model-agnostic interpretation techniques if certain types of explanations are of interest (e.g., counterfactual explanations).
    \end{itemize}

\end{frame}

\begin{frame}{Further Comments}

    \begin{itemize}
    \itemsep1em
        \item Some argue that one should always use interpretable models in the first place \lit{Rudin 2019}{https://www.nature.com/articles/s42256-019-0048-x}
        \begin{itemize}
            \item \ldots and not try to explain uninterpretable models post-hoc.
            \item Can sometimes work out by spending enough time and energy on data pre-processing. %feat. engineering and data cleaning
        \end{itemize}
        \pause
        %\footnote[frame]{Rudin, C. Stop explaining black box machine learning models for high stakes decisions and use interpretable models instead. Nat Mach Intell 1, 206–215 (2019).}
        % {https://www.nature.com/articles/s42256-019-0048-x}
       %\item Interpretable models also have the potential for a high predictive performance, but require more knowledge and time spent on data preprocessing.
       \item[$\leadsto$] Drawback: Hard to achieve for data for which end-to-end learning is crucial\\ (e.g., images and text)
        \pause
        %\item One can assume a trade-off between interpretability and model performance which generally (but not always) holds.
        \item There might be a trade-off between interpretability and model performance (but not always).
        \pause
        \item \textbf{Recommendation:}
        \begin{itemize}
            \item Start with most simple model that makes sense for application at hand
            \item Gradually increase complexity if performance is insufficient\\
            $\leadsto$ will usually lower interpretability and require additional interpretation methods
            \item Choose the most simple, sufficient model (Occam's razor)
        \end{itemize} 
    \end{itemize}

\end{frame}


\endlecture
\end{document}
