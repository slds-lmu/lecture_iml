\documentclass[11pt,compress,t,notes=noshow, aspectratio=169, xcolor=table]{beamer}

\usepackage{../../style/lmu-lecture}
% Defines macros and environments
\usepackage{bbm}
% basic latex stuff
\newcommand{\pkg}[1]{{\fontseries{b}\selectfont #1}} %fontstyle for R packages
\newcommand{\lz}{\vspace{0.5cm}} %vertical space
\newcommand{\dlz}{\vspace{1cm}} %double vertical space
\newcommand{\oneliner}[1] % Oneliner for important statements
{\begin{block}{}\begin{center}\begin{Large}#1\end{Large}\end{center}\end{block}}


%new environments
\newenvironment{vbframe}  %frame with breaks and verbatim
{
 \begin{frame}[containsverbatim,allowframebreaks]
}
{
\end{frame}
}

\newenvironment{vframe}  %frame with verbatim without breaks (to avoid numbering one slided frames)
{
 \begin{frame}[containsverbatim]
}
{
\end{frame}
}

\newenvironment{blocki}[1]   % itemize block
{
 \begin{block}{#1}\begin{itemize}
}
{
\end{itemize}\end{block}
}

\newenvironment{fragileframe}[2]{  %fragile frame with framebreaks
\begin{frame}[allowframebreaks, fragile, environment = fragileframe]
\frametitle{#1}
#2}
{\end{frame}}


\newcommand{\myframe}[2]{  %short for frame with framebreaks
\begin{frame}[allowframebreaks]
\frametitle{#1}
#2
\end{frame}}

\newcommand{\remark}[1]{
  \textbf{Remark:} #1
}


\newenvironment{deleteframe}
{
\begingroup
\usebackgroundtemplate{\includegraphics[width=\paperwidth,height=\paperheight]{../style/color/red.png}}
 \begin{frame}
}
{
\end{frame}
\endgroup
}
\newenvironment{simplifyframe}
{
\begingroup
\usebackgroundtemplate{\includegraphics[width=\paperwidth,height=\paperheight]{../style/color/yellow.png}}
 \begin{frame}
}
{
\end{frame}
\endgroup
}\newenvironment{draftframe}
{
\begingroup
\usebackgroundtemplate{\includegraphics[width=\paperwidth,height=\paperheight]{../style/color/green.jpg}}
 \begin{frame}
}
{
\end{frame}
\endgroup
}
% https://tex.stackexchange.com/a/261480: textcolor that works in mathmode
\makeatletter
\renewcommand*{\@textcolor}[3]{%
  \protect\leavevmode
  \begingroup
    \color#1{#2}#3%
  \endgroup
}
\makeatother


\providecommand{\tightlist}{%
  \setlength{\itemsep}{0pt}\setlength{\parskip}{0pt}}

%\setbeamerfont{footnote}{size=\tiny}
\usepackage[hang,flushmargin]{footmisc}
\renewcommand*{\footnotelayout}{\tiny}
\renewcommand*{\thefootnote}{} %\fnsymbol{footnote}

% https://tex.stackexchange.com/questions/30720/footnote-without-a-marker
% \makeatletter
% \def\blfootnote{\gdef\@thefnmark{}\@footnotetext}
% \makeatother

% https://tex.stackexchange.com/questions/357717/beamer-allowframebreaks-option-and-vertical-spacing-when-using-lists-itemize
% \setbeamertemplate{frametitle continuation}{%
%     (\insertcontinuationcount)%
%     \ifnum\insertcontinuationcount>1%
%     \vspace*{\topsep}%
%     \else%
%     %
%     \fi%
% }


\title{Interpretable Machine Learning}
% \author{LMU}
%\institute{\href{https://compstat-lmu.github.io/lecture_iml/}{compstat-lmu.github.io/lecture\_iml}}
\date{}

\begin{document}

\newcommand{\titlefigure}{figure/whitebox}
\newcommand{\learninggoals}{
\item Definition of GLMs
\item Logistic regression as example
\item Interpretation in logistic regression
}

\lecturechapter{Generalized Linear Models}
\lecture{Interpretable Machine Learning}


%------------------------------------------------------------------
%------------------------------------------------------------------
\begin{frame}{Generalized Linear Model (GLM)}

\textbf{Problem}: Target variable given the features not always normally distributed $\leadsto$ LM not suitable
\begin{itemize}
    \item Target is binary (e.g., disease classification)\\
    $\leadsto$ Bernoulli / Binomial distribution
    \item Target is count variable (e.g., number of sold products)\\
    $\leadsto$ Poisson distribution
    \item Time until an event occurs (e.g., time until death)\\
    $\leadsto$ Gamma distribution
\end{itemize}
\medskip
\pause
\textbf{Solution}: GLMs - extend LMs by allowing other distributions from exponential family
%$$g(\E (y \mid \xv)) = \beta_0 + \beta_1 x_1 + \beta_2 x_2 + \ldots + \beta_p x_p$$
% \begin{align*}
% &g(\E (y \mid \xv)) = 
% %\beta_0 + \beta_1 x_1 + \beta_2 x_2 + \ldots + \beta_p x_p = 
% \xv^\top \beta\\
% \Leftrightarrow \; & \E (y \mid \xv) = g^{-1}(\xv^\top \beta)
% \end{align*}
$$g(\E (y \mid \xv)) = \xv^\top \beta \; \Leftrightarrow \; \E (y \mid \xv) = g^{-1}(\xv^\top \beta)$$
\vspace*{-0.5cm}
    \begin{itemize}[<+->]
        \item Link function $g$ links linear predictor $\xv^\top \beta$ to $\E$ of specified distribution of $y \mid \xv$\\
        $\leadsto$ LM is special case: Gaussian distribution $y \mid \xv$ with $g$ as identity function 
        \item Link function $g$ and distribution need to be specified by the user
        \item Interaction and polynomial effects can be manually added in the same way as in LMs
        \item Note: Interpretation of weights depend on specified link function and distribution
    \end{itemize}
 %   \medskip
 %   Non-Gaussian outputs via Generalized Linear Models (GLMs):
  %  \begin{itemize}
   %     \item link function $g$ -- can be freely chosen
    %    \item exponential family defining $\E_Y$ -- can be freely chosen
     %   \item weighted sum $X^\top W$
    %\end{itemize}
    %\medskip 
    %\pause
    %Interaction effects via feature engineering:
    %\begin{itemize}
    %    \item E.g., feature expansion: $\beta_{x_i,x_j} x_i \cdot x_j$
    %\end{itemize}
\end{frame}
 	
\begin{frame}{GLM - Logistic Regression}
\begin{itemize}
    \item Logistic regression $\hat{=}$ GLM with Bernoulli distribution and logit link function: 
    $$g(x) = \log\left(\frac{x}{1 - x}\right) \; \Rightarrow \; g^{-1}(x) = \frac{1}{1+\exp(-x)} $$
    \item Logistic regression models $\E(y \mid \xv) = P(y = 1)$, i.e., probabilities for binary classification by
    $$P(y = 1) = g^{-1}(\xv^\top \beta) = \frac{1}{1 + \exp(-( \xv^\top \beta ))} $$
    %\item Logistic regression models $\E(y \mid \xv) = P(y = 1)$, i.e., probabilities for binary classification 
    %\item Logistic regression formula:
    %$$P(y = 1) =\frac{1}{1 + \exp(-( \beta_0 + \beta_1 x_1 + \beta_2 x_2 + \ldots + \beta_p x_p ))} $$
    \item Typically, we set the threshold to $0.5$ to predict classes, e.g.,
        \begin{itemize}
            \item Class 1 if $P(y=1) > 0.5$
            \item Class 0 if $P(y=1) \leq 0.5$
        \end{itemize}
\end{itemize}
\end{frame}


% \begin{frame}{Generalized Linear Models (GLMs)}

% \begin{align*}
% g\left(\mathbb{E}_Y(Y \vert X)\right) &= X^T\beta \\
% \mathbb{E}_Y \left(Y \vert X\right) &= g^{-1}(X^T\beta)
% \end{align*}
% \begin{itemize}[<+->]
% %\setlength\itemsep{1em}
% %\item GLMs are a framework for target distributions of  exponential families (e.g., Gaussian, Binomial, Poisson, Gamma). Gaussian target distribution $+$ identity link $\hat =$ linear / polynomial model.
% \item GLM framework assumes distribution of $Y$ is a member of exponential families, e.g.: % (e.g., Gaussian, Binomial, Poisson)
% \begin{itemize}[<.->]
%     \item Gaussian target distribution $+$ identity link $\hat =$ linear / polynomial model.
%     \item Binomial target distribution $+$ logit link $log\left(\frac{\mathbb{E}_Y(Y \vert X)}{1 - \mathbb{E}_Y(Y \vert X)}\right)$ $\hat =$ logistic regression.
% \end{itemize}
% %\item GLMs keeps linear predictor $X^T\beta$ (explains $\mathbb{E}_Y(Y \vert X)$ through a more flexible link function $g$).
% \item Link function $g$ describes how $X^T\beta$ relates to the expected, conditional target $\mathbb{E}_Y(Y \vert X)$.
% \begin{itemize}[<.->]
%     \item Interpretations as for linear / polynomial models but w.r.t. transformed target via $g$.
%     \item Linear terms become non-linear through the link function (except for identity link)!
% \end{itemize}
% %\item If the target is distributed binomially, a natural link function is the logit link $log\left(\frac{\mathbb{E}_Y(Y \vert X)}{1 - \mathbb{E}_Y(Y \vert X)}\right)$ (logistic regression).
% \item We need to specify the correct model equation, target distribution, and link function to receive a good model fit.
% %\item %As the model equation is still known, interpretations are possible the same way as for polynomial regression models.
% %However, even linear terms become non-linear through the link function (if the identity link is not used)!
% \end{itemize}
% \end{frame}
%------------------------------------------------------------------
%------------------------------------------------------------------

\begin{frame}[c]{GLM - Logistic Regression - Interpretation}

    \begin{itemize}
        \item Weights $\beta_j$ relate to log odds $\leadsto$ Effects are again linear (but w.r.t. log odds)
        $$log\text{ }odds = \log \left(\frac{P(y = 1)}{1 - P(y=1)}\right) = \log \left(\frac{P(y = 1)}{P(y=0)}\right) = \beta_0 + \beta_1 x_1 + \beta_2 x_2 + \ldots + \beta_p x_p  $$
        \pause
        \item Odds can be caulculated by 
        $$odds = \left(\frac{P(y = 1)}{P(y=0)}\right) = \exp(\beta_0 + \beta_1 x_1 + \beta_2 x_2 + \ldots + \beta_p x_p)  $$
        \item \textbf{Interpretation}: Changing $x_j$ by one unit, changes the \textbf{odds ratio} by a \textbf{factor} of $\exp(\beta_j)$
        $$odds\text{ }ratio = \frac{odds_{x_j+1}}{odds} = \frac{\exp(\beta_0 + \beta_1 x_1 + \ldots + \beta_j (x_j+1) + \ldots + \beta_p x_p)}{\exp(\beta_0 + \beta_1 x_1 + \ldots + \beta_j x_j + \ldots + \beta_p x_p)} = \exp{(\beta_j)} $$
        \pause
        \item Interpretation for different feature types is the same as for linear regression (however, linear interpretation only possible w.r.t. log odds $\leadsto$ difficult to comprehend)
    \end{itemize}	

\end{frame}

\begin{frame}{GLM - Logistic Regression - Example}

\begin{itemize}
    \item Create a binary target variable for bike rental data:
    \begin{itemize}
        \item Class 1: ``high number of bike rentals'' - more than the 70\% quantile (i.e., \code{cnt} > 5531)
        \item Class 0: ``low to medium number of bike rentals'' (i.e., \code{cnt} $\leq$ 5531)
    \end{itemize}
    \item Fit a logistic regression model (GLM with Bernoulli distribution and logit link)
\end{itemize}

\begin{columns}[T]
\begin{column}{0.45\textwidth}
\begin{table}[ht]
\centering
\tiny
\begin{tabular}{rrrrr}
  \hline
 & Weights & Std. Error & z value & Pr($>$$|$z$|$) \\ 
  \hline
(Intercept) & -8.5 & 1.2 & -7.1 & 0.00 \\ 
  seasonSPRING & 1.7 & 0.6 & 2.9 & 0.00 \\ 
  seasonSUMMER & -0.9 & 0.8 & -1.1 & 0.26 \\ 
  seasonFALL & -0.6 & 0.6 & -1.2 & 0.25 \\ 
  temp & 0.3 & 0.0 & 7.4 & 0.00 \\ 
  hum & -0.1 & 0.0 & -5.0 & 0.00 \\ 
  windspeed & -0.1 & 0.0 & -3.0 & 0.00 \\ 
  days\_since\_2011 & 0.0 & 0.0 & 11.6 & 0.00 \\ 
   \hline
\end{tabular}
\end{table}
\pause
\textbf{Interpretation}
\begin{itemize}
    %\item If the temperature increases by 1 degree Celsius then the log odds of high number of bike rentals increase linearly by 0.3 c.p.
    \item If \code{temp} increases by 1$^\circ C$, the odds for ``high number of bike rentals'' increase by the factor $\exp (0.3) = 1.3$, c.p.
\end{itemize}
\end{column}
\hfill
\pause
\begin{column}{0.5\textwidth}
\includegraphics[width = \textwidth]{figure/logistic_marginal_temp.pdf}
\end{column}
\end{columns}

\end{frame}

%------------------------------------------------------------------
%------------------------------------------------------------------

\endlecture
\end{document}