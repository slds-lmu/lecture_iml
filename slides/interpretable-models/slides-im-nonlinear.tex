\documentclass[11pt,compress,t,notes=noshow, aspectratio=169, xcolor=table]{beamer}

\usepackage{../../style/lmu-lecture}
% Defines macros and environments
\usepackage{bbm}
% basic latex stuff
\newcommand{\pkg}[1]{{\fontseries{b}\selectfont #1}} %fontstyle for R packages
\newcommand{\lz}{\vspace{0.5cm}} %vertical space
\newcommand{\dlz}{\vspace{1cm}} %double vertical space
\newcommand{\oneliner}[1] % Oneliner for important statements
{\begin{block}{}\begin{center}\begin{Large}#1\end{Large}\end{center}\end{block}}


%new environments
\newenvironment{vbframe}  %frame with breaks and verbatim
{
 \begin{frame}[containsverbatim,allowframebreaks]
}
{
\end{frame}
}

\newenvironment{vframe}  %frame with verbatim without breaks (to avoid numbering one slided frames)
{
 \begin{frame}[containsverbatim]
}
{
\end{frame}
}

\newenvironment{blocki}[1]   % itemize block
{
 \begin{block}{#1}\begin{itemize}
}
{
\end{itemize}\end{block}
}

\newenvironment{fragileframe}[2]{  %fragile frame with framebreaks
\begin{frame}[allowframebreaks, fragile, environment = fragileframe]
\frametitle{#1}
#2}
{\end{frame}}


\newcommand{\myframe}[2]{  %short for frame with framebreaks
\begin{frame}[allowframebreaks]
\frametitle{#1}
#2
\end{frame}}

\newcommand{\remark}[1]{
  \textbf{Remark:} #1
}


\newenvironment{deleteframe}
{
\begingroup
\usebackgroundtemplate{\includegraphics[width=\paperwidth,height=\paperheight]{../style/color/red.png}}
 \begin{frame}
}
{
\end{frame}
\endgroup
}
\newenvironment{simplifyframe}
{
\begingroup
\usebackgroundtemplate{\includegraphics[width=\paperwidth,height=\paperheight]{../style/color/yellow.png}}
 \begin{frame}
}
{
\end{frame}
\endgroup
}\newenvironment{draftframe}
{
\begingroup
\usebackgroundtemplate{\includegraphics[width=\paperwidth,height=\paperheight]{../style/color/green.jpg}}
 \begin{frame}
}
{
\end{frame}
\endgroup
}
% https://tex.stackexchange.com/a/261480: textcolor that works in mathmode
\makeatletter
\renewcommand*{\@textcolor}[3]{%
  \protect\leavevmode
  \begingroup
    \color#1{#2}#3%
  \endgroup
}
\makeatother


\providecommand{\tightlist}{%
  \setlength{\itemsep}{0pt}\setlength{\parskip}{0pt}}

%\setbeamerfont{footnote}{size=\tiny}
\usepackage[hang,flushmargin]{footmisc}
\renewcommand*{\footnotelayout}{\tiny}
\renewcommand*{\thefootnote}{} %\fnsymbol{footnote}

% https://tex.stackexchange.com/questions/30720/footnote-without-a-marker
% \makeatletter
% \def\blfootnote{\gdef\@thefnmark{}\@footnotetext}
% \makeatother

% https://tex.stackexchange.com/questions/357717/beamer-allowframebreaks-option-and-vertical-spacing-when-using-lists-itemize
% \setbeamertemplate{frametitle continuation}{%
%     (\insertcontinuationcount)%
%     \ifnum\insertcontinuationcount>1%
%     \vspace*{\topsep}%
%     \else%
%     %
%     \fi%
% }


\title{Interpretable Machine Learning}
% \author{LMU}
%\institute{\href{https://compstat-lmu.github.io/lecture_iml/}{compstat-lmu.github.io/lecture\_iml}}
\date{}

\begin{document}

\newcommand{\titlefigure}{figure/whitebox}
\newcommand{\learninggoals}{
\item Generalized additive model
\item Decision trees
\item Other interpretable models}

\lecturechapter{Interpretable Nonlinear Models}
\lecture{Interpretable Machine Learning}

%------------------------------------------------------------------
%------------------------------------------------------------------

\begin{frame}{Generalized Additive Model (GAM) \citebutton{Hastie and Tibshirani (1986)}{https://www.jstor.org/stable/2245459}}

\textbf{Problem}: Relationship between features and target variable not linear $\leadsto$ standard LM not suitable 

\medskip
\pause
\textbf{Solution}: 
 \begin{itemize}
        \item Feature transformations (e.g., exp or log)
        \item Categorization of features (i.e., intervals / buckets of feature values)
        \item GAMs:
        $$g(\E (y \mid \xv)) = \beta_0 + f_1(x_1) + f_2(x_2) + \ldots + f_p(x_p)$$
    
    \begin{itemize}
        \item Instead of linear terms $\beta_j x_j$ use flexible functions $f_j(x_j)$ $\leadsto$ splines
        \item Preserves additive structure and allows to model non-linear effects
        \item Splines have a smoothness parameter to control flexibility (prevent overfitting)\\
        $\leadsto$ Needs to be chosen, e.g., via cross-validation
    \end{itemize}
    \end{itemize}
    
    

   

\end{frame}


\begin{frame}{Generalized Additive Model (GAM) - Example}
    \begin{itemize}
    \item Fit a GAM with smooth splines for four numeric features of bike rental data 
\end{itemize}

\begin{columns}[T]
\begin{column}{0.45\textwidth}
\begin{table}[ht]
\centering
\tiny
\begin{tabular}{rrrrr}
  \hline
 & edf & Ref.df & F & p-value \\ 
  \hline
s(temp) & 5.8 & 7.0 & 57.2 & 0.00 \\ 
  s(hum) & 4.0 & 5.1 & 68.0 & 0.00 \\ 
  s(windspeed) & 1.7 & 2.1 & 50.1 & 0.00 \\ 
  s(days\_since\_2011) & 8.3 & 8.8 & 154.4 & 0.00 \\ 
   \hline
\end{tabular}
\end{table}


\textbf{Interpretation}
\begin{itemize}
    %\item If the temperature increases by 1 degree Celsius then the log odds of high number of bike rentals increase linearly by 0.3 c.p.
    \item Edf (effective degrees of freedom) represents complexity of smoothness
    \item Interpretation needs to be done visually and relative to average prediction\\
    $\leadsto$ more flexible and better model fit but less interpretable than LM
\end{itemize}
\end{column}
\hfill
\begin{column}{0.5\textwidth}
\includegraphics[width = \textwidth]{figure/gam_effects.pdf}
\end{column}
\end{columns}
\end{frame}
%------------------------------------------------------------------
%------------------------------------------------------------------

\begin{frame}{Decision Trees \citebutton{Breiman et al. (1984)}{https://doi.org/10.1201/9781315139470}}

\textbf{Problem}: Can we model non-linear effects and interactions automatically (without manual specification as in GLMs or GAMs)?\\
%Relationship between features and target are non-linear or interactions are present\\
\medskip
%\pause
\textbf{Idea}: Decision tree -- Split data in different subsets depending on cut-off values in features 
$$
\hat f(x) = \sum_{m=1}^M c_m \mathds{1}_{\{x \in \mathcal{R}_m\}}\text{,  where $c_m$ is a constant and $\mathcal{R}_m$ the $m$-th leaf node of the tree}
$$

\begin{itemize}
    \item Finding best split point (CART): Greedy search for the point that minimizes the variance of $y$ (regression) or the Gini index (classification)
    \item Able to handle mixed feature spaces and missing values
\end{itemize}
\medskip
\pause
\textbf{Interpretation}
\begin{itemize}
    \item Directly by following the tree structure (i.e., sequence of decision rules)
    \item Feature importance by (scaled) score of how much splitting criterion (e.g. variance) is reduced compared to a parent node
\end{itemize}



\end{frame}

\begin{frame}{Decision Trees - Example}
\begin{itemize}
    \item Fit decision tree with tree depth of 3 on bike data
    \item E.g., mean prediction for the first 105 days since 2011 is 1798
    \item Feature \code{days\_since\_2011} shows highest feature importance (explains most of variance)
\end{itemize}
\begin{columns}[T]
\begin{column}{0.4\textwidth}
\begin{table}[ht]
\centering
\begin{tabular}{lr}
  \hline
Feature & Importance \\ 
  \hline
days\_since\_2011 & 68.03 \\ 
  temp & 20.54 \\ 
  season & 6.56 \\ 
  hum & 3.58 \\ 
  windspeed & 1.29 \\ 
   \hline
\end{tabular}
\end{table}
\end{column}
\begin{column}{0.6\textwidth}
  \includegraphics[width = \textwidth]{figure/tree.pdf} 
\end{column}
\end{columns}
 
\end{frame}
%------------------------------------------------------------------
%------------------------------------------------------------------

%\begin{frame}[c]{Decision Rules}

%\texttt{IF COND$_1$ AND COND$_2$ AND ... THEN value}

%\begin{itemize}
%    \item \texttt{COND$_i$} can be of the form \texttt{feature <op> value} where \texttt{<op>} can be for example $\{=, <, > \}$
%\end{itemize}

%\pause
%\medskip

%Properties:
%\begin{description}
%    \item{Support} Fraction of observations to support appliance of rule
%    \item{Accuracy} for predicting the correct class under the condition(s)
%\end{description}

%$\leadsto$ often trade-off between these two

%\pause
%\medskip

%$\leadsto$ many different ways to learn a set of rules (incl. a default rule if none of the rules are met)

%\end{frame}

%------------------------------------------------------------------
%------------------------------------------------------------------


\begin{frame}[c]{Other Interpretable Models}

\textbf{RuleFit} \citebutton{Friedman and Popescu 2008}{https://arxiv.org/abs/0811.1679}
\begin{itemize}
    \item Combination of linear models and decision trees 
    \item Allows for feature interactions and non-linearities
\end{itemize}

\textbf{Decision Rules} \citebutton{Holte 1993}{https://doi.org/10.1023/A:1022631118932}
\begin{itemize}
    \item Simple ``if -- then'' statements - very intuitive and easy-to-interpret
    \item Common assumption: only classification tasks and categorical features
\end{itemize}

\textbf{NaiveBayes}
%$$P (C_k \mid x ) = \frac{1}{Z} P(C_k) \prod_{i=1}^{n} P(x_i \mid C_k) $$
\begin{itemize}
    \item Product of probabilities for a class on the value of each feature
    \item Strong independence assumption
\end{itemize}


\textbf{k-Nearest Neighbor}
\begin{itemize}
    \item (Closely related to case-based reasoning)
    \item Average of the outcome of neighbors -- local explanation
\end{itemize}

\end{frame}


\endlecture
\end{document}