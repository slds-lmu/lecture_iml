\documentclass[11pt,compress,t,notes=noshow, aspectratio=169, xcolor=table]{beamer}

\usepackage{../../style/lmu-lecture}
% Defines macros and environments
\usepackage{bbm}
% basic latex stuff
\newcommand{\pkg}[1]{{\fontseries{b}\selectfont #1}} %fontstyle for R packages
\newcommand{\lz}{\vspace{0.5cm}} %vertical space
\newcommand{\dlz}{\vspace{1cm}} %double vertical space
\newcommand{\oneliner}[1] % Oneliner for important statements
{\begin{block}{}\begin{center}\begin{Large}#1\end{Large}\end{center}\end{block}}


%new environments
\newenvironment{vbframe}  %frame with breaks and verbatim
{
 \begin{frame}[containsverbatim,allowframebreaks]
}
{
\end{frame}
}

\newenvironment{vframe}  %frame with verbatim without breaks (to avoid numbering one slided frames)
{
 \begin{frame}[containsverbatim]
}
{
\end{frame}
}

\newenvironment{blocki}[1]   % itemize block
{
 \begin{block}{#1}\begin{itemize}
}
{
\end{itemize}\end{block}
}

\newenvironment{fragileframe}[2]{  %fragile frame with framebreaks
\begin{frame}[allowframebreaks, fragile, environment = fragileframe]
\frametitle{#1}
#2}
{\end{frame}}


\newcommand{\myframe}[2]{  %short for frame with framebreaks
\begin{frame}[allowframebreaks]
\frametitle{#1}
#2
\end{frame}}

\newcommand{\remark}[1]{
  \textbf{Remark:} #1
}


\newenvironment{deleteframe}
{
\begingroup
\usebackgroundtemplate{\includegraphics[width=\paperwidth,height=\paperheight]{../style/color/red.png}}
 \begin{frame}
}
{
\end{frame}
\endgroup
}
\newenvironment{simplifyframe}
{
\begingroup
\usebackgroundtemplate{\includegraphics[width=\paperwidth,height=\paperheight]{../style/color/yellow.png}}
 \begin{frame}
}
{
\end{frame}
\endgroup
}\newenvironment{draftframe}
{
\begingroup
\usebackgroundtemplate{\includegraphics[width=\paperwidth,height=\paperheight]{../style/color/green.jpg}}
 \begin{frame}
}
{
\end{frame}
\endgroup
}
% https://tex.stackexchange.com/a/261480: textcolor that works in mathmode
\makeatletter
\renewcommand*{\@textcolor}[3]{%
  \protect\leavevmode
  \begingroup
    \color#1{#2}#3%
  \endgroup
}
\makeatother


\providecommand{\tightlist}{%
  \setlength{\itemsep}{0pt}\setlength{\parskip}{0pt}}

%\setbeamerfont{footnote}{size=\tiny}
\usepackage[hang,flushmargin]{footmisc}
\renewcommand*{\footnotelayout}{\tiny}
\renewcommand*{\thefootnote}{} %\fnsymbol{footnote}

% https://tex.stackexchange.com/questions/30720/footnote-without-a-marker
% \makeatletter
% \def\blfootnote{\gdef\@thefnmark{}\@footnotetext}
% \makeatother

% https://tex.stackexchange.com/questions/357717/beamer-allowframebreaks-option-and-vertical-spacing-when-using-lists-itemize
% \setbeamertemplate{frametitle continuation}{%
%     (\insertcontinuationcount)%
%     \ifnum\insertcontinuationcount>1%
%     \vspace*{\topsep}%
%     \else%
%     %
%     \fi%
% }


\title{Interpretable Machine Learning}
% \author{LMU}
%\institute{\href{https://compstat-lmu.github.io/lecture_iml/}{compstat-lmu.github.io/lecture\_iml}}
\date{}

\def\firstrowcolor{}
\def\secondrowcolor{}
\def\thirdrowcolor{}
\def\fourthrowcolor{}

\begin{document}

\newcommand{\titlefigure}{figure/whitebox}
\newcommand{\learninggoals}{
\item Examples of popular interpretable models
%\item Properties of some interpretable models
\item Focus on how to interpret them, not on math details}

\lecturechapter{Overview of Interpretable Models}
\lecture{Interpretable Machine Learning}




\begin{frame}[c]{Linear Regression}
% https://towardsdatascience.com/assumptions-of-linear-regression-fdb71ebeaa8b
% https://www.statology.org/linear-regression-assumptions/
% https://link.springer.com/book/10.1007/978-3-642-34333-9
%\textbf{Model formula}
    %$$\mathbb{E}_Y(Y \vert X) = \beta_0 + \beta_1 x_1 + \beta_2 x_2 + \ldots + \beta_p x_p + \epsilon = X^T\mathbf{\beta} + \epsilon$$
$$y = \beta_0 + \beta_1 x_1 + \beta_2 x_2 + \dots + \beta_p x_p + \epsilon = \xv^\top \beta + \epsilon$$

    \begin{itemize}
        %\item $\mathbb{E}_Y(Y \vert X)$ expected value of target given features $X$
        %\item $y$ target / output
        \item $\beta_j$: weight of input feature $x_j$ (with intercept $\beta_0$)\\
        $\leadsto$ model consists of $p+1$ weights
        %\item $\epsilon$ remaining error (e.g., because of noise)
        %\item Model equation is additive and identical across entire input space
        %\pause
        \item Polynomial regression extends equation above by
        \begin{itemize}
        \item \textbf{higher order main effects} which have their own weights (e.g., quadratic: $\beta_{x_j^2} \cdot x_j^2$)
        \item \textbf{interaction effects} (e.g., 2-way interaction: $\beta_{x_i, x_j} \cdot x_i \cdot x_j$)
        \end{itemize}
    \end{itemize}
   \vspace*{0.2cm} 
   \pause
    \textbf{Model assumptions} \citebutton{Examples with R and Python}{https://towardsdatascience.com/assumptions-of-linear-regression-fdb71ebeaa8b}
    \begin{itemize}
    \item Predictions are a \textbf{linear} combination of features $\leadsto$ model equation is additive% and identical across entire input space
    \item Error term $\epsilon$ and $y \vert \xv$ are \textbf{normally} distributed with \textbf{constant variance} (homoscedastic), i.e.,\\
    \centerline{$\epsilon \sim N(0, \sigma^2) \; \Rightarrow \; (y \vert \xv) \sim N(\xv^\top \beta, \sigma^2)$}
    $\leadsto$ if violated, inference-based metrics (t-statistic, p-values, confidence intervals) are invalid
    %\item Error terms are assumed to have a \textbf{constant variance} over the entire feature space %(homoscedasticity)
    \item Further assumptions: Independence of observations (e.g., no autocorrelation), independence of features $x_j$ with error term $\epsilon$, no multicollinearity 
    % free of measurement error assumption: https://indigo.uic.edu/articles/thesis/Measurement_Error_in_Generalized_Linear_Models/17025971/1/files/31488719.pdf
    % \item Note: For inference-based metrics (t-statistic, p-values, confidence intervals) to be valid, the error term needs to be normally distributed, i.e., $\epsilon \sim N(0, \sigma^2) \; \Rightarrow \; (y \vert x) \sim N(x^T \beta, \sigma^2)$\\
%$\leadsto$ Restricts use of LMs in practice as distribution of error is a prior assumption about data
        % \item Properties and assumptions:
        % \begin{itemize}
        %     \item linear
        %     \item normality assumption of the target % not true...
        %     \item homoscedastic (i.e., constant variance)
        %     \item independence of features
        %     \item fixed features (i.e., free of noise)
        %     \item no strong correlation of features
        % \end{itemize} 
    \end{itemize}

\end{frame}

%------------------------------------------------------------------
%------------------------------------------------------------------

\begin{frame}[c]{Linear Regression - Interpretation}

$$y = \beta_0 + \beta_1 x_1 + \beta_2 x_2 + \dots + \beta_p x_p + \epsilon = \xv^\top \beta + \epsilon$$

    Interpretation of weights (\textbf{feature effects}) depend on type of feature:
    \begin{itemize}
        \item \textbf{Numerical $x_j$}: Increasing $x_j$ by one unit changes outcome by $\beta_j$, ceteris paribus (c.p.)
        \pause
        \item \textbf{Binary $x_j$}: Weight $\beta_j$ is active or not (multiplication with 1 or 0) where 0 is reference category
        \pause
        \item %Categorical: One-hot-encoding of $L-1$ new features for $L$ categories (dummy encoding) \\
        \textbf{Categorical $x_j$ with $L$ categories}: Create $L-1$ one-hot-encoded features $x_{j,1}, \hdots, x_{j,L-1}$ (each having its own weight), left out category is reference ($\hat =$ dummy encoding)
        %$x_{j,l}, \forall l \in \{1, \hdots, L-1\}$ (with weight $\beta_{j,l}$), left out category is reference ($\hat =$ dummy encoding)
        \\
        $\leadsto$ Interpretation:
        Outcome changes by $\beta_{j,l}$ for category $l$ compared to reference cat.,  c.p.\pause
        % (for any of the $L-1$ categories):
        %Predicted outcome changes for $l$-th category compared to the reference category by its weight $\beta_{j,l}$ c.p.\pause
        %Predicted outcome changes for category $x_{j,l}$ compared to the reference category by $\beta_j$ c.p.\pause
        \item \textbf{Intercept $\beta_0$}: Expected outcome if all feature values are set to 0 %(baseline) %reflects expected features values if features were standardised (0-mean, 1-stdev)
        \item Note: In case of higher order or interaction effects, weights cannot be interpreted in isolation
    \end{itemize}	
    \pause
    \textbf{Feature importance}:
    \begin{itemize}
        \item Absolute t-statistic value: Estimated weight scaled with its standard error (reliability of the estimate) -- high absolute t-values indicate important (i.e. significant) features
    \end{itemize}
    $$t_{\hat\beta_j} = \left| \tfrac{\hat\beta_j}{SE(\hat\beta_j)} \right|$$
\end{frame}

%------------------------------------------------------------------
%------------------------------------------------------------------

% \begin{frame}{Linear and Polynomial Regression}

% \begin{align*}
% \mathbb{E}_Y(Y \vert X) &= \beta_0 + \beta_1 x_1 + \dots + \beta_p x_p + \epsilon \\
%  &= X^T\beta + \mathcal{E}
% \end{align*}

% \begin{itemize}
% \itemsep1em
% \item Model equation is identical across the entire feature space.
% %\item The predictive power of LMs is determined by specifying the correct model structure.
% \item Polynomial regression extends the LM by non-linear effects.
% %A polynomial regression model is an extension of the LM that includes higher order terms or interactions.
% %This enables us to model non-linear data while making use of the entire arsenal of LM functionality.
% \item We can exactly determine feature effects (e.g., beta coefficients, effect plots) and importance scores (e.g., p-values, t-statistics).
% %By knowing the model equation, we can exactly determine feature effects (e.g., beta coefficients, effect plots) and importance scores (e.g., p-values, t-statistics).
% \item For higher order effects or interactions, beta coefficients cannot be interpreted in isolation.
% \item Note: For inference-based metrics (p-values, confidence intervals) to be valid, error term needs to be normally distributed with zero mean, i.e., $\epsilon \sim N(0, \sigma^2) \; \Rightarrow \; (y \vert x) \sim N(x^T \beta, \sigma^2)$.\\
% $\leadsto$ Restricts use of LMs in practice as distribution of error is a prior assumption about data.
% \end{itemize}
% \end{frame}

\begin{frame}{Example: Linear Regression - Main Effects}

\textbf{Bike data}: predict number of rented bikes using 4 numeric and 1 categorical feature (season)
% \begin{footnotesize}
% $$
% \hat y = \hat \beta_0 + \hat \beta_1 \mathds{1}_{(seas = SPRING)} + \hat \beta_2 \mathds{1}_{(seas = SUMMER)} + \hat \beta_3 \mathds{1}_{(seas = FALL)} + \hat \beta_4 temp + \hat \beta_5 hum + \hat \beta_6 windspeed + \hat \beta_7 days\_since\_2011
% $$
% \end{footnotesize}
\begin{columns}[T, totalwidth=\linewidth]
\begin{column}{0.52\textwidth}
%\vspace*{-0.3cm}
\begin{align*}
\hat y = 
& \hat \beta_0 + 
\hat \beta_1 \mathds{1}_{x_{season} = SPRING} +
\hat \beta_2 \mathds{1}_{x_{season} = SUMMER} +\\
& 
\hat \beta_3 \mathds{1}_{x_{season} = FALL} + 
\hat \beta_4 x_{temp} + 
\hat \beta_5 x_{hum} + \\
& 
\hat \beta_6 x_{windspeed} + 
\hat \beta_7 x_{days\_since\_2011}
\end{align*}
\end{column}
\begin{column}{0.47\textwidth}
  \centering
\begin{scriptsize}
%\begin{overlayarea}{\textwidth}{\textheight}
%\begin{table}[ht]
\only<2>{\def\firstrowcolor{\rowcolor{lightgray}}}
\only<3>{\def\secondrowcolor{\rowcolor{lightgray}}}
\only<4>{\def\thirdrowcolor{\rowcolor{lightgray}}}
\begin{tabular}{rrrrr}
  \hline
 & Weights & SE & t-stat. & p-val. \\
 \hline
\firstrowcolor (Intercept) & 3229.3 & 220.6 & 14.6 & 0.00 \\ 
\secondrowcolor seasonSPRING & 862.0 & 129.0 & 6.7 & 0.00 \\ 
  seasonSUMMER & 41.6 & 170.2 & 0.2 & 0.81 \\ 
  seasonFALL & 390.1 & 116.6 & 3.3 & 0.00 \\ 
\thirdrowcolor temp & 120.5 & 7.3 & 16.5 & 0.00 \\ 
  hum & -31.1 & 2.6 & -12.1 & 0.00 \\ 
  windspeed & -56.9 & 7.1 & -8.0 & 0.00 \\ 
  days\_since\_2011 & 4.9 & 0.2 & 26.9 & 0.00 \\
   \hline
\end{tabular}
%\end{table}
%\end{overlayarea}
\end{scriptsize}
\end{column}
\end{columns}
%\vspace*{-0.3cm}
\pause

\lz
\begin{columns}[T, totalwidth=\linewidth]
\begin{column}<5>{0.46\textwidth}
\textbf{Vis.}: Boxplot of $\hat\beta_j x_j$-values (scale invariant)\\
%weights multiplied by actual feature value (better comparable due to different scales)
$\leadsto$ Comparable feature contributions w.r.t. $\hat y$

\includegraphics[width = \textwidth]{figure/plot_lin_effect.pdf}
%   \begin{center}
%   % Effect of $i$-th observation $= \beta_j x_j^{(i)}$\\

%     %$\leadsto$ Better comparability due to different scales
%   \end{center}
%\pause
%\verbatiminput{figure/lm_output.txt}
\end{column}\hfill
\begin{column}{0.54\textwidth}  %%<--- here

\begin{itemize}[<+->]
    \item \textbf{Interpretation intercept}:
    If all feature values are 0 (and season is \code{WINTER} $\hat =$ reference cat.), the expected number of bike rentals is $\hat\beta_0 = 3229.3$
    \item \textbf{Interpretation categorical}: Rentals in \code{SPRING} are by $\hat\beta_1 = 862$ higher than in \code{WINTER}, c.p.
    \item \textbf{Interpretation numerical}: Rentals increase by $\hat\beta_4 = 120.5$ if \code{temp} increases by 1 $^{\circ}$C, c.p.
    %If the temperature increases by 1 degree Celsius, the number of bike rentals increases by 120.5 c.p.

\end{itemize}
\end{column}
\end{columns}
\end{frame}


\begin{frame}{Example: Linear Regression - Including Interactions}
\textbf{Example}: Interaction between \code{temp} and \code{season} will affect marginal effect of \code{temp}% with (right) and without (left) interaction.
\begin{columns}[T, totalwidth=\linewidth]
\begin{column}{0.65\linewidth}
\includegraphics[width = \textwidth]{figure/lm_main_vs_interaction_effects.pdf}
\end{column}
\begin{column}{0.35\linewidth}
\begin{scriptsize}
%\begin{table}[ht]
\only<2->{\def\firstrowcolor{\rowcolor{lightgray}}}
\only<3>{\def\secondrowcolor{\rowcolor{lightgray}}}
\only<4>{\def\thirdrowcolor{\rowcolor{lightgray}}}
\only<5>{\def\fourthrowcolor{\rowcolor{lightgray}}}
\centering
\begin{tabular}{rr}
  \hline
 & Weights \\ 
  \hline
(Intercept) & 3453.9 \\ 
  seasonSPRING & 1317.0 \\ 
  seasonSUMMER & 4894.1 \\ 
  seasonFALL & -114.2 \\ 
  \firstrowcolor temp & 160.5 \\ 
  hum & -37.6 \\ 
  windspeed & -61.9 \\ 
  days\_since\_2011 & 4.9 \\
  \hline
  \secondrowcolor seasonSPRING:temp & -50.7 \\ 
  \thirdrowcolor seasonSUMMER:temp & -222.0 \\ 
  \fourthrowcolor seasonFALL:temp & 27.2 \\ 
   \hline
\end{tabular}
%\end{table}
\end{scriptsize}
\end{column}
\end{columns}
\vfill
\pause
\textbf{Interpretation}: If \code{temp} increases by 1 $^{\circ}$C, bike rentals
\begin{itemize}[<+->]
    \item increase by 160.5 in \code{WINTER} (reference)
    \item increase by 109.8 (= 160.5 - 50.7) in \code{SPRING}
    \item decrease by -61.5 (= 160.5 - 222) in \code{SUMMER}
    \item increase by 187.7 (= 160.5 + 27.2) in \code{FALL}
\end{itemize} %\\\vspace*{0.2cm}
%\textbf{Note:} Temperature ranges (on x-axis) depend on season $\leadsto$ not acknowledged by LM
%Value ranges of temperature differ depending on the season $\leadsto$ not acknowledged by LM
\end{frame}


\begin{frame}{Linear Regression - Including Polynomials}
\textbf{Example}: Adding 2nd order polynomial for \code{temp} \only<2>{(left) and interaction with \code{season} (right)}
\begin{columns}[T, totalwidth=\linewidth]
\begin{column}{0.65\linewidth}
\includegraphics[width=0.5\textwidth, trim=0cm 0.1cm 10.4cm 0cm, clip]{figure/poly_main_vs_interaction_effects.pdf}\invisible<1>{\includegraphics[width=0.5\textwidth, trim=10cm 0.1cm 0.4cm 0cm, clip]{figure/poly_main_vs_interaction_effects.pdf}}
%\includegraphics[width = \textwidth]{figure/poly_main_vs_interaction_effects.pdf}
%\begin{columns}
%\begin{center}
%\adjincludegraphics[width=\textwidth, trim={0 0 {.55\width} 0}, clip]{figure/poly_main_vs_interaction_effects.pdf}

\textbf{Interpretation}: Not linear anymore!
\begin{itemize}
    \item[$\leadsto$] \code{temp} depends on two weights: $280.2 \cdot x_{temp} - 5.6 \cdot x_{temp}^2$
    \item<2>[$\leadsto$] Non-linear and interaction effects make the model more flexible but also less interpretable (more weights)
    \item<2>[$\leadsto$] Non-linear and interaction effects need to be specified manually (inconvenient) - ML models do this automatically
\end{itemize}

\end{column}
\begin{column}{0.35\linewidth}
%\hfill
\begin{scriptsize}
% \begin{table}[ht]
% %\centering

\def\firstrowcolor{\rowcolor{lightgray}}%
\def\secondrowcolor{\rowcolor{spring}}%
\def\thirdrowcolor{\rowcolor{summer}}%
\def\fourthrowcolor{\rowcolor{fall}}%

\only<1>{%
\begin{tabular}{rr}
  \hline
 & Weights \\ 
  \hline
(Intercept) & 3094.1 \\ 
  seasonSPRING & 619.2 \\ 
  seasonSUMMER & 284.6 \\ 
  seasonFALL & 123.1 \\ 
  hum & -36.4 \\ 
  windspeed & -65.7 \\ 
  days\_since\_2011 & 4.7 \\ 
  \firstrowcolor temp & 280.2 \\ 
  \firstrowcolor temp$^2$ & -5.6 \\ 
  \hline
\end{tabular}%
}
% \end{table}
\only<2>{%
\definecolor{winter}{RGB} {243,117,108}
\definecolor{spring}{RGB} {121,174,65}
\definecolor{summer}{RGB} {25,188,195}
\definecolor{fall}{RGB} {166,128,185}

\def\firstrowcolor{\rowcolor{winter}}%
\begin{tabular}{rr}
  \hline
 & Weights \\ 
  \hline
(Intercept) & 3802.1 \\ 
  seasonSPRING & -1345.1 \\ 
  seasonSUMMER & -6006.3 \\ 
  seasonFALL & -681.4 \\ 
  hum & -38.9 \\ 
  windspeed & -64.1 \\ 
  days\_since\_2011 & 4.8 \\ 
 \firstrowcolor temp & 39.1 \\ 
 \firstrowcolor temp$^2$ & 8.6 \\ 
  \hline
  \hline
  \secondrowcolor seasonSPRING:temp & 407.4 \\ 
 \secondrowcolor seasonSPRING:temp$^2$ & -18.7 \\ 
  \thirdrowcolor seasonSUMMER:temp & 801.1 \\ 
 \thirdrowcolor seasonSUMMER:temp$^2$ & -27.2 \\ 
  \fourthrowcolor seasonFALL:temp & 217.4 \\ 
 \fourthrowcolor seasonFALL:temp$^2$ & -11.3 \\ 
   \hline
\end{tabular}%
}%
%Table: Weights for main effect model
\end{scriptsize}
%\end{center}
\end{column}
\end{columns}
\end{frame}


\begin{frame}{Regularization via LASSO \citebutton{Tibshirani (1996)}{https://doi.org/10.1111/j.2517-6161.1996.tb02080.x}}
\begin{itemize}
    \item LASSO adds an $L_1$-norm penalization term ($\lambda||\beta||_1$) to the least squares optimization problem\\
    $\leadsto$ Shrinks some feature weights to zero (feature selection)\\
    $\leadsto$ Sparser models (with fewer features) are more interpretable
    %Aim: Feature selection (sparsity) and regularization of feature weights
    \item Penalization parameter $\lambda$ must be chosen (e.g., by CV) %or depending on the number of features to keep
\end{itemize}
$$
min_{\beta} \bigg(\underbrace{\frac{1}{n} \sum_{i=1}^{n} (y^{(i)} - {\xi}^{\top} \beta)^2}_\text{Least square estimate for LM} + \lambda||\beta||_1\bigg)
$$
%\vspace*{0.2cm}\\
%\vspace*{-0.4cm}
\pause
\begin{columns}[T, totalwidth=\linewidth]
\begin{column}{0.65\textwidth}
\textbf{Example}
\begin{itemize}
    \item LASSO with main effects and interaction \code{temp} with \code{season}
    \item $\lambda$ is chosen such that 6 features are selected (not zero)
    \item For categorical features, LASSO shrinks weights of single categories separately (due to dummy encoding)\\
    %only weights of single categories are set to zero (due to dummy encoding) not the whole feature\\
    $\leadsto$ No feature selection of whole categorical features\\
    %May be problematic for categorical features as only weights of single categories are set to zero (due to dummy encoding) instead of the whole categorical feature\\ %, polynomials and interactions
    $\leadsto$ Solution: group LASSO \citebutton{Yuan and Lin (2006)}{https://doi.org/10.1111/j.1467-9868.2005.00532.x}
\end{itemize}
\end{column}
\begin{column}{0.35\textwidth}
\tiny
\begin{table}[ht]
\centering
\begin{tabular}{rr}
  \hline
 & Weights \\ 
  \hline
(Intercept) & 3135.2 \\ 
  seasonSPRING & 767.4 \\ 
  seasonSUMMER & 0.0 \\ 
  seasonFALL & 0.0 \\ 
  temp & 116.7 \\ 
  hum & -28.9 \\ 
  windspeed & -50.5 \\ 
  days\_since\_2011 & 4.8 \\ 
  \hline
  seasonSPRING:temp & 0.0 \\ 
  seasonSUMMER:temp & 0.0 \\ 
  seasonFALL:temp & 30.2 \\ 
   \hline
\end{tabular}
\end{table}
% \begin{table}[ht]
% \centering
% \begin{tabular}{rr}
%   \hline
%  & Weights \\ 
%   \hline
% (Intercept) & 2665.50 \\ 
%   seasonSPRING & 489.34 \\ 
%   seasonSUMMER & 0.00 \\ 
%   seasonFALL & 0.00 \\ 
%   hum & -19.44 \\ 
%   windspeed & -35.54 \\ 
%   days\_since\_2011 & 4.71 \\ 
%   poly(temp, 2)1 & 109.25 \\ 
%   poly(temp, 2)2 & 0.00 \\ 
%   \hline
% \end{tabular}
% \end{table}
\end{column}
\end{columns}

\end{frame}
%------------------------------------------------------------------
%------------------------------------------------------------------
\begin{frame}{Generalized Linear Model (GLM)}

\textbf{Problem}: Target variable given the features not always normally distributed $\leadsto$ LM not suitable
\begin{itemize}
    \item Target is binary (e.g., disease classification) $\leadsto$ Bernoulli / Binomial distribution
    \item Target is count variable (e.g., number of sold products) $\leadsto$ Poisson distribution
    \item Time until an event occurs (e.g., time until death) $\leadsto$ Gamma distribution
\end{itemize}
\medskip
\pause
\textbf{Solution}: GLMs - extend LMs by allowing other distributions from exponential family
%$$g(\E (y \mid \xv)) = \beta_0 + \beta_1 x_1 + \beta_2 x_2 + \ldots + \beta_p x_p$$
% \begin{align*}
% &g(\E (y \mid \xv)) = 
% %\beta_0 + \beta_1 x_1 + \beta_2 x_2 + \ldots + \beta_p x_p = 
% \xv^\top \beta\\
% \Leftrightarrow \; & \E (y \mid \xv) = g^{-1}(\xv^\top \beta)
% \end{align*}
$$g(\E (y \mid \xv)) = \xv^\top \beta \; \Leftrightarrow \; \E (y \mid \xv) = g^{-1}(\xv^\top \beta)$$
\vspace*{-0.5cm}
    \begin{itemize}[<+->]
        \item Link function $g$ links weighted sum of features (linear predictor) to $\E$ of specified distribution\\
        $\leadsto$ LM is a special case: Gaussian distribution with $g$ as identity function 
        \item Link function $g$ and distribution need to be specified by the user
        \item Interaction and polynomial effects can be manually added in the same way as in LMs
        \item Note: Interpretation of weights depend on specified link function and distribution
    \end{itemize}
 %   \medskip
 %   Non-Gaussian outputs via Generalized Linear Models (GLMs):
  %  \begin{itemize}
   %     \item link function $g$ -- can be freely chosen
    %    \item exponential family defining $\E_Y$ -- can be freely chosen
     %   \item weighted sum $X^\top W$
    %\end{itemize}
    %\medskip 
    %\pause
    %Interaction effects via feature engineering:
    %\begin{itemize}
    %    \item E.g., feature expansion: $\beta_{x_i,x_j} x_i \cdot x_j$
    %\end{itemize}
\end{frame}
 	
\begin{frame}{GLM - Logistic Regression}
\begin{itemize}
    \item Logistic regression $\hat{=}$ GLM with Bernoulli distribution and logit link function: 
    $$g(x) = \log\left(\frac{x}{1 - x}\right) \; \Rightarrow \; g^{-1}(x) = \frac{1}{1+\exp(-x)} $$
    \item Logistic regression models $\E(y \mid \xv) = P(y = 1)$, i.e., probabilities for binary classification by
    $$P(y = 1) = g^{-1}(\xv^\top \beta) = \frac{1}{1 + \exp(-( \xv^\top \beta ))} $$
    %\item Logistic regression models $\E(y \mid \xv) = P(y = 1)$, i.e., probabilities for binary classification 
    %\item Logistic regression formula:
    %$$P(y = 1) =\frac{1}{1 + \exp(-( \beta_0 + \beta_1 x_1 + \beta_2 x_2 + \ldots + \beta_p x_p ))} $$
    \item Typically, we set the threshold to $0.5$ to predict classes, e.g.,
        \begin{itemize}
            \item Class 1 if $P(y=1) > 0.5$
            \item Class 0 if $P(y=1) \leq 0.5$
        \end{itemize}
\end{itemize}
\end{frame}


% \begin{frame}{Generalized Linear Models (GLMs)}

% \begin{align*}
% g\left(\mathbb{E}_Y(Y \vert X)\right) &= X^T\beta \\
% \mathbb{E}_Y \left(Y \vert X\right) &= g^{-1}(X^T\beta)
% \end{align*}
% \begin{itemize}[<+->]
% %\setlength\itemsep{1em}
% %\item GLMs are a framework for target distributions of  exponential families (e.g., Gaussian, Binomial, Poisson, Gamma). Gaussian target distribution $+$ identity link $\hat =$ linear / polynomial model.
% \item GLM framework assumes distribution of $Y$ is a member of exponential families, e.g.: % (e.g., Gaussian, Binomial, Poisson)
% \begin{itemize}[<.->]
%     \item Gaussian target distribution $+$ identity link $\hat =$ linear / polynomial model.
%     \item Binomial target distribution $+$ logit link $log\left(\frac{\mathbb{E}_Y(Y \vert X)}{1 - \mathbb{E}_Y(Y \vert X)}\right)$ $\hat =$ logistic regression.
% \end{itemize}
% %\item GLMs keeps linear predictor $X^T\beta$ (explains $\mathbb{E}_Y(Y \vert X)$ through a more flexible link function $g$).
% \item Link function $g$ describes how $X^T\beta$ relates to the expected, conditional target $\mathbb{E}_Y(Y \vert X)$.
% \begin{itemize}[<.->]
%     \item Interpretations as for linear / polynomial models but w.r.t. transformed target via $g$.
%     \item Linear terms become non-linear through the link function (except for identity link)!
% \end{itemize}
% %\item If the target is distributed binomially, a natural link function is the logit link $log\left(\frac{\mathbb{E}_Y(Y \vert X)}{1 - \mathbb{E}_Y(Y \vert X)}\right)$ (logistic regression).
% \item We need to specify the correct model equation, target distribution, and link function to receive a good model fit.
% %\item %As the model equation is still known, interpretations are possible the same way as for polynomial regression models.
% %However, even linear terms become non-linear through the link function (if the identity link is not used)!
% \end{itemize}
% \end{frame}
%------------------------------------------------------------------
%------------------------------------------------------------------

\begin{frame}[c]{GLM - Logistic Regression - Interpretation}

    \begin{itemize}
        \item Weights $\beta_j$ relate to log odds $\leadsto$ Effects are again linear (but w.r.t. log odds)
        $$log\text{ }odds = \log \left(\frac{P(y = 1)}{1 - P(y=1)}\right) = \log \left(\frac{P(y = 1)}{P(y=0)}\right) = \beta_0 + \beta_1 x_1 + \beta_2 x_2 + \ldots + \beta_p x_p  $$
        \pause
        \item Odds can be caulculated by 
        $$odds = \left(\frac{P(y = 1)}{P(y=0)}\right) = \exp(\beta_0 + \beta_1 x_1 + \beta_2 x_2 + \ldots + \beta_p x_p)  $$
        \item \textbf{Interpretation}: Changing $x_j$ by one unit, changes the \textbf{odds ratio} by a \textbf{factor} of $\exp(\beta_j)$
        $$odds\text{ }ratio = \frac{odds_{x_j+1}}{odds} = \frac{\exp(\beta_0 + \beta_1 x_1 + \ldots + \beta_j (x_j+1) + \ldots + \beta_p x_p)}{\exp(\beta_0 + \beta_1 x_1 + \ldots + \beta_j x_j + \ldots + \beta_p x_p)} = \exp{(\beta_j)} $$
        \pause
        \item Interpretation for different feature types is the same as for linear regression (however, linear interpretation only possible w.r.t. log odds $\leadsto$ difficult to comprehend)
    \end{itemize}	

\end{frame}

\begin{frame}{GLM - Logistic Regression - Example}

\begin{itemize}
    \item Create a binary target variable for bike rental data:
    \begin{itemize}
        \item Class 1: ``high number of bike rentals'' - more than the 70\% quantile (i.e., \code{cnt} > 5531)
        \item Class 0: ``low to medium number of bike rentals'' (i.e., \code{cnt} $\leq$ 5531)
    \end{itemize}
    \item Fit a logistic regression model (GLM with Bernoulli distribution and logit link)
\end{itemize}

\begin{columns}[T]
\begin{column}{0.45\textwidth}
\begin{table}[ht]
\centering
\tiny
\begin{tabular}{rrrrr}
  \hline
 & Weights & Std. Error & z value & Pr($>$$|$z$|$) \\ 
  \hline
(Intercept) & -8.5 & 1.2 & -7.1 & 0.00 \\ 
  seasonSPRING & 1.7 & 0.6 & 2.9 & 0.00 \\ 
  seasonSUMMER & -0.9 & 0.8 & -1.1 & 0.26 \\ 
  seasonFALL & -0.6 & 0.6 & -1.2 & 0.25 \\ 
  temp & 0.3 & 0.0 & 7.4 & 0.00 \\ 
  hum & -0.1 & 0.0 & -5.0 & 0.00 \\ 
  windspeed & -0.1 & 0.0 & -3.0 & 0.00 \\ 
  days\_since\_2011 & 0.0 & 0.0 & 11.6 & 0.00 \\ 
   \hline
\end{tabular}
\end{table}
\pause
\textbf{Interpretation}
\begin{itemize}
    %\item If the temperature increases by 1 degree Celsius then the log odds of high number of bike rentals increase linearly by 0.3 c.p.
    \item If \code{temp} increases by 1$^\circ C$, the odds for ``high number of bike rentals'' increase by the factor $\exp (0.3) = 1.3$, c.p.
\end{itemize}
\end{column}
\hfill
\pause
\begin{column}{0.5\textwidth}
\includegraphics[width = \textwidth]{figure/logistic_marginal_temp.pdf}
\end{column}
\end{columns}

\end{frame}
%------------------------------------------------------------------
%------------------------------------------------------------------


%------------------------------------------------------------------
%------------------------------------------------------------------

\begin{frame}{Generalized Additive Model (GAM) \citebutton{Hastie and Tibshirani (1986)}{https://www.jstor.org/stable/2245459}}

\textbf{Problem}: Relationship between features and target variable not linear $\leadsto$ standard LM not suitable 

\medskip
\pause
\textbf{Solution}: 
 \begin{itemize}
        \item Feature transformations (e.g., exp or log)
        \item Categorization of features (i.e., intervals / buckets of feature values)
        \item GAMs:
        $$g(\E (y \mid \xv)) = \beta_0 + f_1(x_1) + f_2(x_2) + \ldots + f_p(x_p)$$
    
    \begin{itemize}
        \item Instead of linear terms $\beta_j x_j$ use flexible functions $f_j(x_j)$ $\leadsto$ splines
        \item Preserves additive structure and allows to model non-linear effects
        \item Splines have a smoothness parameter to control flexibility (prevent overfitting)\\
        $\leadsto$ Needs to be chosen, e.g., via cross-validation
    \end{itemize}
    \end{itemize}
    
    

   

\end{frame}


\begin{frame}{Generalized Additive Model (GAM) - Example}
    \begin{itemize}
    \item Fit a GAM with smooth splines for four numeric features of bike rental data 
\end{itemize}

\begin{columns}[T]
\begin{column}{0.45\textwidth}
\begin{table}[ht]
\centering
\tiny
\begin{tabular}{rrrrr}
  \hline
 & edf & Ref.df & F & p-value \\ 
  \hline
s(temp) & 5.8 & 7.0 & 57.2 & 0.00 \\ 
  s(hum) & 4.0 & 5.1 & 68.0 & 0.00 \\ 
  s(windspeed) & 1.7 & 2.1 & 50.1 & 0.00 \\ 
  s(days\_since\_2011) & 8.3 & 8.8 & 154.4 & 0.00 \\ 
   \hline
\end{tabular}
\end{table}


\textbf{Interpretation}
\begin{itemize}
    %\item If the temperature increases by 1 degree Celsius then the log odds of high number of bike rentals increase linearly by 0.3 c.p.
    \item Edf (effective degrees of freedom) represents complexity of smoothness
    \item Interpretation needs to be done visually and relative to average prediction\\
    $\leadsto$ more flexible and better model fit but less interpretable than LM
\end{itemize}
\end{column}
\hfill
\begin{column}{0.5\textwidth}
\includegraphics[width = \textwidth]{figure/gam_effects.pdf}
\end{column}
\end{columns}
\end{frame}
%------------------------------------------------------------------
%------------------------------------------------------------------

\begin{frame}{Decision Trees \citebutton{Breiman et al. (1984)}{https://doi.org/10.1201/9781315139470}}

\textbf{Problem}: Can we model non-linear effects and interactions automatically (without manual specification as in GLMs or GAMs)?\\
%Relationship between features and target are non-linear or interactions are present\\
\medskip
%\pause
\textbf{Idea}: Decision tree -- Split data in different subsets depending on cut-off values in features 
$$
\hat f(x) = \sum_{m=1}^M c_m \mathds{1}_{\{x \in \mathcal{R}_m\}}\text{,  where $c_m$ is a constant and $\mathcal{R}_m$ the $m$-th leaf node of the tree}
$$

\begin{itemize}
    \item Finding best split point (CART): Greedy search for the point that minimizes the variance of $y$ (regression) or the Gini index (classification)
    \item Able to handle mixed feature spaces and missing values
\end{itemize}
\medskip
\pause
\textbf{Interpretation}
\begin{itemize}
    \item Directly by following the tree structure (i.e., sequence of decision rules)
    \item Feature importance by (scaled) score of how much splitting criterion (e.g. variance) is reduced compared to a parent node
\end{itemize}



\end{frame}

\begin{frame}{Decision Trees - Example}
\begin{itemize}
    \item Fit decision tree with tree depth of 3 on bike data
    \item E.g., mean prediction for the first 105 days since 2011 is 1798
    \item Feature \code{days\_since\_2011} shows highest feature importance (explains most of variance)
\end{itemize}
\begin{columns}[T]
\begin{column}{0.4\textwidth}
\begin{table}[ht]
\centering
\begin{tabular}{lr}
  \hline
Feature & Importance \\ 
  \hline
days\_since\_2011 & 68.03 \\ 
  temp & 20.54 \\ 
  season & 6.56 \\ 
  hum & 3.58 \\ 
  windspeed & 1.29 \\ 
   \hline
\end{tabular}
\end{table}
\end{column}
\begin{column}{0.6\textwidth}
  \includegraphics[width = \textwidth]{figure/tree.pdf} 
\end{column}
\end{columns}
 
\end{frame}
%------------------------------------------------------------------
%------------------------------------------------------------------

%\begin{frame}[c]{Decision Rules}

%\texttt{IF COND$_1$ AND COND$_2$ AND ... THEN value}

%\begin{itemize}
%    \item \texttt{COND$_i$} can be of the form \texttt{feature <op> value} where \texttt{<op>} can be for example $\{=, <, > \}$
%\end{itemize}

%\pause
%\medskip

%Properties:
%\begin{description}
%    \item{Support} Fraction of observations to support appliance of rule
%    \item{Accuracy} for predicting the correct class under the condition(s)
%\end{description}

%$\leadsto$ often trade-off between these two

%\pause
%\medskip

%$\leadsto$ many different ways to learn a set of rules (incl. a default rule if none of the rules are met)

%\end{frame}

%------------------------------------------------------------------
%------------------------------------------------------------------

\begin{frame}{Componentwise Gradient Boosting}

\begin{itemize}
%\setlength\itemsep{2em}
\item<1-> Recall: Boosting iteratively combines weak base learners to create a powerful ensemble model
\item<1->
Idea: Boosting with gradient descent using interpretable base learners (e.g., use base learners with single features in each iteration $\leadsto$ coordinate gradient descent)
%The resulting ensemble is also interpretable.
%\pause
\item<2->
Two linear base learners $b_j(x, \theta)$ and $b_j(x, \theta^{\star})$ of the same type, but distinct parameter vectors $\theta$ and $\theta^{\star}$ can be combined in a base learner of the same type:
$$b_j(x, \theta) + b_j(x, \theta^{\star}) = b_j(x, \theta + \theta^{\star})$$
%\pause
\item<3-> In each iteration, a set of base learners is fitted on pseudo residuals. The one with the best fit is added to the previously computed model (using step-size $\nu$):
\medskip
\begin{align*}
\widehat{f}^{[1]}(x) &= \hat{f}_0 + \nu \textcolor{blue}{b_3(x_3, \theta^{[1]})} \\
\widehat{f}^{[2]}(x) &= \widehat{f}^{[1]}(x) + \nu \textcolor{blue}{b_3(x_3, \theta^{[2]})} 
%= \hat{f}_0 + \nu \textcolor{blue}{b_3(x_3, \theta^{[1]})} + \nu \textcolor{blue}{b_3(x_3, \theta^{[2]})}
\\
\widehat{f}^{[3]}(x) &= \widehat{f}^{[2]}(x) + \nu \textcolor{orange}{b_1(x_1, \theta^{[3]})} 
%= \hat{f}_0 + \nu \textcolor{blue}{b_3(x_3, \theta^{[1]})} + \nu \textcolor{blue}{b_3(x_3, \theta^{[2]})} + \nu \textcolor{orange}{b_1(x_1, \theta^{[3]})} 
= \hat{f}_0 + \nu \left(\textcolor{blue}{b_3(x_3, \theta^{[1]} + \theta^{[2]})} + \textcolor{orange}{b_1(x_1, \theta^{[3]})}\right) 
\\
&= \hat{f}_0 + \textcolor{blue}{\hat{f}_3(x_3)} + \textcolor{orange}{\hat{f}_1(x_1)}
\end{align*}
\item<3-> Final model has additive structure (as GAMs), where each component function is interpretable

\end{itemize}
\end{frame}


\begin{frame}{Componentwise Gradient Boosting - Example}

Simple case: Use linear model with single feature (including intercept) as base learner
$$
b_j(x_j, \theta) = x_j\theta + \theta_0 \hspace*{0.2cm}\text{ for } j = 1,\ldots p \hspace*{0.3cm} \leadsto \text{ordinary linear regression}
$$

\begin{itemize}
\item After many iterations, it converges to same solution as least square estimate of LMs
\item Early stopping allows feature selection and might prevent overfitting (regularization)
%\item Specifying loss and link function according to exponential family leads a (regularized) GLM
\item Here: Interpretation of weights as in LM
\end{itemize}
\pause
\begin{columns}[T]
\begin{column}{0.49\textwidth}
\scriptsize
\begin{table}[ht]
\centering
\begin{tabular}{r|r|l}
  %\hline
\textbf{1000 iter. with $\nu = 0.1$} & Intercept & Weights \\ 
  \hline  \hline
days\_since\_2011 & -1791.06 & 4.9 \\ 
  \hline
  hum & 1953.05 & -31.1 \\ 
    \hline
  season &  &  \begin{tabular}[c]{@{}l@{}}
  WINTER: -323.4\\
  SPRING: 539.5\\
  SUMMER: -280.2\\
  FALL: 67.2
  \end{tabular}\\
    \hline
  %season &  & WINTER: -323.4, SPRING: 539.5, SUMMER: -280.2, FALL: 67.2 \\ 
  temp & -1839.85 & 120.4 \\ 
    \hline
  windspeed & 725.70 & -56.9 \\ 
    \hline
  offset & 4504.35 &  \\ 
   %\hline
\end{tabular}
\end{table}
\centering
$\Rightarrow$ Converges to solution of LM
\end{column}
\begin{column}{0.49\textwidth}
\scriptsize
\only<2>{
\begin{table}[ht]
\centering
\begin{tabular}{r|r|l}
  %\hline
 \textbf{20 iter. with $\nu = 0.1$} & Intercept & Weights \\ 
  \hline  \hline
  days\_since\_2011 & -1210.27 & 3.3 \\ 
    \hline
   season &  & 
   \begin{tabular}[c]{@{}l@{}}
  WINTER: -276.9\\
  SPRING: 137.6\\
  SUMMER: 112.8\\
  FALL: 20.3
   \end{tabular}\\
     \hline
  temp & -1118.94 & 73.2 \\ 
    \hline
  offset & 4504.35 &  \\ 
   %\hline
\end{tabular}
\end{table}
\centering
$\Rightarrow$ 3 base learners selected after 20 iter. (feature selection)
}
\only<3>{
Relative frequency of selected base learners across iterations
\includegraphics[width = .95 \textwidth]{figure/compboost_base_linear.pdf}}
\end{column}
\end{columns}
% \begin{itemize}
%     \item Linear base learners for numeric features and categorical base learner for season
%     \item 3 base learners selected after 100 iterations
% \end{itemize}
\end{frame}

\begin{frame}{Componentwise Gradient Boosting - Interpretation}

\begin{itemize}
    \item Fit model on bike data with different base learner types \citebutton{Daniel Schalk et al. 2018}{https://doi.org/10.21105/joss.00967}
    \item Base learners: linear and centered splines for numeric features, categorical for season
    %and categorical base learner for season
\end{itemize}
\begin{columns}[T]
\pause
\begin{column}{0.5\textwidth}
 \includegraphics[width = 0.9\textwidth]{figure/compboost_pfi.pdf}
 \begin{itemize}
     \item Feature importance (risk reduction over iter.)\\
     $\leadsto$ \code{days\_since\_2011} most important
 \end{itemize}
%\scriptsize
%\verbatiminput{figure/mboost_output.txt}
\end{column}
\pause
\begin{column}{0.5\textwidth}  %%<--- here
  \includegraphics[width = 0.9 \textwidth]{figure/compboost_pfe.pdf}
  \begin{itemize}
      \item Partial feature effect for \code{days\_since\_2011}\\
      $\leadsto$ Total effect: Combination of partial effects of linear and centered spline base learners 
  \end{itemize}
\end{column}
\end{columns}
\end{frame}



\begin{frame}[c]{Other Interpretable Models}

\textbf{RuleFit} \citebutton{Friedman and Popescu 2008}{https://arxiv.org/abs/0811.1679}
\begin{itemize}
    \item Combination of linear models and decision trees 
    \item Allows for feature interactions and non-linearities
\end{itemize}

\textbf{Decision Rules} \citebutton{Holte 1993}{https://doi.org/10.1023/A:1022631118932}
\begin{itemize}
    \item Simple ``if -- then'' statements - very intuitive and easy-to-interpret
    \item Common assumption: only classification tasks and categorical features
\end{itemize}

\textbf{NaiveBayes}
%$$P (C_k \mid x ) = \frac{1}{Z} P(C_k) \prod_{i=1}^{n} P(x_i \mid C_k) $$
\begin{itemize}
    \item Product of probabilities for a class on the value of each feature
    \item Strong independence assumption
\end{itemize}


\textbf{k-Nearest Neighbor}
\begin{itemize}
    \item (Closely related to case-based reasoning)
    \item Average of the outcome of neighbors -- local explanation
\end{itemize}

\end{frame}


\endlecture
\end{document}