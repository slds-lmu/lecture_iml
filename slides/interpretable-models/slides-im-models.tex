\documentclass[11pt,compress,t,notes=noshow, aspectratio=169, xcolor=table]{beamer}

\usepackage{../../style/lmu-lecture}
% Defines macros and environments
\usepackage{bbm}
% basic latex stuff
\newcommand{\pkg}[1]{{\fontseries{b}\selectfont #1}} %fontstyle for R packages
\newcommand{\lz}{\vspace{0.5cm}} %vertical space
\newcommand{\dlz}{\vspace{1cm}} %double vertical space
\newcommand{\oneliner}[1] % Oneliner for important statements
{\begin{block}{}\begin{center}\begin{Large}#1\end{Large}\end{center}\end{block}}


%new environments
\newenvironment{vbframe}  %frame with breaks and verbatim
{
 \begin{frame}[containsverbatim,allowframebreaks]
}
{
\end{frame}
}

\newenvironment{vframe}  %frame with verbatim without breaks (to avoid numbering one slided frames)
{
 \begin{frame}[containsverbatim]
}
{
\end{frame}
}

\newenvironment{blocki}[1]   % itemize block
{
 \begin{block}{#1}\begin{itemize}
}
{
\end{itemize}\end{block}
}

\newenvironment{fragileframe}[2]{  %fragile frame with framebreaks
\begin{frame}[allowframebreaks, fragile, environment = fragileframe]
\frametitle{#1}
#2}
{\end{frame}}


\newcommand{\myframe}[2]{  %short for frame with framebreaks
\begin{frame}[allowframebreaks]
\frametitle{#1}
#2
\end{frame}}

\newcommand{\remark}[1]{
  \textbf{Remark:} #1
}


\newenvironment{deleteframe}
{
\begingroup
\usebackgroundtemplate{\includegraphics[width=\paperwidth,height=\paperheight]{../style/color/red.png}}
 \begin{frame}
}
{
\end{frame}
\endgroup
}
\newenvironment{simplifyframe}
{
\begingroup
\usebackgroundtemplate{\includegraphics[width=\paperwidth,height=\paperheight]{../style/color/yellow.png}}
 \begin{frame}
}
{
\end{frame}
\endgroup
}\newenvironment{draftframe}
{
\begingroup
\usebackgroundtemplate{\includegraphics[width=\paperwidth,height=\paperheight]{../style/color/green.jpg}}
 \begin{frame}
}
{
\end{frame}
\endgroup
}
% https://tex.stackexchange.com/a/261480: textcolor that works in mathmode
\makeatletter
\renewcommand*{\@textcolor}[3]{%
  \protect\leavevmode
  \begingroup
    \color#1{#2}#3%
  \endgroup
}
\makeatother


\providecommand{\tightlist}{%
  \setlength{\itemsep}{0pt}\setlength{\parskip}{0pt}}

%\setbeamerfont{footnote}{size=\tiny}
\usepackage[hang,flushmargin]{footmisc}
\renewcommand*{\footnotelayout}{\tiny}
\renewcommand*{\thefootnote}{} %\fnsymbol{footnote}

% https://tex.stackexchange.com/questions/30720/footnote-without-a-marker
% \makeatletter
% \def\blfootnote{\gdef\@thefnmark{}\@footnotetext}
% \makeatother

% https://tex.stackexchange.com/questions/357717/beamer-allowframebreaks-option-and-vertical-spacing-when-using-lists-itemize
% \setbeamertemplate{frametitle continuation}{%
%     (\insertcontinuationcount)%
%     \ifnum\insertcontinuationcount>1%
%     \vspace*{\topsep}%
%     \else%
%     %
%     \fi%
% }


\title{Interpretable Machine Learning}
% \author{LMU}
%\institute{\href{https://compstat-lmu.github.io/lecture_iml/}{compstat-lmu.github.io/lecture\_iml}}
\date{}

\begin{document}

\newcommand{\titlefigure}{figure/whitebox}
\newcommand{\learninggoals}{
\item What characteristics do different interpretable models have?
\item How can we interpret these models?
%\item Why should we use interpretable models at all?
\item Examples for different interpretable models.}

\lecturechapter{Interpretable Models}
\lecture{Interpretable Machine Learning}




\begin{frame}[c]{Linear Regression - Characteristics}

%\textbf{Model formula}
    %$$\mathbb{E}_Y(Y \vert X) = \beta_0 + \beta_1 x_1 + \beta_2 x_2 + \ldots + \beta_p x_p + \epsilon = X^T\mathbf{\beta} + \epsilon$$
\begin{align*}
y &= \beta_0 + \beta_1 x_1 + \beta_2 x_2 + \dots + \beta_p x_p + \epsilon \\
 &= X^T\beta + \epsilon
\end{align*}

    \begin{itemize}
        %\item $\mathbb{E}_Y(Y \vert X)$ expected value of target given features $X$
        %\item $y$ target / output
        \item $\beta_j$ weight of input feature $x_j$ $\leadsto$ model consists of $p+1$ weights $\beta_j$ (including intercept)
        %\item $\epsilon$ remaining error (e.g., because of noise)
        %\item Model equation is additive and identical across entire input space
        %\pause
        \item Polynomial regression extends equation above by higher order main effects which have their own weights (e.g., $\beta_{x_j^2} \cdot x_j^2$) and interaction effects (e.g., 2-way interaction: $\beta_{x_i, x_j} \cdot x_i \cdot x_j$)
    \end{itemize}
   \vspace*{0.2cm} 
    \textbf{Model assumptions}
    \begin{itemize}
    \item Predictions are a \textbf{linear} combination of features - model equation is additive% and identical across entire input space
    \item error term and with that the target outcome given the features are \textbf{normally} distributed, i.e., 
    \centerline{$\epsilon \sim N(0, \sigma^2) \; \Rightarrow \; (y \vert x) \sim N(x^T \beta, \sigma^2)$ with \textbf{constant variance}}\\
    $\leadsto$ if violated, inference-based metrics (t-statistic, p-values, confidence intervals) are invalid
    %\item Error terms are assumed to have a \textbf{constant variance} over the entire feature space %(homoscedasticity)
    \item Further assumptions: Observations are independent of each other, features are fixed (free of measurement errors), no multicollinearity
       % \item Note: For inference-based metrics (t-statistic, p-values, confidence intervals) to be valid, the error term needs to be normally distributed, i.e., $\epsilon \sim N(0, \sigma^2) \; \Rightarrow \; (y \vert x) \sim N(x^T \beta, \sigma^2)$\\
%$\leadsto$ Restricts use of LMs in practice as distribution of error is a prior assumption about data
        % \item Properties and assumptions:
        % \begin{itemize}
        %     \item linear
        %     \item normality assumption of the target % not true...
        %     \item homoscedastic (i.e., constant variance)
        %     \item independence of features
        %     \item fixed features (i.e., free of noise)
        %     \item no strong correlation of features
        % \end{itemize} 
    \end{itemize}

\end{frame}

%------------------------------------------------------------------
%------------------------------------------------------------------

\begin{frame}[c]{Linear Regression - Interpretation}

\begin{align*}
y &= \beta_0 + \beta_1 x_1 + \beta_2 x_2 + \dots + \beta_p x_p + \epsilon \\
 &= X^T\beta + \epsilon
\end{align*}

    Weight interpretation (\textbf{feature effects}) depending on feature types:
    \begin{itemize}
        \item Numerical: Increasing feature value of $x_j$ by one unit changes predicted outcome by $\beta_j$ c.p.
        %\item Binary: Either weight $\beta_j$ is active or not (multiplication with 1 or 0) where 0 is regarded as reference category.
        \item Categorical: One-hot-encoding of $L-1$ new features for $L$ categories (dummy encoding). \\
        $\leadsto$ Interpretation (for any of the L-1 categories): Predicted outcome changes for category $x_j$ compared to the reference category by $\beta_j$ c.p.
        \item Intercept $\beta_0$: Expected baseline prediction if all features are set to 0. %reflects expected features values if features were standardised (0-mean, 1-stdev)
        \item Note: For higher order or interaction effects, coefficients cannot be interpreted in isolation.
    \end{itemize}	
    \pause
    \textbf{Feature importance}:
    \begin{itemize}
        \item absolute t-statistic value: estimated weight scaled with its standard error (i.e., how certain are we about the correct value?) -- high absolute t-values indicate that the feature is important
    \end{itemize}
    $$t_{\beta_j} = \frac{\beta_j}{SE(\beta_j)} $$
\end{frame}

%------------------------------------------------------------------
%------------------------------------------------------------------

% \begin{frame}{Linear and Polynomial Regression}

% \begin{align*}
% \mathbb{E}_Y(Y \vert X) &= \beta_0 + \beta_1 x_1 + \dots + \beta_p x_p + \epsilon \\
%  &= X^T\beta + \mathcal{E}
% \end{align*}

% \begin{itemize}
% \itemsep1em
% \item Model equation is identical across the entire feature space.
% %\item The predictive power of LMs is determined by specifying the correct model structure.
% \item Polynomial regression extends the LM by non-linear effects.
% %A polynomial regression model is an extension of the LM that includes higher order terms or interactions.
% %This enables us to model non-linear data while making use of the entire arsenal of LM functionality.
% \item We can exactly determine feature effects (e.g., beta coefficients, effect plots) and importance scores (e.g., p-values, t-statistics).
% %By knowing the model equation, we can exactly determine feature effects (e.g., beta coefficients, effect plots) and importance scores (e.g., p-values, t-statistics).
% \item For higher order effects or interactions, beta coefficients cannot be interpreted in isolation.
% \item Note: For inference-based metrics (p-values, confidence intervals) to be valid, error term needs to be normally distributed with zero mean, i.e., $\epsilon \sim N(0, \sigma^2) \; \Rightarrow \; (y \vert x) \sim N(x^T \beta, \sigma^2)$.\\
% $\leadsto$ Restricts use of LMs in practice as distribution of error is a prior assumption about data.
% \end{itemize}
% \end{frame}

\begin{frame}{Linear Regression - Example: Main Effects}

\textbf{Bike rental dataset}: predict number of rented bikes depending on 4 numeric features and one categorical feature (season)
\begin{footnotesize}
$$
\hat f = \hat \beta_0 + \hat \beta_1 \mathbbm{1}_{(seas = SPRING)} + \hat \beta_2 \mathbbm{1}_{(seas = SUMMER)} + \hat \beta_3 \mathbbm{1}_{(seas = FALL)} + \hat \beta_4 temp + \hat \beta_5 hum + \hat \beta_6 windspeed + \hat \beta_7 days\_since\_2011
$$
\end{footnotesize}
\vspace*{-0.3cm}
\begin{columns}[T]
\begin{column}{0.52\textwidth}
\begin{tiny}
\begin{table}[ht]
\centering
\begin{tabular}{rrrrr}
  \hline
 & Weights & Std. Error & t value & Pr($>$$|$t$|$) \\ 
  \hline
(Intercept) & 3229 & 221 & 15 & 0.00 \\ 
  seasonSPRING & 862 & 129 & 7 & 0.00 \\ 
  seasonSUMMER & 42 & 170 & 0 & 0.81 \\ 
  seasonFALL & 390 & 117 & 3 & 0.00 \\ 
  temp & 120 & 7 & 17 & 0.00 \\ 
  hum & -31 & 3 & -12 & 0.00 \\ 
  windspeed & -57 & 7 & -8 & 0.00 \\ 
  days\_since\_2011 & 5 & 0 & 27 & 0.00 \\ 
   \hline
\end{tabular}
\end{table}
\end{tiny}
\begin{itemize}
    \item \textbf{Interpretation categorical}: Number of bike rentals in spring increases by 862 compared to winter c.p.
    \item \textbf{Interpretation numerical}: If the temperature increases by 1 degree Celsius, then the number of bike rentals increases by 120 c.p.
\end{itemize}
%\verbatiminput{figure/lm_output.txt}
\end{column}\hfill
\begin{column}{0.47\textwidth}  %%<--- here
  \includegraphics[width = \textwidth]{slides/interpretable-models/figure/plot_lin_effect.pdf}
  \begin{center}
    Effect of $i$-th observation $= \beta_j x_j^{(i)}$\\
    $\leadsto$ better comparability due to different scales
  \end{center}
   
\end{column}
\end{columns}
\end{frame}


\begin{frame}{Linear Regression - Example: Including Interactions}
We add an interaction between temperature and season and visualize the marginal effect of temperature with (right) and without (left) interaction.
\begin{columns}[T]
\begin{column}{0.25\textwidth}
\begin{tiny}
\begin{table}[ht]
\centering
\begin{tabular}{rr}
  \hline
 & Weights \\ 
  \hline
(Intercept) & 3453.88 \\ 
  seasonSPRING & 1317.00 \\ 
  seasonSUMMER & 4894.10 \\ 
  seasonFALL & -114.24 \\ 
  temp & 160.46 \\ 
  hum & -37.59 \\ 
  windspeed & -61.88 \\ 
  days\_since\_2011 & 4.85 \\ 
  seasonSPRING:temp & -50.68 \\ 
  seasonSUMMER:temp & -222.02 \\ 
  seasonFALL:temp & 27.24 \\ 
   \hline
\end{tabular}
\end{table}
\end{tiny}
\end{column}
\begin{column}{0.7\textwidth}
\includegraphics[width = \textwidth]{slides/interpretable-models/figure/lm_main_vs_interaction_effects.pdf}
\end{column}
\end{columns}
\vfill
\textbf{Interpretation}: If temperature increases by 1 degree Celsius, then number of bike rentals increases by 160 in winter and by 110 (160 + (-50)) in spring. In summer it is decreasing by -62 (160 + (-222)).\\\vspace*{0.2cm}
\textbf{Note:} Value ranges of temperature differ depending on the season $\leadsto$ not acknowledged by LM
\end{frame}


\begin{frame}{Linear Regression - Example: Including Polynomials}
We add a 2nd order polynomial for termperature to the main effect (left) and main \& interaction effect (right) model.
\begin{columns}[T]
\begin{column}{0.25\textwidth}
\begin{tiny}
\begin{table}[ht]
\centering
\begin{tabular}{rr}
  \hline
 & Weights \\ 
  \hline
(Intercept) & 3094.14 \\ 
  seasonSPRING & 619.24 \\ 
  seasonSUMMER & 284.57 \\ 
  seasonFALL & 123.14 \\ 
  hum & -36.38 \\ 
  windspeed & -65.65 \\ 
  days\_since\_2011 & 4.68 \\ 
  poly(temp, 2)1 & 280.15 \\ 
  poly(temp, 2)2 & -5.58 \\ 
   \hline
\end{tabular}
\end{table}
   Table: Weights for main effect model

\end{tiny}

\end{column}
\begin{column}{0.7\textwidth}
\includegraphics[width = \textwidth]{slides/interpretable-models/figure/poly_main_vs_interaction_effects.pdf}
\end{column}
\end{columns}
\vfill
\textbf{Interpretation}: Not linear anymore!\\ $\leadsto$ effect size depends on temperature value: $280.2 \cdot temp - 5.6 \cdot temp^2$\\
$\leadsto$ the more non-linear and interaction effects are added the less interpretable becomes the model
\end{frame}


\begin{frame}{Sparse Linear Regression - LASSO}
\begin{itemize}
    \item Sparser models are more interpretable
    \item LASSO adds an L1-norm penalization term ($\lambda||\beta||_1$) to the optimization problem
    \item Aim: Feature selection (sparsity) and regularization of feature weights
    \item The penalization parameter $\lambda$ needs to be chosen (e.g. by CV or depending on the number of features to keep)
\end{itemize}


$$
min_{\beta} \left(\frac{1}{n} \sum_{i=1}^{n} (y^{(i)} - x^{(i)\top}\beta)^2 + \lambda||\beta||_1\right)
$$
\vspace*{0.2cm}\\
\textbf{Example}
\vspace*{-0.4cm}
\begin{columns}
\begin{column}{0.65\textwidth}
\begin{itemize}
    \item we penalize the main effect model with polynomial term for temp
    \item $\lambda$ is chosen such that 5 features are selected
    \item may be problematic for categorical features, polynomials and interactions $\leadsto$ group LASSO 
\end{itemize}
\end{column}
\begin{column}{0.35\textwidth}
\tiny
\begin{table}[ht]
\centering
\begin{tabular}{rr}
  \hline
 & beta \\ 
  \hline
(Intercept) & 0.00 \\ 
  seasonSPRING & 489.34 \\ 
  seasonSUMMER & 0.00 \\ 
  seasonFALL & 0.00 \\ 
  hum & -19.44 \\ 
  windspeed & -35.54 \\ 
  days\_since\_2011 & 4.71 \\ 
  poly(temp, 2, raw = TRUE)1 & 109.25 \\ 
  poly(temp, 2, raw = TRUE)2 & 0.00 \\ 
   \hline
\end{tabular}
\end{table}
\end{column}
\end{columns}

\end{frame}
%------------------------------------------------------------------
%------------------------------------------------------------------
\begin{frame}{Generalized Linear Model (GLM) - Characteristics}

\textbf{Problem}: Target variable given the features are not normally distributed $\leadsto$ linear regression is not suitable
\begin{itemize}
    \item Target is categorical (e.g. medical diagnosis)
    \item Target is a count variable (e.g. number of sold products)
    \item time until an event occurs (e.g. time until death)
\end{itemize}
\medskip
\textbf{Solution}: GLMs - extend LMs by allowing other distributions from the exponential family
$$g(\mathrm{E}_Y (y\mid x)) = \beta_0 + \beta_1 x_1 + \beta_2 x_2 + \ldots + \beta_p x_p$$
\vspace*{-0.4cm}
    \begin{itemize}
        \item Link function $g$ links weighted sum (linear predictor) to specified distribution
        \item Link function and probability distribution need to be specified by the user
        \item LM is a special case by choosing the identity function for $g$ and the Gaussian distribution
        \item Interpretation depends on specified link function and distribution
        \item Interactions can be manually added in the same way as in the LM
    \end{itemize}
    
 %   \medskip
 %   Non-Gaussian outputs via Generalized Linear Models (GLMs):
    
  %  \begin{itemize}
   %     \item link function $g$ -- can be freely chosen
    %    \item exponential family defining $\mathrm{E}_Y$ -- can be freely chosen
     %   \item weighted sum $X^\top W$
    %\end{itemize}
    
    %\medskip 
    %\pause
    %Interaction effects via feature engineering:
    %\begin{itemize}
    %    \item E.g., feature expansion: $\beta_{x_i,x_j} x_i \cdot x_j$
    %\end{itemize}
    
\end{frame}
 	
\begin{frame}{Logistic Regression - Characteristics}

\begin{itemize}
    \item Logistic regression models probabilities (between 0 and 1) for binary classification tasks 
    \item It is a GLM with a Bernoulli distribution and a logit link function
    \item Logistic regression formula:
    $$P(y = 1) =\frac{1}{1 + \exp(-( \beta_0 + \beta_1 x_1 + \beta_2 x_2 + \ldots + \beta_p x_p ))} $$
    \item Typically, we set the threshold to $0.5$ to predict 
        \begin{itemize}
            \item Class 1 if $P(y=1) > 0.5$
            \item Class 0 if $P(y=1) \leq 0.5$
        \end{itemize}
\end{itemize}
    

	

\end{frame}

%------------------------------------------------------------------
%------------------------------------------------------------------

\begin{frame}[c]{Logistic Regression - Interpretation}

    
    
    
    

    \begin{itemize}
        \item weights relate to formula in log odds - again linear in log odds
        $$log\text{ }odds = \log \left(\frac{P(y = 1)}{P(y=0)}\right) = \beta_0 + \beta_1 x_1 + \beta_2 x_2 + \ldots + \beta_p x_p  $$
        \item Odds can be caulculated by 
        $$odds = \left(\frac{P(y = 1)}{P(y=0)}\right) = \exp(\beta_0 + \beta_1 x_1 + \beta_2 x_2 + \ldots + \beta_p x_p)  $$
        \item[$\leadsto$] change by one unit changes the odds ratio by a \alert{factor} of $\exp(\beta_J)$
        $$odds\text{ }ratio = \frac{odds_{x_j+1}}{odds} = \frac{\exp(\beta_0 + \beta_1 x_1 + \ldots + \beta_j (x_j+1) + \ldots + \beta_p x_p)}{\exp(\beta_0 + \beta_1 x_1 + \ldots + \beta_j x_j + \ldots + \beta_p x_p)} = \exp{(\beta_j)} $$
        \item Interpretation for different feature types is the same as for linear regression (however, linear interpretation only possible for log odds $\leadsto$ difficult to comprehend)
    \end{itemize}	

\end{frame}

\begin{frame}{Logistic Regression - Example}

\begin{itemize}
    \item Create a binary target variable for bike rental data (Class 1: high number of bike rentals - more than the 70\% quantile of given lables and Class 0: low to medium number of bike rentals)
    \item Fit a logistic regression with Bernoulli distribution and logit link
\end{itemize}

\begin{columns}[T]
\begin{column}{0.45\textwidth}
\begin{table}[ht]
\centering
\tiny
\begin{tabular}{rrrrr}
  \hline
 & Weights & Std. Error & z value & Pr($>$$|$z$|$) \\ 
  \hline
(Intercept) & -8.5 & 1.2 & -7.1 & 0.00 \\ 
  seasonSPRING & 1.7 & 0.6 & 2.9 & 0.00 \\ 
  seasonSUMMER & -0.9 & 0.8 & -1.1 & 0.26 \\ 
  seasonFALL & -0.6 & 0.6 & -1.2 & 0.25 \\ 
  temp & 0.3 & 0.0 & 7.4 & 0.00 \\ 
  hum & -0.1 & 0.0 & -5.0 & 0.00 \\ 
  windspeed & -0.1 & 0.0 & -3.0 & 0.00 \\ 
  days\_since\_2011 & 0.0 & 0.0 & 11.6 & 0.00 \\ 
   \hline
\end{tabular}
\end{table}

\textbf{Interpretation}
\begin{itemize}
    %\item If the temperature increases by 1 degree Celsius then the log odds of high number of bike rentals increase linearly by 0.3 c.p.
    \item If the temperature increases by 1 degree Celsius then the odds for high number of bike rentals increase by the factor $\exp (0.3) = 1.3$ c.p.
\end{itemize}
\end{column}
\hfill
\begin{column}{0.5\textwidth}
\includegraphics[width = \textwidth]{slides/interpretable-models/figure/logistic_maginal_temp.pdf}
\end{column}
\end{columns}

\end{frame}
%------------------------------------------------------------------
%------------------------------------------------------------------


%------------------------------------------------------------------
%------------------------------------------------------------------

\begin{frame}{Generalized Additive Models (GAMs) - Characteristics}

\textbf{Problem}: Relationship between features and the target variable is not linear $\leadsto$ standard linear regression is not suitable 

\medskip
\textbf{Solution}: 
 \begin{itemize}
        \item feature transformations (e.g., exp or log)
        \item Categorization of features (i.e., intervals / buckets of feature values)
        \item GAMs:
        $$g(\mathrm{E}_Y (y\mid x)) = \beta_0 + f_1(x_1) + f_2(x_2) + \ldots + f_p(x_p)$$
    
    \begin{itemize}
        \item instead of $\beta_j x_j$ use flexible functions $f_j(x_j)$ $\leadsto$ splines
        \item preserves additive structure of feature effects but does also allow nonlinear effects
        \item smoothness parameter to control flexibility (prevent overfitting) needs to be tuned
    \end{itemize}
    \end{itemize}
    
    

   

\end{frame}


\begin{frame}{Generalized Additive Models (GAMs) - Example}
    \begin{itemize}
    \item Fit a GAM with smooth splines for all four numeric features of bike rental data 
\end{itemize}

\begin{columns}[T]
\begin{column}{0.45\textwidth}
\begin{table}[ht]
\centering
\tiny
\begin{tabular}{rrrrr}
  \hline
 & edf & Ref.df & F & p-value \\ 
  \hline
s(temp) & 5.8 & 7.0 & 57.2 & 0.00 \\ 
  s(hum) & 4.0 & 5.1 & 68.0 & 0.00 \\ 
  s(windspeed) & 1.7 & 2.1 & 50.1 & 0.00 \\ 
  s(days\_since\_2011) & 8.3 & 8.8 & 154.4 & 0.00 \\ 
   \hline
\end{tabular}
\end{table}


\textbf{Interpretation}
\begin{itemize}
    %\item If the temperature increases by 1 degree Celsius then the log odds of high number of bike rentals increase linearly by 0.3 c.p.
    \item edf (effective degrees of freedom) represents complexity of the smooth
    \item Interpretation needs to be done visually and relative to the average prediction\\
    $\leadsto$ more flexibility and better model fit but less interpretability
\end{itemize}
\end{column}
\hfill
\begin{column}{0.5\textwidth}
\includegraphics[width = \textwidth]{slides/interpretable-models/figure/gam_effects.pdf}
\end{column}
\end{columns}
\end{frame}
%------------------------------------------------------------------
%------------------------------------------------------------------

\begin{frame}{Decision Trees - Characteristics \& Interpretation}

\textbf{Problem}: Relationship between features and target are nonlinear or feature interactions are present

\medskip
\textbf{Solution}: Decision tree: Splitting data in different subsets depending on cutoff values in features
$$
\hat f(x) = \sum_{m=1}^M c_m \mathbbm{1}_{\{x \in \mathcal{R}_m\}}\text{ ,  where $\mathcal{R}_m$ is the $m$-th leaf node of the tree}
$$

\begin{itemize}
    \item Determination of split point (CART): at split point that minimizes the variance of $y$ (regression) or the Gini index (classification)
    \item Able to handle mixed feature space
\end{itemize}


\textbf{Interpretation}
\begin{itemize}
    \item Directly by following the tree (i.e., sequence of rules)
    \item Feature importance by (scaled) score of how much the splitting criterion (e.g. variance) was reduced compared to the parent
\end{itemize}



\end{frame}

\begin{frame}{Decision Trees - Example}
\begin{itemize}
    \item Fit a decision tree with tree depth of 3
    \item E.g., the mean prediction for days the first 105 days since 2011 is 1798
    \item The feature days\_since\_2011 shows the highest feature importance (explains most of the variance)
\end{itemize}
\begin{columns}[T]
\begin{column}{0.4\textwidth}
\begin{table}[ht]
\centering
\begin{tabular}{lr}
  \hline
Feature & Importance \\ 
  \hline
days\_since\_2011 & 68.03 \\ 
  temp & 20.54 \\ 
  season & 6.56 \\ 
  hum & 3.58 \\ 
  windspeed & 1.29 \\ 
   \hline
\end{tabular}
\end{table}
\end{column}
\begin{column}{0.6\textwidth}
  \includegraphics[width = \textwidth]{slides/interpretable-models/figure/tree.pdf} 
\end{column}
\end{columns}
 
\end{frame}
%------------------------------------------------------------------
%------------------------------------------------------------------

%\begin{frame}[c]{Decision Rules}

%\texttt{IF COND$_1$ AND COND$_2$ AND ... THEN value}

%\begin{itemize}
%    \item \texttt{COND$_i$} can be of the form \texttt{feature <op> value} where \texttt{<op>} can be for example $\{=, <, > \}$
%\end{itemize}

%\pause
%\medskip

%Properties:
%\begin{description}
%    \item{Support} Fraction of observations to support appliance of rule
%    \item{Accuracy} for predicting the correct class under the condition(s)
%\end{description}

%$\leadsto$ often trade-off between these two

%\pause
%\medskip

%$\leadsto$ many different ways to learn a set of rules (incl. a default rule if none of the rules are met)

%\end{frame}

%------------------------------------------------------------------
%------------------------------------------------------------------

\begin{frame}[c]{Other Interpretable Models}

\textbf{RuleFit} \lit{Friedman and Popescu 2008}{https://arxiv.org/abs/0811.1679}
\begin{itemize}
    \item Combination of linear models and decision trees 
    \item Allows for feature interactions and non-linearities
\end{itemize}

\textbf{Decision Rules}

\textbf{NaiveBayes}
%$$P (C_k \mid x ) = \frac{1}{Z} P(C_k) \prod_{i=1}^{n} P(x_i \mid C_k) $$
\begin{itemize}
    \item product of probabilities for a class on the value of each feature
    \item strong independence assumption
\end{itemize}


\textbf{k-Nearest Neighbor}
\begin{itemize}
    \item (closely related to case-based reasoning)
    \item Average of of the outcome of neighbors -- local explanation
\end{itemize}

\textbf{Model-based Boosting}

\end{frame}


\endlecture
\end{document}