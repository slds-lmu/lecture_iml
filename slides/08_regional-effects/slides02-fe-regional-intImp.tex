
\begin{frame}{Interaction quantification}

    \begin{columns}[T, totalwidth=\textwidth]
        \begin{column}{0.59\textwidth}
    
    \textbf{On parent node level (for {\color{mygreen}$\mathcal{N}_{parent}$}):}
    %Relative interaction importance for each parent node :
    $$
       intImp({\color{mygreen}{\mathcal{N}_{parent}}}) = \frac{\mathcal{R}({\color{mygreen}\mathcal{N}_{parent}}) - (\mathcal{R}({\color{amber}\mathcal{N}_{left}}) + \mathcal{R}({\color{red}\mathcal{N}_{right}}))}{\mathcal{R}({\color{blue}\mathcal{N}})} 
    $$
    \textbf{Interpretation:}
    Reduction of ICE curve variance after one split of $\mathcal{N}_{parent}$ into $\mathcal{N}_{left}$ and $\mathcal{N}_{right}$ 
    relative to the ICE curve variance in the root node $\mathcal{N}$.
    
    \medskip
    
    \begin{center}
          \resizebox{0.5\linewidth}{!}{
    \begin{tikzpicture}
        \usetikzlibrary{arrows}
        \usetikzlibrary{shapes}
    
        % Define node styles
        \tikzset{
            treenode/.style={draw, rectangle, minimum size=1cm, text centered},
            N/.style={draw, rectangle, minimum size=1cm, text centered, color=blue, text=blue},
            Np/.style={draw, rectangle, minimum size=1cm, text centered, color=mygreen, text=mygreen},
            Nl/.style={draw, rectangle, minimum size=1cm, text centered, color=amber, text=amber},
            Nr/.style={draw, rectangle, minimum size=1cm, text centered, color=red, text=red}
        }
    
        % Nodes
        \node [N] (a0) {$\mathcal{N}$}; % Top node
        \node [treenode, below right=1cm and 0.7cm of a0] (a1) {}; % Bottom right node
        \node [Np, below left=0.7cm and 1cm of a0] (a2) {$\mathcal{N}_{parent}$}; % Bottom center node
        \node [Nl, below=0.7cm of a2, xshift=-1cm] (a3) {$\mathcal{N}_{left}$}; % Bottom left node
        \node [Nr, below=0.7cm of a2, xshift=1cm] (a4) {$\mathcal{N}_{right}$}; % Bottom right node from center
    
        % Lines
        \draw [thick] (a0) -- (a1);
        \draw [thick] (a0) -- (a2);
        \draw [thick] (a2) -- (a3);
        \draw [thick] (a2) -- (a4);
    \end{tikzpicture}}
    
    \end{center}
        \end{column}
    \pause
        \begin{column}{0.4\textwidth}
      %\centering\includegraphics[width = 0.6\textwidth]{figure/tree_expl.png}
     \centering
    \includegraphics[width = 0.95\textwidth]{figure/sim1_fake.png}
    
    Split reduces 83.5\% of variance.
        \end{column}
    \end{columns}
    
    \end{frame}
    
    
    \begin{frame}{Interaction quantification}
        
    \begin{columns}[T, totalwidth=\textwidth]%, totalwidth=\textwidth]
        \begin{column}{0.46\textwidth}
    
    \textbf{On feature level (for $x_j$):} 
    %Relative interaction importance for feature $x_j$:
    %We denote this subset by $\mathcal{B}_j \subset \mathcal{B}_P$ and the relative interaction importance by
    \medskip
    
    \centerline{$\textstyle
       intImp_j = \sum\nolimits_{i \in \mathcal{B}_j} intImp(\mathcal{N}_i)$}
    
    \medskip
    
    %\textbf{$\bm{intImp(\mathcal{N}_P)}$: risk reduction after one split relative to the root node risk}
    %where $\mathcal{B}_j$ indexes parent nodes where splits are based on $x_j$.
    where $\mathcal{B}_j$ indexes parent nodes split by $x_j$.
    
    \medskip
    
    \textbf{Interpretation:}
    Overall reduction of ICE curve variance due to splits by $X_j$ (in \%).
    
    \medskip
    
    \textbf{Example:} For $X_1 \Rightarrow$  {\color{orange} $\mathcal{B}_1 = \{0, 2, 3\}$}
        \begin{center}
          \resizebox{1\textwidth}{!}{
    \begin{tikzpicture}[scale=0.7, transform shape]
        \tikzset{treenode/.style={draw, rectangle, font=\LARGE}}
        \tikzset{line/.style={draw, thick}}
    
        % Root node
        \node [treenode, fill=orange] (a0) {$\mathcal{N}_0$};
    
        % Level 1 nodes
        \node [treenode, below=1cm of a0, xshift=-3cm] (a1) {$\mathcal{N}_1$};
        \node [treenode, fill=orange, below=1cm of a0, xshift=3cm] (a2) {$\mathcal{N}_2$};
    
        % Level 2 nodes
        \node [treenode, fill=orange, below=1cm of a1, xshift=-1.5cm] (a3) {$\mathcal{N}_3$};
        \node [treenode, below=1cm of a1, xshift=1.5cm] (a4) {$\mathcal{N}_4$};
        \node [treenode, below=1cm of a2, xshift=-1.5cm] (a5) {$\mathcal{N}_5$};
        \node [treenode, below=1cm of a2, xshift=1.5cm] (a6) {$\mathcal{N}_6$};
    
        % Level 3 nodes
        \node [treenode, below=1cm of a3, xshift=-0.75cm] (a7) {$\mathcal{N}_7$};
        \node [treenode, below=1cm of a3, xshift=0.75cm] (a8) {$\mathcal{N}_8$};
        \node [treenode, below=1cm of a4, xshift=-0.75cm] (a9) {$\mathcal{N}_9$};
        \node [treenode, below=1cm of a4, xshift=0.75cm] (a10) {$\mathcal{N}_{10}$};
        \node [treenode, below=1cm of a5, xshift=-0.75cm] (a11) {$\mathcal{N}_{11}$};
        \node [treenode, below=1cm of a5, xshift=0.75cm] (a12) {$\mathcal{N}_{12}$};
        \node [treenode, below=1cm of a6, xshift=-0.75cm] (a13) {$\mathcal{N}_{13}$};
        \node [treenode, below=1cm of a6, xshift=0.75cm] (a14) {$\mathcal{N}_{14}$};
    
        % Edges and labels
        \path [line] (a0.south) -- + (0,-0.5cm) -| (a1.north) node [pos=0.4, above, color=orange] {$X_1 < 5$};
        \path [line] (a0.south) -- + (0,-0.5cm) -| (a2.north) node [pos=0.4, above, color=orange] {$X_1 \geq 5$};
    
        \path [line] (a1.south) -- + (0,-0.5cm) -| (a3.north) node [pos=0.3, above, yshift = -0.05cm] {$X_2 < 3$};
        \path [line] (a1.south) -- + (0,-0.5cm) -| (a4.north) node [pos=0.3, above, yshift = -0.05cm] {$X_2 \geq 3$};
    
        \path [line] (a2.south) -- + (0,-0.5cm) -| (a5.north) node [pos=0.3, above, color=orange, yshift = -0.05cm] {$X_1 < 7$};
        \path [line] (a2.south) -- + (0,-0.5cm) -| (a6.north) node [pos=0.3, above, color=orange, yshift = -0.05cm] {$X_1 \geq 7$};
    
        \path [line] (a3.south) -- + (0,-0.5cm) -| (a7.north) node [pos=0.5, above, color=orange, yshift = -0.05cm] {\small $X_1 < 2$};
        \path [line] (a3.south) -- + (0,-0.5cm) -| (a8.north) node [pos=0.5, above, color=orange, yshift = -0.05cm] {\small $X_1 \geq 2$};
    
        \path [line] (a4.south) -- + (0,-0.5cm) -| (a9.north) node [pos=0.5, above, yshift = -0.05cm] {\small $X_3 < 4$};
        \path [line] (a4.south) -- + (0,-0.5cm) -| (a10.north) node [pos=0.5, above, yshift = -0.05cm] {\small $X_3 \geq 4$};
    
        \path [line] (a5.south) -- + (0,-0.5cm) -| (a11.north) node [pos=0.5, above, yshift = -0.05cm] {\small $X_2 < 5$};
        \path [line] (a5.south) -- + (0,-0.5cm) -| (a12.north) node [pos=0.5, above, yshift = -0.05cm] {\small $X_2 \geq 5$};
    
        \path [line] (a6.south) -- + (0,-0.5cm) -| (a13.north) node [pos=0.5, above, yshift = -0.05cm] {\small $X_4 < 7$};
        \path [line] (a6.south) -- + (0,-0.5cm) -| (a14.north) node [pos=0.5, above, yshift = -0.05cm] {\small $X_4 \geq 7$};
    
    \end{tikzpicture}}
        \end{center}
        \end{column}
    \pause
        \begin{column}{0.52\textwidth}
    
    
        \centering
        \includegraphics[width=\textwidth]{figure/sim1}
         %\begin{minipage}[t]{.5\textwidth}
        %   \includegraphics[width=0.37\textwidth]{figure/sim1_dt_split1.png}
        %   \scalebox{1}{
        %   \hspace{15pt} 
        %   \begin{tikzpicture}
        %   \usetikzlibrary{arrows}
        %     \usetikzlibrary{shapes}
        %      \tikzset{treenode/.style={draw}}
        %      \tikzset{line/.style={draw, thick}}
        %     \node [treenode](a0) {} ; [below=1pt,at=(4,0)]  {};
        %      \node [treenode, below=0.3cm, at=(a0.south), xshift=-1.3cm]  (a1) {};
        %      \node [treenode, below=0.3cm, at=(a0.south), xshift=-0.2cm]  (a2) {};
        %      \path [line] (a0.south) -- + (0,-0.2cm) -| (a1.north) node [midway, above] {};
        %      \path [line] (a0.south) -- +(0,-0.2cm) -|  (a2.north) node [midway, above] {};
        %   \end{tikzpicture}
        %   \hspace{25pt}
        %   \begin{tikzpicture}
        %   \usetikzlibrary{arrows}
        %     \usetikzlibrary{shapes}
        %      \tikzset{treenode/.style={draw}}
        %      \tikzset{line/.style={draw, thick}}
        %     \node [treenode] (a01) {};[below=5pt,at=(node1.south) , xshift=0cm]
        %      \node [treenode, below=0.3cm, at=(a01.south), xshift=0.1cm]  (a1) {};
        %      \node [treenode, below=0.3cm, at=(a01.south), xshift=1.1cm]  (a2) {};
        %      \path [line] (a01.south) -- + (0,-0.2cm) -| (a1.north) node [midway, above] {};
        %      \path [line] (a01.south) -- +(0,-0.2cm) -|  (a2.north) node [midway, above] {};
        %   \end{tikzpicture}
        %   }
        % \includegraphics[width=0.37\textwidth]{figure/sim1_dt_split2_1.png}
        % \includegraphics[width=0.37\textwidth]{figure/sim1_dt_split2_2.png}
        
     \vspace{-200px}
        \scriptsize
     \hspace{130px}
     \setlength{\tabcolsep}{1pt}
     \begin{tabular}{|c|c|}
        \hline
           $x_j$ & $intImp_j$  \\\hline
           \rowcolor{ForestGreen!70}
            $\xv_3$     & 0.835 \\
            \rowcolor{YellowGreen!50}
            $\xv_1$     &  0.14\\\hline
        \end{tabular}\\
     \hspace{138px}= 0.975
          %\footnotesize\\
        %\vspace{150px}
    
        \end{column}
    \end{columns}
    
    \end{frame}
    
    
    
    
    
    \begin{frame}{Outperforming SOTA}
    
    \textbf{Simulation setting}
    \begin{itemize}
        \item Draw 1000 i.i.d. samples from $X_1, \ldots , X_4 \sim \mathcal{U}(-1,1)$
        \item True underlying function: $f(\xv) = \sum\nolimits_{j=1}^4 \xv_j + \xv_1 \xv_2 + \xv_2 \xv_3 + \xv_1 \xv_3 + \xv_1 \xv_2 \xv_3 + \epsilon$ % , \epsilon \sim \mathbbm{N}(0, 0.01 \sigma(\mu(\xv))^2)
        \item Fit a correctly specified linear model (interactions with $\xv_4$ are excluded)
        \item 30 repetitions, measure interaction strength between $\xv_2$ and all other 3 features
    \end{itemize}
    
    \textbf{Which methods are sensitive to changes in main effect sizes or feature correlations?}
    
    
        \begin{table}[thb]
    \vspace{.1in}
        \label{tab:simSummary}
        \begin{center}
        \begin{tabular}{|p{5.4cm}|p{1.6cm}|p{1.8cm}|p{1.6cm}|p{1.6cm}|}
        \hline
           Pitfall & REPID & H-Statistic & Greenwell & SHAP  \\\hline
           sensitive to changes of main effect & No & Yes & Yes & No\\\hline
           sensitive to changes of correlation between $\xv_j$ and other features & No & Yes & No & Yes\\
      \hline
        \end{tabular}
        \end{center}
    \end{table}
    \vspace*{0.2cm}
    
    
    
    
    \end{frame}
    
    \begin{frame}{Outperforming SOTA}
    \centerline{
    \includegraphics[width=0.4\textwidth, page = 1]{figure/sim_sensi_linear.pdf}
    \includegraphics[width=0.4\textwidth, page = 2]{figure/sim_sensi_linear.pdf}
    }
    
       \begin{itemize}
           \item \textbf{Left (initial setting)}: Interaction strength of $\xv_1$:$\xv_2$ and $\xv_3$:$\xv_2$ similar; $\xv_4$:$\xv_2$ no interaction
           \item \textbf{Right}: Set main effect $\beta_1$ = 0.1
           
       \begin{itemize}
           \item \textbf{Expectation}: Interaction strengths should not change
           \item \textbf{Fail}: H-statistic ($\xv_1$:$\xv_2$ > $\xv_3$:$\xv_2$) and Greenwell ($\xv_1$:$\xv_2$ < $\xv_3$:$\xv_2$) 
       \end{itemize}
       \end{itemize}
    \end{frame}
    
    \begin{frame}{Outperforming SOTA}
    \centerline{
    \includegraphics[width=0.4\textwidth, page = 1]{figure/sim_sensi_linear.pdf}
    \includegraphics[width=0.4\textwidth, page = 3]{figure/sim_sensi_linear.pdf}
    }
      \begin{itemize}
            \item \textbf{Left (initial setting)}: Interaction strength of $\xv_1$:$\xv_2$ and $\xv_3$:$\xv_2$ similar; $\xv_4$:$\xv_2$ no interaction
           \item \textbf{Right}: Increase correlation $\rho(\xv_1, \xv_2) = 0.9$
              \begin{itemize}
           \item \textbf{Expectation}: Interaction strengths should not change
           \item \textbf{Fail}: H-statistic ($\xv_1$:$\xv_2$ < $\xv_3$:$\xv_2$) and Shapley ($\xv_1$:$\xv_2$ > $\xv_3$:$\xv_2$)
       \end{itemize}
           \item[$\rightarrow$] \textbf{REPID is the only method which always leads to correct rankings for these settings}  
       \end{itemize}
       
    \end{frame}
    
    
    \begin{frame}{Conclusion}
    
    \begin{columns}[T, totalwidth = \textwidth]
        \begin{column}{0.53\textwidth}
    
     \textbf{Summary of Contributions (REPID)}:
     \begin{itemize}
        \item Regional effects in interpretable regions
        \item Additive decomposition of feature effect 
        %\item Prove meaningfulness of objective - only feature interactions between $\xv_j$ and $\xv_{-j}$ minimized %(measures only feature interactions)
        \item Quantify feature interactions 
        \item Outperforms SOTA interaction indices
        %H-Statistics, Greenwell's and Shapley
    \end{itemize}
    
     \textbf{Summary of Contributions (GADGET)}:
     \begin{itemize}
     \item Unique regions for multiple features
     \item Additive decomposition of prediction function%\\
     \item Extension to ALE and Shapley Dependence
    % $\Rightarrow$ TODO: Investigate usefulness as ``approximation'' when terminal regions contain interactions
     \item Test to identify significant interactions
    \end{itemize}
    
    \textbf{Further Directions}: 
    
    Pruning, GADGET as a predictor, comparing regions across models, efficient implementation, more efficient testing and splitting approach, $\dots$
     
            % \begin{itemize}
            %     \item $f(X) = 3X_1\mathbbm{1}_{X_3>0} - 3X_1\mathbbm{1}_{X_3\leq 0} + X_3 + \epsilon$
            %     \item Features uniformly distributed
            %     \item Left: features independent
            %     \item Right: $X_1$ and $X_3$ highly correlated
            %     \item First split: $\xv_3 \leq 0$ (blue) and $\xv_3 > 0$ (orange)
            % \end{itemize}
        \end{column}
        \begin{column}{0.45\textwidth}
        \centering
        \includegraphics[width=0.95\textwidth]{figure/sim1}
        \vspace{-8pt}
            \begin{columns}[T, totalwidth = \linewidth]
         \footnotesize
                \begin{column}{0.125\linewidth}
                \centering
                 $\hat{f}_2^{PD}(X_2)$ %, X_1 > 0 \land X_3 = 0\)\\
             \end{column}
             \begin{column}{0.175\linewidth}
             \centering
                 $\approx 8X_2$ %, X_1 > 0 \land X_3 = 0\)\\
             \end{column}
            \begin{column}{0.225\linewidth}
    \centering
                $\approx 16X_2$ %, X_1 \leq 0 \land X_3 = 0\)
             \end{column}
            \begin{column}{0.225\linewidth}
            \centering
                $ \approx -8X_2$ %, X_1 \leq 0 \land X_3 \neq 0\)
             \end{column}        
             \begin{column}{0.225\linewidth}
             \centering
                 $\approx 0$%, X_1 > 0 \land X_3 \neq 0\)\\
             \end{column}
            \begin{column}{0.025\linewidth}
    
             \end{column}
         \end{columns}
        \end{column}
    
    \end{columns}
    \bigskip
    \end{frame}
    
    
    
    
    
    