\documentclass[11pt,compress,t,notes=noshow, aspectratio=169, xcolor=table]{beamer}

\usepackage{../../style/lmu-lecture}
% Defines macros and environments
\usepackage{bbm}
% basic latex stuff
\newcommand{\pkg}[1]{{\fontseries{b}\selectfont #1}} %fontstyle for R packages
\newcommand{\lz}{\vspace{0.5cm}} %vertical space
\newcommand{\dlz}{\vspace{1cm}} %double vertical space
\newcommand{\oneliner}[1] % Oneliner for important statements
{\begin{block}{}\begin{center}\begin{Large}#1\end{Large}\end{center}\end{block}}


%new environments
\newenvironment{vbframe}  %frame with breaks and verbatim
{
 \begin{frame}[containsverbatim,allowframebreaks]
}
{
\end{frame}
}

\newenvironment{vframe}  %frame with verbatim without breaks (to avoid numbering one slided frames)
{
 \begin{frame}[containsverbatim]
}
{
\end{frame}
}

\newenvironment{blocki}[1]   % itemize block
{
 \begin{block}{#1}\begin{itemize}
}
{
\end{itemize}\end{block}
}

\newenvironment{fragileframe}[2]{  %fragile frame with framebreaks
\begin{frame}[allowframebreaks, fragile, environment = fragileframe]
\frametitle{#1}
#2}
{\end{frame}}


\newcommand{\myframe}[2]{  %short for frame with framebreaks
\begin{frame}[allowframebreaks]
\frametitle{#1}
#2
\end{frame}}

\newcommand{\remark}[1]{
  \textbf{Remark:} #1
}


\newenvironment{deleteframe}
{
\begingroup
\usebackgroundtemplate{\includegraphics[width=\paperwidth,height=\paperheight]{../style/color/red.png}}
 \begin{frame}
}
{
\end{frame}
\endgroup
}
\newenvironment{simplifyframe}
{
\begingroup
\usebackgroundtemplate{\includegraphics[width=\paperwidth,height=\paperheight]{../style/color/yellow.png}}
 \begin{frame}
}
{
\end{frame}
\endgroup
}\newenvironment{draftframe}
{
\begingroup
\usebackgroundtemplate{\includegraphics[width=\paperwidth,height=\paperheight]{../style/color/green.jpg}}
 \begin{frame}
}
{
\end{frame}
\endgroup
}
% https://tex.stackexchange.com/a/261480: textcolor that works in mathmode
\makeatletter
\renewcommand*{\@textcolor}[3]{%
  \protect\leavevmode
  \begingroup
    \color#1{#2}#3%
  \endgroup
}
\makeatother


\providecommand{\tightlist}{%
  \setlength{\itemsep}{0pt}\setlength{\parskip}{0pt}}

%\setbeamerfont{footnote}{size=\tiny}
\usepackage[hang,flushmargin]{footmisc}
\renewcommand*{\footnotelayout}{\tiny}
\renewcommand*{\thefootnote}{} %\fnsymbol{footnote}

% https://tex.stackexchange.com/questions/30720/footnote-without-a-marker
% \makeatletter
% \def\blfootnote{\gdef\@thefnmark{}\@footnotetext}
% \makeatother

% https://tex.stackexchange.com/questions/357717/beamer-allowframebreaks-option-and-vertical-spacing-when-using-lists-itemize
% \setbeamertemplate{frametitle continuation}{%
%     (\insertcontinuationcount)%
%     \ifnum\insertcontinuationcount>1%
%     \vspace*{\topsep}%
%     \else%
%     %
%     \fi%
% }


\title{Interpretable Machine Learning}
% \author{LMU}
%\institute{\href{https://compstat-lmu.github.io/lecture_iml/}{compstat-lmu.github.io/lecture\_iml}}
\date{}

\begin{document}

% Set style/preamble.Rnw as parent.
\newcommand{\vertiii}[1]{{\left\vert\kern-0.25ex\left\vert\kern-0.25ex\left\vert #1 
    \right\vert\kern-0.25ex\right\vert\kern-0.25ex\right\vert}}
% Load all R packages and set up knitr

% This file loads R packages, configures knitr options and sets preamble.Rnw as 
% parent file
% IF YOU MODIFY THIS, PLZ ALSO MODIFY setup.Rmd ACCORDINGLY...

% Defines macros and environments
 \newcommand{\titlefigure}{figure/AEloanApplication.png} 
\newcommand{\learninggoals}{
\item Compare adversarial examples to counterfactual explanations
\item See an example where both coincident}

\lecturechapter{\Large{Adversarial Examples and counterfactual explanations}}
\lecture{Interpretable Machine Learning}

% ------------------------------------------------------------------------------

\begin{vbframe}{ADE and Counterfactual Explanations}
It seems as if ADEs and counterfactual explanations (CEs) are defined similarly. Both ADEs and CEs describe inputs close to a given input $\xv$ that gets a different assignment. What are their differences?
\begin{itemize}
    \item Counterfactuals do not have to be misclassified.
%    \item Counterfactuals should be maximally close to $\xv$.
    \item Different notions of distance $\|\cdot\|$ are applied, e.g., $p_{2,\infty}$-norm for ADEs or $p_{0,1}$-norm for CEs.
    \item Informal difference I: ADEs are mostly considered for high-dimensional data, while CEs are mostly considered in the context of low-dimensional data.
    \item Informal difference II: ADEs hide changes while CEs highlight them.
\end{itemize}
\end{vbframe}

\begin{vbframe}{Shared Example \citebutton{Ballet (2019)}{https://arxiv.org/pdf/1911.03274.pdf}}
\begin{itemize}
\item \textbf``If you had two more pets, your loan application would have been granted" is an example of both ADEs and CEs.
\end{itemize}
\begin{figure}[h]
\centering
\includegraphics[width=0.6\linewidth]{figure/AEloanApplication.png}\\
   \centering
  {Decision boundary of a classifier deciding loan applications. ADE via ``number of pets''}
  \label{fig:mnist}
\end{figure} 
\end{vbframe}

\endlecture
\end{document}

