\documentclass{article}
\usepackage{hyperref}       % hyperlinks
\usepackage{graphicx} % Required for inserting images
\usepackage{xcolor}         % colors

%%%%%%%% MY PACKAGES

\usepackage[utf8]{inputenc} % allow utf-8 input
\usepackage[T1]{fontenc}    % use 8-bit T1 fonts
\usepackage{url}            % simple URL typesetting
\usepackage{booktabs}       % professional-quality tables
\usepackage{amsfonts}       % blackboard math symbols
\usepackage{amsmath}
\usepackage{nicefrac}       % compact symbols for 1/2, etc.
\usepackage{microtype}      % microtypography
\usepackage{xcolor}         % colors
%\usepackage[pdftex]{graphicx}
%\usepackage{ngerman} % für deutsche Anführungszeichen
%\usepackage[inline]{enumitem} % inline enumeration
%\usepackage{dsfont}         % Real numbers symbol etc. => wieder entfernt, nehme stattdessen \mathbb{R} aus dem amsfonts package wie von neurips vorgeschlagen
\usepackage{caption}        % For subfigures and captions
\usepackage{subcaption}     % For subfigures and captions
\usepackage{algorithm}
\usepackage{algpseudocode}
% \usepackage[linesnumbered,ruled,vlined]{algorithm2e}        % For algos with referencable line numbers
\usepackage{tikz}
\usetikzlibrary{shapes.geometric}
\usepackage{multirow}
%\usepackage{pdfpages}
\usepackage{amsthm, amssymb}
\usepackage{bm}
\usepackage{pifont}         % For checkmark symbols with \ding{...}
%\usepackage{ulem}
\usepackage{enumitem}   % Custom enumerate labels
\usepackage[textsize=small]{todonotes}


\newtheorem{definition}{Definition}[section] % Defines the definition environment
\newtheorem{proposition}[definition]{Proposition} % Defines the proposition environment, numbering it along with definitions
\newtheorem{example}[definition]{Example}

% Set paragraph indent to zero
\setlength{\parindent}{0pt}

\include{latex-math/basic-math}
\include{latex-math/basic-ml}

\begin{document}

\newtheorem{theorem}{Theorem}
\newtheorem{proof_sketch}{Proof}
\newtheorem{proofsketch}{Proof (Step-by-Step)}

\textit{TO DO: all the proofs there are very brief or partially false, see the solution to the homework sheet for a correct and complete solution for all axioms}

\section*{Set-based Shapley Values for Privilege Score Contributions}

\textbf{Setup.} 
$k \hat{=} |P|$. Let the Shapley value for $j \in P$ be
\[
\phi_j \;=\;
\sum_{S \subseteq P \setminus \{j\}}
\frac{|S|!\,\bigl(k - |S| -1\bigr)!}{k!}
\;
\bigl[v(S \cup \{j\}) - v(S)\bigr].
\]

\begin{theorem}[Efficiency]
Let $\{\phi_j\}_{j \in P}$ be the Shapley values induced by $v$. Then
\[
\sum_{j=1}^k \phi_j
\;=\;
v(P).
\]
\end{theorem}

\begin{proof_sketch}
The Shapley value for player $j \in P$ is 
\[
\phi_j 
\;=\;
\sum_{S \subseteq P \setminus \{j\}}
\frac{|S|!\,\bigl(k - |S| - 1\bigr)!}{k!}
\;\Bigl[\,v\bigl(S \cup \{j\}\bigr) \;-\; v(S)\Bigr].
\]
Hence
\[
\sum_{j=1}^k \phi_j
\;=\;
\sum_{j=1}^k
\sum_{S \subseteq P \setminus \{j\}}
\frac{|S|!\,(k - |S| - 1)!}{k!}
\;\bigl[v(S \cup \{j\}) \;-\; v(S)\bigr].
\]
Re-index over nonempty subsets $T \subseteq P$ by setting $T = S \cup \{j\}$. This shows each $v(T)$ with $T\neq \varnothing$ arises exactly $|T|$ times in the sum (once for each $j \in T$), but for subsets $T \neq P$, there is also a corresponding negative occurrence that cancels out the positive one. Concretely:
\[
\bigl[v(T) - v(T \setminus \{i\})\bigr]
\quad \text{and} \quad
\bigl[v(T \cup \{i'\}) - v(T)\bigr]
\]
yield $+\,v(T)$ and $-\,v(T)$, respectively, for some $i,i'\notin T$. 
All intermediate $T \neq P$ vanish in pairs, leaving only the net term $v(P) - v(\varnothing)$. Because $v(\varnothing)=0$, the total sum equals $v(P)$. 
\[
\sum_{j=1}^k \phi_j \;=\; v(P).
\]
\end{proof_sketch}

\begin{proofsketch}
\noindent \textbf{Step 1: Split the sum into pieces.}
\[
\sum_{j=1}^k \phi_j 
\;=\;
\sum_{j=1}^k 
\;\sum_{\,S \subseteq P\setminus\{j\}}
\;\frac{|S|!\,\bigl(k - |S| -1\bigr)!}{k!}
\;\Bigl[\,v\bigl(S \cup \{j\}\bigr) - v(S)\Bigr].
\]
Each pair $(j,S)$ contributes a difference $v(S \cup \{j\}) - v(S)$. Let $T = S \cup \{j\}$.

\medskip
\noindent
\textbf{Step 2: Identify how $v(P)$ appears exactly once overall.}

\begin{itemize}
    \item Whenever $T = P$, we must have $j \in P$ and $S = P \setminus \{j\}$. There are exactly $k$ such pairs \((j,S)\) because $j$ can be any of the $k$ elements in $P$.
    \item For each such pair, $\lvert S\rvert = k-1$, so the coefficient becomes
    \[
       \frac{(k-1)!\,(k - (k-1) - 1)!}{k!}
       \;=\;
       \frac{(k-1)!\,(0)!}{k!}
       \;=\;
       \frac{1}{k}.
    \]
    \item Hence each of the $k$ occurrences of $v(P)$ is multiplied by $\tfrac{1}{k}$, summing to $\bigl(\frac{1}{k}\times k\bigr)=1$. Thus \emph{overall}, $v(P)$ is counted once in total. 
\end{itemize}

\medskip
\noindent
\textbf{Step 3: Show that all $v(T)$ with $T \neq P$ cancel.}

For each nonempty $T \subset P$:
% \begin{itemize}
%     \item \textit{Positive occurrence.} If $T = S \cup \{j\}$ for some $j \in T$, then $v(T)$ appears with a $+$ sign in the term $v(S \cup \{j\}) - v(S)$. 
%     \item \textit{Negative occurrence.} If \(T \subset P\) is not the full set, there is at least one element \(j' \notin T\). Hence, we can form the larger set \(T \cup \{j'\}\). In the Shapley sum, this larger set contributes a term
% \[
% v\bigl(T \cup \{j'\}\bigr) \;-\; v(T),
% \]
% which includes precisely the \textit{negative occurrence} of \(v(T)\).  

%     \item Check the multinomial coefficient: Consider a fixed nonempty $T$. Its size is $|T|$. In the inner sum, $T$ arises precisely once for the unique $j \in T$ and $S = T \setminus \{j\}$. The corresponding coefficient is
% \[
% \frac{|T\setminus\{j\}|!\,\bigl(k - |T\setminus\{j\}| -1\bigr)!}{k!}
% \;=\;
% \frac{(|T|-1)!\,\bigl(k - (|T|-1) -1\bigr)!}{k!}
% \;=\;
% \frac{(|T|-1)!\,(k - |T|)!}{k!}.
% \]
% Multiplying by $|T|$ (because there are $|T|$ ways to pick $j$ in $T$ as shown in Step 1) yields
% \[
% |T| \;\cdot\; \frac{(|T|-1)!\,(k - |T|)!}{k!}
% \;=\;
% \frac{|T|! \,(k - |T|)!}{k!}
% \;=\;
% 1.
% \]
% Thus each $v(T)$ with $T\neq \varnothing$ appears with a total net weight of $+1$ (when summing across all $j\in T$).
%     \item Every nonempty \(T \neq P\) appears positively in some difference \(v(T) - v(\cdot)\) and also appears negatively in a difference \(v(\cdot) - v(T)\) as part of a larger set. These two occurrences nullify each other, ensuring that \(v(T)\) does not remain in the final sum.
% \end{itemize}
\begin{itemize}
    \item \textbf{Positive occurrence.} 
    If $T = S \cup \{j\}$ for some $j \in T$, then $v(T)$ appears \emph{positively} in the difference 
    \[
      v(S \cup \{j\}) \;-\; v(S).
    \]

    \item \textbf{Negative occurrence.} 
    If \(T \subset P\) is not the full set, then there exists at least one element \(j' \notin T\). We can thus form the larger set \(T \cup \{j'\}\). In the Shapley sum, this larger set contributes
    \[
      v\bigl(T \cup \{j'\}\bigr) \;-\; v(T),
    \]
    which contains \(-\,v(T)\) as the \emph{negative} appearance of \(v(T)\).

    \item \textbf{Multinomial coefficient.} 
    Fix a nonempty $T$. Let $|T|$ denote its size. Inside the sum, $T$ appears once for each $j \in T$ with $S = T \setminus \{j\}$. The coefficient in front of $v(T)$ for such a pair $(j,S)$ is
    \[
      \frac{|T\setminus\{j\}|!\,\bigl(k - |T\setminus\{j\}| -1\bigr)!}{k!}
      \;=\;
      \frac{(|T|-1)!\,(k - |T|)!}{k!}.
    \]
    Since there are $|T|$ possible ways to choose $j\in T$, multiplying by $|T|$ yields
    \textit{TO DO The following calculation is wrong, this fraction yields $\frac{1}{\binom{k}{|T|}}$, for exactly the reason shown in the proof}
    \[
      |T| 
      \;\cdot\; 
      \frac{(|T|-1)!\,(k - |T|)!}{k!}
      \;=\;
      \frac{|T|!\,(k - |T|)!}{k!}
      \;=\;
      \frac{k!}{k!}
      \;=\;
      1.
    \]
    Where the identity $|T|!\,(k - |T|)! = k!$ holds because to permute $k$ distinct elements, 
we can imagine first choosing which $|T|$ are in one group and ordering them 
(in $|T|!$ ways), then ordering the remaining $k - |T|$ (in $(k - |T|)!$ ways). 
Overall, this accounts for all $k!$ permutations. 
    Therefore, each nonempty $T$ receives a total \emph{positive} weight of exactly $+1$ when summing over all $j \in T$.

    \item \textbf{Cancellation.} 
    \textit{TO DO: The following is completely handwaivy, it is totally unclear why the weights are the same, so that these terms really cancel out to 0}
    Every nonempty \(T \neq P\) also appears \emph{negatively} as part of a larger set’s difference, ensuring one positive and one negative occurrence of $v(T)$. Consequently, these contributions cancel each other out, so \(v(T)\) does not remain in the final sum unless $T=P$.
\end{itemize}
\medskip
\noindent
\textbf{Step 4: Conclusion.}

Since $v(\varnothing) = 0$ does not contribute and $v(P)$ remains once in total, we get
\[
\sum_{j=1}^k \phi_j 
\;=\;
\underbrace{v(P)}_{\text{one net positive}} 
\;+\; 
\underbrace{\sum_{T \neq P} \bigl[v(T)\ \text{terms cancel}\bigr]}_{0} 
\;=\;
v(P).
\]
\qedhere
\end{proofsketch}

% \begin{proofsketch}
% \noindent \textbf{Step 1: Expand the total Shapley sum.}

% \[
% \sum_{j=1}^k \phi_j
% \;=\;
% \sum_{j=1}^k
% \;\sum_{\,S \subseteq P \setminus \{j\}}
% \frac{|S|!\,(k-|S|-1)!}{k!}
% \;\Bigl[v\bigl(S \cup \{j\}\bigr) - v(S)\Bigr].
% \]

% \noindent \textbf{Step 2: Re-index to track each $v(T)$ term.}

% Each nonempty subset \(T \subseteq P\) appears exactly \(\lvert T\rvert\) times positively in the expanded total Shapley sum from Step 1. More precisely, for each \(j \in T\), there is a unique set \(S = T \setminus \{j\}\) such that \(T = S \cup \{j\}\). Consequently, each \(v(T)\) appears exactly once in the difference \(v(S \cup \{j\}) - v(S)\) if and only if \(S = T \setminus \{j\}\).

% \textbf{Example:} 
% Let $P = \{1,2,3\}$ and focus on $T = \{1,3\}$.  
% In the Shapley sum, we form terms $v(S \cup \{j\}) - v(S)$ for each $j \in P$ and each $S \subseteq P \setminus \{j\}$. 
% We list all such $(j,S)$ pairs:

% \begin{itemize}
%   \item \textbf{Case $j=1$}, so $S \subseteq \{2,3\}$:
%     \begin{itemize}
%       \item $S = \varnothing$: $S \cup \{1\} = \{1\} \implies v(\{1\}) - v(\varnothing).$
%       \item $S = \{2\}$: $S \cup \{1\} = \{1,2\} \implies v(\{1,2\}) - v(\{2\}).$
%       \item \textbf{$S = \{3\}$:} $S \cup \{1\} = \{1,3\} \implies \textbf{v(\{1,3\}) - v(\{3\})}.$
%       \item $S = \{2,3\}$: $S \cup \{1\} = \{1,2,3\} \implies v(\{1,2,3\}) - v(\{2,3\}).$
%     \end{itemize}

%   \item \textbf{Case $j=2$}, so $S \subseteq \{1,3\}$:
%     \begin{itemize}
%       \item $S = \varnothing$: $S \cup \{2\} = \{2\} \implies v(\{2\}) - v(\varnothing).$
%       \item $S = \{1\}$: $S \cup \{2\} = \{1,2\} \implies v(\{1,2\}) - v(\{1\}).$
%       \item $S = \{3\}$: $S \cup \{2\} = \{2,3\} \implies v(\{2,3\}) - v(\{3\}).$
%       \item $S = \{1,3\}$: $S \cup \{2\} = \{1,2,3\} \implies v(\{1,2,3\}) - v(\{1,3\}).$
%     \end{itemize}

%   \item \textbf{Case $j=3$}, so $S \subseteq \{1,2\}$:
%     \begin{itemize}
%       \item $S = \varnothing$: $S \cup \{3\} = \{3\} \implies v(\{3\}) - v(\varnothing).$
%       \item \textbf{$S = \{1\}$:} $S \cup \{3\} = \{1,3\} \implies \textbf{v(\{1,3\}) - v(\{1\})}.$
%       \item $S = \{2\}$: $S \cup \{3\} = \{2,3\} \implies v(\{2,3\}) - v(\{2\}).$
%       \item $S = \{1,2\}$: $S \cup \{3\} = \{1,2,3\} \implies v(\{1,2,3\}) - v(\{1,2\}).$
%     \end{itemize}
% \end{itemize}

% \noindent
% \textbf{Observation:} $T = \{1,3\}$ occurs \emph{only} in $|T| = 2$ cases, i.e. for the pairs
% \[
% (j,S) = (1,\{3\}) \quad \text{and} \quad (j,S) = (3,\{1\}),
% \]
% matching the principle: $T$ appears once for each $j\in T$, with $S = T \setminus\{j\}$.  
% No other $(j,S)$ pair produces $T=\{1,3\}$, so $T$ cannot arise for $j=2$ or for different $S$.



% \noindent \textbf{Step 3: Check the multinomial coefficient contribution.}

% Consider a fixed nonempty $T$. Its size is $|T|$. In the inner sum, $T$ arises precisely once for the unique $j \in T$ and $S = T \setminus \{j\}$. The corresponding coefficient is
% \[
% \frac{|T\setminus\{j\}|!\,\bigl(k - |T\setminus\{j\}| -1\bigr)!}{k!}
% \;=\;
% \frac{(|T|-1)!\,\bigl(k - (|T|-1) -1\bigr)!}{k!}
% \;=\;
% \frac{(|T|-1)!\,(k - |T|)!}{k!}.
% \]
% Multiplying by $|T|$ (because there are $|T|$ ways to pick $j$ in $T$ as shown in Step 1) yields
% \[
% |T| \;\cdot\; \frac{(|T|-1)!\,(k - |T|)!}{k!}
% \;=\;
% \frac{|T|! \,(k - |T|)!}{k!}
% \;=\;
% 1.
% \]
% Thus each $v(T)$ with $T\neq \varnothing$ appears with a total net weight of $+1$ (when summing across all $j\in T$).

% \noindent \textbf{Step 4: Telescoping.}

% \begin{itemize}
%     \item $v(\varnothing)$ never appears positively, because $\varnothing$ cannot be formed by adding some $j$.
%     \item $v(P)$ never appears negatively, because $P$ cannot be “further extended.”
%     \item All intermediate $v(T)$ terms with $T \neq \varnothing,P$ appear exactly once positively and once negatively, and hence cancel out except for $v(P) - v(\varnothing)$.
% \end{itemize}

% \noindent \textbf{Step 5: Conclude.}

% Since $v(\varnothing) = 0$, the only remaining term is $v(P)$. Hence
% \[
% \sum_{j=1}^k \phi_j \;=\; v(P).
% \]
% \qedhere
% \end{proofsketch}


\section*{Order-Based Shapley Values for Privilege Score Contributions}

\paragraph{Setup.}
Let $P = \{1,\dots,k\}$ be the set of $k$ players.  
We aim to allocate $v(P)$ among the $k$ players (arrows) in a principled way.

\medskip

\paragraph{Permutation-Based Definition of Shapley Values.}
A \emph{permutation} of $P$ is any ordering $\pi = (\pi(1), \pi(2), \dots, \pi(k))$.  
Let $\mathrm{Pred}_\pi(j)$ be the set of players that appear \emph{before} $j$ in the permutation $\pi$.  
Then the \textbf{Shapley value} of player $j$ is
\[
\phi_j 
\;=\;
\frac{1}{k!}\;
\sum_{\pi \in \mathfrak{S}_P}
\Bigl[v\!\bigl(\mathrm{Pred}_\pi(j) \cup \{\,j\}\bigr)
\;-\;
v\!\bigl(\mathrm{Pred}_\pi(j)\bigr)\Bigr],
\]
where $\mathfrak{S}_P$ is the set of all $k!$ permutations of $P$.  
In words, we look at the “marginal contribution” of $j$ each time it arrives in a permutation (given whichever players arrived before it), and then average over all permutations.

\begin{theorem}[Efficiency]
Let $\{\phi_j\}_{j \in P}$ be the order-based Shapley values induced by $v$. Then
\[
\sum_{j=1}^k \phi_j
\;=\;
v(P).
\]
\end{theorem}

\begin{proofsketch}
\noindent
\textbf{Step 1: Sum the Shapley values over all players.}

\[
\sum_{j=1}^k \phi_j 
\;=\;
\sum_{j=1}^k
\frac{1}{k!}
\sum_{\pi \in \mathfrak{S}_P}
\Bigl[v\bigl(\mathrm{Pred}_\pi(j)\cup\{j\}\bigr)
- 
v\bigl(\mathrm{Pred}_\pi(j)\bigr)\Bigr].
\]
Swap the sums:
\[
=
\frac{1}{k!}
\sum_{\pi \in \mathfrak{S}_P}
\sum_{j=1}^k
\Bigl[v\bigl(\mathrm{Pred}_\pi(j)\cup\{j\}\bigr)
- 
v\bigl(\mathrm{Pred}_\pi(j)\bigr)\Bigr].
\]

\medskip
\noindent
\textbf{Step 2: Telescoping within each permutation.}

Fix a particular permutation $\pi$.  List its elements in order:
\[
\bigl(\pi(1), \pi(2), \dots, \pi(k)\bigr).
\]
Within this permutation, the inner sum
\(\sum_{j=1}^k
\bigl[v(\mathrm{Pred}_\pi(j)\cup\{j\}) - v(\mathrm{Pred}_\pi(j))\bigr]\)
can be viewed as a chain of marginal contributions:
\[
v\bigl(\{\pi(1)\}\bigr) - v(\varnothing)
\;+\;
v\bigl(\{\pi(1), \pi(2)\}\bigr) - v\bigl(\{\pi(1)\}\bigr)
\;+\;\dots\;+\;
v\bigl(\{\pi(1),\ldots,\pi(k)\}\bigr) - v\bigl(\{\pi(1),\ldots,\pi(k-1)\}\bigr).
\]
All intermediate terms telescope, leaving exactly
\[
v\bigl(\{\pi(1),\ldots,\pi(k)\}\bigr)
\;-\;
v(\varnothing)
\;=\;
v(P)
\;-\;
0
\;=\;
v(P).
\]

\medskip
\noindent
\textbf{Step 3: Average over all permutations.}

Since every permutation $\pi$ yields exactly $v(P)$ in the telescoped sum, we have
\[
\sum_{j=1}^k \phi_j
\;=\;
\frac{1}{k!}
\sum_{\pi \in \mathfrak{S}_P} 
v(P)
\;=\;
\frac{1}{k!} \times \bigl(k! \cdot v(P)\bigr)
\;=\;
v(P).
\]
Hence
\[
\sum_{j=1}^k \phi_j 
\;=\;
v(P).
\]
\end{proofsketch}

\end{document}
