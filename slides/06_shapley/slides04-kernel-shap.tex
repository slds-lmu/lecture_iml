\documentclass[11pt,compress,t,notes=noshow, aspectratio=169, xcolor=table]{beamer}
\newcommand{\btVFill}{\vskip0pt plus 1filll}
\newcommand\hmmax{0}
\newcommand\bmmax{0}
\usepackage{../../style/lmu-lecture}
\usepackage{siunitx}
% Defines macros and environments
\usepackage{bbm}
% basic latex stuff
\newcommand{\pkg}[1]{{\fontseries{b}\selectfont #1}} %fontstyle for R packages
\newcommand{\lz}{\vspace{0.5cm}} %vertical space
\newcommand{\dlz}{\vspace{1cm}} %double vertical space
\newcommand{\oneliner}[1] % Oneliner for important statements
{\begin{block}{}\begin{center}\begin{Large}#1\end{Large}\end{center}\end{block}}


%new environments
\newenvironment{vbframe}  %frame with breaks and verbatim
{
 \begin{frame}[containsverbatim,allowframebreaks]
}
{
\end{frame}
}

\newenvironment{vframe}  %frame with verbatim without breaks (to avoid numbering one slided frames)
{
 \begin{frame}[containsverbatim]
}
{
\end{frame}
}

\newenvironment{blocki}[1]   % itemize block
{
 \begin{block}{#1}\begin{itemize}
}
{
\end{itemize}\end{block}
}

\newenvironment{fragileframe}[2]{  %fragile frame with framebreaks
\begin{frame}[allowframebreaks, fragile, environment = fragileframe]
\frametitle{#1}
#2}
{\end{frame}}


\newcommand{\myframe}[2]{  %short for frame with framebreaks
\begin{frame}[allowframebreaks]
\frametitle{#1}
#2
\end{frame}}

\newcommand{\remark}[1]{
  \textbf{Remark:} #1
}


\newenvironment{deleteframe}
{
\begingroup
\usebackgroundtemplate{\includegraphics[width=\paperwidth,height=\paperheight]{../style/color/red.png}}
 \begin{frame}
}
{
\end{frame}
\endgroup
}
\newenvironment{simplifyframe}
{
\begingroup
\usebackgroundtemplate{\includegraphics[width=\paperwidth,height=\paperheight]{../style/color/yellow.png}}
 \begin{frame}
}
{
\end{frame}
\endgroup
}\newenvironment{draftframe}
{
\begingroup
\usebackgroundtemplate{\includegraphics[width=\paperwidth,height=\paperheight]{../style/color/green.jpg}}
 \begin{frame}
}
{
\end{frame}
\endgroup
}
% https://tex.stackexchange.com/a/261480: textcolor that works in mathmode
\makeatletter
\renewcommand*{\@textcolor}[3]{%
  \protect\leavevmode
  \begingroup
    \color#1{#2}#3%
  \endgroup
}
\makeatother


\providecommand{\tightlist}{%
  \setlength{\itemsep}{0pt}\setlength{\parskip}{0pt}}

%\setbeamerfont{footnote}{size=\tiny}
\usepackage[hang,flushmargin]{footmisc}
\renewcommand*{\footnotelayout}{\tiny}
\renewcommand*{\thefootnote}{} %\fnsymbol{footnote}

% https://tex.stackexchange.com/questions/30720/footnote-without-a-marker
% \makeatletter
% \def\blfootnote{\gdef\@thefnmark{}\@footnotetext}
% \makeatother

% https://tex.stackexchange.com/questions/357717/beamer-allowframebreaks-option-and-vertical-spacing-when-using-lists-itemize
% \setbeamertemplate{frametitle continuation}{%
%     (\insertcontinuationcount)%
%     \ifnum\insertcontinuationcount>1%
%     \vspace*{\topsep}%
%     \else%
%     %
%     \fi%
% }


\title{Interpretable Machine Learning}
% \author{LMU}
%\institute{\href{https://compstat-lmu.github.io/lecture_iml/}{compstat-lmu.github.io/lecture\_iml}}
\date{}

\begin{document}

% TODO
\newcommand{\titlefigure}{figure_man/exSHAP.png}
\newcommand{\learninggoals}{
\item Get an intuition of additive feature attributions
\item Understand the concept of Kernel SHAP
\item Ability to interpret SHAP plots
\item Global SHAP methods
}

\lecturechapter{SHAP (SHapley Additive exPlanation) Values}
\lecture{Interpretable Machine Learning}

%\begin{frame}{Example}

%Given the following example from the bike sharing data set

%\begin{table}[h]
%\centering
%\begin{tabular}{l rrrrr || r}
%  \hline
%  && temperature & humidity & windspeed & year & prediction\\ 
%  \hline
% example $x_{ex}$ && 24.27 & 58.5 & 13.96 & 2011 & 6825 \\ 
% \hline
%\end{tabular}
%\end{table}

%we are searching for Shapley values such that
%\begin{equation}
%\begin{array}{lllllcr}
%\phi_0 &+ \phi_{temp} &+ \phi_{hum} &+ \phi_{windspeed} &+ \phi_{yr} & = &\hat{y} \\
%4469 &+ 1809 &+ 450 &+ 241 &- 144 & = & 6825
%\end{array}
%\end{equation}

%\begin{figure}
 %   \centering
 %   \includegraphics[width=\columnwidth]{figure_man/exSHAP.png}
%\end{figure}
%\end{frame}

\begin{frame}{Kernel SHAP - In 5 Steps}

\textbf{Definition:} A kernel-based, model-agnostic method to compute Shapley values via local surrogate models (e.g. linear model)\\
\vspace{1cm}
\begin{enumerate}
    \item Sample coalition vectors  \(\zv'\!\in\!\{0,1\}^p\)
    %\begin{onlyenv}<1>
   % $$z_{k}^{\prime} \in\{0,1\}^{M}, \quad k \in\{1, \ldots, K\}$$
    %\end{onlyenv}
    
    \item Map coalition vectors to original feature space and predict
    %Transfer coalitions into feature space \& get predictions by applying ML model
    
 %   \begin{onlyenv}<2>
  %  $$\hat{f}: \hat{f}\left(h_{x}\left(z_{k}^{\prime}\right)\right)$$
   % \end{onlyenv}
    
    \item Compute kernel weights for surrogate model
 %   \begin{onlyenv}<3>
 %   $$\pi_{x}\left(z^{\prime}\right)=\frac{(M-1)}{\left(\begin{array}{c} M \\\left|z^{\prime}\right|\end{array}\right)\left|z^{\prime}\right|\left(M-\left|z^{\prime}\right|\right)}$$
 %   \end{onlyenv}
    
    \item Fit a weighted linear model 
  %  \begin{onlyenv}<4>
  %  $$L\left(\hat{f}, g, \pi_{x}\right)=\sum_{z^{\prime} \in Z}\left[\hat{f}\left(h_{x}\left(z^{\prime}\right)\right)-g\left(z^{\prime}\right)\right]^{2} \pi_{x}\left(z^{\prime}\right)$$
  %  \end{onlyenv}

    \item Return Shapley values
%    \begin{onlyenv}<5>
%    $$(\phi_1, \ldots, \phi_M)$$
%    \end{onlyenv}
    
    
\end{enumerate}

\end{frame}



% \begin{frame}{Kernel SHAP in 5 Steps}

% \textbf{Goal:} Estimate Shapley values for a fixed instance \(\xv\) without model restrictions.

% \medskip
% \begin{enumerate}
%   \item \textbf{Sample coalitions}  
%         Draw \(K\) binary masks  
%         \(\zv'^{(k)}\!\in\!\{0,1\}^p\) (\(k=1,\dots,K\)); each mask marks a subset of “known” features.

%   \item \textbf{Create coalition inputs \& predict}  
%         Build \(\tilde{\xv}^{(k)}\) by  
%         \(\tilde{x}^{(k)}_j = x_j\) if \(z'^{(k)}_j=1\); otherwise impute.  
%         Evaluate the model: \(y^{(k)} = \fh\!\bigl(\tilde{\xv}^{(k)}\bigr)\).

%   \item \textbf{Assign kernel weights}  
%         \[
%           \pi_x(\zv') = \frac{p-1}{\binom{p}{|\zv'|}\,|\zv'|\,(p-|\zv'|)}
%         \]

%   \item \textbf{Fit weighted surrogate}  
%         Weighted least squares for  
%         \(g(\zv') = \phi_0 + \sum_{j=1}^p \phi_j z'_j\):
%         \[
%           \min_{\phi}\sum_{k=1}^K \bigl[y^{(k)} - g(\zv'^{(k)})\bigr]^2\,\pi_x(\zv'^{(k)})
%         \]

%   \item \textbf{Return Shapley estimates}  
%         Coefficients \(\phi_j(\xv)\) (and \(\phi_0\)) are the Kernel-SHAP attributions.
% \end{enumerate}

% \end{frame}


\begin{frame}{Kernel SHAP - In 5 Steps}


\textbf{Step 1: Sample coalition vectors}
\begin{itemize}
    \item Sample K coalitions from the simplified (binary) feature space
    $$\mathbf{z}^{\prime (k)} \in\{0,1\}^{p}, \quad k \in\{1, \ldots, K\}$$
    \item Example (3 features) $\Rightarrow$ $2^p = 2^3 = 8$ coalitions (without sampling)
    %For our simple example, we have in total $2^p = 2^3 = 8$ coalitions (without sampling)
\end{itemize}

\begin{table}[]
    \centering
     \begin{tabular}{l |c|ccc}
  Coalition  & $\mathbf{z}^{\prime (k)}$ &  hum & temp & ws \\
  \hline 
  $\varnothing$ & $\mathbf{z}^{\prime (1)}$ & 0 & 0 & 0  \\
  hum & $\mathbf{z}^{\prime (2)}$ & 1 & 0 & 0  \\
  temp &  $\mathbf{z}^{\prime (3)}$ & 0 & 1 & 0  \\
  ws &   $\mathbf{z}^{\prime (4)}$ & 0 & 0 & 1  \\
  hum, temp & $\mathbf{z}^{\prime (5)}$ & 1 & 1 & 0  \\
  temp, ws & $\mathbf{z}^{\prime (6)}$ & 0 & 1 & 1  \\
  hum, ws &   $\mathbf{z}^{\prime (7)}$ & 1 & 0 & 1  \\
  hum, temp, ws & $\mathbf{z}^{\prime (8)}$ & 1 & 1 & 1  \\
  
 
  \end{tabular}
\end{table}

\end{frame}

\begin{frame}{Kernel SHAP - In 5 Steps}


\textbf{Step 2: Map coalition vectors to original feature space and predict}
\begin{itemize}
   % \item $$\hat{f}: \hat{f}\left(h_{x}\left(z_{k}^{\prime}\right)\right)$$
   \item $\mathbf{z}^{\prime (k)}$ is 1 if features are are part of the $k$-th coalition, 0 if they are absent
   \item To calculate predictions for these coalitions, we need to define a function which maps the binary feature space back to the original feature space
\end{itemize}


\begin{tikzpicture}
\centering

\fontsize{8}{12}

\node (tab1) {%
       \begin{tabular}{l |cccc}
  $\xv^{coalition}$ &  hum & temp & ws \\
  \hline 
  $\xv^{\{\varnothing\}}$ & $\varnothing$ & $\varnothing$ &$\varnothing$  \\
   $\xv^{\{hum\}}$ & 51.6 & $\varnothing$ & $\varnothing$  \\
    $\xv^{\{temp\}}$ & $\varnothing$ & 5.1 & $\varnothing$  \\
     $\xv^{\{ws\}}$ & $\varnothing$ & $\varnothing$ & 17.0  \\
     $\xv^{\{hum, temp\}}$ & 51.6 & 5.1 & $\varnothing$  \\
     $\xv^{\{temp, ws\}}$ &$\varnothing$ & 5.1 & 17.0  \\
     $\xv^{\{hum, ws\}}$ & 51.6 & $\varnothing$ & 17.0  \\
  $\xv^{\{hum, temp, ws\}}$ &51.6 & 5.1 & 17.0   \\
  
 
  \end{tabular}};

\node [left=of tab1] (tab2) {%
     \begin{tabular}{l |c|ccc}
  Coalition & $\mathbf{z}^{\prime (k)}$ &  hum & temp & ws \\
  \hline 
  $\varnothing$ & $\mathbf{z}^{\prime (1)}$ & 0 & 0 & 0  \\
  hum & $\mathbf{z}^{\prime (2)}$ & 1 & 0 & 0  \\
  temp &  $\mathbf{z}^{\prime (3)}$ & 0 & 1 & 0  \\
  ws &   $\mathbf{z}^{\prime (4)}$ & 0 & 0 & 1  \\
  hum, temp & $\mathbf{z}^{\prime (5)}$ & 1 & 1 & 0  \\
  temp, ws & $\mathbf{z}^{\prime (6)}$ & 0 & 1 & 1  \\
  hum, ws &   $\mathbf{z}^{\prime (7)}$ & 1 & 0 & 1  \\
  hum, temp, ws & $\mathbf{z}^{\prime (8)}$ & 1 & 1 & 1  \\
   
 
  \end{tabular}};
\draw[->]
(tab2.north) to[out=10,in=170] node[below]{} (tab1.north) ;
\end{tikzpicture}


   




\end{frame}


\begin{frame}{Kernel SHAP - In 5 Steps}


\textbf{Step 2: Map coalition vectors to original feature space and predict}
\begin{itemize}
   % \item $$\hat{f}: \hat{f}\left(h_{x}\left(z_{k}^{\prime}\right)\right)$$
    \item Define 
$h_x\left(\mathbf{z}^{\prime (k)}\right)=\mathbf{z}^{(k)} \text { where } h_x:\{0,1\}^{p} \rightarrow \R^{p}$
 maps 1’s to feature values of observation $\xv$ for features part of the $k$-th coalition and 0's to feature values of a \color{orange}{randomly sampled observation} \color{black}for features absent in the $k$-th coalition
  % \item Absent feature values are replaced by feature values of a \color{orange}{random observation} \color{black} of the dataset (permuted) $\leadsto$ permute feature values several times
  (feature values are permuted multiple times) 
   \item Predict with ML model on this dataset $\hat{f}: \hat{f}\left(h_{x}\left(\mathbf{z}^{\prime (k)}\right)\right)$
\end{itemize}


\begin{tikzpicture}
\centering

\fontsize{7}{12}

\node (tab1) {%
       \begin{tabular}{l |ccc | c}
  $\mathbf{z}^{(k)}$ &  hum & temp & ws & $\hat{f}\left(h_{x}\left(\mathbf{z}^{\prime (k)}\right)\right)$\\
  \hline 
  $\mathbf{z}^{(1)}$ & \color{orange}{64.3} & \color{orange}{28.0} & \color{orange}{14.5} & 6211 \\
   $\mathbf{z}^{(2)}$ & 51.6 & \color{orange}{28.0} & \color{orange}{14.5} & 5586  \\
    $\mathbf{z}^{(3)}$ & \color{orange}{64.3} & 5.1 & \color{orange}{14.5}  & 3295\\
     $\mathbf{z}^{(4)}$ & \color{orange}{64.3} & \color{orange}{28.0} & 17.0 &5762 \\
     $\mathbf{z}^{(5)}$ & 51.6 & 5.1 & \color{orange}{14.5}  & 2616\\
     $\mathbf{z}^{(6)}$ &\color{orange}{64.3} & 5.1 & 17.0  & 2900\\
     $\mathbf{z}^{(7)}$ & 51.6 & \color{orange}{28.0} & 17.0 & 5411 \\
  $\mathbf{z}^{(8)}$ &51.6 & 5.1 & 17.0 & 2573  \\
  
 
  \end{tabular}};

\node [left=of tab1] (tab2) {%
     \begin{tabular}{l |c|ccc}
  Coalition & $\mathbf{z}^{\prime (k)}$ &  hum & temp & ws \\
  \hline 
  $\varnothing$ & $\mathbf{z}^{\prime (1)}$ & 0 & 0 & 0  \\
  hum & $\mathbf{z}^{\prime (2)}$ & 1 & 0 & 0  \\
  temp &  $\mathbf{z}^{\prime (3)}$ & 0 & 1 & 0  \\
  ws &   $\mathbf{z}^{\prime (4)}$ & 0 & 0 & 1  \\
  hum, temp & $\mathbf{z}^{\prime (5)}$ & 1 & 1 & 0  \\
  temp, ws & $\mathbf{z}^{\prime (6)}$ & 0 & 1 & 1  \\
  hum, ws &   $\mathbf{z}^{\prime (7)}$ & 1 & 0 & 1  \\
  hum, temp, ws & $\mathbf{z}^{\prime (8)}$ & 1 & 1 & 1  \\
  
  \end{tabular}};
\draw[->]
(tab2.north) to[out=10,in=170] node[below]{$h_x(\mathbf{z}^{\prime (k)})$} (tab1.north) ;
\end{tikzpicture}


   




\end{frame}

\begin{frame}{Kernel shap - in 5 steps}
\textbf{Step 3: Compute kernel weights for surrogate model
\only<2>{\citebutton{see shapley\_kernel\_proof.pdf}{https://proceedings.neurips.cc/paper/2017/file/8a20a8621978632d76c43dfd28b67767-Supplemental.zip}}}\\\medskip
%\textbf{Intuition}: We learn most about individual features if we can study their effects in isolation or at maximal interaction:
%Small coalitions (few 1’s) and large coalitions (i.e. many 1’s) get the largest weights\\NB: the figure is independent from the running example

\begin{onlyenv}<1>
\textbf{Intuition:}  
We learn most about a feature’s effect when:
\begin{itemize}
  \item it appears \textbf{in isolation} (small coalition), or
  \item in \textbf{near-complete context} (large coalition).
\end{itemize}
\(\Rightarrow\) SHAP assigns highest weights to very small and very large coalitions.

\medskip
\textit{Note:} The figure below is illustrative and not tied to the running example.

\begin{figure}
    \centering
    \includegraphics[width=0.5\columnwidth]{figure_man/kernel-weights.pdf}
    %\caption{Examplary dependence between kernel weights and coalition size for a data set p = 10 features}   
\end{figure}
\end{onlyenv}


\begin{onlyenv}<2>
\vspace{1cm}
\begin{exampleblock}{}
\[
\tikzmark{pi}\pi_{x}\left(\mathbf{z}^{\prime (k)}\right)=\frac{(
\tikzmark{M}p-1)}{\left(\begin{array}{c} p \\\left|\mathbf{z}^{\prime (k)}\right|\end{array}\right)\left|
\tikzmark{z}\mathbf{z}^{\prime (k)}\right|\left(p-\left|\mathbf{z}^{\prime (k)}\right|\right)}
\]
\begin{tikzpicture}[
  remember picture,
  overlay,
  expl/.style={draw=blue,fill=white,rounded corners,text width=3cm},
  arrow/.style={blue,ultra thick,->,>=latex}
]
\node[expl] 
  (piex) 
  at (2,2.5cm)
  {$\pi_x(\mathbf{z}^{\prime (k)})$: kernel weight for coalition $\mathbf{z}^{\prime (k)}$};
\node[expl] 
  (Mex) 
  at (8,3cm)
  {$p$: Number of features in $\xv$};
\node[expl] 
  (zex) 
  at (6,-1cm)
  {$\mid \mathbf{z}^{\prime (k)}\mid$: coalition size / sum of 1s in $\mathbf{z}^{\prime (k)}$};
\draw[arrow]
  (piex.south) to[out=270,in=135] ([xshift= 0.5ex, yshift=2ex]{pic cs:pi}); 
\draw[arrow]
  (Mex.south) to[out=270,in=90] ([xshift= 0.5ex, yshift=2ex]{pic cs:M}); 
\draw[arrow]
  (zex.north) to[out=90,in=250] ([xshift= 0.5ex, yshift=-1ex]{pic cs:z}); 
\end{tikzpicture}
\end{exampleblock}
\end{onlyenv}


% \begin{onlyenv}<3>

% $$\pi_{x}\left(z^{\prime}\right)=\frac{(M-1)}{\left(\begin{array}{c} M \\\left|z^{\prime}\right|\end{array}\right)\left|z^{\prime}\right|\left(M-\left|z^{\prime}\right|\right)}$$

% \begin{itemize}
%     \item If a coalition consists of a single feature, we can learn about this feature’s isolated main effect on the prediction
%     \item If a coalition consists of all but one feature, we can learn about this feature’s total effect (main effect plus feature interactions)
%     \item If a coalition consists of half the features, we learn little about an individual feature’s contribution, as there are many possible coalitions with half of the features
% \end{itemize}
% \end{onlyenv}

% \begin{onlyenv}<4>
% \vspace{1cm}
% \textbf{Limited Budget $K$}: Can we be a bit smarter about the sampling of coalitions, than just randomly drawing?
% \begin{itemize}
%     \item The smallest and largest coalitions take up most of the weight\\ We get better Shapley value estimates by using some of the sampling budget K to include these high-weight coalitions
%     \item We start with all possible coalitions with 1 and M-1 features, which makes 2 times M coalitions in total\\ When we have enough budget left (current budget is K - 2M), we can include coalitions with 2 features and with M-2 features and so on.
%     \item From the remaining coalition sizes, we sample with readjusted weights
% \end{itemize}
% \end{onlyenv}
  
\end{frame}


\begin{frame}{Kernel shap - in 5 steps}
\textbf{Step 3: Compute kernel weights for surrogate model}\\\medskip
%\textbf{Purpose}: to include this knowledge in the local surrogate model (linear regression), we calculate weights for each coalition which are the observations of the linear regression
\textbf{Purpose:} Assign weights to coalition vectors when fitting the local surrogate (linear regression) to account for their "importance"
\only<1>{    $$\pi_{x}\left(\mathbf{z}^{\prime}\right)=\frac{(p-1)}{\left(\begin{array}{c} p \\|\mathbf{z}^{\prime}|\end{array}\right)|\mathbf{z}^{\prime}|(p-|\mathbf{z}^{\prime}|)} \leadsto \pi_x\left(\mathbf{z}^{\prime} = (1,0,0)\right)=\frac{(3-1)}{\left(\begin{array}{c} 3 \\1\end{array}\right)1\left(3-1\right)} = \frac{1}{3}$$
}

\begin{table}[]
    \centering
        \begin{tabular}{l |c|ccc|c}
 Coalition & $\mathbf{z}^{\prime (k)}$ &  hum & temp & ws & weight\\
  \hline 
  $\varnothing$ & $\mathbf{z}^{\prime (1)}$ & 0 & 0 & 0 & $\infty$ \\
  hum & $\mathbf{z}^{\prime (2)}$ & 1 & 0 & 0 & 0.33 \\
  temp &  $\mathbf{z}^{\prime (3)}$ & 0 & 1 & 0 & 0.33 \\
  ws &   $\mathbf{z}^{\prime (4)}$ & 0 & 0 & 1 & 0.33  \\
  hum, temp & $\mathbf{z}^{\prime (5)}$ & 1 & 1 & 0 & 0.33 \\
  temp, ws & $\mathbf{z}^{\prime (6)}$ & 0 & 1 & 1 & 0.33 \\
  hum, ws &   $\mathbf{z}^{\prime (7)}$ & 1 & 0 & 1 & 0.33 \\
  hum, temp, ws & $\mathbf{z}^{\prime (8)}$ & 1 & 1 & 1 & $\infty$ \\
  
 
  \end{tabular}
\end{table}
% \medskip
\only<2>{
$\leadsto$ all the finite weights being equal to the same value (0.33) is the result of coalition size being equal to 3. In general the weight distribution is not uniform. 
\\$\leadsto$ weights for empty and full set are infinity (since they result in division by 0 when calculating $\pi_{x}\left(\mathbf{z}^{\prime}\right)$) and not used as observations for the linear regression\\ $\leadsto$ instead constraints are used such that properties (local accuracy and missingness) are satisfied
}

  
\end{frame}

%\begin{frame}{Coalition Mapping}
%We define a coalition $z^{\prime}$, by describing a function 

%$$
%h\left(z^{\prime}\right)=z \text { where } h:\{0,1\}^{M} \rightarrow \mathbb{R}^{p}
%$$


%\begin{onlyenv}<1>
%\vspace{1cm}
%\begin{itemize}
%    \item Coalition $z^{\prime} \in \{0, 1\}^M$ is the  vector, indicating if feature $j$ contributes to the prediction 
%    \item $h(\cdot)$ represent a function that maps 1’s to the corresponding value from the observation x that we want to explain: $h(\cdot)$ connects our coalition vector to the underlying data 
%\end{itemize}
%\end{onlyenv}

%\begin{onlyenv}<2->
%\begin{tikzpicture}
%\centering

%\node<2> (tab1) {%
%  \begin{tabular}{l |cccc}
%  observation & temp & hum & ws & yr\\
%  \hline 
%  $x_{ex}$ & 24.7 & 58.5 & 13.96 & 2011\\
 % \\
 % \\
  %\end{tabular}};
%\node<3-> (tab1) {%
%  \begin{tabular}{l |cccc}
%  observation & temp & hum & ws & yr\\
%  \hline 
%  $x_{ex}$ & 24.7 & 58.5 & 13.96 & 2011\\
%  $z_{temp, yr}$ & 24.7 & $\varnothing$ & $\varnothing$ & 2011\\
%  $z_{yr}$ & $\varnothing$ & $\varnothing$ & $\varnothing$ & 2011\\
%  \end{tabular}};
%\node<2> [left=of tab1] (tab2) {%
%  \begin{tabular}{l |cccc}
%  Coalition & temp & hum & ws & yr\\
%  \hline 
%  $x^{\prime}$ & 1 & 1 & 1 & 1 \\
%  \\
%  \\
%  \end{tabular}};
%\node<3-> [left=of tab1] (tab2) {%
%  \begin{tabular}{l |cccc}
%  Coalition & temp & hum & ws & yr\\
%  \hline 
%  $x^{\prime}$ & 1 & 1 & 1 & 1 \\
%  $z^{\prime}_{temp, yr}$ & 1 & 0 & 0 %& 1 \\
%  $z^{\prime}_{yr}$ & 0 & 0 & 0 & 1 %\\
%  \end{tabular}};
%\draw<2->[->]
%(tab2.north) to[out=30,in=150] node[below]{$h(\cdot)$} (tab1.north) ;
%\end{tikzpicture}
%\end{onlyenv}
%\begin{onlyenv}<3->
%\begin{itemize}
%    \item $h(\cdot)$ maps 1’s to the %corresponding value from the observation x that we want to explain
%    \item<4> it maps 0’s to the values of another observation that we sample from the data
%    \item<4>  we equate “feature value is absent” with “feature value is replaced by random feature value from data”
%\end{itemize}
%\end{onlyenv}
%\end{frame}


\begin{frame}{Kernel shap - in 5 steps}
\textbf{Step 4: Fit a weighted linear model}\\\medskip
\textbf{Aim}: Estimate a weighted linear model with Shapley values being the coefficients $\phi_j$
$$
g\left(\mathbf{z}^{\prime (k)}\right)=
\phi_{0}+\sum_{j=1}^{p}
 \phi_{j} z_{j}^{\prime (k)} \only<2>{\leadsto g\left(\mathbf{z}^{\prime (k)}\right)=
4515 +
 34 \cdot z_{1}^{\prime (k)} - 1654 \cdot z_{2}^{\prime (k)} - 323 \cdot z_{3}^{\prime (k)} }
$$


\only<1>{
and minimize by WLS using the weights $\pi_{x}$ of step 3
    $$L\left(\hat{f}, g, \pi_{x}\right)=\sum_{k = 1}^K\left[\hat{f}\left(h_{x}\left(\mathbf{z}^{\prime (k)}\right)\right)-g\left(\mathbf{z}^{\prime (k)}\right)\right]^{2} \pi_{x}\left(\mathbf{z}^{\prime (k)}\right)$$

with $\phi_0 = \E(\fh)$ and $\phi_p = \fh(x) - \sum_{j=0}^{p-1} \phi_j$ we receive a $p-1$ dimensional linear regression problem
}

\only<2>{
\begin{table}[]
    \centering
        \begin{tabular}{l |ccc|c|c}
  $\mathbf{z}^{\prime (k)}$ &  hum & temp & ws & weight & $\fh$\\
  \hline 
   $\mathbf{z}^{\prime (2)}$ & 1 & 0 & 0 & 0.33 & 4635\\
    $\mathbf{z}^{\prime (3)}$ & 0 & 1 & 0 & 0.33 & 3087\\
     $\mathbf{z}^{\prime (4)}$ & 0 & 0 & 1 & 0.33 & 4359\\
     $\mathbf{z}^{\prime (5)}$ & 1 & 1 & 0 & 0.33 & 3060\\
     $\mathbf{z}^{\prime (6)}$ & 0 & 1 & 1 &0.33 & 2623\\
     $\mathbf{z}^{\prime (7)}$ & 1 & 0 & 1 & 0.33 & 4450\\
      \multicolumn{1}{c}{} & \multicolumn{3}{c}{\upbracefill}&\multicolumn{1}{c}{} &\multicolumn{1}{c}{\upbracefill}\\[-1ex]
    \multicolumn{1}{c}{} & \multicolumn{3}{c}{$\scriptstyle input$}&\multicolumn{1}{c}{} &  \multicolumn{1}{c}{$\scriptstyle output$}\\
  
 
  \end{tabular}
\end{table}
}


  
\end{frame}


\begin{frame}{Kernel shap - in 5 steps}
\textbf{Step 5: Return SHAP values}\\\medskip
\textbf{Intuition}: Estimated Kernel SHAP values are equivalent to Shapley values 
\begin{align*}
g(\mathbf{z}^{\prime (8)}) &= \fh(h_x(\mathbf{z}^{\prime (8)}) ) = 4515 + 34 \cdot 1 - 1654 \cdot 1 - 323 \cdot 1\\&= \underbrace{\E(\fh)}_{\phi_0} + \phi_{hum} + \phi_{temp} + \phi_{ws} = \fh(\xv) = 2573
\end{align*}

\begin{figure}
    \centering
    \includegraphics[width=\columnwidth]{figure_man/exSHAP.png}
\end{figure}

\end{frame}
%\begin{frame}{Marginal Contribution}

%\begin{itemize}
%    \begin{onlyenv}<1>
%    \item Consider coalition $z^{\prime}$ as indicator function for our shapley values $\phi$
%    \end{onlyenv}
%    \begin{onlyenv}<2>
%    \item This connects the coalition vector $z^{\prime}$ to the respective marginal contribution
%    \end{onlyenv}
%    \begin{onlyenv}<3>
%    \item To estimate the marginal contribution, we can transfer the coalition to the data space by $h(z^{\prime})$
%    \end{onlyenv}
%    \begin{onlyenv}<4->
%    \item $\fh(h(z^{\prime}))$ connects the coalitions directly to the marginal distribution.
%    \end{onlyenv}
%\end{itemize}

%\vspace{1cm}

%\begin{tikzpicture}
%\centering

%\node<1-2> (tab1) {%
%  \begin{tabular}{l |cccc}
%  Coalition & temp & hum & ws & yr\\
%  \hline 
%  $x^{\prime}$ & 1 & 1 & 1 & 1 \\
%  $z^{\prime}_{temp, yr}$ & 1 & 0 & 0 & 1 \\
%  $z^{\prime}_{yr}$ & 0 & 0 & 0 & 1 \\
%  \end{tabular}};
%\node<2-> [right=of tab1] (tab2) {%
%\begin{tabular}{l | cccc}
%  & temp & hum & ws & yr\\
%  \hline 
%  $g(x^{\prime})$ & $\phi_{temp}$ + & $\phi_{hum}$ + & $\phi_{ws}$ + & $\phi_{yr}$ \\
%  $g(z^{\prime}_{temp, yr})$ & $\phi_{temp}$ + &  &  & $\phi_{yr}$\\
%   $g(z^{\prime}_{yr})$ & &  &  & $\phi_{yr}$ \\
%  \end{tabular}};
%\node<3-> [left=of tab2] (tab) {%
%  \begin{tabular}{l |cccc}
%  observation & temp & hum & ws & yr\\
%  \hline 
%  $x_{ex}$ & 24.7 & 58.5 & 13.96 & %2011\\
%  $z_{temp, yr}$ & 24.7 & %$\varnothing$ & $\varnothing$ & %2011\\
%  $z_{yr}$ & $\varnothing$ & %$\varnothing$ & $\varnothing$ & 2011\\
%  \end{tabular}};
%\draw<2>[->]
%(tab1.south) to[out=320,in=200] node[above]{$\sum \mathbb{I}_{[z^{\prime}_i == 1]} \phi_i$} (tab2.south) ;
%\draw<3->[->]
%(tab2.south) to[out=200,in=330] node[above]{$\fh(h(z^{\prime}))$} (tab1.south) ;
%\end{tikzpicture}

%\begin{onlyenv}<4>
%\begin{equation}
%\begin{array}{lllc}
  
%  g(x^{\prime}) &= \phi_{temp} + \phi_{hum} + \phi_{ws} + &\phi_{yr} &= 6825\\
%  g(z^{\prime}_{temp, yr}) &= \phi_{temp} + &\phi_{yr} &= 6134\\
%   g(z^{\prime}_{yr}) &= &\phi_{yr} &= 4325\\
%\end{array}
%\end{equation}
%\end{onlyenv}

%\begin{onlyenv}<5>
%\vspace{0.5cm}

%\textbf{Notice:}\\ We created a coalition data set $Z^{\prime}$ here by sampling multiple coalitions from observation $\xv$ that is evaluable with the prediction function $\fh$
%\end{onlyenv}

%\end{frame}



\endlecture
\end{document}
