\documentclass[11pt,compress,t,notes=noshow, aspectratio=169, xcolor=table]{beamer}
% \definecolor{gold}{RGB}{218,165,32}       % classic goldenrod/gold
% \definecolor{darkyellow}{RGB}{204,153,0}  % darker golden tone
% Define colors in LaTeX preamble using xcolor
\definecolor{playerred}{RGB}{204, 0, 0}      % #CC0000
\definecolor{playeryellow}{RGB}{241, 194, 50}  % #FFCC00
\definecolor{playerblue}{RGB}{60, 120, 216}   % #3366CC

% Define a generic circle with customizable fill, text color, and content
\NewDocumentCommand{\colorcircle}{m m m}{%
  \tikz[baseline=(char.base)]{%
    \node[shape=circle, draw=none, fill=#1, minimum size=0.75em, inner sep=0pt] (char) {%
      \textcolor{#2}{\textbf\scriptsize #3}%
    };%
  }%
}

\usepackage{pifont}
\usepackage{../../style/lmu-lecture}
% Defines macros and environments
\usepackage{bbm}
% basic latex stuff
\newcommand{\pkg}[1]{{\fontseries{b}\selectfont #1}} %fontstyle for R packages
\newcommand{\lz}{\vspace{0.5cm}} %vertical space
\newcommand{\dlz}{\vspace{1cm}} %double vertical space
\newcommand{\oneliner}[1] % Oneliner for important statements
{\begin{block}{}\begin{center}\begin{Large}#1\end{Large}\end{center}\end{block}}


%new environments
\newenvironment{vbframe}  %frame with breaks and verbatim
{
 \begin{frame}[containsverbatim,allowframebreaks]
}
{
\end{frame}
}

\newenvironment{vframe}  %frame with verbatim without breaks (to avoid numbering one slided frames)
{
 \begin{frame}[containsverbatim]
}
{
\end{frame}
}

\newenvironment{blocki}[1]   % itemize block
{
 \begin{block}{#1}\begin{itemize}
}
{
\end{itemize}\end{block}
}

\newenvironment{fragileframe}[2]{  %fragile frame with framebreaks
\begin{frame}[allowframebreaks, fragile, environment = fragileframe]
\frametitle{#1}
#2}
{\end{frame}}


\newcommand{\myframe}[2]{  %short for frame with framebreaks
\begin{frame}[allowframebreaks]
\frametitle{#1}
#2
\end{frame}}

\newcommand{\remark}[1]{
  \textbf{Remark:} #1
}


\newenvironment{deleteframe}
{
\begingroup
\usebackgroundtemplate{\includegraphics[width=\paperwidth,height=\paperheight]{../style/color/red.png}}
 \begin{frame}
}
{
\end{frame}
\endgroup
}
\newenvironment{simplifyframe}
{
\begingroup
\usebackgroundtemplate{\includegraphics[width=\paperwidth,height=\paperheight]{../style/color/yellow.png}}
 \begin{frame}
}
{
\end{frame}
\endgroup
}\newenvironment{draftframe}
{
\begingroup
\usebackgroundtemplate{\includegraphics[width=\paperwidth,height=\paperheight]{../style/color/green.jpg}}
 \begin{frame}
}
{
\end{frame}
\endgroup
}
% https://tex.stackexchange.com/a/261480: textcolor that works in mathmode
\makeatletter
\renewcommand*{\@textcolor}[3]{%
  \protect\leavevmode
  \begingroup
    \color#1{#2}#3%
  \endgroup
}
\makeatother


\providecommand{\tightlist}{%
  \setlength{\itemsep}{0pt}\setlength{\parskip}{0pt}}

%\setbeamerfont{footnote}{size=\tiny}
\usepackage[hang,flushmargin]{footmisc}
\renewcommand*{\footnotelayout}{\tiny}
\renewcommand*{\thefootnote}{} %\fnsymbol{footnote}

% https://tex.stackexchange.com/questions/30720/footnote-without-a-marker
% \makeatletter
% \def\blfootnote{\gdef\@thefnmark{}\@footnotetext}
% \makeatother

% https://tex.stackexchange.com/questions/357717/beamer-allowframebreaks-option-and-vertical-spacing-when-using-lists-itemize
% \setbeamertemplate{frametitle continuation}{%
%     (\insertcontinuationcount)%
%     \ifnum\insertcontinuationcount>1%
%     \vspace*{\topsep}%
%     \else%
%     %
%     \fi%
% }


%\newcommand{\Sm}{S_{j}^\tau}%{Pre(\tau,j)} % S(\tau,j)

\title{Interpretable Machine Learning}
% \author{LMU}
%\institute{\href{https://compstat-lmu.github.io/lecture_iml/}{compstat-lmu.github.io/lecture\_iml}}
\date{}

\begin{document}

\newcommand{\titlefigure}{figure/Shapley_1.png}
\newcommand{\learninggoals}{%
\item Learn cooperative games and value functions

\item Define the marginal contribution of a player

\item Study Shapley value as a fair payout solution

\item Compare order and set definitions
}

\lecturechapter{Shapley Values}
\lecture{Interpretable Machine Learning}

% License of titlefigure: free pixabay license
% https://pixabay.com/de/vectors/ergebnis-geld-gesch%C3%A4ft-gehalt-5567652/


% \begin{frame}{Game Theory}
% \begin{itemize}
% \itemsep1em
%   \item Game theory is the study of strategic games between players
%   \item Term \enquote{game} not restricted to actual games (e.g., chess or poker) but to any series of interactions between actors or agents with gains and losses of quantifiable utility value
%   \item Often used in social context where players correspond to people or organizations, e.g., warfare, provision of public goods, auctions and bargaining, formation of cartels, interrogation practices.
%   \item Example of prisoner's dilemma: Two prisoners A and B are interrogated in two separate rooms with no means of communication between each other.
%     \begin{itemize}
%         \item If they betray each other, each person serves 2 years in prison.
%         \item If one person remains silent and one betrays their partner, person staying silent receives 3 years in prison while betrayer is set free.
%         \item If both remain silent, both serve 1 year in prison.
%     \end{itemize}
%     Conditional on strategy of remaining partner, best personal outcome is achieved by betrayal.
%     \\
%     $\Rightarrow$ Best aggregate outcome not achieved (both staying silent). 
% \end{itemize}
% \end{frame}


% \begin{frame}{Cooperative Games in Game Theory \citebutton{Shapley (1951)}{https://www.rand.org/pubs/research_memoranda/RM0670.html}}
% \begin{itemize}[<+->]
% %\itemsep1em
%   \item Game theory: Study of strategic games between players, where \enquote{game} refers to interactions between \enquote{players} involving quantifiable gains and losses
%   \item Cooperative games: For all possible players $P = \{1, \hdots, p\}$, each subset of players $\SsubP$ forms a coalition -- each coalition $S$ achieves a certain payout
%   %\item A value function $v(S): 2^{|P|}\mapsto \R$ describes the payout (or gain) achieved by any coalition $S \subseteq P$
%   \item Value function $v: 2^P\mapsto \R$ maps all $2^{|P|}$ possible coalitions to their payout/gain
%   \item $v(S)$ denotes the payout of coalition $S \subseteq P$; payout of empty coalition must be 0 ($v(\emptyset) = 0$). $v(P)$ denotes the total achievable payout.
%   %\\
%   %\textit{Note:} The payout or gain not necessarily has a positive meaning. In the example of the prisoner's dilemma, the payout is the total number of years spent in prison.
%   \item We denote by $\phi_j$ the individual payout of player $j \in P$, referred to later as the Shapley value. 
%   \item As some players may contribute differently, we want to fairly divide the total achievable payout $v(P)$ among the players according to a player's individual contribution

%   %\item What would be properties of a fair distribution of the payout?
% \end{itemize}
% \end{frame}

\begin{frame}{Cooperative Games in Game Theory \citebutton{Shapley (1951)}{https://www.rand.org/pubs/research_memoranda/RM0670.html}}
\begin{itemize}%[<+->]
  \item \textbf{Game theory:} Studies strategic interactions among "players" (who act to maximize their utility), where outcomes depend on collective behavior
  \item \textbf{Cooperative games:} Any subset $S \subseteq P = \{1, \ldots, p\}$ can form a coalition to cooperate in a game, each achieving a payout $v(S)$
  %For all possible players $P = \{1, \hdots, p\}$, any subset $\SsubP$ forms a coalition -- each coalition $S$ achieves a certain payout
  \pause
  %For a player set $P = \{1, \ldots, p\}$, any subset $S \subseteq P$ can form a \emph{coalition}.
  \item \textbf{Value function:} $v: 2^P \rightarrow \mathbb{R}$ assigns each coalition $S$ a payout $v(S)$ 
  \begin{itemize}
      \item Convention: $v(\emptyset) = 0 \leadsto$  Empty coalitions generate no gain
      \item $v(P)$: Total achievable payout when all players cooperate\\
      $\leadsto$ Forms the game's budget to be fairly distributed
  \end{itemize}%Convention: $v(\emptyset) = 0$ and total payout  $v(P)$ to be fairly distributed among players.
    \item \textbf{Marginal contribution:} Measure how much value player \(j\) adds to coalition \( S \) by

    \medskip
    
    \centerline{$\Delta (j, S) := v(S \cup \{j\}) - v(S) \quad \; \left(\text{for all } j \in P \; S \subseteq P \setminus \{j\}\right)$}

    \medskip
    \pause
  \item \textbf{Challenge:} Players vary in their contributions \& how they influence each other
  \item \textbf{Goal:} Fairly distribute $v(P)$ among players by accounting for player interactions\\
  $\leadsto$ Assign each player $j \in P$ a fair share $\phi_j$ (\textbf{Shapley value}) 
\end{itemize}
\end{frame}


\begin{frame}{Cooperative Games - No Interactions}
% Figure Source: https://docs.google.com/presentation/d/1-bK90Gv1vIDr61s1PfgC51Kvic2Avnb7v8PgoO9iN0I/edit?usp=sharing
%\begin{center}
\includegraphics<1>[page=1, width = \textwidth]{figure/Shapley.pdf}%
\only<2-3>{
%\vspace{-\baselineskip}
\vspace*{-7pt}
\begin{columns}[T, totalwidth=\textwidth]
    \begin{column}{0.459\textwidth}
\includegraphics[page=1, width = \textwidth, trim=0px 75px 390px 0px, clip]{figure/Shapley.pdf}%
    \end{column}
    \begin{column}{0.541\textwidth}
        {\footnotesize
        \begin{tabular}{ll|ll|l}
\toprule
\scriptsize Player & \scriptsize Coalition $S$ & \scriptsize $v(S \cup \{j\})$ & \scriptsize $v(S)$ & \scriptsize $\Delta (j, S)$ \\
\hline
\colorcircle{playerred}{white}{1} & $\emptyset$ & 1000 & 0 & 1000 \\
\colorcircle{playerred}{white}{1} & $\{\colorcircle{playeryellow}{white}{2}\}$ & 3000 & 2000 & 1000 \\
\colorcircle{playerred}{white}{1} & $\{\colorcircle{playerblue}{white}{3}\}$ & 4000 & 3000 & 1000 \\
\colorcircle{playerred}{white}{1} & $\{\colorcircle{playeryellow}{white}{2},\colorcircle{playerblue}{white}{3}\}$ & 6000 & 5000 & 1000 \\
\hline
\colorcircle{playeryellow}{white}{2} & $\emptyset$ & 2000 & 0 & 2000 \\
\colorcircle{playeryellow}{white}{2} & $\{\colorcircle{playerred}{white}{1}\}$ & 3000 & 1000 & 2000 \\
\colorcircle{playeryellow}{white}{2} & $\{\colorcircle{playerblue}{white}{3}\}$ & 5000 & 3000 & 2000 \\
\colorcircle{playeryellow}{white}{2} & $\{\colorcircle{playerred}{white}{1},\colorcircle{playerblue}{white}{3}\}$ & 6000 & 4000 & 2000 \\
\hline
\colorcircle{playerblue}{white}{3} & $\emptyset$ & 3000 & 0 & 3000 \\
\colorcircle{playerblue}{white}{3} & $\{\colorcircle{playerred}{white}{1}\}$ & 4000 & 1000 & 3000 \\
\colorcircle{playerblue}{white}{3} & $\{\colorcircle{playeryellow}{white}{2}\}$ & 5000 & 2000 & 3000 \\
\colorcircle{playerblue}{white}{3} & $\{\colorcircle{playerred}{white}{1},\colorcircle{playeryellow}{white}{2}\}$ & 6000 & 3000 & 3000 \\
\bottomrule
\end{tabular}

}
% \begin{itemize}
% \item No interactions: player contributions are additive for any coalition $S$ 
% \item Each player adds same value in all $S$
% \item Fair payout = average marginal contrib.
% \item Total payout splits exactly by individual contributions: \colorcircle{playerblue}{white}{3} = $\tfrac{1}{2}$, \colorcircle{playeryellow}{white}{2} = $\tfrac{1}{3}$, \colorcircle{playerred}{white}{1} = $\tfrac{1}{6}$ 
% \end{itemize}
    \end{column}
\end{columns}
\only<3>{
\begin{itemize}
  \item \textbf{No interactions:} Each player contributes the same fixed value to each coalition\\
  $\leadsto$ Player \colorcircle{playerred}{white}{1} always adds 1000, \colorcircle{playeryellow}{white}{2} adds 2000, and \colorcircle{playerblue}{white}{3} adds 3000\\
  $\leadsto$ Marginal contributions are constant across all coalitions \( S \)
  %Constant marginal gains of each player to every coalition $S$
  \item \textbf{Conclusion:} Fair payout $=$ average marginal contribution across all $S$\\
  $\leadsto$ Total value $v(P) = 6000$ splits proportionally by individual contributions:

\centerline{$
  \colorcircle{playerred}{white}{1} = \tfrac{1}{6}, \quad
  \colorcircle{playeryellow}{white}{2} = \tfrac{1}{3}, \quad
  \colorcircle{playerblue}{white}{3} = \tfrac{1}{2}
$}
\end{itemize}
}
%\vspace*{5pt}
}
\includegraphics<4>[page=2, width = \textwidth]{figure/Shapley.pdf}%

\visible<1>{\centering \textbf{Question:} What are the individual marginal contributions and what is a fair payout?}% if contributions are $\tfrac{1}{2}$ (blue), $\tfrac{1}{3}$ (yellow), $\tfrac{1}{6}$ (red)?}

\visible<4>{\centering $\Rightarrow$ Fair payouts are trivial without interactions}
%\end{center}
\end{frame}




% \begin{frame}{Cooperative Games - No Interactions}
% \scriptsize
% \begin{tabular}{ll|ll|l}
% \toprule
% Player & Coalition $S$ & $v(S)$ & $v(S \cup \{j\})$ & $\Delta (j, S)$ \\
% \hline
% \colorcircle{playerred}{white}{1}& $\emptyset$ & 0 & 1000 & 1000 \\
% \colorcircle{playerred}{white}{1}& $\{\colorcircle{playeryellow}{white}{2}\}$ & 2000 & 3000 & 1000 \\
% \colorcircle{playerred}{white}{1}& $\{\colorcircle{playerblue}{white}{3}\}$ & 3000 & 4000 & 1000 \\
% \colorcircle{playerred}{white}{1} & $\{\colorcircle{playeryellow}{white}{2},\colorcircle{playerblue}{white}{3}\}$ & 5000 & 6000 & 1000 \\
% \hline
% \colorcircle{playeryellow}{white}{2} & $\emptyset$ & 0 & 2000 & 2000 \\
% \colorcircle{playeryellow}{white}{2} & $\{\colorcircle{playerred}{white}{1}\}$ & 1000 & 3000 & 2000 \\
% \colorcircle{playeryellow}{white}{2} & $\{\colorcircle{playerblue}{white}{3}\}$ & 3000 & 5000 & 2000 \\
% \colorcircle{playeryellow}{white}{2} & $\{\colorcircle{playerred}{white}{1},\colorcircle{playerblue}{white}{3}\}$ & 4000 & 6000 & 2000 \\
% \hline
% \colorcircle{playerblue}{white}{3} & $\emptyset$ & 0 & 3000 & 3000 \\
% \colorcircle{playerblue}{white}{3} & $\{\colorcircle{playerred}{white}{1}\}$ & 1000 & 4000 & 3000 \\
% \colorcircle{playerblue}{white}{3} & $\{\colorcircle{playeryellow}{white}{2}\}$ & 2000 & 5000 & 3000 \\
% \colorcircle{playerblue}{white}{3} & $\{\colorcircle{playerred}{white}{1},\colorcircle{playeryellow}{white}{2}\}$ & 3000 & 6000 & 3000 \\
% \bottomrule
% \end{tabular}
% \end{frame}



% \begin{frame}{Cooperative Games with Interactions}
% \begin{center}
% \includegraphics<1>[page=3, width = 0.8\textwidth]{figure/Shapley.pdf}

% \only<1>{$\Rightarrow$ Unclear how to fairly distribute payouts when players interact}

% \includegraphics<2>[page=4]{figure/Shapley.pdf}
% \end{center}
% \end{frame}

\begin{frame}{Cooperative Games - Interactions}
%\begin{center}
\includegraphics[page=3, width = \textwidth]{figure/Shapley.pdf}

\centering 
$\Rightarrow$ Unclear how to fairly distribute payouts when players interact
%\end{center}
\end{frame}


\begin{frame}{Cooperative Games - Interactions}
%\vspace{-\baselineskip}
\vspace*{-7pt}
\begin{columns}[T, totalwidth=\textwidth]
    \begin{column}{0.459\textwidth}
\includegraphics[page=3, width = \textwidth, trim=0px 75px 390px 0px, clip]{figure/Shapley.pdf}%
    \end{column}
    \begin{column}{0.541\textwidth}
        {\footnotesize
\begin{tabular}{ll|ll|l}
\toprule
\scriptsize Player & \scriptsize Coalition $S$ & \scriptsize $v(S \cup \{j\})$ & \scriptsize $v(S)$ & \scriptsize $\Delta (j, S)$ \\
\hline
\colorcircle{playerred}{white}{1} & $\emptyset$ & 1000 & 0 & 1000 \\
\colorcircle{playerred}{white}{1} & $\{\colorcircle{playeryellow}{white}{2}\}$ & 4000 & 2000 & 2000 \\
\colorcircle{playerred}{white}{1} & $\{\colorcircle{playerblue}{white}{3}\}$ & 4000 & 3000 & 1000 \\
\colorcircle{playerred}{white}{1} & $\{\colorcircle{playeryellow}{white}{2},\colorcircle{playerblue}{white}{3}\}$ & 6000 & 3000 & 3000 \\
\hline
\colorcircle{playeryellow}{white}{2} & $\emptyset$ & 2000 & 0 & 2000 \\
\colorcircle{playeryellow}{white}{2} & $\{\colorcircle{playerred}{white}{1}\}$ & 4000 & 1000 & 3000 \\
\colorcircle{playeryellow}{white}{2} & $\{\colorcircle{playerblue}{white}{3}\}$ & 3000 & 3000 & 0 \\
\colorcircle{playeryellow}{white}{2} & $\{\colorcircle{playerred}{white}{1},\colorcircle{playerblue}{white}{3}\}$ & 6000 & 4000 & 2000 \\
\hline
\colorcircle{playerblue}{white}{3} & $\emptyset$ & 3000 & 0 & 3000 \\
\colorcircle{playerblue}{white}{3} & $\{\colorcircle{playerred}{white}{1}\}$ & 4000 & 1000 & 3000 \\
\colorcircle{playerblue}{white}{3} & $\{\colorcircle{playeryellow}{white}{2}\}$ & 3000 & 2000 & 1000 \\
\colorcircle{playerblue}{white}{3} & $\{\colorcircle{playerred}{white}{1},\colorcircle{playeryellow}{white}{2}\}$ & 6000 & 4000 & 2000 \\
\bottomrule
\end{tabular}
}
% \begin{itemize}
% \item Marginal contributions differ across coalitions
% \item What is a fair payout here?
% \end{itemize}

    \end{column}
\end{columns}
\pause
\begin{itemize}
  \item \textbf{With interactions:} Players contribute different amounts depending on coalition\\
  %$\leadsto$ \colorcircle{playerblue}{white}{3} adds 3000 alone, but 0 when \colorcircle{playeryellow}{white}{2} is present\\
  $\leadsto$ Marginal contributions vary across coalitions \( S \) (e.g., due to overlap, synergy)
  
  \item Averaging over subsets does not recover total payout $v(P)$ $\leadsto$ unfair payout distr. \\
  $\leadsto$ average contrib. \colorcircle{playerred}{white}{1} = 1750,
  \colorcircle{playeryellow}{white}{2} = 1750, 
  \colorcircle{playerblue}{white}{3} = 2250
  do not sum to $v(P) = 6000$

  \item Value a player adds depends on joining order, not just who else is in the coalition\\
  $\leadsto$ Shapley values fairly average over all possible joining orders
\end{itemize}

\end{frame}

\begin{frame}{Cooperative Games - Interactions}

\begin{columns}[T, totalwidth=\linewidth]
\begin{column}{0.4\textwidth}
%\begin{center}
\includegraphics[page=3, width = \textwidth, trim=0px 50px 420px 50px, clip]{figure/Shapley.pdf}
%\end{center}
    \end{column}
    \begin{column}{0.5\textwidth}
        \textbf{Ordering 1:} \colorcircle{playerblue}{white}{3} $\rightarrow$ \colorcircle{playeryellow}{white}{2} $\rightarrow$ \colorcircle{playerred}{white}{1}
\begin{itemize}
  \item[\colorcircle{playerblue}{white}{3}] joins alone: 3 \ding{73}
  \item[\colorcircle{playeryellow}{white}{2}] joins: total = 3 \ding{73}, marginal = 0
  \item[\colorcircle{playerred}{white}{1}] joins: total = 6 \ding{73}, marginal = +3
\end{itemize}

\vspace{0.25em}

\textbf{But what if \colorcircle{playerred}{white}{1} joins before \colorcircle{playeryellow}{white}{2}?}

\vspace{0.25em}

\textbf{Ordering 2:} \colorcircle{playerblue}{white}{3} $\rightarrow$ \colorcircle{playerred}{white}{1} $\rightarrow$ \colorcircle{playeryellow}{white}{2}
\begin{itemize}
  \item[\colorcircle{playerblue}{white}{3}] joins alone: 3 \ding{73}
  \item[\colorcircle{playerred}{white}{1}] joins: total = 4 \ding{73}, marginal = +1
  \item[\colorcircle{playeryellow}{white}{2}] joins: total = 6 \ding{73}, marginal = +2
\end{itemize}
    \end{column}
\end{columns}
\pause
\begin{itemize}
  \item \textbf{Order sensitivity:} A player's marginal contribution depends on when they join $S$
  %\textbf{Overlapping skills:} Different join orders lead to different marginal contributions
  \item \textbf{Shapley value:} Averages each player’s contribution over all possible join orders
  %\textbf{Fairness:} Shapley values consider every possible way the team could have formed and average each player's contribution across all possible join orders
  %\item This results in fair credit assignment by accounting for
  \begin{itemize}
    \item[$\leadsto$] Resolves redundancy (e.g., \colorcircle{playerblue}{white}{3}’s contribution/skill overlaps with \colorcircle{playeryellow}{white}{2}’s)
    \item[$\leadsto$] Accounts for order sensitivity (e.g., \colorcircle{playerred}{white}{1} brings more value if added last)
    \item[$\leadsto$] Ensures fairness (no player is advantaged or penalized by order of joining)% of evaluation.
  \end{itemize}
\end{itemize}

% \begin{itemize}
%   \item \textbf{Order sensitivity:} A player's marginal contribution depends on when they join $S$
%   \item \textbf{Shapley value:} Averages each player’s contribution over all possible join orders
%   \begin{itemize}
%     \item[$\leadsto$] Resolves redundancy (e.g., \colorcircle{playerblue}{white}{3} adds less if \colorcircle{playerred}{white}{1} is present)
%     \item[$\leadsto$] Corrects for join-time advantage/disadvantage
%   \end{itemize}
% \end{itemize}
\end{frame}




\begin{frame}{Shapley Values - Illustration}
\begin{itemize}
    \item Generate all possible joining orders of players (all permutations of full set $P$)
    \item For each order: track player $j$-th marginal contribution when $j$ joins a coalition
    \item<2-> Shapley value of $j$: Average this marginal contribution over all joining orders
    % and coalitions where $j$ joined
%\only<1>{\item Here, player $2$ enters the coalition after player $1$, resulting in a value change of $v(\{1,2\}) - v(\{1\}) = 24-12 = 12$ with a overall coalition value of $v(\{1,2,3\}) = 36$}
\only<1>{\item[] \phantom{\textbf{Example:} Compute payout difference after player $1$ enters coalition $\leadsto$ average}}%
\only<2>{\item \textbf{Example:} Compute payout difference after player $1$ enters coalition $\leadsto$ average}%
\only<3>{\item \textbf{Example:} Compute payout difference after player $2$ enters coalition $\leadsto$ average}%
\only<4>{\item \textbf{Example:} Compute payout difference after player $3$ enters coalition $\leadsto$ average}%
\end{itemize}

\begin{center}
\includegraphics<1>[page=9, width = 0.9\textwidth]{figure/Shapley.pdf}%
\includegraphics<2>[page=10, width = 0.9\textwidth]{figure/Shapley.pdf}%
\includegraphics<3>[page=11, width = 0.9\textwidth]{figure/Shapley.pdf}%
\includegraphics<4>[page=12, width = 0.9\textwidth]{figure/Shapley.pdf}%
\includegraphics<5>[page=13, width = 0.9\textwidth]{figure/Shapley.pdf}%
\end{center}

% \begin{center}
%   \only<1>{
%     \includegraphics[page=1, width=0.6\textwidth]{figure_man/shapley_feature_effect}
%   }
%   \only<2>{
%     \includegraphics[page=2, width=0.6\textwidth]{figure_man/shapley_feature_effect}
%   }
% \end{center}
\end{frame}




% \begin{frame}{Cooperative Games with Interactions}
% \textbf{Question:} What is a fair payout for a specific player (e.g., \enquote{yellow})?\\
% \textbf{Idea:} Compute that player's marginal contribution across all coalitions

% \begin{columns}[T, totalwidth=\linewidth]
% \begin{column}{0.53\textwidth}
% \includegraphics[page=4, trim=150px 120px 164px 10px, clip]{figure/Shapley.pdf}

% %\includegraphics[page=4, trim=170px 15px 200px 305px, clip, width = 0.8\textwidth]{figure/Shapley.pdf}
% %\scriptsize
% %\hspace{-100px}
% \begin{itemize}
% \itemsep0em
% %\addtolength{\itemindent}{-1cm}
%     \item Compute coalition payouts without \enquote{yellow} and after adding \enquote{yellow}
%     \item Take the difference to obtain marginal contribution of  \enquote{yellow}
%     \item Average these contributions across all coalitions using weighted aggregation
% \end{itemize}
% \end{column}
% \begin{column}{0.47\textwidth}
% \pause
% \textbf{Note:} Weights depend on coalition size; weight is highest for small and large $|S|$ and lowest in the middle
% % \textbf{Note:} Each contribution is weighted by number of possible orders of its coalition\\
% % $\leadsto$ More players in $S$ $\Rightarrow$ more ways to order players before adding \enquote{yellow}
% %The more players in a coalition, the more possible orderings inside the coalition %more possibilities of ordering the players inside the coalition
% %\includegraphics[page=15, width = \textwidth, trim=122px 10px 82px 10px, clip]{figure/Shapley.pdf}
% \includegraphics[page=7, width = \textwidth, trim=230px 0px 25px 15px, clip]{figure/Shapley.pdf}
% \end{column}
% \end{columns}
% \end{frame}

% \begin{frame}{Cooperative Games without Interactions}
% \begin{figure}
%     \centering
%     \includegraphics{figure/Shapley_1.png}
% \end{figure}
% \end{frame}

% \begin{frame}{Fair Payouts are Trivial Without Interactions}
% \begin{figure}
%     \centering
%     \includegraphics{figure/Shapley_2.png}
% \end{figure}

% \end{frame}
% \begin{frame}{Cooperative Games with Interactions}
% \begin{figure}
%     \centering
%     \includegraphics{figure/Shapley_3.png}
% \end{figure}
%\end{frame}

% \begin{frame}{What is a fair payout for player \enquote{yellow}?}

% \begin{figure}
%     \centering
%     \includegraphics{figure/Shapley_4.png}
% \end{figure}

% \end{frame}



\begin{frame}{Shapley Value - Order Definition}
%The Shapley value was introduced as summation over sets $\SsubPnoj$, but it can be equivalently defined as a summation of all orders of players: 
%(which explains where the factor $\frac{|S|!(|P| - |S| - 1)!}{|P|!}$ comes from):

\textbf{The Shapley value order definition} averages the marginal contribution of a player across all possible player orderings:

$$\phi_j = \frac{1}{|P|!} \sum_{\tau \in \Pi} (v(\Stau \cup \{j\}) - v(\Stau))$$
  
\begin{itemize}[<+->]
  \item $\Pi$: Set of all permutations (joining orders) of the players -- there are \(|P|!\) in total
  %\item Recall the order definition: 
  \item $\Stau$: Set of players before \(j\) joins, for each ordering \(\tau = (\tau^{(1)}, \dots, \tau^{(p)})\)\\
  %Set of players before player $j$ in order $\tau = (\tau^{(1)}, \dots, \tau^{(p)})$  where $\tau^{(i)}$ is $i$-th element \\ % j, \dots, 
  %\only<3>{\textbf{For $P = \{1,2,3\}$:} $\Pi = \{({\color{red} 1,2},3), (1,3,2), ({\color{red} 2,1},3), (2,3,1), (3,1,2), (3,2,1)\}$  \\}%
  \only<1->{\textbf{E.g.:} 
  $\Pi = \footnotesize \{
  (\colorcircle{playerred}{white}{1},\colorcircle{playeryellow}{white}{2},\colorcircle{playerblue}{white}{3}),
  (\colorcircle{playerred}{white}{1},\colorcircle{playerblue}{white}{3},\colorcircle{playeryellow}{white}{2}),
 (\colorcircle{playeryellow}{white}{2},\colorcircle{playerred}{white}{1},\colorcircle{playerblue}{white}{3}),
  (\colorcircle{playeryellow}{white}{2},\colorcircle{playerblue}{white}{3},\colorcircle{playerred}{white}{1}),
  (\colorcircle{playerblue}{white}{3},\colorcircle{playerred}{white}{1},\colorcircle{playeryellow}{white}{2}),
  (\colorcircle{playerblue}{white}{3},\colorcircle{playeryellow}{white}{2},\colorcircle{playerred}{white}{1}) 
\}$
  %$\Pi = \{(1,2,3), (1,3,2), (2,1,3), (2,3,1), (3,1,2), (3,2,1)\}$ 
  \\}%
  \phantom{$\Rightarrow$} $\leadsto$ For joining order $\tau = \footnotesize(\colorcircle{playeryellow}{white}{2},\colorcircle{playerred}{white}{1},\colorcircle{playerblue}{white}{3})$ and player $j=\footnotesize\colorcircle{playerblue}{white}{3}$ $\Rightarrow$ $\Stau = \footnotesize\{\colorcircle{playeryellow}{white}{2},\colorcircle{playerred}{white}{1}\}$\\
  \phantom{$\Rightarrow$} $\leadsto$ For joining order $\tau = (\footnotesize\colorcircle{playerblue}{white}{3},\colorcircle{playerred}{white}{1},\colorcircle{playeryellow}{white}{2})$ and player  $j=\footnotesize\colorcircle{playerred}{white}{1}$ $\Rightarrow$ $\Stau = \footnotesize\{\colorcircle{playerblue}{white}{3}\}$
  %\phantom{$\Rightarrow$} $\leadsto$ For joining order $\tau = (3,1,2)$ and player of interest $j=3$ $\Rightarrow$ $\Stau = \emptyset$
  %\item In order definition, we sum the marginal contribution twice for orders that yield set $S = \{1,2\}$\\
  %\item \textbf{Note:} Marginal contribution of orders that yield set {\color{red} $S = \{1,2\}$} is summed twice\\
  %$\leadsto$ In set definition, it has the weight $\tfrac{2! (3 - 2 - 1)!}{3!} = \tfrac{2 \cdot 0!}{6} = \tfrac{2}{6}$
  %\item<3-> \textbf{Note:} Subsets may appear in multiple orderings $\Rightarrow$ duplicated elements in sum\\
  %$\leadsto$ For \(j = 3\), joining order \((1,2,3)\) and \((2,1,3)\) yields subset \({\color{red} \Stau= \{1,2\}}\) %and are included separately in the sum
\item<3-> Order definition allows to approximate Shapley values by sampling permutations \\
\(\leadsto\) Sample a fixed number \( M \ll |P|! \) of random permutations and average:

\[
\phi_j \approx \frac{1}{M} \sum_{\tau \in \Pi_M} \left(v(\Stau \cup \{j\}) - v(\Stau)\right)
\]

where \( \Pi_M \subset \Pi \) is the random sample of \( M \) player orderings
  % Order definition allows to approximate Shapley values by sampling permutations \\
  % $\leadsto$ randomly sample a fixed number of $M$ permutations and average them:
  % %to approximate the Shapley values
  % %instead of producing all $|P|!$ permutations, a fixed number of $M< |P|!$ permutations can be sampled and averaged to approximate the Shapley values
  % $$\phi_j = \frac{1}{M} \sum_{\tau \in \Pi_M} (v(\Stau \cup \{j\}) - v(\Stau))$$
  %   where $\Pi_M \subset \Pi$ is a random subset of $\Pi$ containing only $M$ orders of players\\
  %   %$\leadsto$ $\Stau$: For permutation $\tau$, the set of players before player $j$ in order $\tau$ 
\end{itemize}

\end{frame}


\begin{frame}{From Order Definition to Set Definition}

\begin{itemize}
\item<1-> \textbf{Note:} The same subset \(\Stau\) can occur in multiple permutations (joining orders)\\
\(\leadsto\) Its marginal contribution is included multiple times in the sum in $\phi_j$

\item<1-> \textbf{Example (for set of players \(P = \{\colorcircle{playerred}{white}{1},\colorcircle{playeryellow}{white}{2},\colorcircle{playerblue}{white}{3}\}\), player of interest \(j = \colorcircle{playerblue}{white}{3}\)):}

\smallskip

%\(\Pi = \{({\color{red}1,2},3), (1,3,2), ({\color{red}2,1},3), (2,3,1), (3,1,2), (3,2,1)\}\)
\(\Pi = \{
  \underline{\color{black}(\colorcircle{playerred}{white}{1},\colorcircle{playeryellow}{white}{2},\colorcircle{playerblue}{white}{3})},\ 
  (\colorcircle{playerred}{white}{1},\colorcircle{playerblue}{white}{3},\colorcircle{playeryellow}{white}{2}),\ 
 \underline{\color{black}(\colorcircle{playeryellow}{white}{2},\colorcircle{playerred}{white}{1},\colorcircle{playerblue}{white}{3})},\ 
  (\colorcircle{playeryellow}{white}{2},\colorcircle{playerblue}{white}{3},\colorcircle{playerred}{white}{1}),\ 
  (\colorcircle{playerblue}{white}{3},\colorcircle{playerred}{white}{1},\colorcircle{playeryellow}{white}{2}),\ 
  (\colorcircle{playerblue}{white}{3},\colorcircle{playeryellow}{white}{2},\colorcircle{playerred}{white}{1}) 
\}\)

\smallskip

\(\leadsto\) In both \(\underline{\color{black}(\colorcircle{playerred}{white}{1},\colorcircle{playeryellow}{white}{2},\colorcircle{playerblue}{white}{3})}\) and \( \underline{\color{black}(\colorcircle{playeryellow}{white}{2},\colorcircle{playerred}{white}{1},\colorcircle{playerblue}{white}{3})}\), player \colorcircle{playerblue}{white}{3} joins after coalition \({\color{black} \Stau = \{\colorcircle{playerred}{white}{1},\colorcircle{playeryellow}{white}{2}\}}\)

\(\Rightarrow\) Marginal contribution \(v(\{\colorcircle{playerred}{white}{1},\colorcircle{playeryellow}{white}{2},\colorcircle{playerblue}{white}{3}\}) - v(\{\colorcircle{playerred}{white}{1},\colorcircle{playeryellow}{white}{2}\})\) occurs twice in $\phi_j$
% \item<1->  \textbf{Note:} Subsets may appear in multiple orderings $\Rightarrow$ duplicated elements in sum\\
%   $\leadsto$ For \(j = 3\), joining order \((1,2,3)\) and \((2,1,3)\) yields subset \({\color{red} \Stau= \{1,2\}}\) %and are included separately in the sum
%   %\item<1-> Order and set definition are equivalent
%   \textbf{For $P = \{1,2,3\}$:} $\Pi = \{({\color{red} 1,2},3), (1,3,2), ({\color{red} 2,1},3), (2,3,1), (3,1,2), (3,2,1)\}$  
  \item<1-> \textbf{Reason:} Each subset $S$ appears in $|S|!(|P| - |S| - 1)!$ orderings before $j$ joins\\
  %The number of orders which yield the same coalition $S$ is $|S|!(|P| - |S| - 1)!$\\
  $\Rightarrow$ There are $|S|!$ possible orders of players within coalition $S$\\
  $\Rightarrow$ There are $(|P| - |S| - 1)!$ possible orders of players without $S$ and $j$
  \centerline{
  \begin{tabular}{|c|c|c|c|c|c|c|}
    \multicolumn{3}{c}{\enspace\raisebox{-3.3ex}[0pt][2.6ex]{$ \overbrace{\vphantom{-}\hspace{9em}}^{|S|! \text{ permutations}}$}} &
    \multicolumn{1}{c}{} &
    \multicolumn{3}{c}{\enspace\raisebox{-3.3ex}[0pt][2.6ex]{$ \overbrace{\vphantom{-}\hspace{9em}}^{(|P| - |S| - 1)! \text{ permutations}}$}}\\
    \hline
    $\tau^{(1)}$ & \ldots & $\tau^{(|S|)}$ & $\tau^{(|S| + 1)}$ & $\tau^{(|S| + 2)}$ & \ldots & $\tau^{(p)}$ \\
    \hline
    \multicolumn{3}{c}{\enspace\raisebox{1.3ex}[0pt][2.6ex]{$ \underbrace{\vphantom{-}\hspace{9em}}^{}$}} &
    \multicolumn{1}{c}{\enspace\raisebox{1.3ex}[0pt][2.6ex]{$ \underbrace{\vphantom{-}\hspace{4em}}^{}$}} &
    \multicolumn{3}{c}{\enspace\raisebox{1.3ex}[0pt][2.6ex]{$ \underbrace{\vphantom{-}\hspace{9em}}^{}$}}\\
    \multicolumn{3}{c}{Players before player $j$} & \multicolumn{1}{c}{player $j$} & \multicolumn{3}{c}{Players after player $j$} \\
  \end{tabular}}
  
\end{itemize}

\end{frame}


% How to weight differences in payout
\begin{frame}{From Order Definition to Set Definition}
%   \begin{center}
%   \only<1>{\includegraphics{figure/Shapley_7.png}}%
%   \only<2>{ \includegraphics{figure/Shapley_8.png}}%
%   \only<3>{ \includegraphics{figure/Shapley_9.png}}%
%   \end{center}
  
\begin{center}
\includegraphics<1>[page=5, width = 0.8\textwidth]{figure/Shapley.pdf}%
\includegraphics<2>[page=6, width = 0.8\textwidth]{figure/Shapley.pdf}%
\includegraphics<3>[page=7, width = 0.8\textwidth]{figure/Shapley.pdf}%
\end{center}
    \begin{itemize}
      \item \textbf{Order view:} Each of the \( |P|! \) permutations contributes one term with weight \( \tfrac{1}{|P|!} \)
      \item Same subset \( S \subseteq P \setminus \{j\} \) can appear before \( j \) in multiple orders\\
      $\leadsto$ e.g., S = \{\colorcircle{playerblue}{playerblue}{3}, \colorcircle{playerred}{playerred}{1}\} = \{\colorcircle{playerred}{playerred}{1}, \colorcircle{playerblue}{playerblue}{3}\}
      \item \textbf{Set view:} Group by unique subsets \( S \), not permutations
      \item Each \( S \) occurs in \( |S|!(|P|-|S|-1)! \) orderings
      \(\leadsto\) Weight:
      $
      \frac{|S|!(|P| - |S| - 1)!}{|P|!}
      $
    \end{itemize}
\end{frame}


\begin{frame}{Shapley Value - Set definition}
Shapley value via \textbf{set definition} (weighting via multinomial coefficient): 
  $$\phi_j = \sum_{\SsubPnoj} \frac{|S|!(|P| - |S| - 1)!}{|P|!}(v(\Scupj) - v(S))$$

  The coefficient gives the probability that, when randomly arranging all $|P|$ players, the exact set $S$ appears before player $j$, and the remaining players appear afterward.
\centerline{
  \begin{tabular}{|c|c|c|c|c|c|c|}
    \multicolumn{3}{c}{\enspace\raisebox{-3.3ex}[0pt][2.6ex]{$ \overbrace{\vphantom{-}\hspace{9em}}^{|S|! \text{ permutations}}$}} &
    \multicolumn{1}{c}{\enspace\raisebox{-3.3ex}[0pt][1.1ex]{$ \overbrace{\vphantom{-}\hspace{3em}}^{\text{player } j}$}} &
    \multicolumn{3}{c}{\enspace\raisebox{-3.3ex}[0pt][2.6ex]{$ \overbrace{\vphantom{-}\hspace{9em}}^{(|P| - |S| - 1)! \text{ permutations}}$}}\\
    \hline
    $\tau^{(1)}$ & \ldots & $\tau^{(|S|)}$ & $\tau^{(|S| + 1)}$ & $\tau^{(|S| + 2)}$ & \ldots & $\tau^{(|P|)}$ \\
    \hline
  \end{tabular}}

\pause
    % code for the figure (https://colab.research.google.com/drive/1kOEZkT2jgjgaK9OFRbnfx8-_C8z7t8v3?usp=sharing)
    \begin{minipage}[c]{0.44\textwidth}
        \includegraphics[width=\textwidth]{figure_man/multinom_coef_over_size.png}
    \end{minipage}
    %\hspace{0.01\textwidth} 
    \begin{minipage}[c]{0.55\textwidth}
\begin{itemize}
  \item \( |S| = 0 \): player \( j \) joins first\\ \( \Rightarrow \) many permutations \( \Rightarrow \) high weight
  \item \( |S| = |P| - 1 \): player \( j \) joins last\\
  \( \Rightarrow \) many permutations \( \Rightarrow \) high weight
  \item Middle-sized \( |S| \): fewer exact matches \\
  \( \Rightarrow \) lower weight
  \item Result: U-shaped weight distribution %over \( |S| \) forms a 
\end{itemize}
        % \small
        % \begin{itemize}
        %   \item Joining early yields a larger weight, because it is harder to contribute when the coalition is still small.
        %   \item Joining last leads to a high weight, since it is challenging to add value to an almost-complete coalition.
        %   \item Being in the middle results in a lower weight.
        % \end{itemize}
    \end{minipage}
\end{frame}

% \begin{frame}{Shapley Value - Set definition}

% This idea refers to the \textbf{Shapley value} which assigns a payout value to each player according to its marginal contribution in all possible coalitions.
% % which  provide a unique solution for the payout attribution problem.
  
% \begin{itemize}[<+->]
%   %\item Shapley values were proposed by Lloyd Shapley in 1951 for cooperative games (game theory).
%   % given axioms of efficiency, symmetry, dummy and additivity.
%   %\item The \textbf{Shapley value} of player $j$ assigns a payout value according to the marginal contribution of each player in all possible coalitions\\
%   %$\leadsto$ %To compute the Shapley payout for a player $j$, we 
%   \item Let $v(\Scupj) - v(S)$ be the marginal contribution of player $j$ to coalition $S$\\
%   $\leadsto$ measures how much a player $j$ increases the value of a coalition $S$
%   \item Average marginal contributions for all possible coalitions $\SsubPnoj$ \\
%   $\leadsto$ order of how players join the coalition matters $\Rightarrow$ different weights depending on size of $S$\\
%   %$\leadsto$ marginal contributions are weighted differently
%   \item Shapley value via \textbf{set definition} (weighting via multinomial coefficient): 
%   $$\phi_j = \sum_{\SsubPnoj} \frac{|S|!(|P| - |S| - 1)!}{|P|!}(v(\Scupj) - v(S))$$
    
%   %\item Shapley values are the \texthetit{only} solution for the attribution with the specified axioms.
% %   \item Equivalent Shapley value definition via \textbf{orders}: $$\phi_j = \frac{1}{|P|!} \sum_{\tau \in \Pi} (v(\Stau \cup \{j\}) - v(\Stau))$$
% %     $\leadsto$ $\Pi$: All possible $|P|!$ orders/permutations of players\\
% %     $\leadsto$ $\Stau$: The set of players before player $j$ in order $\tau = (\tau^{(1)}, \dots, \tau^{(p)})$ % j, \dots, 
% \end{itemize}

% \end{frame}

% \begin{frame}{Shapley Value - Multinomial coefficient}
% \[
% \frac{|S|!(|P| - |S| - 1)!}{|P|!}
% \]
%   The coefficient tells us the probability that, when randomly arranging all $|P|$ players, the exact set $S$ appears before player $j$, and the remaining players appear afterward.
% \centerline{
%   \begin{tabular}{|c|c|c|c|c|c|c|}
%     \multicolumn{3}{c}{\enspace\raisebox{-3.3ex}[0pt][2.6ex]{$ \overbrace{\vphantom{-}\hspace{9em}}^{|S|! \text{ permutations}}$}} &
%     \multicolumn{1}{c}{\enspace\raisebox{-3.3ex}[0pt][1.1ex]{$ \overbrace{\vphantom{-}\hspace{3em}}^{\text{player } j}$}} &
%     \multicolumn{3}{c}{\enspace\raisebox{-3.3ex}[0pt][2.6ex]{$ \overbrace{\vphantom{-}\hspace{9em}}^{(|P| - |S| - 1)! \text{ permutations}}$}}\\
%     \hline
%     $\tau^{(1)}$ & \ldots & $\tau^{(|S|)}$ & $\tau^{(|S| + 1)}$ & $\tau^{(|S| + 2)}$ & \ldots & $\tau^{(|P|)}$ \\
%     \hline
%   \end{tabular}}

% \pause
% \begin{figure}[h!]
%     \centering
%     % code for the figure (https://colab.research.google.com/drive/1kOEZkT2jgjgaK9OFRbnfx8-_C8z7t8v3?usp=sharing)
%     \begin{minipage}[b]{0.4\textwidth}
%         \includegraphics[width=\textwidth]{figure_man/multinom_coef_over_size.png}
%     \end{minipage}
%     \hspace{0.05\textwidth} 
%     \begin{minipage}[b]{0.5\textwidth}
%         \small
%         \begin{itemize}
%           \item Joining early yields a larger weight, because it is harder to contribute when the coalition is still small.
%           \item Joining last leads to a high weight, since it is challenging to add value to an almost-complete coalition.
%           \item Being in the middle results in a lower weight.
%         \end{itemize}
%     \end{minipage}
% \end{figure}
% \end{frame}

% \begin{frame}{How to Weight Differences in Payout?}

% \begin{itemize}
%     %\itemsep2em
%     %\item Form a coalition one player at a time -- each player receives its contribution to the total payout, i.e., the increase in total payout when a player joins a coalition
%     % \item The Shapley value is the players' average contribution over all possible formations of coalitions\\
%     % $\leadsto$ order of how players join the coalition matters\\
%     % $\leadsto$ marginal contributions are weighted
%     %\item Each contribution is weighted proportionally to the number of possible orders of its coalition -- the more players in a coalition, the more possibilities of ordering the players inside the coalition
%     \item Shapley value via \textbf{set definition} (weighting via multinomial coefficient): 
%     $$\phi_j = \sum_{\SsubPnoj} \frac{|S|!(|P| - |S| - 1)!}{|P|!}(v(\Scupj) - v(S))$$
%     \item Shapley value via \textbf{order definition}: $$\phi_j = \frac{1}{|P|!} \sum_{\tau \in \Pi} (v(\Stau \cup \{j\}) - v(\Stau))$$
    
%     with $\Pi$ being all possible $|P|!$ orders/permutations of players and $\Stau$ being the set of players before player $j$ in order $\tau$ 
    
%     %$\phi_j = \frac{1}{|P|!} \sum_{\tau \in \Pi} (v(S_j(\tau) \cup \{j\}) - v(S_j(\tau)))$
% \end{itemize}

% \end{frame}

% \begin{frame}{How to Weight Differences in Payout?}

% \begin{figure}
%     \centering
%     \includegraphics{figure/Shapley_5.png}
% \end{figure}

% \end{frame}



% \begin{frame}{From Game Theory To Machine Learning}

% \begin{figure}
%     \centering
%     \includegraphics{figure/Shapley_5.png}
% \end{figure}

% \end{frame}

% \begin{frame}{Shapley Values}

%   Shapley values provide a unique solution to the attribution problem while satisfying all axioms:
%     \vspace{0.25cm}
% \begin{itemize}
%   \itemsep1em
%   \item Shapley values were proposed by Lloyd Shapley in 1951.
%   \item The Shapley value assigns a value to each player according to the marginal contribution of each player in all possible coalitions.
%   \item $\phi_j = \sum_{\SsubPnoj} \frac{|S|!(|P| - |S| - 1)!}{|P|!}(v(\Scupj) - v(S))$
%   \item $v(\Scupj) - v(S)$ is the marginal contribution of player $j$ to coalition $S$.
%   \item To compute the Shapley payout for a player, we average, for all possible coalitions, how much the player would increase the value of the coalition (=marginal contribution).
%   \item Shapley values are the \textit{only} solution for the attribution with the specified axioms.
% \end{itemize}

% \footnote{Shapley, Lloyd S. (August 21, 1951). "Notes on the n-Person Game -- II: The Value of an n-Person Game" (PDF). Santa Monica, Calif.: RAND Corporation.}

% \end{frame}


\begin{frame}{Axioms of Fair Payouts}
\textbf{What makes a payout fair?} The Shapley value provides a fair payout $\phi_j$ for each player \( j \in P \) and uniquely satisfies the following axioms for any value function \( v\):

\medskip

\begin{itemize}[<+->]
  \item \textbf{Efficiency}: Total payout $v(P)$ is fully allocated to players:
  \[
  \textstyle \sum_{j \in P} \phi_j = v(P)
  \]

  \item \textbf{Symmetry}: Indistinguishable players $j,k \in P$ receive equal shares:
  \[
  \text{If } v(S \cup \{j\}) = v(S \cup \{k\}) \text{ for all } S \subseteq P \setminus \{j,k\}, \text{ then } \phi_j = \phi_k
  \]

  \item \textbf{Null Player (Dummy)}: Players who contribute nothing receive nothing:
  \[
  \text{If } v(S \cup \{j\}) = v(S) \text{ for all } S \subseteq P \setminus \{j\}, \text{ then } \phi_j = 0
  \]

  \item \textbf{Additivity}: For two separate games with value functions \( v_1, v_2 \), define a combined game with \( v(S) = v_1(S) + v_2(S) \) for all \( S \subseteq P \). Then:
  \[
  \phi_{j, v_1 + v_2} = \phi_{j, v_1} + \phi_{j, v_2}\]
  $\leadsto$ Payout of combined game = payout of the two separate games
  %A combined game is one where each coalition's value is the sum of its values in two separate games.
\end{itemize}
\end{frame}




% \begin{frame}{Axioms of Fair Payouts}
%  Why is this a fair payout solution?
%  \\
%  One possibility to define fair payouts are the following axioms for a given value function $v$:
%   \vspace{0.25cm}
%   \begin{itemize}[<+->]
%   \itemsep1em
%     \item \textbf{Efficiency}: Player contributions add up to the total payout of the game:
%       $\sum\nolimits_{j=1}^p\phi_j = v(P)$
%     % symmetry - "any" confused me a bit since we actually look into coalitions that do not include both j and k. Maybe we can state this more as "for any coalition it make no difference whether we add $j$ or $k$"
%     \item \textbf{Symmetry}: For any players $j,k \in P$ who contribute the same to any coalition $S$ get the same payout: \\ 
%       If $v(\Scupj) = v(\Scupk)$ for all $\SsubP \setminus\{j,k\}$, then $\phi_j=\phi_k$
%     \item \textbf{Dummy/Null Player}: Payout is 0 for players who do not contribute to the value of any coalition: \\
%       If $v(\Scupj)=v(S)\quad  \forall \quad \SsubP \setminus \{j\}$, then $\phi_j=0$
%     \item \textbf{Additivity}: For games with value functions $v_1, v_2$, their combined value function $v(S) = v_1(S) + v_2(S)$ implies that the payout $\phi_{j,v}$ is $\phi_{j,v_1} + \phi_{j, v_2}$
%   \end{itemize}
%   \vspace{0.5cm}
  

% \end{frame}


% \begin{frame}{Banzhaf Value}

% \textbf{Banzhaf value:} Average marginal contribution of \(j\) across all \(S\) \textbf{with equal weight}
      
% \[
% \beta_j = \frac{1}{2^{p-1}} \sum_{S \subseteq P \setminus \{j\}} \Delta(j,S) = \frac{1}{2^{p-1}} \sum_{S \subseteq P \setminus \{j\}} \left( v(S \cup \{j\}) - v(S) \right)
% \]

% \vspace{1em}
% \begin{columns}[T, totalwidth=\textwidth]
  
%       \begin{column}{0.45\textwidth}
%         %\small
% \begin{tabular}{|c|c|c|c|}
% \hline
% $j$ & $S$ & $\Delta(j,S)$ & Weight \\
% \hline
% \multirow{4}{*}{3} 
%   & $\emptyset$         & $+3000$ & $1/4$ \\
%   & $\{1\}$             & $+2000$ & $1/4$ \\
%   & $\{2\}$             & $+3000$ & $1/4$ \\
%   & $\{1,2\}$           & $+2000$ & $1/4$ \\
% \hline
% \multirow{4}{*}{2} 
%   & $\emptyset$         & $+2000$ & $1/4$ \\
%   & $\{1\}$             & $+1000$ & $1/4$ \\
%   & $\{3\}$             & $+3000$ & $1/4$ \\
%   & $\{1,3\}$           & $+2000$ & $1/4$ \\
% \hline
% \multirow{4}{*}{1} 
%   & $\emptyset$         & $+1000$ & $1/4$ \\
%   & $\{2\}$             & $+1000$ & $1/4$ \\
%   & $\{3\}$             & $+4000$ & $1/4$ \\
%   & $\{2,3\}$           & $+3000$ & $1/4$ \\
% \hline
% \end{tabular}
%     \end{column} 
%     \begin{column}{0.55\textwidth}
  

% \textbf{Banzhaf values:} \\ 
% \(\beta_1 = 2250\), 
% \(\beta_2 = 2000\), 
% \(\beta_3 = 2500\)

% \vspace{0.5em}

% \textbf{Total:} \(\beta_1 + \beta_2 + \beta_3 = 6750 \neq v(P) = 6000\)

% \vspace{0.5em}

% \textbf{Conclusion:} Payout exceeds total value $v(P)$
% %Banzhaf values do \textbf{not enforce efficiency} -- payout exceeds total value.

%     \end{column}
% \end{columns}
% \vspace{1em}

% \end{frame}

% \begin{frame}{Proof of Axioms}
%   The Shapley values fulfills all the 4 axioms.
%   Symmetry, Dummy and Additivity are relatively easy to proof:
% \begin{itemize}
%     % See also: https://math.stackexchange.com/questions/2747088/shapley-value-is-efficient
%   \item \textbf{Symmetry}: Let's assume coalition $\SsubPnojk$ and $v(\Scupj) = v(\Scupk)$.
%     \begin{itemize}
%         \item Then all marginal contributions are equal: $v(\Scupj) -  v(S) = v(\Scupk) -  v(S), \forall \SsubPnojk$
%         \item Consequently, $\phi_j = \phi_k$.
%     \end{itemize}
%   \item \textbf{Dummy}: Let's assume we have player $j$ such that for all $\SsubP$, we have $v(S) = v(\Scupj)$. Then, each marginal contribution of player $j$ is zero, and therefore $\phi_j = 0$.
%   \item \textbf{Additivity}:  Assume two games $v_1$ and $v_2$ and a third game which is the sum of both $v(S) = v_1(S) + v_2(S)$. The marginal contribution for all $\SsubPnoj$ for game $v$ can be expressed as $v(\Scupj) - v(S) = v_1(\Scupj) - v_1(S) + v_2(\Scupj) - v_2(S)$. Since the Shapley value is additive in the marginal contributions, we can split the sum into two sums so that $\phi_{j,v} = \phi_{j, v_1} + \phi_{j,v_2}$.
% \end{itemize}
% \end{frame}

% \begin{frame}{Proof of Axioms}

% Efficiency requires a bit more effort, see proof sketch:
%   \begin{itemize}
%   \item \textbf{Efficiency}: $v(P)$ exactly appears once per player ($=p$ times) for coalition $S = P \setminus \{j\}$ with the weight $\frac{|P - 1|!(|P| - |P - 1| - 1)!}{|P|!} = \frac{1}{p}$ each. The values for all other coalitions $v(S), \SsubP \{j,k\}$ appear with both minus and plus signs that cancel each other out.
% \end{itemize}
% \end{frame}


% \begin{frame}{Linearity Axiom Corollary}
%   \begin{itemize}
%   \item The Shapley values are also linear:
%   \item For a game with payout $v(S) = \alpha v_1(S) + v_2(S)$, the Shapley values are $\alpha \phi_{j,v_1} + \phi_{j,v_2}$.
%   \item The multiplication with $\alpha$ works since we can pull out $\alpha$ from the sum of marginal contributions, so that for $v(S) = \alpha v_1(S)$ the Shapley value is $\alpha \phi_{j,v_1}$.
%   \item We already know that Shapley values are additive which proofs the linearity.
%   \end{itemize}
% \end{frame}



% \begin{frame}{Applications of Shapley Value}

%   \begin{itemize}
%       \item Game theory
%       \item Economics (e.g., cost allocation)
%       \item Marketing (e.g., social network analysis to discover influencers)
%       \item ...
%       \item Machine learning
%       \begin{itemize}
%          \item Feature selection: Attribute loss reduction to features.
%          \item Quantify data value: Attribute loss reduction to data points.
%          \item \textbf{Explain individual predictions}.
%       \end{itemize}
%   \end{itemize}

%   % Economics
%   \tiny{Moulin, Hervé. "An application of the Shapley value to fair division with money." Econometrica: Journal of the Econometric Society (1992): 1331-1349.}
% % Network analysis
%   \tiny{Narayanam, Ramasuri, and Yadati Narahari. "A shapley value-based approach to discover influential nodes in social networks." IEEE Transactions on Automation Science and Engineering 8.1 (2010): 130-147.}
%   % Feature Selection
%   \tiny{Cohen, Shay B., Eytan Ruppin, and Gideon Dror. "Feature Selection Based on the Shapley Value." IJCAI. Vol. 5. 2005.}
%   % Value of data
%   \tiny{Ghorbani, Amirata, and James Zou. "Data shapley: Equitable valuation of data for machine learning." International Conference on Machine Learning. PMLR, 2019.}
% \end{frame}

\endlecture
\end{document}
