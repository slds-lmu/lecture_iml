\documentclass[11pt,compress,t,notes=noshow, aspectratio=169, xcolor=table]{beamer}

\usepackage{../../style/lmu-lecture}
% Defines macros and environments
\usepackage{bbm}
% basic latex stuff
\newcommand{\pkg}[1]{{\fontseries{b}\selectfont #1}} %fontstyle for R packages
\newcommand{\lz}{\vspace{0.5cm}} %vertical space
\newcommand{\dlz}{\vspace{1cm}} %double vertical space
\newcommand{\oneliner}[1] % Oneliner for important statements
{\begin{block}{}\begin{center}\begin{Large}#1\end{Large}\end{center}\end{block}}


%new environments
\newenvironment{vbframe}  %frame with breaks and verbatim
{
 \begin{frame}[containsverbatim,allowframebreaks]
}
{
\end{frame}
}

\newenvironment{vframe}  %frame with verbatim without breaks (to avoid numbering one slided frames)
{
 \begin{frame}[containsverbatim]
}
{
\end{frame}
}

\newenvironment{blocki}[1]   % itemize block
{
 \begin{block}{#1}\begin{itemize}
}
{
\end{itemize}\end{block}
}

\newenvironment{fragileframe}[2]{  %fragile frame with framebreaks
\begin{frame}[allowframebreaks, fragile, environment = fragileframe]
\frametitle{#1}
#2}
{\end{frame}}


\newcommand{\myframe}[2]{  %short for frame with framebreaks
\begin{frame}[allowframebreaks]
\frametitle{#1}
#2
\end{frame}}

\newcommand{\remark}[1]{
  \textbf{Remark:} #1
}


\newenvironment{deleteframe}
{
\begingroup
\usebackgroundtemplate{\includegraphics[width=\paperwidth,height=\paperheight]{../style/color/red.png}}
 \begin{frame}
}
{
\end{frame}
\endgroup
}
\newenvironment{simplifyframe}
{
\begingroup
\usebackgroundtemplate{\includegraphics[width=\paperwidth,height=\paperheight]{../style/color/yellow.png}}
 \begin{frame}
}
{
\end{frame}
\endgroup
}\newenvironment{draftframe}
{
\begingroup
\usebackgroundtemplate{\includegraphics[width=\paperwidth,height=\paperheight]{../style/color/green.jpg}}
 \begin{frame}
}
{
\end{frame}
\endgroup
}
% https://tex.stackexchange.com/a/261480: textcolor that works in mathmode
\makeatletter
\renewcommand*{\@textcolor}[3]{%
  \protect\leavevmode
  \begingroup
    \color#1{#2}#3%
  \endgroup
}
\makeatother


\providecommand{\tightlist}{%
  \setlength{\itemsep}{0pt}\setlength{\parskip}{0pt}}

%\setbeamerfont{footnote}{size=\tiny}
\usepackage[hang,flushmargin]{footmisc}
\renewcommand*{\footnotelayout}{\tiny}
\renewcommand*{\thefootnote}{} %\fnsymbol{footnote}

% https://tex.stackexchange.com/questions/30720/footnote-without-a-marker
% \makeatletter
% \def\blfootnote{\gdef\@thefnmark{}\@footnotetext}
% \makeatother

% https://tex.stackexchange.com/questions/357717/beamer-allowframebreaks-option-and-vertical-spacing-when-using-lists-itemize
% \setbeamertemplate{frametitle continuation}{%
%     (\insertcontinuationcount)%
%     \ifnum\insertcontinuationcount>1%
%     \vspace*{\topsep}%
%     \else%
%     %
%     \fi%
% }


%\title{iML: Ante-hoc Methods for Neural Networks}
%\subtitle{Instance-wise Feature Selection}
\title{Interpretable Machine Learning}
\date{}

\begin{document}
	%\maketitle
	\graphicspath{ {./figure/} }

\newcommand{\titlefigure}{figure/bild9}
\newcommand{\learninggoals}{
\item Instance-wise feature selection
\item Explain then predict models
\item Optimizing using explanation data}

\lecturechapter{Instance-wise Feature Selection}
\lecture{Interpretable Machine Learning}

 
\begin{frame}{Instance-wise Feature Selection}
\begin{itemize}
    \item Select a subset of features conditioned or based on the input instance
    \begin{itemize}
        \item Two instances might not have the same feature mask
    \end{itemize}
    \bigskip
    \item Instance-wise feature selection similar to feature attribution — important features are
selected
\item Unambiguous with respect to explanation (more on that later)
\end{itemize}
    
\end{frame}

\begin{frame}{Instance-wise Feature Selection - Example in Language}


\centerline{\includegraphics[width=0.15\linewidth,left]{bild5}}

\end{frame}

\begin{frame}{Instance-wise Feature Selection - Example in Language}

\centerline{\includegraphics[width=0.6\linewidth,left]{bild6}}

\end{frame}

\begin{frame}{Instance-wise Feature Selection - Example in Language}
    \includegraphics[scale=.36, left]{bild7}
\end{frame}

\begin{frame}{Instance-wise Feature Selection - Example in Language}


    \begin{figure}
    \includegraphics[width=0.6\linewidth]{bild8}
    \end{figure}
\end{frame}

\begin{frame}{Instance-wise Feature Selection - Example in Language}
    \begin{figure}
    \includegraphics[width=0.6\linewidth]{bild9}
    \end{figure}
\end{frame}

\begin{frame}{Instance-wise Feature Selection - Example in Language}
    \begin{figure}
    \includegraphics[width=0.6\linewidth]{bild10}
    \end{figure}
\end{frame}

\begin{frame}{Instance-wise Feature Selection - Example in Language}
    \begin{figure}
    \includegraphics[width=0.6\linewidth]{bild11}
    \end{figure}
\end{frame}

\begin{frame}{Instance-wise Feature Selection - Example in Language}
    \begin{figure}
    \includegraphics[width=0.6\linewidth]{bild12}
    \end{figure}
\end{frame}

\begin{frame}{Explanation Data}
    \begin{figure}
    \includegraphics[width=0.6\linewidth]{bild13}
    \end{figure}
\end{frame}

\begin{frame}{Explanation Data}
    \begin{figure}
    \includegraphics[width=0.6\linewidth]{bild14}
    \end{figure}
\end{frame}


\begin{frame}{Explanation Data}
    \begin{figure}
    \includegraphics[width=0.7\linewidth]{bild15}
    \end{figure}
\end{frame}

\begin{frame}{Explain then Predict using Explanation data}
\begin{itemize}
    \item Selecting features conditioned on individual instances results in better task performance in
comparison to global feature selection
    \bigskip
    \item Instance-wise feature selection has higher inherent sparsity
    \bigskip
    \item The output of the feature-selection stage is the explanation
    \bigskip
    \item The predictor depends solely on the masked input and therefore unambiguous with respect to
explanation
\end{itemize}
    
\end{frame}



\endlecture
\end{document}