\documentclass[11pt,compress,t,notes=noshow, aspectratio=169, xcolor=table]{beamer}

\usepackage{../../style/lmu-lecture}
% Defines macros and environments
\usepackage{bbm}
% basic latex stuff
\newcommand{\pkg}[1]{{\fontseries{b}\selectfont #1}} %fontstyle for R packages
\newcommand{\lz}{\vspace{0.5cm}} %vertical space
\newcommand{\dlz}{\vspace{1cm}} %double vertical space
\newcommand{\oneliner}[1] % Oneliner for important statements
{\begin{block}{}\begin{center}\begin{Large}#1\end{Large}\end{center}\end{block}}


%new environments
\newenvironment{vbframe}  %frame with breaks and verbatim
{
 \begin{frame}[containsverbatim,allowframebreaks]
}
{
\end{frame}
}

\newenvironment{vframe}  %frame with verbatim without breaks (to avoid numbering one slided frames)
{
 \begin{frame}[containsverbatim]
}
{
\end{frame}
}

\newenvironment{blocki}[1]   % itemize block
{
 \begin{block}{#1}\begin{itemize}
}
{
\end{itemize}\end{block}
}

\newenvironment{fragileframe}[2]{  %fragile frame with framebreaks
\begin{frame}[allowframebreaks, fragile, environment = fragileframe]
\frametitle{#1}
#2}
{\end{frame}}


\newcommand{\myframe}[2]{  %short for frame with framebreaks
\begin{frame}[allowframebreaks]
\frametitle{#1}
#2
\end{frame}}

\newcommand{\remark}[1]{
  \textbf{Remark:} #1
}


\newenvironment{deleteframe}
{
\begingroup
\usebackgroundtemplate{\includegraphics[width=\paperwidth,height=\paperheight]{../style/color/red.png}}
 \begin{frame}
}
{
\end{frame}
\endgroup
}
\newenvironment{simplifyframe}
{
\begingroup
\usebackgroundtemplate{\includegraphics[width=\paperwidth,height=\paperheight]{../style/color/yellow.png}}
 \begin{frame}
}
{
\end{frame}
\endgroup
}\newenvironment{draftframe}
{
\begingroup
\usebackgroundtemplate{\includegraphics[width=\paperwidth,height=\paperheight]{../style/color/green.jpg}}
 \begin{frame}
}
{
\end{frame}
\endgroup
}
% https://tex.stackexchange.com/a/261480: textcolor that works in mathmode
\makeatletter
\renewcommand*{\@textcolor}[3]{%
  \protect\leavevmode
  \begingroup
    \color#1{#2}#3%
  \endgroup
}
\makeatother


\providecommand{\tightlist}{%
  \setlength{\itemsep}{0pt}\setlength{\parskip}{0pt}}

%\setbeamerfont{footnote}{size=\tiny}
\usepackage[hang,flushmargin]{footmisc}
\renewcommand*{\footnotelayout}{\tiny}
\renewcommand*{\thefootnote}{} %\fnsymbol{footnote}

% https://tex.stackexchange.com/questions/30720/footnote-without-a-marker
% \makeatletter
% \def\blfootnote{\gdef\@thefnmark{}\@footnotetext}
% \makeatother

% https://tex.stackexchange.com/questions/357717/beamer-allowframebreaks-option-and-vertical-spacing-when-using-lists-itemize
% \setbeamertemplate{frametitle continuation}{%
%     (\insertcontinuationcount)%
%     \ifnum\insertcontinuationcount>1%
%     \vspace*{\topsep}%
%     \else%
%     %
%     \fi%
% }


%\title{iML: Ante-hoc Methods for Neural Networks}
%\subtitle{Learning to explain}
\title{Interpretable Machine Learning}
\date{}

\begin{document}
%	\maketitle
	\graphicspath{ {./figure/} }

\newcommand{\titlefigure}{figure/bild16}
\newcommand{\learninggoals}{
\item Optimization problems in hard-masking
\item Generative masking
\item Sampling-based instance-wise feature selection}

\lecturechapter{Learning to Explain}
\lecture{Interpretable Machine Learning}

 
\begin{frame}[c]{Instance-wise Feature Selection}
    \begin{itemize}
        \item What happens when we do not have explanation data?
        \bigskip
        \item Need to use the task-specific supervision signal to create explanations
        \bigskip
        \item Key principle: Given an instance, automatically learn to select features during inference
from the task-specific supervised signal
    \end{itemize}
\end{frame}	
	
\begin{frame}{Instance-wise Feature Selection}
    \begin{itemize}
        \item Key principle: Given an instance, automatically learn to select features during inference
        \item The selected features can be implemented as a binary mask over the original feature
space
\item Selector network selects the mask, predictor network predicts using the masked input
    \end{itemize}
    \begin{figure}
        \centering
        \includegraphics[scale=.45]{bild16}
    \end{figure}
\end{frame}
	
	
\begin{frame}{Problems in optimisation}
    \begin{itemize}
        \item Selector network selects the mask, predictor network predicts using the masked input
        \item Binary masking introduces discontinuity in the neural network
\item Discontinuity $\rightarrow$ gradient-based optimisation is not possible
\bigskip
\item How can we learn the parameters of such a network using gradient-based optimization?
    \end{itemize}
    \begin{figure}
        \centering
        \includegraphics[scale=.43]{bild16}
    \end{figure}
\end{frame}	

\begin{frame}{Generative masks}
    \begin{itemize}
        \item Masks are generated from a probability distribution
        \item Instance-wise feature selection as finding the expectation of the predictor function
distributed according to human interpretability
    \end{itemize}
    \bigskip
   \begin{equation*}
             \centering
    \mathcal{F}(\boldsymbol{\theta}):=\int p(\mathbf{m} ; \mathbf{x}, \boldsymbol{\theta}) f(\mathbf{m} \odot \mathbf{x} ; \boldsymbol{\phi}) \mathrm{d} \mathbf{x}=\mathbb{E}_{p(\mathbf{m} ; \mathbf{x}, \boldsymbol{\theta})}[f(\mathbf{m} \odot \mathbf{x} ; \boldsymbol{\phi})]
\end{equation*}
    
    
    
    %\begin{figure}
     %   \includegraphics[scale=.4]{bild17}
    %\end{figure}
\end{frame}
	
	
\begin{frame}{Instance-wise Feature Selection}
\begin{itemize}
    \item The distribution over explanations is parameterized by a neural network
    \item The predictor network is also parameterized by a neural network
\end{itemize}
\bigskip
\begin{equation*}
             \centering
    \mathcal{F}(\boldsymbol{\theta}):=\int p(\mathbf{m} ; \boldsymbol{\theta}) f(\mathbf{m} \odot \mathbf{x} ; \boldsymbol{\phi}) \mathrm{d} \mathbf{x}=\mathbb{E}_{p(\mathbf{m} ; \boldsymbol{\theta})}[f(\mathbf{m} \odot \mathbf{x} ; \boldsymbol{\phi})]
\end{equation*}

\bigskip

%\begin{figure}
%    \includegraphics[scale=.4]{bild18}
%\end{figure}
\begin{itemize}
    \item Predictor network accepts a masked input
\end{itemize}
\end{frame}

\begin{frame}{Monte Carlo Sampling}
 
$$
\mathcal{F}(\boldsymbol{\theta}):=\int p(\mathbf{m} ; \mathbf{x}, \boldsymbol{\theta}) f(\mathbf{m} \odot \mathbf{x} ; \boldsymbol{\phi}) \mathrm{d} \mathbf{x}=\mathbb{E}_{p(\mathbf{m} ; \mathbf{x}, \boldsymbol{\theta})}[f(\mathbf{m} \odot \mathbf{x} ; \boldsymbol{\phi})] 
$$


    \begin{itemize}
        \item \textbf{Trick:} $\nabla_{\boldsymbol{\theta}} \log p(\mathbf{x} ; \boldsymbol{\theta})=\frac{\nabla_{\boldsymbol{\theta}} p(\mathbf{x} ; \boldsymbol{\theta})}{p(\mathbf{x} ; \boldsymbol{\theta})}$
    \end{itemize}
\medskip

$\leadsto$ In a simplified notation (ignoring $\mathbf{m}$), we get the following:
$$
\smallskip
    \begin{aligned}
        \,\,\,\,\,\,\,\,\,\,\,\,\,\,\,\,
        \boldsymbol{\eta}:=\nabla_{\boldsymbol{\theta}} \mathcal{F}(\boldsymbol{\theta}) &=\nabla_{\boldsymbol{\theta}} \mathbb{E}_{p(\mathbf{x} ; \boldsymbol{\theta})}[f(\mathbf{x} ; \boldsymbol{\phi})] \\
        &=\nabla_{\boldsymbol{\theta}} \int p(\mathbf{x} ; \boldsymbol{\theta}) f(\mathbf{x}) d \mathbf{x}=\int f(\mathbf{x}) \nabla_{\boldsymbol{\theta}} p(\mathbf{x} ; \boldsymbol{\theta}) d \mathbf{x} \\
        &=\int p(\mathbf{x} ; \boldsymbol{\theta}) f(\mathbf{x}) \nabla_{\boldsymbol{\theta}} \log p(\mathbf{x} ; \boldsymbol{\theta}) d \mathbf{x} \\
        &=\mathbb{E}_{p(\mathbf{x} ; \boldsymbol{\theta})}\left[f(\mathbf{x}) \nabla_{\boldsymbol{\theta}} \log p(\mathbf{x} ; \boldsymbol{\theta})\right]
    \end{aligned}
$$
\end{frame}

\begin{frame}{Monte Carlo Estimator}
   
\begin{equation*}
     \mathcal{F}(\boldsymbol{\theta}):=\int p(\mathbf{m} ; \mathbf{x}, \boldsymbol{\theta}) f(\mathbf{m} \odot \mathbf{x} ; \boldsymbol{\phi}) \mathrm{d} \mathbf{x}=\mathbb{E}_{p(\mathbf{m} ; \mathbf{x}, \boldsymbol{\theta})}[f(\mathbf{m} \odot \mathbf{x} ; \boldsymbol{\phi})]
\end{equation*}

$$
\begin{aligned}
\boldsymbol{\eta}:=\nabla_{\boldsymbol{\theta}} \mathcal{F}(\boldsymbol{\theta})=\nabla_{\boldsymbol{\theta}} \mathbb{E}_{p(\mathbf{x} ; \boldsymbol{\theta})}[f(\mathbf{x} ; \boldsymbol{\phi})] &=\mathbb{E}_{p(\mathbf{x} ; \boldsymbol{\theta})}\left[f(\mathbf{x}) \nabla_{\boldsymbol{\theta}} \log p(\mathbf{x} ; \boldsymbol{\theta})\right] \\
&=\frac{1}{N} \sum_{n=1}^N f\left(\hat{\mathbf{x}}^{(n)}\right) \nabla_{\boldsymbol{\theta}} \log p\left(\hat{\mathbf{x}}^{(n)} ; \boldsymbol{\theta}\right) ; \quad \hat{\mathbf{x}}^{(n)} \sim p(\mathbf{x} ; \boldsymbol{\theta})
\end{aligned}
$$

%\begin{figure}
%\includegraphics[scale=.35]{bild20}
%\end{figure}


    \begin{itemize}
        \item Sample N masks from the probability distribution p
        \item Compute the weighted avg. of the samples where:
        \begin{itemize}
            \item weight = derivative of the log prob. of the sample mask
        \end{itemize}
        \item update the parameters of the selector network using this weighted average
    \end{itemize}
\end{frame}

\begin{frame}{Reducing variance}
  
  \begin{equation*}
     \mathcal{F}(\boldsymbol{\theta}):=\int p(\mathbf{m} ; \mathbf{x}, \boldsymbol{\theta}) f(\mathbf{m} \odot \mathbf{x} ; \boldsymbol{\phi}) \mathrm{d} \mathbf{x}=\mathbb{E}_{p(\mathbf{m} ; \mathbf{x}, \boldsymbol{\theta})}[f(\mathbf{m} \odot \mathbf{x} ; \boldsymbol{\phi})]
\end{equation*}

  \bigskip
  \begin{itemize}
      \item Monte Carlo estimators suffer from the problem of high variance
      \item Solution: introduce a constant baseline value $\beta$
  \end{itemize}
  \bigskip
 
 
 \begin{equation*}
     \eta = \mathbb{E}_{p(x;\theta)}[(f(x) - \beta)\nabla_\theta \log p(x;\theta)]
 \end{equation*} 
  
 % \begin{figure}
%\includegraphics[scale=.5]{bild21}
%  \end{figure}
  
  
\end{frame}

\begin{frame}[c]{Conclusion}
    \begin{itemize}
        \item Prefer simple models for better interpretability
        \item Regularisation for enforcing sparsity in the parameter space
        \item Feature selection for enforcing sparsity in the feature space
        \item Instance-wise feature selection selects different features based on different instances
        \item Selector and predictor architecture for instance-wise feature selection
        \item Optimisation using without explanation data requires tricks like Monte-Carlo sampling
with gradients
    \end{itemize}
\end{frame}

\begin{frame}[c]{References}
    \begin{itemize}
        \item “Learning to Explain: An Information-Theoretic Perspective on Model Interpretation” — J.
Chen, Song, M.J. Wainwright, M. I. Jordan. ICML 2018.
\begin{itemize}
    \item \url{http://proceedings.mlr.press/v80/chen18j/chen18j.pdf}
\end{itemize}
\bigskip
\item “Explain and Predict, and then Predict again" — Z Zhang, K Rudra, A Anand. WSDM 2020.
\begin{itemize}
    \item \url{https://arxiv.org/pdf/2101.04109.pdf}
\end{itemize}
\bigskip
\item “INVASE: Instance-wise Variable Selection using Neural Networks” J. Yoon, J. Jordon, M.
Schaar. ICLR 2019.
\begin{itemize}
    \item \url{https://openreview.net/pdf?id=BJg_roAcK7}
\end{itemize}
    \end{itemize}
\end{frame}
\endlecture
\end{document}	