% \documentclass[11pt,compress,t,notes=noshow, aspectratio=169, xcolor=table]{beamer}
% % deactivate beamer navigation
% %\setbeamertemplate{navigation symbols}{}
% %\usepackage{geometry}
% %\geometry{papersize={180mm, 135mm}, top=-1.5mm} % 210mm, 297mm

% % set path of iml lecture here, e.g., relative to the current dir
% \newcommand{\pathiml}{../../}
% \usepackage{\pathiml/style/lmu-lecture}
% \usepackage{pax}
% \setbeamertemplate{frametitle}{\expandafter\uppercase\expandafter\insertframetitle}
% %\useoutertheme{metropolis}
% % remove section slides
% \AtBeginSection[]
% {
%   \begin{frame}<beamer>
%     \frametitle{Interpretable Machine Learning: Shapley values}
%     \tableofcontents[currentsection, subsectionstyle=hide]
%   \end{frame}
% }
% % includepdf slides, pagecommad will set counter for framenumber
% \usepackage{pdfpages}
% \includepdfset{trim=0mm 0mm 0mm 0mm, pagecommand={\global\setcounter{framenumber}{\value{page}}}}
% % trim=0mm 6mm 0mm 0mm, offset=0 15,
% % add footer:
% \usepackage{framed, color}
% \usepackage{xcolor}
% %\iffalse
% \setbeamertemplate{footline}[text line]{%
%     \noindent\hspace*{\dimexpr-\oddsidemargin-1in\relax}%
%      \colorbox{white}{
%      \makebox[\dimexpr\paperwidth-2\fboxsep\relax]{
%      \color{black}
%      \begin{minipage}[c][4.5ex][c]{0.5\linewidth}
%       \secname
%      \end{minipage}
%      \hfill\begin{minipage}[c][4.5ex][c]{0.5\linewidth}
%       \flushright
%       \insertframenumber{}~/~\inserttotalframenumber~~
%      \end{minipage}
%      }}%
%   \hspace*{-\paperwidth}
% }
% %\fi

% \begin{document}
% \setbeamercolor{background canvas}{bg=}

% % General remark: hyperlinks in included pdfs are not clickable anymore in the combined pdf unless you use pax to extract annotations from original files and complile this file locally (see https://tex.stackexchange.com/questions/497624/merging-multiple-pdf-files-without-breaking-hyperlinks)

% \section{Shapley Values - Game Theory}
% \includepdf[pages={1-last}]{\pathiml/slides-pdf/slides-shapley-game-theory.pdf}

% \section{Shapley Values - Local Feature Effects}
% \includepdf[pages={1-last}]{\pathiml/slides-pdf/slides-shapley-ml.pdf}

% \section{Shapley Values - Local Feature Effects with SHAP}
% \includepdf[pages={1-23}]{\pathiml/slides-pdf/slides-shap.pdf}

% \section{Shapley Values - Global Feature Effects with SHAP}
% \includepdf[pages={28-last}]{\pathiml/slides-pdf/slides-shap.pdf}

% \end{document}



\documentclass[11pt,compress,t,notes=noshow, aspectratio=169, xcolor=table]{beamer}

\usepackage{../../style/lmu-lecture}
% Defines macros and environments
\usepackage{bbm}
% basic latex stuff
\newcommand{\pkg}[1]{{\fontseries{b}\selectfont #1}} %fontstyle for R packages
\newcommand{\lz}{\vspace{0.5cm}} %vertical space
\newcommand{\dlz}{\vspace{1cm}} %double vertical space
\newcommand{\oneliner}[1] % Oneliner for important statements
{\begin{block}{}\begin{center}\begin{Large}#1\end{Large}\end{center}\end{block}}


%new environments
\newenvironment{vbframe}  %frame with breaks and verbatim
{
 \begin{frame}[containsverbatim,allowframebreaks]
}
{
\end{frame}
}

\newenvironment{vframe}  %frame with verbatim without breaks (to avoid numbering one slided frames)
{
 \begin{frame}[containsverbatim]
}
{
\end{frame}
}

\newenvironment{blocki}[1]   % itemize block
{
 \begin{block}{#1}\begin{itemize}
}
{
\end{itemize}\end{block}
}

\newenvironment{fragileframe}[2]{  %fragile frame with framebreaks
\begin{frame}[allowframebreaks, fragile, environment = fragileframe]
\frametitle{#1}
#2}
{\end{frame}}


\newcommand{\myframe}[2]{  %short for frame with framebreaks
\begin{frame}[allowframebreaks]
\frametitle{#1}
#2
\end{frame}}

\newcommand{\remark}[1]{
  \textbf{Remark:} #1
}


\newenvironment{deleteframe}
{
\begingroup
\usebackgroundtemplate{\includegraphics[width=\paperwidth,height=\paperheight]{../style/color/red.png}}
 \begin{frame}
}
{
\end{frame}
\endgroup
}
\newenvironment{simplifyframe}
{
\begingroup
\usebackgroundtemplate{\includegraphics[width=\paperwidth,height=\paperheight]{../style/color/yellow.png}}
 \begin{frame}
}
{
\end{frame}
\endgroup
}\newenvironment{draftframe}
{
\begingroup
\usebackgroundtemplate{\includegraphics[width=\paperwidth,height=\paperheight]{../style/color/green.jpg}}
 \begin{frame}
}
{
\end{frame}
\endgroup
}
% https://tex.stackexchange.com/a/261480: textcolor that works in mathmode
\makeatletter
\renewcommand*{\@textcolor}[3]{%
  \protect\leavevmode
  \begingroup
    \color#1{#2}#3%
  \endgroup
}
\makeatother


\providecommand{\tightlist}{%
  \setlength{\itemsep}{0pt}\setlength{\parskip}{0pt}}

%\setbeamerfont{footnote}{size=\tiny}
\usepackage[hang,flushmargin]{footmisc}
\renewcommand*{\footnotelayout}{\tiny}
\renewcommand*{\thefootnote}{} %\fnsymbol{footnote}

% https://tex.stackexchange.com/questions/30720/footnote-without-a-marker
% \makeatletter
% \def\blfootnote{\gdef\@thefnmark{}\@footnotetext}
% \makeatother

% https://tex.stackexchange.com/questions/357717/beamer-allowframebreaks-option-and-vertical-spacing-when-using-lists-itemize
% \setbeamertemplate{frametitle continuation}{%
%     (\insertcontinuationcount)%
%     \ifnum\insertcontinuationcount>1%
%     \vspace*{\topsep}%
%     \else%
%     %
%     \fi%
% }


\newcommand\tab[1][1cm]{\hspace*{#1}}
\tikzset{main node/.style={rectangle,draw,minimum size=1cm,inner sep=4pt},}

\usepackage[export]{adjustbox}
\usepackage[most]{tcolorbox}

\newtcolorbox{BlueBox}[2][]{%
   enhanced,
   colback   = blue!5!white,
   colframe  = blue!65!black,
   arc       = 1mm,
   outer arc = 1mm,
   fonttitle = \Large\slshape\textbf,
   center title,
   title     = #2,
   #1}

\title{Interpretable Machine Learning}
% \author{LMU}
%\institute{\href{https://compstat-lmu.github.io/lecture_iml/}{compstat-lmu.github.io/lecture\_iml}}
\date{}

\newcommand{\mysectionslide}{
\begin{frame}[plain]
    %\global\advance\c@lecture by -1
    \vspace*{0.5 cm}
    \LARGE\bfseries\inserttitle
    \vspace*{0.5 cm}

    \ifx\lecturesection\@empty\relax\else%
    {\lecturesection}%
    \fi%

\ifcsname learninggoals\endcsname
  {\vspace*{0.5 cm}
  \begin{minipage}{0.4\textwidth}
  \ifcsname titlefigure\endcsname
    {\begin{center}
    \begin{figure}[!b]
    \includegraphics[width=0.9\textwidth, keepaspectratio]{\titlefigure}
     \ifcsname titlecaption\endcsname
     \caption*{\titlecaption}
     \fi
    \end{figure}
    \end{center}}
  \else
    $\;$
  \fi
  \end{minipage}
  \begin{minipage}{0.55\textwidth}
  \normalsize
  Learning goals
   \normalfont
   \footnotesize
  \begin{itemize}
  \learninggoals
  \end{itemize}
  \end{minipage}}
\fi

  \end{frame}
}


\begin{document}

\newcommand{\titlefigure}{../../slides/04_shapley/figure/Shapley_1.png}
\newcommand{\learninggoals}{%
\item Learn what game theory is
\item Understand the concept behind cooperative games
\item Understand the Shapley value in game theory
}

\lecturechapter{Shapley Values}
\lecture{Interpretable Machine Learning}

% License of titlefigure: free pixabay license
% https://pixabay.com/de/vectors/ergebnis-geld-gesch%C3%A4ft-gehalt-5567652/


% \begin{frame}{Game Theory}
% \begin{itemize}
% \itemsep1em
%   \item Game theory is the study of strategic games between players
%   \item Term \enquote{game} not restricted to actual games (e.g., chess or poker) but to any series of interactions between actors or agents with gains and losses of quantifiable utility value
%   \item Often used in social context where players correspond to people or organizations, e.g., warfare, provision of public goods, auctions and bargaining, formation of cartels, interrogation practices.
%   \item Example of prisoner's dilemma: Two prisoners A and B are interrogated in two separate rooms with no means of communication between each other.
%     \begin{itemize}
%         \item If they betray each other, each person serves 2 years in prison.
%         \item If one person remains silent and one betrays their partner, person staying silent receives 3 years in prison while betrayer is set free.
%         \item If both remain silent, both serve 1 year in prison.
%     \end{itemize}
%     Conditional on strategy of remaining partner, best personal outcome is achieved by betrayal.
%     \\
%     $\Rightarrow$ Best aggregate outcome not achieved (both staying silent). 
% \end{itemize}
% \end{frame}


\begin{frame}{Cooperative Games in Game Theory \citebutton{Shapley (1951)}{https://www.rand.org/pubs/research_memoranda/RM0670.html}}
\begin{itemize}[<+->]
%\itemsep1em
  \item Game theory is the study of strategic games between players, \enquote{game} refers to any series of interactions between actors / agents with gains and losses of quantifiable utility value
  \item Cooperative games: For all possible players $P = \{1, \hdots, p\}$, each subset of players $\SsubP$ forms a coalition -- each coalition $S$ achieves a certain payout
  %\item A value function $v(S): 2^{|P|}\mapsto \R$ describes the payout (or gain) achieved by any coalition $S \subseteq P$
  \item A value function $v: 2^{|P|}\mapsto \R$ maps all $2^{|P|}$ possible coalitions to their payout (or gain)
  \item $v(S)$ is the payout of coalition $S \subseteq P$ (payout of empty coalition must be zero: $v(\emptyset) = 0$)
  %\\
  %\textit{Note:} The payout or gain not necessarily has a positive meaning. In the example on the prisoner's dilemma, the payout is the total number of years spend in prison.
  \item As some players contribute more than others, we want to fairly divide the total achievable payout $v(P)$ among the players according to a player's individual contribution
  \item We call the individual payout per player $\phi_j$, $j \in P$ (later: Shapley value)
  %\item What would be properties of a fair distribution of the payout?
\end{itemize}
\end{frame}

\begin{frame}{Cooperative Games without Interactions}
% Figure Source: https://docs.google.com/presentation/d/1-bK90Gv1vIDr61s1PfgC51Kvic2Avnb7v8PgoO9iN0I/edit?usp=sharing
\begin{center}
\includegraphics<1>[page=1, width = 0.8\textwidth]{../../slides/04_shapley/figure/Shapley.pdf}%
\includegraphics<2>[page=2, width = 0.8\textwidth]{../../slides/04_shapley/figure/Shapley.pdf}%

\visible<2>{$\Rightarrow$ Fair Payouts are Trivial Without Interactions}
\end{center}
\end{frame}

% \begin{frame}{Cooperative Games with Interactions}
% \begin{center}
% \includegraphics<1>[page=3, width = 0.8\textwidth]{figure/Shapley.pdf}

% \only<1>{$\Rightarrow$ Unclear how to fairly distribute payouts when players interact}

% \includegraphics<2>[page=4]{figure/Shapley.pdf}
% \end{center}
% \end{frame}

\begin{frame}{Cooperative Games with Interactions}
\begin{center}
\includegraphics[page=3, width = 0.8\textwidth]{../../slides/04_shapley/figure/Shapley.pdf}

$\Rightarrow$ Unclear how to fairly distribute payouts when players interact
\end{center}
\end{frame}

\begin{frame}{Cooperative Games with Interactions}
\textbf{Question:} What is a fair payout for player \enquote{yellow}?\\
\textbf{Idea:} Compute marginal contribution of the player of interest across different coalitions

\begin{columns}[T, totalwidth=\linewidth]
\begin{column}{0.53\textwidth}
\includegraphics[page=4, trim=150px 120px 160px 10px, clip]{../../slides/04_shapley/figure/Shapley.pdf}

%\includegraphics[page=4, trim=170px 15px 200px 305px, clip, width = 0.8\textwidth]{figure/Shapley.pdf}
\scriptsize
%\hspace{-100px}
\begin{itemize}
\itemsep0em
%\addtolength{\itemindent}{-1cm}
    \item Compute the total payout of each coalition
    \item Compute difference in payouts for each coalition with and without player \enquote{yellow} (= marginal contribution)
    \item Average marginal contributions using appropriate weights
\end{itemize}
\end{column}
\begin{column}{0.47\textwidth}
\pause
\textbf{Note:} Each marginal contribution is weighted w.r.t. number of possible orders of its coalition\\
$\leadsto$ More players in $S$ $\Rightarrow$ more orderings of $S$
%The more players in a coalition, the more possible orderings inside the coalition %more possibilities of ordering the players inside the coalition
%\includegraphics[page=15, width = \textwidth, trim=122px 10px 82px 10px, clip]{figure/Shapley.pdf}
\includegraphics[page=7, width = \textwidth, trim=182px 10px 0px 10px, clip]{../../slides/04_shapley/figure/Shapley.pdf}
\end{column}
\end{columns}
\end{frame}

% \begin{frame}{Cooperative Games without Interactions}
% \begin{figure}
%     \centering
%     \includegraphics{figure/Shapley_1.png}
% \end{figure}
% \end{frame}

% \begin{frame}{Fair Payouts are Trivial Without Interactions}
% \begin{figure}
%     \centering
%     \includegraphics{figure/Shapley_2.png}
% \end{figure}

% \end{frame}
% \begin{frame}{Cooperative Games with Interactions}
% \begin{figure}
%     \centering
%     \includegraphics{figure/Shapley_3.png}
% \end{figure}
%\end{frame}

% \begin{frame}{What is a fair payout for player \enquote{yellow}?}

% \begin{figure}
%     \centering
%     \includegraphics{figure/Shapley_4.png}
% \end{figure}

% \end{frame}



\begin{frame}{Shapley Value - Set definition}

This idea refers to the \textbf{Shapley value} which assigns a payout value to each player according to its marginal contribution in all possible coalitions.
% which  provide a unique solution for the payout attribution problem.
  
\begin{itemize}[<+->]
  %\item Shapley values were proposed by Lloyd Shapley in 1951 for cooperative games (game theory).
  % given axioms of efficiency, symmetry, dummy and additivity.
  %\item The \textbf{Shapley value} of player $j$ assigns a payout value according to the marginal contribution of each player in all possible coalitions\\
  %$\leadsto$ %To compute the Shapley payout for a player $j$, we 
  \item Let $v(\Scupj) - v(S)$ be the marginal contribution of player $j$ to coalition $S$\\
  $\leadsto$ measures how much a player $j$ increases the value of a coalition $S$
  \item Average marginal contributions for all possible coalitions $\SsubPnoj$ \\
  $\leadsto$ order of how players join the coalition matters $\Rightarrow$ different weights depending on size of $S$\\
  %$\leadsto$ marginal contributions are weighted differently
  \item Shapley value via \textbf{set definition} (weighting via multinomial coefficient): 
  $$\phi_j = \sum_{\SsubPnoj} \frac{|S|!(|P| - |S| - 1)!}{|P|!}(v(\Scupj) - v(S))$$
  %\item Shapley values are the \textit{only} solution for the attribution with the specified axioms.
%   \item Equivalent Shapley value definition via \textbf{orders}: $$\phi_j = \frac{1}{|P|!} \sum_{\tau \in \Pi} (v(\Stau \cup \{j\}) - v(\Stau))$$
%     $\leadsto$ $\Pi$: All possible $|P|!$ orders/permutations of players\\
%     $\leadsto$ $\Stau$: The set of players before player $j$ in order $\tau = (\tau^{(1)}, \dots, \tau^{(p)})$ % j, \dots, 
\end{itemize}

\end{frame}

% \begin{frame}{How to Weight Differences in Payout?}

% \begin{itemize}
%     %\itemsep2em
%     %\item Form a coalition one player at a time -- each player receives its contribution to the total payout, i.e., the increase in total payout when a player joins a coalition
%     % \item The Shapley value is the players' average contribution over all possible formations of coalitions\\
%     % $\leadsto$ order of how players join the coalition matters\\
%     % $\leadsto$ marginal contributions are weighted
%     %\item Each contribution is weighted proportionally to the number of possible orders of its coalition -- the more players in a coalition, the more possibilities of ordering the players inside the coalition
%     \item Shapley value via \textbf{set definition} (weighting via multinomial coefficient): 
%     $$\phi_j = \sum_{\SsubPnoj} \frac{|S|!(|P| - |S| - 1)!}{|P|!}(v(\Scupj) - v(S))$$
%     \item Shapley value via \textbf{order definition}: $$\phi_j = \frac{1}{|P|!} \sum_{\tau \in \Pi} (v(\Stau \cup \{j\}) - v(\Stau))$$
    
%     with $\Pi$ being all possible $|P|!$ orders/permutations of players and $\Stau$ being the set of players before player $j$ in order $\tau$ 
    
%     %$\phi_j = \frac{1}{|P|!} \sum_{\tau \in \Pi} (v(S_j(\tau) \cup \{j\}) - v(S_j(\tau)))$
% \end{itemize}

% \end{frame}

% \begin{frame}{How to Weight Differences in Payout?}

% \begin{figure}
%     \centering
%     \includegraphics{figure/Shapley_5.png}
% \end{figure}

% \end{frame}



% \begin{frame}{From Game Theory To Machine Learning}

% \begin{figure}
%     \centering
%     \includegraphics{figure/Shapley_5.png}
% \end{figure}

% \end{frame}

% \begin{frame}{Shapley Values}

%   Shapley values provide a unique solution to the attribution problem while satisfying all axioms:
%     \vspace{0.25cm}
% \begin{itemize}
%   \itemsep1em
%   \item Shapley values were proposed by Lloyd Shapley in 1951.
%   \item The Shapley value assigns a value to each player according to the marginal contribution of each player in all possible coalitions.
%   \item $\phi_j = \sum_{\SsubPnoj} \frac{|S|!(|P| - |S| - 1)!}{|P|!}(v(\Scupj) - v(S))$
%   \item $v(\Scupj) - v(S)$ is the marginal contribution of player $j$ to coalition $S$.
%   \item To compute the Shapley payout for a player, we average, for all possible coalitions, how much the player would increase the value of the coalition (=marginal contribution).
%   \item Shapley values are the \textit{only} solution for the attribution with the specified axioms.
% \end{itemize}

% \footnote{Shapley, Lloyd S. (August 21, 1951). "Notes on the n-Person Game -- II: The Value of an n-Person Game" (PDF). Santa Monica, Calif.: RAND Corporation.}

% \end{frame}



\begin{frame}{Shapley Value - Order Definition}
The Shapley value was introduced as summation over sets $\SsubPnoj$, but it can be equivalently defined as a summation of all orders of players: 
%(which explains where the factor $\frac{|S|!(|P| - |S| - 1)!}{|P|!}$ comes from):

$$\phi_j = \frac{1}{|P|!} \sum_{\tau \in \Pi} (v(\Stau \cup \{j\}) - v(\Stau))$$
  
\begin{itemize}[<+->]
  \item $\Pi$: All possible orders of players (we have $|P|!$ in total)
  %\item Recall the order definition: 
  \item $\Stau$: Set of players before player $j$ in order $\tau = (\tau^{(1)}, \dots, \tau^{(p)})$  where $\tau^{(i)}$ is $i$-th element \\ % j, \dots, 
  $\Rightarrow$ Example: Players $1,2,3$ $\Rightarrow$ 
  \only<3>{$\Pi = \{({\color{red} 1,2},3), (1,3,2), ({\color{red} 2,1},3), (2,3,1), (3,1,2), (3,2,1)\}$}%
  \only<1-2>{$\Pi = \{(1,2,3), (1,3,2), (2,1,3), (2,3,1), (3,1,2), (3,2,1)\}$}%
  \\
  \phantom{$\Rightarrow$} $\leadsto$ For order $\tau = (2,1,3)$ and player of interest $j=3$ $\Rightarrow$ $\Stau = \{2, 1\}$\\
  \phantom{$\Rightarrow$} $\leadsto$ For order $\tau = (3,1,2)$ and player of interest $j=1$ $\Rightarrow$ $\Stau = \{3\}$\\
  \phantom{$\Rightarrow$} $\leadsto$ For order $\tau = (3,1,2)$ and player of interest $j=3$ $\Rightarrow$ $\Stau = \emptyset$
  %\item In order definition, we sum the marginal contribution twice for orders that yield set $S = \{1,2\}$\\
  \item Order definition: Marginal contribution of orders that yield set {\color{red} $S = \{1,2\}$} is summed twice\\
  $\leadsto$ In set definition, it has the weight $\tfrac{2! (3 - 2 - 1)!}{3!} = \tfrac{2 \cdot 0!}{6} = \tfrac{2}{6}$
\end{itemize}

\end{frame}


% \begin{frame}{Shapley Value - Comments on Order Definition}

% \begin{itemize}
%   \item<1-> Order and set definition are equivalent
%   \item<1-> Reason: The number of orders which yield the same coalition $S$ is $|S|!(|P| - |S| - 1)!$\\
%   $\Rightarrow$ There are $|S|!$ possible orders of players within coalition $S$\\
%   $\Rightarrow$ There are $(|P| - |S| - 1)!$ possible orders of players without $S$ and $j$
%   \centerline{
%   \begin{tabular}{|c|c|c|c|c|c|c|}
%     \multicolumn{3}{c}{\enspace\raisebox{-3.3ex}[0pt][2.6ex]{$ \overbrace{\vphantom{-}\hspace{9em}}^{|S|! \text{ permutations}}$}} &
%     \multicolumn{1}{c}{} &
%     \multicolumn{3}{c}{\enspace\raisebox{-3.3ex}[0pt][2.6ex]{$ \overbrace{\vphantom{-}\hspace{9em}}^{(|P| - |S| - 1)! \text{ permutations}}$}}\\
%     \hline
%     $\tau^{(1)}$ & \ldots & $\tau^{(|S|)}$ & $\tau^{(|S| + 1)}$ & $\tau^{(|S| + 2)}$ & \ldots & $\tau^{(p)}$ \\
%     \hline
%     \multicolumn{3}{c}{\enspace\raisebox{1.3ex}[0pt][2.6ex]{$ \underbrace{\vphantom{-}\hspace{9em}}^{}$}} &
%     \multicolumn{1}{c}{\enspace\raisebox{1.3ex}[0pt][2.6ex]{$ \underbrace{\vphantom{-}\hspace{4em}}^{}$}} &
%     \multicolumn{3}{c}{\enspace\raisebox{1.3ex}[0pt][2.6ex]{$ \underbrace{\vphantom{-}\hspace{9em}}^{}$}}\\
%     \multicolumn{3}{c}{Players before player $j$} & \multicolumn{1}{c}{player $j$} & \multicolumn{3}{c}{Players after player $j$} \\
%   \end{tabular}}
%   \item<2-> Relevance of the order definition: Approximate Shapley values by sampling permutations \\
%   $\leadsto$ randomly sample a fixed number of $M$ permutations and average them:
%   %to approximate the Shapley values
%   %instead of producing all $|P|!$ permutations, a fixed number of $M< |P|!$ permutations can be sampled and averaged to approximate the Shapley values
%   $$\phi_j = \frac{1}{M} \sum_{\tau \in \Pi_M} (v(\Stau \cup \{j\}) - v(\Stau))$$
%     where $\Pi_M \subset \Pi$ is a random subset of $\Pi$ containing only $M$ orders of players\\
%     %$\leadsto$ $\Stau$: For permutation $\tau$, the set of players before player $j$ in order $\tau$ 
% \end{itemize}

% \end{frame}


% How to weight differences in payout
\begin{frame}{Weights for Marginal Contribution - Illustration}
%   \begin{center}
%   \only<1>{\includegraphics{figure/Shapley_7.png}}%
%   \only<2>{ \includegraphics{figure/Shapley_8.png}}%
%   \only<3>{ \includegraphics{figure/Shapley_9.png}}%
%   \end{center}
  
\begin{center}
\includegraphics<1>[page=5, width = 0.8\textwidth]{../../slides/04_shapley/figure/Shapley.pdf}%
\includegraphics<2>[page=6, width = 0.8\textwidth]{../../slides/04_shapley/figure/Shapley.pdf}%
\includegraphics<3>[page=7, width = 0.8\textwidth]{../../slides/04_shapley/figure/Shapley.pdf}%
\end{center}
\end{frame}


\begin{frame}{Shapley Values - Illustration}
\begin{itemize}
    \item Shapley value of player $j$ is the marginal contribution to the value when it enters any coalition
    \item Produce all possible joining orders of player coalitions
%\only<1>{\item Here, player $2$ enters the coalition after player $1$, resulting in a value change of $v(\{1,2\}) - v(\{1\}) = 24-12 = 12$ with a overall coalition value of $v(\{1,2,3\}) = 36$}
\only<1>{\item[] \phantom{Measure and average the difference in payout after player $1$ enters the coalition}}%
\only<2>{\item Measure and average the difference in payout after player $1$ enters the coalition}%
\only<3>{\item Measure and average the difference in payout after player $2$ enters the coalition}%
\only<4>{\item Measure and average the difference in payout after player $3$ enters the coalition}%
\end{itemize}

\begin{center}
\includegraphics<1>[page=9, width = 0.7\textwidth]{../../slides/04_shapley/figure/Shapley.pdf}%
\includegraphics<2>[page=10, width = 0.7\textwidth]{../../slides/04_shapley/figure/Shapley.pdf}%
\includegraphics<3>[page=11, width = 0.7\textwidth]{../../slides/04_shapley/figure/Shapley.pdf}%
\includegraphics<4>[page=12, width = 0.7\textwidth]{../../slides/04_shapley/figure/Shapley.pdf}%
\includegraphics<5>[page=13, width = 0.7\textwidth]{../../slides/04_shapley/figure/Shapley.pdf}%
\end{center}

% \begin{center}
%   \only<1>{
%     \includegraphics[page=1, width=0.6\textwidth]{figure_man/shapley_feature_effect}
%   }
%   \only<2>{
%     \includegraphics[page=2, width=0.6\textwidth]{figure_man/shapley_feature_effect}
%   }
% \end{center}
\end{frame}


\begin{frame}{Axioms of Fair Payouts}
 Why is this a fair payout solution?
 \\
 One possibility to define fair payouts are the following axioms for a given value function $v$:
  \vspace{0.25cm}
  \begin{itemize}[<+->]
  \itemsep1em
    \item \textbf{Efficiency}: Player contributions add up to the total payout of the game:
      $\sum\nolimits_{j=1}^p\phi_j = v(P)$
    \item \textbf{Symmetry}: Players $j,k \in P$ who contribute the same to any coalition get the same payout: \\
      If $v(\Scupj) = v(\Scupk)$ for all $\SsubP \setminus\{j,k\}$, then $\phi_j=\phi_k$
    \item \textbf{Dummy/Null Player}: Payout is 0 for players who don't contribute to the value of any coalition: \\
      If $v(\Scupj)=v(S)\quad  \forall \quad \SsubP \setminus \{j\}$, then $\phi_j=0$
    \item \textbf{Additivity}: For a game $v$ with combined payouts $v(S) = v_1(S) + v_2(S)$, the payout is the sum of payouts: $\phi_{j,v} = \phi_{j,v_1} + \phi_{j, v_2}$
  \end{itemize}
  \vspace{0.5cm}
 \end{frame}

%\endlecture

%%%%%%%%%%%%%%%%%%%%%%%%%%%%%%%%%%%%%%%%%%%%%%%%%%%%%%%%%%
\renewcommand{\titlefigure}{../../slides/04_shapley/figure_man/bike-sharing03.png}
\renewcommand{\learninggoals}{
\item See model predictions as a cooperative game
\item Transfer the Shapley value concept from game theory to machine learning
}

\lecturechapter{Shapley Values for Local Explanations}
%\lecture{Interpretable Machine Learning}
\mysectionslide

% \begin{frame}{Shapley Values for Local Explanations}
%     \begin{columns}
%     \begin{column}{0.5\textwidth}
%     \vspace{1.5cm}
%     \begin{center}
%         \includegraphics[]{../../slides/04_shapley/figure_man/bike-sharing03.png}
%     \end{center}
%     \end{column}
%     \begin{column}{0.5\textwidth}
%      \vspace*{2cm}
     
%     \textbf{Learning goals}
%     \begin{itemize}
%         \item See model predictions as a cooperative game
%         \item Transfer the Shapley value concept from game theory to machine learning
%     \end{itemize}
%     \end{column}
%     \end{columns}
% \end{frame}

\begin{frame}{From Game Theory To Machine Learning}

\begin{figure}
    \centering
    \includegraphics{../../slides/04_shapley/figure/Shapley_6.png}
\end{figure}

\end{frame}

\begin{frame}{From Game Theory to Machine Learning}
\begin{itemize}[<+->]
    %\itemsep1em
    \item Game: Make prediction $\fh(x_1, x_2, \ldots, x_p)$ for a single observation $\xv$
    \item Players: Features $x_j, j \in \pset$ which cooperate to produce a prediction\\
    $\leadsto$ How can we make a prediction with a subset of features without changing the model?
    \\ 
    $\leadsto$ PD function: $\fh_{S}(\xv_S) := \int_{X_{-S}} \fh(\xv_S, X_{-S})d \P_{X_{-S}}$ (``removing'' by marginalizing over $-S$)
    \item  Value function / payout of coalition $S \subseteq P$ for observation $\xv$:
    $$v(S) =  \fh_{S} (\xv_S) - \E_{\xv}(\fh(\xv)), \text{ where } %\xv_S = \{x_j\}_{j \in S} \text{ and } 
    \fh_S: \Xspace_S \mapsto \Yspace$$
    $\leadsto$ subtraction of $\E_{\xv}(\fh(\xv))$ ensures that $v$ is a value function with $v(\emptyset) = 0$
    \centerline{\includegraphics[width=0.5\textwidth, trim=20 0 0 100, clip]{../../slides/04_shapley/figure_man/shapley_valuefct}}
    \item Marginal contribution: $v(\Scupj) - v(S) =  \fh_{\Scupj} (\xv_{\Scupj}) - \fh_{S} (\xv_{S})$\\
    $\leadsto$ $\E_{\xv}(\fh(\xv))$ cancels out due to the subtraction of value functions
\end{itemize}
\end{frame}

% \begin{frame}{Shapley Values}
%   We can use Shapley values to explain individual predictions $\fh(\xv)$ of a machine learning model $\fh$:
% \begin{itemize}
%   \item Players $\hat{=}$ feature values of $i$-th observation $x_j, j \in \pset$.
%   \item Features cooperate to produce a prediction $\fh(x_1, x_2, \ldots, x_p)$.
%   \item The value function / payout of coalition $S$ for observation $\xv$ is
%     $$v(S) =  \fh_{S} (\xv_S) - \E_{\xv}(\fh(\xv)),$$ 
%     where $\xv_S = \{x_j\}_{j \in S}$ and $\fh_S: \Xspace_S \mapsto \Yspace$.
% \item The marginal prediction $\fh_S$ is defined as $\fh_{S}(\xv_S) := \int_{X_{-S}} \fh(\xv_S, X_{-S})d \P_{X_{-S}}$
% \item We have already seen the marginal prediction in action in the PDP.
% \item The subtraction of $\E_{\xv}(\fh(\xv))$ is necessary so that $v$ is a value function with $v(\emptyset) = 0$.
% \item By using the marginal prediction, we have defined what it means for features to be \enquote{missing} for the prediction: We remove it by integrating over its distribution.
% \end{itemize}
% \begin{center}
% \vspace{-0.3cm}
% \includegraphics[width=0.6\textwidth]{figure_man/shapley_valuefct}
% \end{center}

% \begin{itemize}
%  \item Shapley values tell us what the payout of each feature is, i.e., how each feature contributes to the overall prediction of a specific observation.
%     \item The Shapley value is the average marginal contribution of a feature towards the prediction \textbf{across all possible feature coalitions}.
%     \item The sum of Shapley values over all features yields the difference between the average prediction of all data points (baseline) and the selected individual prediction.
%   \end{itemize}
% \end{frame}

\begin{frame}{Shapley Value - Definition \citebutton{Shapley (1953)}{https://doi.org/10.7249/P0295} \citebutton{Strumbelj et al. (2014)}{https://doi.org/10.1007/s10115-013-0679-x}}
  %Using the order definition, the Shapley value for feature $j$ and a given observation $\xv$ is computed by:
  Shapley value $\phi_j$ of feature $j$ for observation $\xv$ via \textbf{order definition}:

     $$ \phi_j(\xv)  = \frac{1}{|P|!} \sum_{\tau \in \Pi} \underbrace{\fh_{\Stauj}(\xv_{\Stauj}) - \fh_{\Stau}(\xv_{\Stau})}_{\text{marginal contribution of feature $j$}} $$ %S \subseteq P\setminus \{j\}
     
     
\begin{itemize}
    
  \item Interpretation: Feature $x_j$ contributed $\phi_j$ to difference between $\fh(\xv)$ and average prediction\\
  $\leadsto$ Note: Marginal contributions and Shapley values can be negative
  %$x_j$ contributed $\phi_j$ to prediction $\fh(\xv)$ compared to average prediction
  % of Shapley value $\phi_j$ for feature $j$ and observation $\xv$:\\
  %Feature value $x_j$ contributed amount $\phi_j$ to prediction $\fh(\xv)$ compared to the average prediction
   \item For exact computation of $\phi_j(\xv) $, the PD function 
   %$\fh_{S}(\xv_S) =  \frac{1}{n} \sum_{i=1}^n \fh_S^{(i)}(\xv_S) = \frac{1}{n} \sum_{i=1}^n \fh(\xv_S, \xv_{-S}^{(i)})$ 
   $\fh_{S}(\xv_S) = \frac{1}{n} \sum_{i=1}^n \fh(\xv_S, \xv_{-S}^{(i)})$ 
   for any set of features $S$ can be used which yields
   %$$ \phi_j(\xv) = \frac{1}{|P|! \cdot n} \sum_{\tau \in \Pi} \sum_{i = 1}^n \fh_{\Smj}^{(i)}(\xv_{\Smj}) -  \fh_{\Sm }^{(i)}(\xv_{\Sm}) $$
    $$ \phi_j(\xv) = \frac{1}{|P|! \cdot n} \sum_{\tau \in \Pi} \sum_{i = 1}^n
   \fh(\xv_{\Stauj}, \xv_{\minusStauj}^{(i)}) - \fh(\xv_{\Stau}, \xv_{- \Stau}^{(i)})
   $$
   $\leadsto$ Note: $\fh_S$ marginalizes over all other features $-S$ using all observations $i = 1, \ldots, n$
   
\end{itemize}
\lz
%\tiny
%Shapley, Lloyd S. 1953. $"$A Value for N-Person Games.$"$\\
%\vspace{0.2cm}
%Strumbelj, Erik, Igor Kononenko, Erik Strumbelj, and Igor Kononenko. 2014. $"$Explaining prediction models and individual predictions with feature contributions.$"$

\end{frame}


\begin{frame}{Estimation: A practical problem}
  \begin{itemize}[<+->]
  %\itemsep1em
      \item Exact Shapley value computation is problematic for high-dimensional feature spaces\\
      $\leadsto$ For 10 features, there are already $|P|! = 10! \approx 3.6$ million possible orders of features
      \item Additional problem due to estimation of the marginal prediction $\fh_{\Stau}$: Averaging over the entire data set for each coalition $\Stau$ introduced by $\tau$ can be very expensive for large data sets
      \item Solution to both problems is sampling:
      Instead of averaging over $|P|! \cdot n$ terms, we approximate it using a limited amount of $M$ random samples of $\tau$ to build coalitions $\Stau$
      %We calculate the Shapley value over $M$ samples -- for each sample, we sample one order of features and one data point to replace missing features
      \item $M$ is a tradeoff between accuracy of the Shapley value and computational costs\\
      $\leadsto$ The higher $M$, the closer to the exact Shapley values, but the more costly the computation
      %- the higher $M$, the closer we get to the true Shapley values, but the more costly the computation becomes
  \end{itemize}
\end{frame}

\newcommand{\xk}{\mathbf{x}^{(k)}}

% \begin{frame}{Approximation Algorithm \citebutton{Strumbelj et al. (2014)}{https://doi.org/10.1007/s10115-013-0679-x}}
% Estimation of $\phi_j$ for observation $\xv$ of model $\fh$ fitted on data $\D$ using sample size $M$:
% %\vspace{0.25cm}
%   \begin{enumerate}[<+->]
%       \item For $m = 1, \ldots, M$ \textbf{do}:
%       \begin{enumerate}
%         \item Select random order / permutation of feature indices $\tau = (\tau^{(1)}, \ldots, \tau^{(p)}) \in \Pi$
%         %\item Select random data point $\xv^{(k)} \in X$.
%         %\item Let $\Sm := \Sm$ be the set of features before $j$ of order $\tau$ forming a coalition
%         \item Determine coalition $\Sm := \Stau$, i.e., the set of features before feature $j$ in order $\tau$
%         \item Select random data point $\mathbf{z}^{(m)} \in \D$%\\
%         %$\leadsto$ used to marginalize over
%         %\item Order feature values in $\xv$ according to $\tau$: $\xv_{\tau} = (x_{\tau^{(1)}}, \ldots, x_{\tau^{(j)}}, \ldots, x_{\tau^{(p)}})$
%         %\item Order feature values in $\mathbf{z}^{(m)}$ according to $\tau$: $\mathbf{z}_{\tau}^{(m)} = (z_{\tau^{(1)}}^{(m)}, \ldots, z_{\tau^{(j)}}^{(m)}, \ldots, z_{\tau^{(p)}}^{(m)})$
%         \item Construct two artificial observations by replacing feature values from $\xv$ with $\mathbf{z}^{(m)}$:
%           \begin{itemize}
%           \setlength\itemsep{.5em}
%             \item % 
%             $ \xjp  = (\underbrace{x_{\tau^{(1)}}, \ldots, x_{\tau^{(|\Sm|-1)}}, x_{j}}_{\xv_{\Smj}}, \underbrace{z_{\tau^{(|\Sm|+1)}}^{(m)} , \ldots, z_{\tau^{(p)}}^{(m)}}_{\mathbf{z}^{(m)}_{\minusSmj}} )$
%             takes features $\Smj$ from $\xv$
%             %(with value of feature $j$ from $\xv$)
%             \item % \xjm =
%             $ \xjm = (\underbrace{x_{\tau^{(1)}}, \ldots, x_{\tau^{(|\Sm|-1)}}}_{\xv_{\Sm}}, \underbrace{z_{j}^{(m)} , z_{\tau^{(|\Sm|+1)}}^{(m)} , \ldots, z_{\tau^{(p)}}^{(m)}}_{\mathbf{z}^{(m)}_{-\Sm}} )$
%             takes features $\Sm$ from $\xv$
%             %(without value of feature $j$ from $\xv$)
%           \end{itemize}
%         \item Compute difference $\phi_j^m = \fh(\xjp) - \fh(\xjm)$ \\
%         $\leadsto$ $\fh_{\Sm}(\xv_{\Sm})$ is approximated by $\fh(\xv_{-j}^{(m)})$ and  $\fh_{\Smj}(\xv_{\Smj})$ by  $\fh(\xv_{+j}^{(m)})$ over $M$ iters
%         %$\hat =$ A random sample from the marginal contribution of feature $j$ to coalition $\Sm$
%         \end{enumerate}
        
%     \item Compute Shapley value $\phi_j = \frac{1}{M}\sum_{m=1}^M \phi_j^m$ %= \frac{1}{M} \sum_{m = 1}^M   \fh(\xjp ) -  \fh(\xjm ) $
%   \end{enumerate}
% %\pause
% %Each $\phi_j^m$ is a random sample from the marginal contribution of feature $j$ for a sampled coalition introduced by $\tau$.
   
%   %$$ \phi_j(\xv) = \frac{1}{|P|! \cdot n} \sum_{\tau \in \Pi} \sum_{i = 1}^n  \fh(\xv_{\Smj}, \xv_{\minusSmj}^{(i)}) - \fh(\xv_{\Sm}, \xv_{- \Sm}^{(i)})  $$
   
%   % \vspace{0.25cm}
%     %\tiny{Strumbelj, Erik, Igor Kononenko, Erik Strumbelj, and Igor Kononenko. 2014. $"$Explaining prediction models and individual predictions with feature contributions.$"$}

% \end{frame}

\begin{frame}{Shapley value approximation - Illustration}

\begin{exampleblock}{Definition}
% \[
% \tikzmark{phi}\phi_{\tikzmark{j}j}(%\tikzmark{v}\fh,
% \tikzmark{x}\xv)=\frac{1}{M}\tikzmark{M}\sum_{m=1}^M \left[
% \fh_{\Sm \tikzmark{SJ}\cup j}
% \left(\xv_{\Sm \tikzmark{SJ}\cup j}\right)-
% \fh_{\tikzmark{S}\Sm}\left(\xv_{\tikzmark{S}\Sm}\right)\right]
% \]
$$
\tikzmark{phi}\phi_{\tikzmark{j}j}(%\tikzmark{v}\fh,
\tikzmark{x}\xv)=\frac{1}{M}\tikzmark{M}\sum_{m=1}^M \left[
\fh(\xv_{+j}\tikzmark{SJ}^{(m)}) -
\fh(\xv_{-j}\tikzmark{S}^{(m)})\right]
$$
\begin{tikzpicture}[
  remember picture,
  overlay,
  expl/.style={draw=blue,fill=white,rounded corners,text width=3.7cm},
  arrow/.style={blue,ultra thick,->,>=latex}
]
%%%% Explain Formula I
% \draw<2-3> [draw=white, fill=white, opacity=0]
%       (5.8,0) -- (11,0) -- (11,2) -- (5.8,2) --cycle;
% \node<1>[expl] 
%   (phiex) 
%   at (6,2cm)
%   {Shapley value of feat. $j$};
% \node<2>[expl] 
%   (jex) 
%   at (2,0cm)
%   {for feature $j$};
% \node<3>[expl] 
%   (vex) 
%   at (5,-0.7cm)
%   {$\fh$: pred. function};
\node<1>[expl] 
  (xex) 
  at (6,2cm)
  {$\xv$: obs. of interest};
% \draw<1>[arrow]
%   (phiex.south) to[out=270,in=90] ([xshift= 1ex, yshift=1.5ex]{pic cs:phi});  
% \draw<2>[arrow]
%   (jex.north) to[out=90,in=270] ([yshift=-0.5ex]{pic cs:j});  
% \draw<3>[arrow]
%   (vex.east) to[out=0,in=270] ([xshift= 1ex, yshift=-0.5ex]{pic cs:v});  
\draw<1>[arrow]
  (xex.west) to[out=180,in=135] ([xshift= 1ex, yshift=1ex]{pic cs:x}); 
  
 %%%% Explain Formula II
% \draw<3-5> [draw=white, fill=white, opacity=0.6]
%       (4,0) -- (6.8,0) -- (6.8,2) -- (4,2) --cycle;
\node<1>[expl] 
  (Sex) 
  at (12,2cm)
  {$\xv$ with feature values in $\Sm$ (other are replaced)};
\node<2>[expl] 
  (Sex) 
  at (13,1cm)
  {Contribution of feature $j$ to coalition $\Sm$};
 \node<1>[expl] 
  (SJex) 
  at (10,-0.8cm)
  {$\xv$ with feature values in $\Smj$};
  \draw<1>[arrow]
  (Sex.south) to[out=270,in=90] ([xshift= 1ex, yshift=1.5ex]{pic cs:S});
 \draw<2>[arrow]
  (Sex.north) to[out=90,in=90] ([xshift= 1ex, yshift=1.5ex]{pic cs:S});
  \draw<1>[arrow]
  (SJex.north) to[out=90,in=270] ([yshift=-0.5ex]{pic cs:SJ}); 
  \draw<2> [decorate,decoration={brace, amplitude=5pt,mirror,raise=4ex}]  (7,1.1) --  (10.4,1.1)
  node[midway,yshift=-3.4em]{\scriptsize $:= \Delta(j, \Sm)$};
  
  \draw<2> [draw=white, fill=white, opacity=0.6]
       (4,0) -- (7,0) -- (7, 2) -- (4,2) --cycle;
  %%%% Explain Formula III
  \draw<3> [draw=white, fill=white, opacity=0.6]
       (4,0) -- (5.7,0) -- (5.7, 2) -- (4,2) --cycle;
  \draw<3> [draw=white, fill=white, opacity=0.6]
        (7,0) -- (12,0) -- (12,2) -- (7,2) --cycle;
  \node<3>[expl] 
  (Mex) 
  at (12,2cm)
  {average the contributions of feature $j$};
  \draw<3>[arrow]
  (Mex.west) to[out=180,in=90] ([xshift= 0.5ex, yshift=4ex]{pic cs:M}); 
\end{tikzpicture}
\end{exampleblock}
%\vspace{0.5cm}
\begin{itemize}
    \begin{onlyenv}<1>
    \vspace{50pt}
    % \item The Shapley value assigns a value to each feature $j$ according to the marginal contribution of each player in all possible coalitions
    \end{onlyenv}
    %\invisible<2>{\item The Shapley value assigns a value to each feature $j$ according to the marginal contribution of each player in all possible coalitions}
    %\begin{onlyenv}<3>
    %\item A value function $v(S): 2^{|P|}\mapsto \R$ describes the payout (or gain) achieved by any coalition $\forall S \subseteq P$. The value of the empty coalition must be zero: $v(\emptyset) = 0$.
    %\item The payout of coalition $S$ for observation $\xv$ is 
    %$$v(\xv_S) =  \fh_{S} (\xv_S) - \E (\fh(\xv))$$ 
    %i.e., the difference of the marginal prediction of $\xv_S$ and the average prediction.
    %\end{onlyenv}
    %\begin{onlyenv}<4>
    %\item Example observation from bike sharing data
    %\end{onlyenv}
    % Part II
    %\begin{onlyenv}<5>
    %\item $\Sm$ is a randomly selected set of features: $\Sm \subseteq P \setminus j$ where $P$ denotes the quantity of all features in $X$
    %\item Let $x_{\Sm}$ be a random chosen subset of features in $x_S$ that is hold fix
    %\end{onlyenv}
     %   \begin{onlyenv}<6>
    %\item Let $S_{\lnot m}$ denote the subset of $P$ that is not in $\Sm$.
    %\item Draw the values of all other features $x_{S_{\lnot m}}$ randomly from the set of available values of the regarding feature to calculate $v(x_{\Sm})$
    %\end{onlyenv}
    \begin{onlyenv}<2>
     \item $\Delta(j, \Sm) = \fh(\xjp) -
\fh(\xjm)$ is the marginal contribution of feature $j$ to coalition $\Sm$
    \item Here: Feature \textit{year} contributes +700 bike rentals if it joins coalition $\Sm = \{temp, hum\}$ %compared to the expected prediction conditioned on features in coalition $\Sm$ (while other features were taken from a random obs.)
    \end{onlyenv}
    \begin{onlyenv}<3>
    \item Compute marginal contribution of feature $j$ towards the prediction across all randomly drawn feature coalitions $S_1, \ldots , \Sm$
    \item Average all $M$ marginal contributions of feature $j$
    \item Shapley value $\phi_j$ is the payout of feature $j$, i.e., how much feature \textit{year} contributed to the overall prediction in bicycle counts of a specific observation $\xv$
    \end{onlyenv}
\end{itemize}
\vspace*{\fill}
%\vspace*{30pt}
\begin{center}
%\includegraphics<3>[width=0.5\textwidth]{figure_man/shapley_valuefct}
% Link https://docs.google.com/presentation/d/14FZZ4zk7IBZv6XnQA0wDVfCDhmjzf4np5uVj7KrIxXg/edit#slide=id.p
%\includegraphics<4>[page=1, width=1\textwidth]{figure_man/data_shapley}
%\includegraphics<5>[page=2, width=1\textwidth]{figure_man/data_shapley}
%\includegraphics<6>[page=3, width=1\textwidth]{figure_man/data_shapley}
%\includegraphics<7>[page=4, width=1\textwidth]{figure_man/data_shapley}
\includegraphics<1>[page=13, width=1\textwidth]{../../slides/04_shapley/figure_man/data_shapley}
\includegraphics<2>[page=14, width=1\textwidth]{../../slides/04_shapley/figure_man/data_shapley}
\includegraphics<3>[page=15, width=1\textwidth]{../../slides/04_shapley/figure_man/data_shapley}
\end{center}

\end{frame}





\begin{frame}{Revisited: Axioms for Fair Attributions}
  We take the general axioms for Shapley Values and apply it to predictions:
  \vspace{0.25cm}
  \begin{itemize}[<+->]
  \itemsep1em
    \item \textbf{Efficiency}: Shapley values add up to the (centered) prediction: %. 
    %That means, unlike, e.g., LIME, we get a dense attribution, and not a sparse one.
    $\sum\nolimits_{j=1}^p\phi_j=\fh(\xv)-\E_{\xv}(\fh(X))$
    \item \textbf{Symmetry}: Two features $j$ and $k$ that contribute the same to the prediction get the same payout\\
    $\leadsto$ interaction effects between features are fairly divided \\
      $\fh_{S\cup\{j\}}(\xv_{\Scupj}) = \fh_{\Scupk}(\xv_{\Scupk})$ for all $S \subseteq P\setminus\{j,k\}$ then $\phi_j=\phi_k$
    \item \textbf{Dummy / Null Player}: Shapley value of a feature that does not influence the prediction is zero $\leadsto$ if a feature was not selected by the model (e.g., tree or LASSO), its Shapley value is zero  \\
      $\fh_{\Scupj}(\xv_{\Scupj})=\fhS(\xv_S)$ for all $S \subseteq P$ then $\phi_j=0$
    \item \textbf{Additivity}:  For a prediction with combined payouts, the
      payout is the sum of payouts: $\phi_j(v_1) + \phi_j(v_2)$ $\leadsto$ Shapley values for model ensembles can be combined
  \end{itemize}
\end{frame}



% \begin{frame}{Additional Estimation Trick}


%   The Shapley value can be estimated more efficiently when certain coalitions are always included in the computation, instead of random sampling:
%   \vspace{0.25cm}
%   \begin{itemize}
%   \itemsep1em
%     \item The coalitions with $S = \emptyset$ (i.e., $|S| = 0$) and $S = \{1, \ldots, p\} \setminus j$ have the highest weights in the Shapley value computation $\leadsto$ including them makes Shapley value more stable with fewer samples %Sample weights have to be adapted for the sampled coalitions afterwards.
%     \item Intuition: Adding a feature to the empty coalition gives information about the \textit{pure} first order effect of the feature, which is the effect without any interactions 
%     \item Adding the feature value to the complete set of feature values gives us the information about the total effect of a feature, which is the sum of the main effect and all interaction effects with other features
%     \item For coalition $S = \emptyset$, there are $0! (|P| - 0 - 1)! = 1 \cdot (|P| - 1)! = (p - 1)!$ orders, which is the same for $S = P \setminus \{j\}$: $|P \setminus \{j\}|! (|P| - |P \setminus \{j\}| - 1)! = (p - 1)! (p - (p-1) - 1)! = (p-1)!$
% \end{itemize}
%  \end{frame}

% \begin{frame}{Additional Estimation Trick}
% An example with $p = 5$ features:
% \vspace{0.25cm}
%     \begin{itemize}
%     \itemsep1em
%         \item There are $5! = 120$ orders in total
%         \item In $(5 - 1)! = 24$ orders, we added feature value $x_j$ to the empty set
%         \item In 24 orders, we added the feature value to the otherwise full feature set
%         \item That means with just two sets, we can already get $\frac{48}{120} = 0.4$ of the contributions to the Shapley value
%         \item Similarly, we could proceed with all coalitions of $\{S: |S| = 1\}$ and $\{S: |S| = p - 1\}$
%         \item When some coalitions are added \enquote{manually}, and the rest are sampled, we have to adapt the weights: Let $w$ be the weight of the \enquote{manually} sampled coalitions, $\hat{\phi}_{j,fixed}$ the part of the Shapley value with only the manual contributions and $\hat{\phi}_{j,sample}$ the Shapley value with the sampled coalitions, then the Shapley value is: $w \cdot \hat{\phi}_{j,fixed} + (1 - w) \hat{\phi}_{j,sample}$
%   \end{itemize}
% \end{frame}

% \begin{frame}{Additional Estimation Trick}
%       \begin{center}
%         \includegraphics[width=0.5\textwidth]{figure/shapley-weights}
%       \end{center}
% \end{frame}

\begin{frame}{Bike Sharing Dataset}

\begin{center}
\includegraphics[width=0.6\textwidth]{../../slides/04_shapley/figure/shapley-bike.pdf}%{figure_man/bike-sharing03.png}%
\end{center}

\begin{itemize}
    \item Shapley values of observation $i = 200$ from the bike sharing data
    \item Difference between model prediction of this observation and the average prediction of the data is fairly distributed among the features (i.e., $4434 - 4507 \approx -73 $)
    \item Feature value temp = 28.5 has the most positive effect, with a contribution (increase of prediction) of about +400
\end{itemize}
\end{frame}

% \begin{frame}{Versions of the Shapley Value}

%   \begin{itemize}
%   \item KernelSHAP formulates the Shapley value solution as a regression problem using a specific kernel function. The authors show paralles to LIME and Deeplift.
%   \item TreeSHAP is a fast Shapley value computation method for tree-based models such as gradient boosted trees.
%  \end{itemize}
%     \tiny{Lundberg, Scott M., and Su-In Lee. "A Unified Approach to Interpreting Model Predictions." Advances in Neural Information Processing Systems 30 (2017): 4765-4774.}
%     \tiny{Lundberg, Scott M., Gabriel G. Erion, and Su-In Lee. "Consistent individualized feature attribution for tree ensembles." arXiv preprint arXiv:1802.03888 (2018).}
% \end{frame}

\begin{frame}{ADVANTAGES AND DISADVANTAGES}
	\textbf{Advantages:}
	\begin{itemize}
	 \item \textbf{Solid theoretical foundation} in game theory
        \item Prediction is \textbf{fairly distributed} among the feature values $\leadsto$ easy to interpret for a user
        \item \textbf{Contrastive explanations} that compare the prediction with the average prediction
	\end{itemize}
\vspace{0.25cm}
	\textbf{Disadvantages:}
	\begin{itemize}
		\item Without sampling, Shapley values need a lot of computing time to
		inspect all possible coalitions
		%\item The Shapley value of a feature value can be easily misinterpreted:
		%It is not the difference of the predicted value after removing the
		%feature from the model training; it is the contribution of a feature
		%value to the difference between the actual prediction and the mean
		%prediction, given the current set of features
		\item Like many other IML methods, Shapley values suffer from the
		inclusion of unrealistic data observations when features are correlated
	\end{itemize}



\end{frame}

%%%%%%%%%%%%%%%%%%%%%%%%%%%%%%%%%%%%%%%%%%%%%%%%%%%%%%%%%%
% TODO
\renewcommand{\titlefigure}{../../slides/04_shapley/figure_man/exSHAP.png}
\renewcommand{\learninggoals}{
\item Get an intuition of additive feature attributions
\item Understand the concept of Kernel SHAP
\item Ability to interpret SHAP plots
\item Global SHAP methods
}

\lecturechapter{SHAP (SHapley Additive exPlanation) Values}

\mysectionslide
% \begin{frame}{SHAP (SHapley Additive exPlanation) Values}
    
%     \vspace{1.5cm}
%     \begin{center}
%         \includegraphics[]{../../slides/04_shapley/figure_man/exSHAP.png}
%     \end{center}

%     \textbf{Learning goals}
%     \begin{itemize}
%         \item Get an intuition of additive feature attributions
%         \item Understand the concept of Kernel SHAP
%         \item Ability to interpret SHAP plots
%         \item Global SHAP methods
%     \end{itemize}
  
% \end{frame}
% \begin{frame}{Shapley Values in ML - A short Recap}
  
%   \textbf{Question:} How much does a feature $j$ contribute to the prediction of a single observation. \\
%   \textbf{Idea:} Use Shapley values from cooperative game theory \\
%   \pause
%   \textbf{Procedure:} 
%   \begin{itemize}
%     \item Compare ``reduced prediction function'' of feature coalition $S$ with $\Scupj$ 
%     \item Iterate over possible coalitions to calculate the marginal contribution of feature $j$ to sample $\xv$
% \end{itemize}

% $$\phi_j 
% %= \frac{1}{|P|!} \sum_{\tau \in \Pi} (v(\Stauj) - v(\Stau)) 
% = \frac{1}{p!} \sum_{\tau \in \Pi}  \underbrace{\fh_{\Stauj}(\xv_{\Stauj}) - \fh_{\Stau}(\xv_{\Stau})}_{\text{marginal contribution of feature $j$}} $$

% \pause
% \textbf{Remember:}

% \begin{itemize}
%     \item $\fh$ is the prediction function, $p$ denotes the number of features
%     \item Non-existent features in a coalition are replaced by values of random feature values 
%     \item Recall $\Stau$ defines the coalition as the set of players before player $j$ in order $\tau = (\tau^{(1)}, \dots, \tau^{(p)})$% states from the feature
    
%   \centerline{
%   \begin{tabular}{|c|c|c|c|c|c|c|}
%     %\multicolumn{3}{c}{\enspace\raisebox{-3.3ex}[0pt][2.6ex]{$ \overbrace{\vphantom{-}\hspace{9em}}^{|S|! \text{ permutations}}$}} &
%     %\multicolumn{1}{c}{} &
%     %\multicolumn{3}{c}{\enspace\raisebox{-3.3ex}[0pt][2.6ex]{$ \overbrace{\vphantom{-}\hspace{9em}}^{(|P| - |S| - 1)! \text{ permutations}}$}}\\
%     \hline
%     $\tau^{(1)}$ & \ldots & $\tau^{(|S|)}$ & $\tau^{(|S| + 1)}$ & $\tau^{(|S| + 2)}$ & \ldots & $\tau^{(p)}$ \\
%     \hline
%     \multicolumn{3}{c}{\enspace\raisebox{1.3ex}[0pt][2.6ex]{$ \underbrace{\vphantom{-}\hspace{9em}}^{}$}} &
%     \multicolumn{1}{c}{\enspace\raisebox{1.3ex}[0pt][2.6ex]{$ \underbrace{\vphantom{-}\hspace{4em}}^{}$}} &
%     \multicolumn{3}{c}{\enspace\raisebox{1.3ex}[0pt][2.6ex]{$ \underbrace{\vphantom{-}\hspace{9em}}^{}$}}\\
%     \multicolumn{3}{c}{$S_j^\tau:$ Players before player $j$} & \multicolumn{1}{c}{player $j$} & \multicolumn{3}{c}{Players after player $j$} \\
%   \end{tabular}}
% \end{itemize}

% \end{frame}


% \begin{frame}{Shapley Values in ML - A short Recap}
  
%   \textbf{Example:} 
%   \begin{itemize}
%       \item Train a random forest on bike sharing data only using features humidity (hum), temperature (temp) and windspeed (ws)
%       \item Calculate Shapley value for an observation $\xv$ with $\fh(\xv) = \color{orange}{2573}$
%       \item Mean prediction is $\E(\fh) = \color{blue}{4515}$
%   \end{itemize}
%   \pause
%   \textbf{Exact Shapley calculation for humidity:} 
%   \begin{table}[T]
%       \centering
%       \begin{tabular}{c|c|c|c|c}
%   $S$    &  $\Scupj$  & $\fh_S$ &  $\fh_{\Scupj}$  & weight\\\hline
%      $\varnothing$&    hum  & \color{blue}{4515} & 4635 & $2/6$\\
%       temp &  temp, hum & 3087 & 3060& $1/6$\\
%       ws &  ws, hum & 4359  & 4450 & $1/6$\\
%       temp, ws &  hum, temp, ws & 2623 & \color{orange}{2573} & $2/6$
         
%       \end{tabular}
%       \label{tab:my_label}
%   \end{table}

% $$
% \phi_{hum} = \frac{2}{6} (4635-4515) + \frac{1}{6} (3060-3087) + \frac{1}{6} (4450-4359) + \frac{2}{6} (2573-2623) = 34
% $$

% \end{frame}


\begin{frame}{From Shapley to SHAP}
%\textbf{Example continued}: Same calculation can be done for temperature and windspeed:
%\begin{itemize}
%    \item $\phi_{temp} = \ldots = -1654$
%    \item $\phi_{ws} = \ldots = -323$
%\end{itemize}

\begin{columns}[T]
\begin{column}{0.5\textwidth}
\textbf{Remember}: Shapley values explain the difference between actual and average prediction:
\begin{eqnarray*}
\color{orange}{2573} \color{black}- \color{blue}{4515} \color{black} = &34 - 1654 - 323 &= - 1942\\
\fh(\xv) - \E(\fh) =& \phi_{hum} + \phi_{temp} + \phi_{ws}&\\
\end{eqnarray*}
$\leadsto$ can be rewritten to
$$
\fh(\xv) = \underbrace{\E(\fh)}_{\phi_0} + \phi_{hum} + \phi_{temp} + \phi_{ws}
$$
%N.B.: This reminds us of a LM with weights $\phi_j$ multiplied by a binary feature value indicating weather a feature is included in a considered coalition.
\end{column}
\begin{column}{0.5\textwidth}
\begin{figure}
    \centering
    \includegraphics[width=0.9\columnwidth]{../../slides/04_shapley/figure/shapley2shap.pdf}
\end{figure}
\end{column}
\end{columns}
\end{frame}



\begin{frame}{SHAP Definition \citebutton{Lundberg et al. 2017}{https://doi.org/10.48550/arXiv.1705.07874}}
\textbf{Aim}: 
Find an additive combination that explains the prediction of an observation $\xv$ by computing the contribution of each feature to the prediction using a (more efficient) estimation procedure. \\\medskip
\only<1-4>{\textbf{Definition}
\begin{itemize}
    \item Define simplified (binary) coalition feature space $\mathbf{Z}^\prime \in \{0,1\}^{K \times p}$ with $K$ rows and $p$ columns 
    \item Rows are referred to as $\mathbf{z}^{\prime (k)} = \{z^{\prime (k)}_1, \ldots, z^{\prime (k)}_p\}$ with $k \in \{1,\ldots, K\}$ (indexes $k$-th coalition)
    \item Columns are referred to as $\mathbf{z}_j$ with $j \in \{1, \ldots, p\}$ being the index of the original feature
\end{itemize}
}
\only<1>{
\textbf{Example}: 
\begin{table}[]
    \centering
     \begin{tabular}{l |c|ccc}
  Coalition  & $\mathbf{z}^{\prime (k)}$ &  hum & temp & ws \\
  \hline 
  $\varnothing$ & $\mathbf{z}^{\prime (1)}$ & 0 & 0 & 0  \\
  hum & $\mathbf{z}^{\prime (2)}$ & 1 & 0 & 0  \\
  temp &  $\mathbf{z}^{\prime (3)}$ & 0 & 1 & 0  \\
  ws &   $\mathbf{z}^{\prime (4)}$ & 0 & 0 & 1  \\
  hum, temp & $\mathbf{z}^{\prime (5)}$ & 1 & 1 & 0  \\
  temp, ws & $\mathbf{z}^{\prime (6)}$ & 0 & 1 & 1  \\
  hum, ws &   $\mathbf{z}^{\prime (7)}$ & 1 & 0 & 1  \\
  hum, temp, ws & $\mathbf{z}^{\prime (8)}$ & 1 & 1 & 1  \\
  \end{tabular}
\end{table}
}

%\vspace{0.5cm}
\begin{exampleblock}{}
\[
g\left(\tikzmark{zp} \mathbf{z}^{\prime (k)}\right)=
\tikzmark{ph0}\phi_{0}+\sum_{j=1}^{p}
\tikzmark{phj} \phi_{j} z_{j}^{\prime (k)}
\]
\begin{tikzpicture}[
  remember picture,
  overlay,
  expl/.style={draw=blue,fill=white,rounded corners,text width=3cm},
  arrow/.style={blue,ultra thick,->,>=latex}
]
\node<2-4>[expl] 
  (zex) 
  at (2,1.5cm)
  {$\mathbf{z}^{\prime (k)}$: \textbf{Coalition} \\ simplified features};
\node<3-4>[expl] 
  (ph0ex) 
  at (5,-.5cm)
  {$\phi_0$: \textbf{Null Output} \\ Average Model Baseline ($\E(\fh))$};
\node<4>[expl] 
  (phjex) 
  at (12,0cm)
  {$\phi_j$: \textbf{Attribution} \\ How much does feature $j$ change the output for coaltion $k$};

\draw<2-4>[arrow]
  (zex.east) to[out=180,in=90] ([xshift= 1ex, yshift=1.9ex]{pic cs:zp});  
\draw<3-4>[arrow]
  (ph0ex.east) to[out=180,in=270] ([xshift= 1ex, yshift=-0.5ex]{pic cs:ph0});  
\draw<4>[arrow]
  (phjex.west) to[out=90,in=270] ([xshift= 1ex, yshift=-0.5ex]{pic cs:phj});
 \node<5>[expl] 
  (zex)  
  at (2,1cm)
  {$g(\mathbf{z}^{\prime (k)})$: \textbf{Marginal Contribution} \\ Contribution of coalition $\mathbf{z}^{\prime (k)}$ to the prediction};
  \draw<5>[arrow]
  (zex.east) to[out=180,in=90] ([xshift= 1ex, yshift=1.9ex]{pic cs:zp});  
\node<5>[expl] 
  (phjsh) 
  at (12,1.5cm)
  {$\phi_j$: \textbf{Shapley Values}};
\draw<5>[arrow]
  (phjsh.west) to[out=90,in=90] ([xshift= 1ex, yshift=2ex]{pic cs:phj}); 
\draw<5> [
    thick,
    decoration={
        brace,
        mirror,
        raise=0.5cm
    },
    decorate
] (7,0.7) -- (9.5,0.7)
node[pos=0.5,below=15pt,black]{\textbf{Additive Feature Attribution}};
\end{tikzpicture}
\end{exampleblock}
\begin{onlyenv}<5>
\vspace{1cm}
\textbf{Problem}\\
How do we estimate the Shapley values $\phi_j$?
\end{onlyenv}

\end{frame}



%\begin{frame}{Example}

%Given the following example from the bike sharing data set

%\begin{table}[h]
%\centering
%\begin{tabular}{l rrrrr || r}
%  \hline
%  && temperature & humidity & windspeed & year & prediction\\ 
%  \hline
% example $x_{ex}$ && 24.27 & 58.5 & 13.96 & 2011 & 6825 \\ 
% \hline
%\end{tabular}
%\end{table}

%we are searching for Shapley values such that
%\begin{equation}
%\begin{array}{lllllcr}
%\phi_0 &+ \phi_{temp} &+ \phi_{hum} &+ \phi_{windspeed} &+ \phi_{yr} & = &\hat{y} \\
%4469 &+ 1809 &+ 450 &+ 241 &- 144 & = & 6825
%\end{array}
%\end{equation}

%\begin{figure}
 %   \centering
 %   \includegraphics[width=\columnwidth]{figure_man/exSHAP.png}
%\end{figure}
%\end{frame}

\begin{frame}{Kernel SHAP - In 5 Steps}

\textbf{Definition:} A kernel-based, model-agnostic method to compute Shapley values via local surrogate models (e.g. linear model)\\
\vspace{1cm}
\begin{enumerate}
    \item Sample coalitions 
    %\begin{onlyenv}<1>
   % $$z_{k}^{\prime} \in\{0,1\}^{M}, \quad k \in\{1, \ldots, K\}$$
    %\end{onlyenv}
    
    \item Transfer coalitions into feature space \& get predictions by applying ML model
    
 %   \begin{onlyenv}<2>
  %  $$\hat{f}: \hat{f}\left(h_{x}\left(z_{k}^{\prime}\right)\right)$$
   % \end{onlyenv}
    
    \item Compute weights through kernel
 %   \begin{onlyenv}<3>
 %   $$\pi_{x}\left(z^{\prime}\right)=\frac{(M-1)}{\left(\begin{array}{c} M \\\left|z^{\prime}\right|\end{array}\right)\left|z^{\prime}\right|\left(M-\left|z^{\prime}\right|\right)}$$
 %   \end{onlyenv}
    
    \item Fit a weighted linear model 
  %  \begin{onlyenv}<4>
  %  $$L\left(\hat{f}, g, \pi_{x}\right)=\sum_{z^{\prime} \in Z}\left[\hat{f}\left(h_{x}\left(z^{\prime}\right)\right)-g\left(z^{\prime}\right)\right]^{2} \pi_{x}\left(z^{\prime}\right)$$
  %  \end{onlyenv}

    \item Return Shapley values
%    \begin{onlyenv}<5>
%    $$(\phi_1, \ldots, \phi_M)$$
%    \end{onlyenv}
    
    
\end{enumerate}

\end{frame}

\begin{frame}{Kernel SHAP - In 5 Steps}


\textbf{Step 1: Sample coalitions}
\begin{itemize}
    \item Sample K coalitions from the simplified feature space
    $$\mathbf{z}^{\prime (k)} \in\{0,1\}^{p}, \quad k \in\{1, \ldots, K\}$$
    \item For our simple example, we have in total $2^p = 2^3 = 8$ coalitions (without sampling)
\end{itemize}

\begin{table}[]
    \centering
     \begin{tabular}{l |c|ccc}
  Coalition  & $\mathbf{z}^{\prime (k)}$ &  hum & temp & ws \\
  \hline 
  $\varnothing$ & $\mathbf{z}^{\prime (1)}$ & 0 & 0 & 0  \\
  hum & $\mathbf{z}^{\prime (2)}$ & 1 & 0 & 0  \\
  temp &  $\mathbf{z}^{\prime (3)}$ & 0 & 1 & 0  \\
  ws &   $\mathbf{z}^{\prime (4)}$ & 0 & 0 & 1  \\
  hum, temp & $\mathbf{z}^{\prime (5)}$ & 1 & 1 & 0  \\
  temp, ws & $\mathbf{z}^{\prime (6)}$ & 0 & 1 & 1  \\
  hum, ws &   $\mathbf{z}^{\prime (7)}$ & 1 & 0 & 1  \\
  hum, temp, ws & $\mathbf{z}^{\prime (8)}$ & 1 & 1 & 1  \\
  
 
  \end{tabular}
\end{table}

\end{frame}

\begin{frame}{Kernel SHAP - In 5 Steps}


\textbf{Step 2: Transfer Coalitions into feature space \& get predictions by applying ML model}
\begin{itemize}
   % \item $$\hat{f}: \hat{f}\left(h_{x}\left(z_{k}^{\prime}\right)\right)$$
   \item $\mathbf{z}^{\prime (k)}$ is 1 if features are part of the $k$-th coalition, 0 if they are absent
   \item To calculate predictions for these coalitions, we need to define a function which maps the binary feature space back to the original feature space
\end{itemize}


\begin{tikzpicture}
\centering


\node (tab1) {%
       \begin{tabular}{l |cccc}
  $\xv^{coalition}$ &  hum & temp & ws \\
  \hline 
  $\xv^{\{\varnothing\}}$ & $\varnothing$ & $\varnothing$ &$\varnothing$  \\
   $\xv^{\{hum\}}$ & 51.6 & $\varnothing$ & $\varnothing$  \\
    $\xv^{\{temp\}}$ & $\varnothing$ & 5.1 & $\varnothing$  \\
     $\xv^{\{ws\}}$ & $\varnothing$ & $\varnothing$ & 17.0  \\
     $\xv^{\{hum, temp\}}$ & 51.6 & 5.1 & $\varnothing$  \\
     $\xv^{\{temp, ws\}}$ &$\varnothing$ & 5.1 & 17.0  \\
     $\xv^{\{hum, ws\}}$ & 51.6 & $\varnothing$ & 17.0  \\
  $\xv^{\{hum, temp, ws\}}$ &51.6 & 5.1 & 17.0   \\
  
 
  \end{tabular}};

\node [left=of tab1] (tab2) {%
     \begin{tabular}{l |c|ccc}
  Coalition & $\mathbf{z}^{\prime (k)}$ &  hum & temp & ws \\
  \hline 
  $\varnothing$ & $\mathbf{z}^{\prime (1)}$ & 0 & 0 & 0  \\
  hum & $\mathbf{z}^{\prime (2)}$ & 1 & 0 & 0  \\
  temp &  $\mathbf{z}^{\prime (3)}$ & 0 & 1 & 0  \\
  ws &   $\mathbf{z}^{\prime (4)}$ & 0 & 0 & 1  \\
  hum, temp & $\mathbf{z}^{\prime (5)}$ & 1 & 1 & 0  \\
  temp, ws & $\mathbf{z}^{\prime (6)}$ & 0 & 1 & 1  \\
  hum, ws &   $\mathbf{z}^{\prime (7)}$ & 1 & 0 & 1  \\
  hum, temp, ws & $\mathbf{z}^{\prime (8)}$ & 1 & 1 & 1  \\
   
 
  \end{tabular}};
\draw[->]
(tab2.north) to[out=10,in=170] node[below]{} (tab1.north) ;
\end{tikzpicture}


   




\end{frame}


\begin{frame}{Kernel SHAP - In 5 Steps}


\textbf{Step 2: Transfer Coalitions into feature space \& get predictions by applying ML model}
\begin{itemize}
   % \item $$\hat{f}: \hat{f}\left(h_{x}\left(z_{k}^{\prime}\right)\right)$$
    \item Define 
$h_x\left(\mathbf{z}^{\prime (k)}\right)=\mathbf{z}^{(k)} \text { where } h_x:\{0,1\}^{p} \rightarrow \R^{p}$
 maps 1’s to feature values of observation $\xv$ for features part of the $k$-th coalition and 0's to feature values of a \color{orange}{randomly sampled observation} \color{black}for features absent in the $k$-th coalition
  % \item Absent feature values are replaced by feature values of a \color{orange}{random observation} \color{black} of the dataset (permuted) $\leadsto$ permute feature values several times
  (feature values are permuted multiple times) 
   \item Predict with ML model on this dataset $\hat{f}: \hat{f}\left(h_{x}\left(\mathbf{z}^{\prime (k)}\right)\right)$
\end{itemize}


\begin{tikzpicture}
\centering


\node (tab1) {%
       \begin{tabular}{l |ccc | c}
  $\mathbf{z}^{(k)}$ &  hum & temp & ws & $\hat{f}\left(h_{x}\left(\mathbf{z}^{\prime (k)}\right)\right)$\\
  \hline 
  $\mathbf{z}^{(1)}$ & \color{orange}{64.3} & \color{orange}{28.0} & \color{orange}{14.5} & 6211 \\
   $\mathbf{z}^{(2)}$ & 51.6 & \color{orange}{28.0} & \color{orange}{14.5} & 5586  \\
    $\mathbf{z}^{(3)}$ & \color{orange}{64.3} & 5.1 & \color{orange}{14.5}  & 3295\\
     $\mathbf{z}^{(4)}$ & \color{orange}{64.3} & \color{orange}{28.0} & 17.0 &5762 \\
     $\mathbf{z}^{(5)}$ & 51.6 & 5.1 & \color{orange}{14.5}  & 2616\\
     $\mathbf{z}^{(6)}$ &\color{orange}{64.3} & 5.1 & 17.0  & 2900\\
     $\mathbf{z}^{(7)}$ & 51.6 & \color{orange}{28.0} & 17.0 & 5411 \\
  $\mathbf{z}^{(8)}$ &51.6 & 5.1 & 17.0 & 2573  \\
  
 
  \end{tabular}};

\node [left=of tab1] (tab2) {%
     \begin{tabular}{l |c|ccc}
  Coalition & $\mathbf{z}^{\prime (k)}$ &  hum & temp & ws \\
  \hline 
  $\varnothing$ & $\mathbf{z}^{\prime (1)}$ & 0 & 0 & 0  \\
  hum & $\mathbf{z}^{\prime (2)}$ & 1 & 0 & 0  \\
  temp &  $\mathbf{z}^{\prime (3)}$ & 0 & 1 & 0  \\
  ws &   $\mathbf{z}^{\prime (4)}$ & 0 & 0 & 1  \\
  hum, temp & $\mathbf{z}^{\prime (5)}$ & 1 & 1 & 0  \\
  temp, ws & $\mathbf{z}^{\prime (6)}$ & 0 & 1 & 1  \\
  hum, ws &   $\mathbf{z}^{\prime (7)}$ & 1 & 0 & 1  \\
  hum, temp, ws & $\mathbf{z}^{\prime (8)}$ & 1 & 1 & 1  \\
  
  \end{tabular}};
\draw[->]
(tab2.north) to[out=10,in=170] node[below]{$h_x(\mathbf{z}^{\prime (k)})$} (tab1.north) ;
\end{tikzpicture}


   




\end{frame}

\begin{frame}{Kernel shap - in 5 steps}
\textbf{Step 3: Compute weights through Kernel \only<2>{\citebutton{see shapley\_kernel\_proof.pdf}{https://proceedings.neurips.cc/paper/2017/file/8a20a8621978632d76c43dfd28b67767-Supplemental.zip}}}\\\medskip
\textbf{Intuition}: We learn most about individual features if we can study their effects in isolation or at maximal interaction:
Small coalitions (few 1’s) and large coalitions (i.e. many 1’s) get the largest weights 


\begin{onlyenv}<1>
\begin{figure}
    \centering
    \includegraphics[width=0.4\columnwidth]{../../slides/04_shapley/figure_man/kernel-weights.pdf}
    %\caption{Examplary dependence between kernel weights and coalition size for a data set p = 10 features}   
\end{figure}
\end{onlyenv}


\begin{onlyenv}<2>
\vspace{1cm}
\begin{exampleblock}{}
\[
\tikzmark{pi}\pi_{x}\left(\mathbf{z}^{\prime (k)}\right)=\frac{(
\tikzmark{M}p-1)}{\left(\begin{array}{c} p \\\left|\mathbf{z}^{\prime (k)}\right|\end{array}\right)\left|
\tikzmark{z}\mathbf{z}^{\prime (k)}\right|\left(p-\left|\mathbf{z}^{\prime (k)}\right|\right)}
\]
\begin{tikzpicture}[
  remember picture,
  overlay,
  expl/.style={draw=blue,fill=white,rounded corners,text width=3cm},
  arrow/.style={blue,ultra thick,->,>=latex}
]
\node[expl] 
  (piex) 
  at (4,0cm)
  {$\pi_x(\mathbf{z}^{\prime (k)})$: kernel weight for coalition $\mathbf{z}^{\prime (k)}$};
\node[expl] 
  (Mex) 
  at (13,2cm)
  {p: Number of features in $\xv$};
\node[expl] 
  (zex) 
  at (8,-1cm)
  {$\mid \mathbf{z}^{\prime (k)}\mid$: coalition size / sum of 1s in $\mathbf{z}^{\prime (k)}$};
\draw[arrow]
  (piex.west) to[out=180,in=135] ([xshift= 0.5ex, yshift=2ex]{pic cs:pi}); 
\draw[arrow]
  (Mex.west) to[out=180,in=135] ([xshift= 0.5ex, yshift=2ex]{pic cs:M}); 
\draw[arrow]
  (zex.north) to[out=180,in=250] ([xshift= 0.5ex, yshift=-1ex]{pic cs:z}); 
\end{tikzpicture}
\end{exampleblock}
\end{onlyenv}


% \begin{onlyenv}<3>

% $$\pi_{x}\left(z^{\prime}\right)=\frac{(M-1)}{\left(\begin{array}{c} M \\\left|z^{\prime}\right|\end{array}\right)\left|z^{\prime}\right|\left(M-\left|z^{\prime}\right|\right)}$$

% \begin{itemize}
%     \item If a coalition consists of a single feature, we can learn about this feature’s isolated main effect on the prediction
%     \item If a coalition consists of all but one feature, we can learn about this feature’s total effect (main effect plus feature interactions)
%     \item If a coalition consists of half the features, we learn little about an individual feature’s contribution, as there are many possible coalitions with half of the features
% \end{itemize}
% \end{onlyenv}

% \begin{onlyenv}<4>
% \vspace{1cm}
% \textbf{Limited Budget $K$}: Can we be a bit smarter about the sampling of coalitions, than just randomly drawing?
% \begin{itemize}
%     \item The smallest and largest coalitions take up most of the weight\\ We get better Shapley value estimates by using some of the sampling budget K to include these high-weight coalitions
%     \item We start with all possible coalitions with 1 and M-1 features, which makes 2 times M coalitions in total\\ When we have enough budget left (current budget is K - 2M), we can include coalitions with 2 features and with M-2 features and so on.
%     \item From the remaining coalition sizes, we sample with readjusted weights
% \end{itemize}
% \end{onlyenv}
  
\end{frame}


\begin{frame}{Kernel shap - in 5 steps}
\textbf{Step 3: Compute weights through Kernel}\\\medskip
\textbf{Purpose}: to include this knowledge in the local surrogate model (linear regression), we calculate weights for each coalition which are the observations of the linear regression
\only<1>{    $$\pi_{x}\left(\mathbf{z}^{\prime}\right)=\frac{(p-1)}{\left(\begin{array}{c} p \\|\mathbf{z}^{\prime}|\end{array}\right)|\mathbf{z}^{\prime}|(p-|\mathbf{z}^{\prime}|)} \leadsto \pi_x\left(\mathbf{z}^{\prime} = (1,0,0)\right)=\frac{(3-1)}{\left(\begin{array}{c} 3 \\1\end{array}\right)1\left(3-1\right)} = \frac{1}{3}$$
}

\begin{table}[]
    \centering
        \begin{tabular}{l |c|ccc|c}
 Coalition & $\mathbf{z}^{\prime (k)}$ &  hum & temp & ws & weight\\
  \hline 
  $\varnothing$ & $\mathbf{z}^{\prime (1)}$ & 0 & 0 & 0 & $\infty$ \\
  hum & $\mathbf{z}^{\prime (2)}$ & 1 & 0 & 0 & 0.33 \\
  temp &  $\mathbf{z}^{\prime (3)}$ & 0 & 1 & 0 & 0.33 \\
  ws &   $\mathbf{z}^{\prime (4)}$ & 0 & 0 & 1 & 0.33  \\
  hum, temp & $\mathbf{z}^{\prime (5)}$ & 1 & 1 & 0 & 0.33 \\
  temp, ws & $\mathbf{z}^{\prime (6)}$ & 0 & 1 & 1 & 0.33 \\
  hum, ws &   $\mathbf{z}^{\prime (7)}$ & 1 & 0 & 1 & 0.33 \\
  hum, temp, ws & $\mathbf{z}^{\prime (8)}$ & 1 & 1 & 1 & $\infty$ \\
  
 
  \end{tabular}
\end{table}
\medskip
\only<2>{
$\leadsto$ weights for empty and full set are infinity and not used as observations for the linear regression\\ $\leadsto$ instead constraints are used such that properties (local accuracy and missingness) are satisfied
}

  
\end{frame}

%\begin{frame}{Coalition Mapping}
%We define a coalition $z^{\prime}$, by describing a function 

%$$
%h\left(z^{\prime}\right)=z \text { where } h:\{0,1\}^{M} \rightarrow \mathbb{R}^{p}
%$$


%\begin{onlyenv}<1>
%\vspace{1cm}
%\begin{itemize}
%    \item Coalition $z^{\prime} \in \{0, 1\}^M$ is the  vector, indicating if feature $j$ contributes to the prediction 
%    \item $h(\cdot)$ represent a function that maps 1’s to the corresponding value from the observation x that we want to explain: $h(\cdot)$ connects our coalition vector to the underlying data 
%\end{itemize}
%\end{onlyenv}

%\begin{onlyenv}<2->
%\begin{tikzpicture}
%\centering

%\node<2> (tab1) {%
%  \begin{tabular}{l |cccc}
%  observation & temp & hum & ws & yr\\
%  \hline 
%  $x_{ex}$ & 24.7 & 58.5 & 13.96 & 2011\\
 % \\
 % \\
  %\end{tabular}};
%\node<3-> (tab1) {%
%  \begin{tabular}{l |cccc}
%  observation & temp & hum & ws & yr\\
%  \hline 
%  $x_{ex}$ & 24.7 & 58.5 & 13.96 & 2011\\
%  $z_{temp, yr}$ & 24.7 & $\varnothing$ & $\varnothing$ & 2011\\
%  $z_{yr}$ & $\varnothing$ & $\varnothing$ & $\varnothing$ & 2011\\
%  \end{tabular}};
%\node<2> [left=of tab1] (tab2) {%
%  \begin{tabular}{l |cccc}
%  Coalition & temp & hum & ws & yr\\
%  \hline 
%  $x^{\prime}$ & 1 & 1 & 1 & 1 \\
%  \\
%  \\
%  \end{tabular}};
%\node<3-> [left=of tab1] (tab2) {%
%  \begin{tabular}{l |cccc}
%  Coalition & temp & hum & ws & yr\\
%  \hline 
%  $x^{\prime}$ & 1 & 1 & 1 & 1 \\
%  $z^{\prime}_{temp, yr}$ & 1 & 0 & 0 %& 1 \\
%  $z^{\prime}_{yr}$ & 0 & 0 & 0 & 1 %\\
%  \end{tabular}};
%\draw<2->[->]
%(tab2.north) to[out=30,in=150] node[below]{$h(\cdot)$} (tab1.north) ;
%\end{tikzpicture}
%\end{onlyenv}
%\begin{onlyenv}<3->
%\begin{itemize}
%    \item $h(\cdot)$ maps 1’s to the %corresponding value from the observation x that we want to explain
%    \item<4> it maps 0’s to the values of another observation that we sample from the data
%    \item<4>  we equate “feature value is absent” with “feature value is replaced by random feature value from data”
%\end{itemize}
%\end{onlyenv}
%\end{frame}


\begin{frame}{Kernel shap - in 5 steps}
\textbf{Step 4: Fit a weighted linear model}\\\medskip
\textbf{Aim}: Estimate a weighted linear model with Shapley values being the coefficients $\phi_j$
$$
g\left(\mathbf{z}^{\prime (k)}\right)=
\phi_{0}+\sum_{j=1}^{p}
 \phi_{j} z_{j}^{\prime (k)} \only<2>{\leadsto g\left(\mathbf{z}^{\prime (k)}\right)=
4515 +
 34 \cdot z_{1}^{\prime (k)} - 1654 \cdot z_{2}^{\prime (k)} - 323 \cdot z_{3}^{\prime (k)} }
$$


\only<1>{
and minimize by WLS using the weights $\pi_{x}$ of step 3
    $$L\left(\hat{f}, g, \pi_{x}\right)=\sum_{k = 1}^K\left[\hat{f}\left(h_{x}\left(\mathbf{z}^{\prime (k)}\right)\right)-g\left(\mathbf{z}^{\prime (k)}\right)\right]^{2} \pi_{x}\left(\mathbf{z}^{\prime (k)}\right)$$

with $\phi_0 = \E(\fh)$ and $\phi_p = \fh(x) - \sum_{j=0}^{p-1} \phi_j$ we receive a $p-1$ dimensional linear regression problem
}

\only<2>{
\begin{table}[]
    \centering
        \begin{tabular}{l |ccc|c|c}
  $\mathbf{z}^{\prime (k)}$ &  hum & temp & ws & weight & $\fh$\\
  \hline 
   $\mathbf{z}^{\prime (2)}$ & 1 & 0 & 0 & 0.33 & 4635\\
    $\mathbf{z}^{\prime (3)}$ & 0 & 1 & 0 & 0.33 & 3087\\
     $\mathbf{z}^{\prime (4)}$ & 0 & 0 & 1 & 0.33 & 4359\\
     $\mathbf{z}^{\prime (5)}$ & 1 & 1 & 0 & 0.33 & 3060\\
     $\mathbf{z}^{\prime (6)}$ & 0 & 1 & 1 &0.33 & 2623\\
     $\mathbf{z}^{\prime (7)}$ & 1 & 0 & 1 & 0.33 & 4450\\
      \multicolumn{1}{c}{} & \multicolumn{3}{c}{\upbracefill}&\multicolumn{1}{c}{} &\multicolumn{1}{c}{\upbracefill}\\[-1ex]
    \multicolumn{1}{c}{} & \multicolumn{3}{c}{$\scriptstyle input$}&\multicolumn{1}{c}{} &  \multicolumn{1}{c}{$\scriptstyle output$}\\
  
 
  \end{tabular}
\end{table}
}


  
\end{frame}


\begin{frame}{Kernel shap - in 5 steps}
\textbf{Step 5: Return SHAP values}\\\medskip
\textbf{Intuition}: Estimated Kernel SHAP values are equivalent to Shapley values 
$$
g(\mathbf{z}^{\prime (8)}) = \fh(h_x(\mathbf{z}^{\prime (8)}) ) = 4515 + 34 \cdot 1 - 1654 \cdot 1 - 323 \cdot 1 = \underbrace{\E(\fh)}_{\phi_0} + \phi_{hum} + \phi_{temp} + \phi_{ws} = \fh(\xv) = 2573
$$

\begin{figure}
    \centering
    \includegraphics[width=\columnwidth]{../../slides/04_shapley/figure_man/exSHAP.png}
\end{figure}

\end{frame}
%\begin{frame}{Marginal Contribution}

%\begin{itemize}
%    \begin{onlyenv}<1>
%    \item Consider coalition $z^{\prime}$ as indicator function for our shapley values $\phi$
%    \end{onlyenv}
%    \begin{onlyenv}<2>
%    \item This connects the coalition vector $z^{\prime}$ to the respective marginal contribution
%    \end{onlyenv}
%    \begin{onlyenv}<3>
%    \item To estimate the marginal contribution, we can transfer the coalition to the data space by $h(z^{\prime})$
%    \end{onlyenv}
%    \begin{onlyenv}<4->
%    \item $\fh(h(z^{\prime}))$ connects the coalitions directly to the marginal distribution.
%    \end{onlyenv}
%\end{itemize}

%\vspace{1cm}

%\begin{tikzpicture}
%\centering

%\node<1-2> (tab1) {%
%  \begin{tabular}{l |cccc}
%  Coalition & temp & hum & ws & yr\\
%  \hline 
%  $x^{\prime}$ & 1 & 1 & 1 & 1 \\
%  $z^{\prime}_{temp, yr}$ & 1 & 0 & 0 & 1 \\
%  $z^{\prime}_{yr}$ & 0 & 0 & 0 & 1 \\
%  \end{tabular}};
%\node<2-> [right=of tab1] (tab2) {%
%\begin{tabular}{l | cccc}
%  & temp & hum & ws & yr\\
%  \hline 
%  $g(x^{\prime})$ & $\phi_{temp}$ + & $\phi_{hum}$ + & $\phi_{ws}$ + & $\phi_{yr}$ \\
%  $g(z^{\prime}_{temp, yr})$ & $\phi_{temp}$ + &  &  & $\phi_{yr}$\\
%   $g(z^{\prime}_{yr})$ & &  &  & $\phi_{yr}$ \\
%  \end{tabular}};
%\node<3-> [left=of tab2] (tab) {%
%  \begin{tabular}{l |cccc}
%  observation & temp & hum & ws & yr\\
%  \hline 
%  $x_{ex}$ & 24.7 & 58.5 & 13.96 & %2011\\
%  $z_{temp, yr}$ & 24.7 & %$\varnothing$ & $\varnothing$ & %2011\\
%  $z_{yr}$ & $\varnothing$ & %$\varnothing$ & $\varnothing$ & 2011\\
%  \end{tabular}};
%\draw<2>[->]
%(tab1.south) to[out=320,in=200] node[above]{$\sum \mathbb{I}_{[z^{\prime}_i == 1]} \phi_i$} (tab2.south) ;
%\draw<3->[->]
%(tab2.south) to[out=200,in=330] node[above]{$\fh(h(z^{\prime}))$} (tab1.south) ;
%\end{tikzpicture}

%\begin{onlyenv}<4>
%\begin{equation}
%\begin{array}{lllc}
  
%  g(x^{\prime}) &= \phi_{temp} + \phi_{hum} + \phi_{ws} + &\phi_{yr} &= 6825\\
%  g(z^{\prime}_{temp, yr}) &= \phi_{temp} + &\phi_{yr} &= 6134\\
%   g(z^{\prime}_{yr}) &= &\phi_{yr} &= 4325\\
%\end{array}
%\end{equation}
%\end{onlyenv}

%\begin{onlyenv}<5>
%\vspace{0.5cm}

%\textbf{Notice:}\\ We created a coalition data set $Z^{\prime}$ here by sampling multiple coalitions from observation $\xv$ that is evaluable with the prediction function $\fh$
%\end{onlyenv}

%\end{frame}





\begin{frame}{Properties}

\textbf{Local Accuracy}
$$
f(\xv)=g\left(\xv^{\prime}\right)=\phi_{0}+\sum_{j=1}^{p} \phi_{j} x_{j}^{\prime}
$$
\begin{onlyenv}<1>
\textbf{Intuition:} If the coalition includes all features ($\xv^{\prime}  \in \{1\}^p $), the attributions $\phi_j$ and the null output $\phi_0$ sum up to the original model output $f(\xv)$\\\medskip
Local accuracy corresponds to the \textbf{axiom of efficiency} in Shapley game theory 

\end{onlyenv}

\begin{onlyenv}<2->
\textbf{Missingness}
$$
x_{j}^{\prime}=0 \Longrightarrow \phi_{j}=0
$$
\end{onlyenv}

\begin{onlyenv}<2>
\textbf{Intution:}  A missing feature gets an attribution of zero
\end{onlyenv}

\begin{onlyenv}<3->
\textbf{Consistency} \\
\end{onlyenv}
\begin{onlyenv}<3>
$\fh_{x}\left(\mathbf{z}^{\prime (k)}\right)=\fh\left(h_{x}\left(\mathbf{z}^{\prime (k)}\right)\right) \text { and } \mathbf{z}^{\prime (k)}_{-j} \text{ denote setting } z_{j}^{\prime (k)}=0$ . For any two
models $\fh$ and $\fh^{\prime}$, if
$$
\fh_{x}^{\prime}\left(\mathbf{z}^{\prime (k)}\right)-\fh_{x}^{\prime}\left(\mathbf{z}^{\prime (k)}_{-j}\right) \geq \fh_{x}\left(\mathbf{z}^{\prime (k)}\right)-\fh_{x}\left(\mathbf{z}^{\prime (k)}_{-j}\right)
$$
for all inputs $\mathbf{z}^{\prime (k)} \in \{0, 1\}^p$, then
$$
\phi_{j}\left(\fh^{\prime}, \mathbf{x}\right) \geq \phi_{j}(\fh, \mathbf{x})
$$
\end{onlyenv}

\begin{onlyenv}<4->
$$
\fh_{x}^{\prime}\left(\mathbf{z}^{\prime (k)}\right)-\fh_{x}^{\prime}\left(\mathbf{z}^{\prime (k)}_{-j}\right) \geq \fh_{x}\left(\mathbf{z}^{\prime (k)}\right)-\fh_{x}\left(\mathbf{z}^{\prime (k)} _{-j}\right) \Longrightarrow \phi_{j}\left(\fh^{\prime}, \xv\right) \geq \phi_{j}(\fh, \xv)
$$

\textbf{Intution:} If a model changes so that the marginal contribution of a feature value increases or stays the same, the Shapley value also increases or stays the same\\\medskip 
From \textbf{consistency} the Shapley \textbf{axioms of additivity, dummy and symmetry} follow
\end{onlyenv}


\end{frame}

% TODO
\renewcommand{\titlefigure}{../../slides/04_shapley/figure_man/global_shap_depend_season.pdf}
\renewcommand{\learninggoals}{
\item Get an intuition of additive feature attributions
\item Understand the concept of Kernel SHAP
\item Ability to interpret SHAP plots
\item Global SHAP methods
}

\lecturechapter{Global SHAP}
\mysectionslide
% \begin{frame}{Global SHAP}
%     \begin{columns}
%     \begin{column}{0.6\textwidth}
%     \vspace{1cm}
%     \begin{center}
%         \includegraphics[]{../../slides/04_shapley/figure_man/global_shap_depend_season.pdf}
%     \end{center}
%     \end{column}
%     \begin{column}{0.4\textwidth}
%      \vspace*{1.5cm}
     
%     \textbf{Learning goals}
%     \begin{itemize}
%         \item Get an intuition of additive feature attributions
% \item Understand the concept of Kernel SHAP
% \item Ability to interpret SHAP plots
% \item Global SHAP methods
%     \end{itemize}
%     \end{column}
%     \end{columns}
% \end{frame}

\begin{frame}{Global SHAP \citebutton{Lundberg et al. 2018}{https://doi.org/10.48550/arXiv.1802.03888}}
\textbf{Idea: }
\begin{itemize}
    \item Run SHAP for every observation and thereby get a matrix of Shapley values
    \item The matrix has one row per data observation and one column per feature
    \item We can interpret the model globally by analyzing the Shapley values in this matrix
\end{itemize}
\vspace{2cm}
$$
\Phi =
\begin{bmatrix}
    \phi_{11} & \phi_{12} & \phi_{13} & \dots  & \phi_{1p} \\
    \phi_{21} & \phi_{22} & \phi_{23} & \dots  & \phi_{2p} \\
    \vdots & \vdots & \vdots & \ddots & \vdots \\
    \phi_{n1} & \phi_{n2} & \phi_{n3} & \dots  & \phi_{np} \\
\end{bmatrix}
$$

 \end{frame}

 \begin{frame}{Feature Importance}

\begin{onlyenv}<1>
\textbf{Idea:} Average the absolute Shapley values of each feature over all observations. This corresponds to calculating averages column by column in $\Phi$
$$
I_{j}=\frac{1}{n} \sum_{i=1}^{n}\left|\phi_{j}^{(i)}\right|
$$
\end{onlyenv}

\begin{onlyenv}<2>
\textbf{Interpretation:}
\begin{itemize}
    \item The features temperature and year have by far the highest influence on the model's prediction
    \item Compared to Shapley values, no effect direction is provided, but instead a feature ranking similar to PFI
    \item However, Shapley FI is based on the model's predictions only while PFI is based on the model's performance (loss)
\end{itemize}
 
\end{onlyenv}




\begin{figure}
    \centering
    \includegraphics[width=0.5\columnwidth]{../../slides/04_shapley/figure_man/global_shap_fi.pdf}
\end{figure}

\end{frame}
 
\begin{frame}{Summary Plot}
\begin{onlyenv}<1>
Combines feature importance with feature effects
\begin{itemize}
    \item Each point is a Shapley value for a feature and an observation
    \item The color represents the value of the feature from low to high
    \item Overlapping points are jittered in y-axis direction
\end{itemize}
\end{onlyenv}

\begin{onlyenv}<2>
\textbf{Interpretation:}\\
\begin{itemize}
    \item Low temperatures have a negative impact while high temperatures lead to more bike rentals
    \item Year: two point clouds for 2011 and 2012 (other categorical features are gray)
    \item A high humidity has a huge, negative impact on the bike rental, while low humidity has a rather minor positive impact on bike rentals
\end{itemize}
 
\end{onlyenv}



\begin{figure}
    \centering
    \includegraphics[width=0.5\columnwidth]{../../slides/04_shapley/figure_man/global_shap_jitter.pdf}
    
\end{figure}
\end{frame} 

\begin{frame}{Dependence Plot}

\begin{onlyenv}<1>
\begin{itemize}
    \item Visualize the marginal contribution of a feature similar to the PDP 
    \item Plot a point with the feature value on the x-axis and the corresponding Shapley value on the y-axis
\end{itemize}

\end{onlyenv}

\begin{onlyenv}<2>
\textbf{Interpretation:}\\
\begin{itemize}
    \item Increasing temperatures induce increasing bike rentals until $25^\circ\text{C}$
    \item If it gets too hot, the bike rentals decrease
\end{itemize}
\end{onlyenv}

\begin{onlyenv}<3>
\textbf{Interpretation:}\\
\begin{itemize}
    \item We can colour the observations by a second feature to detect interactions
    \item Visibly the temperatures interaction with the season is very strong
\end{itemize}
\end{onlyenv}



\begin{onlyenv}<1-2>
\begin{figure}
    \centering
    \includegraphics[width=0.5\columnwidth]{../../slides/04_shapley/figure_man/global_shap_depend.pdf}
\end{figure}
\end{onlyenv}

\begin{onlyenv}<3>
\begin{figure}
    \centering
    \includegraphics[width=0.5\columnwidth]{../../slides/04_shapley/figure_man/global_shap_depend_season.pdf}
\end{figure}
\end{onlyenv}
\end{frame}

\begin{frame}{Discussion}

\textbf{Advantages}

\begin{itemize}
    \item All the advantages of Shapley values
    \item Unify the field of interpretable machine learning in the class of additive feature attribution methods
    \item Has a fast implementation for tree-based models
    \item Various global interpretation methods
\end{itemize}

\medskip

\textbf{Disadvantages}

\begin{itemize}
\item Disadvantages of Shapley values also apply to SHAP
    % \begin{itemize}
    %     \item Shapley values can be misinterpreted and access to data is needed to compute them for new data (not if TreeSHAP is used)
    % \end{itemize}
    \item KernelSHAP is slow (TreeSHAP can be used as a faster alternative for tree-based models \citebutton{Lundberg et al 2018}{https://doi.org/10.48550/arXiv.1802.03888} -- and for an intuitive explanation  \citebutton{see Sukumar: TreeSHAP}{https://medium.com/analytics-vidhya/shap-part-3-tree-shap-3af9bcd7cd9b})
    \item KernelSHAP ignores feature dependence
    %\item TreeSHAP can produce unintuitive feature attributions
    %\item It is possible to create intentionally misleading interpretations with SHAP, which can hide biases \citebutton{Slack et al. 2020}{https://doi.org/10.1145/3375627.3375830}
\end{itemize}


\end{frame}

\endlecture

\end{document}
