\loesung{
\begin{enumerate}[(a)]

%-----------------------------------------------------------------
\item \textbf{Partial-dependence function}  
      For a univariate model
      \(
        \hat f_{\mathrm{PD}}(x)=\hat f(x)
      \)
      (because the expectation over \(X_{-j}\) is vacuous).
      Evaluating the piece-wise constant tree:

      \[
      \boxed{
      \begin{array}{c|cccc}
      x & 1.0 & 2.5 & 3.5 & 5.0 \\ \hline
      \hat f_{\mathrm{PD}}(x) & 2.0 & 2.0 & 5.0 & 8.0
      \end{array}}
      \]

%-----------------------------------------------------------------
\item \textbf{Accumulated local effects (ALE)}

      \vspace{2pt}
      We follow the slide estimator  
      \(
        \widetilde f_{S,\mathrm{ALE}}(x)
        =\sum_{l=1}^{k_S(x)}
          \frac{1}{n_S(l)}
          \sum_{i:x^{(i)}\in I_l}
          \bigl[\hat f(z_l)-\hat f(z_{l-1})\bigr]
      \)
      and then centre by the sample mean.

      \begin{enumerate}[1.]
      \item \emph{Average local effect in each bin}  
            (finite differences already span the interval, so no width factor):

            \[
            \begin{array}{c|c|c|c|c}
            k & I_k=[z_{k-1},z_k] & \hat f(z_{k-1}) & \hat f(z_k) &
            \Delta_k:=\tfrac1{n_S(k)}\!\sum\bigl[\hat f(z_k)-\hat f(z_{k-1})\bigr]\\ \hline
            1 & [1.0,2.5) & 2.0 & 2.0 & 0.0\\
            2 & [2.5,3.5) & 2.0 & 5.0 & 3.0\\
            3 & [3.5,5.0] & 5.0 & 8.0 & 3.0
            \end{array}
            \]

      \item \emph{Un-centred ALE}  
            For \(x\in I_{k_S(x)}\)
            \(
               \widetilde f_{\mathrm{ALE}}(x)
               =\sum_{l=1}^{k_S(x)}\Delta_l
            \).

            \[
            \boxed{
            \begin{array}{c|cccc}
            x & 1.0 & 2.5 & 3.5 & 5.0 \\ \hline
            \widetilde f_{\mathrm{ALE}}(x) & 0 & 0 & 3 & 6
            \end{array}}
            \]

      \item \emph{Centring}  
            Evaluate \(\widetilde f_{\mathrm{ALE}}\) at the five observations:

            \[
              \bigl(0,\;0,\;3,\;6,\;6\bigr)
              \quad\Longrightarrow\quad
              \bar{\widetilde f}=\tfrac{0+0+3+6+6}{5}=3.
            \]

            \[
              f_{\mathrm{ALE}}(x)=\widetilde f_{\mathrm{ALE}}(x)-3.
            \]

            \[
            \boxed{
            \begin{array}{c|cccc}
            x & 1.0 & 2.5 & 3.5 & 5.0 \\ \hline
            f_{\mathrm{ALE}}(x) & -3 & -3 & 0 & 3
            \end{array}}
            \]
      \end{enumerate}

%-----------------------------------------------------------------
\item \textbf{Shape comparison}  
      Compute the pointwise difference:

      \[
      \hat f_{\mathrm{PD}}(x)-f_{\mathrm{ALE}}(x)
      \;=\;
      \bigl[2-(-3),\;2-(-3),\;5-0,\;8-3\bigr]
      \;=\;
      \bigl[5,\,5,\,5,\,5\bigr].
      \]

      Hence a constant vertical shift exists:
      \[
        \boxed{%
        \hat f_{\mathrm{PD}}(x)=f_{\mathrm{ALE}}(x)+C},
        \qquad C=5.
      \]

      \textit{Reason:} For a step-function model the
      un-centred ALE is the cumulative sum of inter-bin jumps,
      while the PDP is that sum plus the leaf value of the first bin
      (\(\theta_0=2\)).  After centring
      (\(f_{\mathrm{ALE}}=\widetilde f_{\mathrm{ALE}}-\bar{\widetilde f}\))
      the residual \(\theta_0+\bar{\widetilde f}=5\) is independent of \(x\),
      so both curves share the same shape.
\end{enumerate}
}