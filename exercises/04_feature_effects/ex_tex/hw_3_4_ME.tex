\aufgabe{Derivative, Forward, and Non-Linearity Effects}{

Let us consider the following non-linear regression model:

\begin{equation}
  \hat{f}(x_1, x_2) = \theta_0 + \theta_1 x_1^2 + \theta_2 x_2^2 + \theta_{1,2} x_1 x_2
  \label{eq:nonlinmod}
\end{equation}

with parameters \( \theta_0 = 0 \), \( \theta_1 = 1 \), \( \theta_2 = 0.5 \), and \( \theta_{1,2} = 2 \).

\begin{enumerate}[a)]
  \item Derive the analytic expression for the \emph{derivative marginal effect} (dME) of feature $x_1$.
  % , i.e.\ compute
  % \[
  %   \text{dME}_1(x_1, x_2) = \frac{\partial \hat{f}(x_1, x_2)}{\partial x_1}.
  % \]

  \item Derive the expression for the \emph{forward marginal effect} (fME) of $x_1$ with step size $h_1 > 0$.
  % , i.e.\
  % \[
  %   \text{fME}_1(x_1, x_2; h_1) = \hat{f}(x_1 + h_1, x_2) - \hat{f}(x_1, x_2).
  % \]

  \item Let \( x_1 = 1 \), \( x_2 = 2 \), and \( h_1 = 1 \). Compute \( \text{dME}_1(1, 2) \) and \( \text{fME}_1(1, 2; h_1 = 1) \) and explain why these two results differ and interpret the additional terms appearing in the fME.
% 
%   \item The \emph{Non-Linearity Measure (NLM)} below quantifies how well the local linear approximation (secant) explains the variation in the model output along the direction of $h_1$. It is defined as:
%   \[
%     \text{NLM} = 1 - \frac{\sum_{i=1}^T \left[\hat{f}(\xv^{(i)}) - g(t_i)\right]^2}{\sum_{i=1}^T \left[\hat{f}(\xv^{(i)}) - \bar{f}\right]^2},
%   \]
%   where:
%   \begin{align*}
%     \xv^{(i)} &= (x_1 + t_i h_1, x_2), \quad \text{with } t_i = \frac{i-1}{T-1}, \quad i = 1, \ldots, T \\
%     g(t_i) &= \hat{f}(x_1, x_2) + t_i \cdot \text{fME}_1(x_1, x_2; h_1) \\
%     \bar{f} &= \frac{1}{T} \sum_{i=1}^T \hat{f}(\xv^{(i)})
%   \end{align*}
%   Compute the NLM for \( x_1 = 1 \), \( x_2 = 2 \), \( h_1 = 0.5 \), and \( T = 10 \). Interpret your result in relation to the fit of a local linear approximation.
  \item \textbf{Non-Linearity Measure (NLM):} We want test how well the secant (local linear approximation) fits the model output along the direction of $x_1$. The NLM defined below quantifies how well the local linear approximation (secant) explains the variation in the model output along the direction of $h_1$. For $T$ grid points $(\mathbf x^{(i)})_{i=1}^{T}$, it is defined as:
\[
  \mathrm{NLM}
  = 1 -
    \frac{\displaystyle
          \underbrace{\sum_{i=1}^{T}
            \bigl[\hat f(\mathbf x^{(i)})-g(t_i)\bigr]^2}_{\text{SSR: error of secant}}}
         {\displaystyle
          \underbrace{\sum_{i=1}^{T}
            \bigl[\hat f(\mathbf x^{(i)})-\bar f\bigr]^2}_{\text{SST: total variance}}}.
\]
\begin{itemize}\setlength\itemsep{0.2em}
  \item $\displaystyle t_i \in (0,1)$ param\-e\-tri\-ses the path (here only in $x_1$ direction):
        \(\mathbf x^{(i)} = (x_1 + t_i h_1,\,x_2)\).
  \item $\displaystyle g(t_i)
        = \hat f(x_1,x_2)
        + t_i\cdot\text{fME}_1(x_1,x_2;h_1)$  
        is the secant (straight line) through the two end-points.
  \item $\displaystyle\bar f
        = \frac1T \sum_{i=1}^{T}\hat f(\mathbf x^{(i)})$
        is the mean prediction on the path.
  \item Hence NLM is an $R^{2}$: $1$ means perfect linear fit,
        values $\ll 1$ reveal curvature.
\end{itemize}

%--------------------------------------------------------------
Compute the NLM for \( x_1 = 1 \), \( x_2 = 2 \), and stepsize \( h_1 = 0.5 \). Interpret your result in relation to the fit of a local linear approximation using the $T = 9$ equidistant points below $t_i = 0.1,0.2,\dots,0.9$: %and $(\theta_0,\theta_1,\theta_2,\theta_{1,2})=(0,1,0.5,2)$.

\begin{center}
\renewcommand{\arraystretch}{1.15}
\begin{tabular}{c|c|c|c|c}
  $i$ & $t_i$ & $x_1^{(i)}=1+0.5t_i$ & $\hat f(\mathbf x^{(i)})$ & $g(t_i)$ \\ \hline %=7+3.25t_i$
  1 & 0.1 & 1.05 & 7.3025 & 7.3250 \\
  2 & 0.2 & 1.10 & 7.6100 & 7.6500 \\
  3 & 0.3 & 1.15 & 7.9225 & 7.9750 \\
  4 & 0.4 & 1.20 & 8.2400 & 8.3000 \\
  5 & 0.5 & 1.25 & 8.5625 & 8.6250 \\
  6 & 0.6 & 1.30 & 8.8900 & 8.9500 \\
  7 & 0.7 & 1.35 & 9.2225 & 9.2750 \\
  8 & 0.8 & 1.40 & 9.5600 & 9.6000 \\
  9 & 0.9 & 1.45 & 9.9025 & 9.9250
\end{tabular}
\end{center}


  \item \textbf{Implementation:} Write three functions in either R or Python:
  \begin{itemize}
    \item A function \texttt{dME(f, x, j)} that computes a central difference approximation of the derivative marginal effect for feature $x_j$ at input $x$.
    \item A function \texttt{fME(f, x, h)} that computes the forward marginal effect using a vector of stepsizes $h$.
    \item A function \texttt{NLM(f, x, h, t\_vals)} that computes the Non-Linearity Measure as described in d) where \texttt{t\_vals} are the grid points used to estimate the NLM.
  \end{itemize}
  Use them to confirm the results from sub-tasks c) and d).
\end{enumerate}
}
