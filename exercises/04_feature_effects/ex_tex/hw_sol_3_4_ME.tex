% \loesung{
% 
% \textbf{Solution: Derivative, Forward, and Non-Linearity Effects}
% 
% \begin{enumerate}[a)]
%   \item \textbf{Derivative Marginal Effect (dME):}
% 
%   Given:
%   \[
%     \hat{f}(x_1, x_2) = \theta_0 + \theta_1 x_1^2 + \theta_2 x_2^2 + \theta_{1,2} x_1 x_2
%   \]
%   The derivative with respect to \( x_1 \) is:
%   \[
%     \text{dME}_1(x_1, x_2) = \frac{\partial \hat{f}}{\partial x_1} = 2 \theta_1 x_1 + \theta_{1,2} x_2
%   \]
% 
%   \item \textbf{Forward Marginal Effect (fME):}
% 
%   We compute:
%   \[
%     \text{fME}_1(x_1, x_2; h_1) = \hat{f}(x_1 + h_1, x_2) - \hat{f}(x_1, x_2)
%   \]
%   Plug in the model:
%   \begin{align*}
%     \hat{f}(x_1 + h_1, x_2) &= \theta_1 (x_1 + h_1)^2 + \theta_2 x_2^2 + \theta_{1,2} (x_1 + h_1) x_2 \\
%     &= \theta_1 (x_1^2 + 2 x_1 h_1 + h_1^2) + \theta_2 x_2^2 + \theta_{1,2} x_1 x_2 + \theta_{1,2} h_1 x_2
%   \end{align*}
%   Subtracting \(\hat{f}(x_1, x_2)\), we obtain:
%   \[
%     \text{fME}_1(x_1, x_2; h_1) = 2 \theta_1 x_1 h_1 + \theta_1 h_1^2 + \theta_{1,2} x_2 h_1
%   \]
% 
%   \item \textbf{Numerical Evaluation:}
% 
%   Given \( x_1 = 1 \), \( x_2 = 2 \), \( h_1 = 0.5 \), and parameters:
%   \(\theta_1 = 1\), \(\theta_{1,2} = 2\)
% 
%   \begin{align*}
%     \text{dME}_1(1, 2) &= 2 \cdot 1 \cdot 1 + 2 \cdot 2 = 2 + 4 = 6 \\
%     \text{fME}_1(1, 2; 0.5) &= 2 \cdot 1 \cdot 1 \cdot 0.5 + 1 \cdot 0.5^2 + 2 \cdot 2 \cdot 0.5 \\
%                            &= 1 + 0.25 + 2 = 3.25
%   \end{align*}
% 
%   \textbf{Interpretation:} The dME overestimates the effect, as it assumes linearity. The fME reflects the actual finite difference. The term \( \theta_1 h_1^2 \) captures curvature in the prediction function.
% 
%   \item \textbf{Non-Linearity Measure (NLM):}
% 
%   Define the 10 evaluation points:
%   \[
%     t_i = \frac{i-1}{9}, \quad x_1^{(i)} = 1 + t_i \cdot 0.5, \quad x_2^{(i)} = 2
%   \]
%   Compute:
%   \begin{itemize}
%     \item \( f_i = \hat{f}(x_1^{(i)}, x_2) \)
%     \item \( \bar{f} = \frac{1}{10} \sum f_i \)
%     \item Secant line: \( g_i = \hat{f}(1, 2) + t_i \cdot \text{fME}_1(1, 2; 0.5) = 6 + 3.25 t_i \)
%   \end{itemize}
%   Then evaluate:
%   \[
%     \text{NLM} = 1 - \frac{\sum_{i=1}^{10} (f_i - g_i)^2}{\sum_{i=1}^{10} (f_i - \bar{f})^2}
%   \]
%   This gives an estimate of how well the linear approximation explains variation along the path. For this example, the NLM is close to—but below—1, indicating mild non-linearity.
% 
% \end{enumerate}
% }

%------------------------------------------------------------
\loesung
{

\vspace{0.5em}
The fitted model is
\[
  \hat{f}(x_1,x_2)=\theta_0+\theta_1 x_1^{2}+\theta_2 x_2^{2}+ \theta_{1,2}x_1x_2,
  \quad
  (\theta_0,\theta_1,\theta_2,\theta_{1,2})=(0,\,1,\,0.5,\,2).
\]

\begin{enumerate}[a)]
%-------------------------------------------------------------------
\item \textbf{Derivative marginal effect (dME)}\\[0.1em]
\[
  \text{dME}_1(x_1,x_2)=
  \frac{\partial\hat f}{\partial x_1}
  =2\theta_1x_1+\theta_{1,2}x_2
  =2x_1+2x_2.
\]

\emph{Discussion}.  
Because the model is quadratic in \(x_1\), the derivative is a \emph{linear}
function of both \(x_1\) and \(x_2\).  
If the surface were exactly linear, this rate would coincide with any finite
difference.

%-------------------------------------------------------------------
\item \textbf{Forward marginal effect (fME)} with step \(h_1>0\)\\[0.1em]

\[
  \text{fME}_1(x_1,x_2;h_1) \;=\;
  \hat f(x_1+h_1,x_2)-\hat f(x_1,x_2).
\]

Expand \(\hat f(x_1+h_1,x_2)\):

\begin{align*}
  \hat f(x_1\!+\!h_1,x_2)
  &=\theta_1(x_1+h_1)^2+\theta_2x_2^2+\theta_{1,2}(x_1+h_1)x_2 \\
  &=\theta_1\bigl[x_1^2+2x_1h_1+h_1^2\bigr]
    \;+\;\theta_2x_2^2
    \;+\;\theta_{1,2}x_1x_2+\theta_{1,2}h_1x_2.
\end{align*}

Subtracting \(\hat f(x_1,x_2)\) gives

\[
  \boxed{\;
  \text{fME}_1(x_1,x_2;h_1)=
  2\theta_1x_1h_1+\theta_1h_1^{2}+\theta_{1,2}x_2h_1
  \;}
  \;=\;
  2x_1h_1+h_1^{2}+2x_2h_1 .
\]

\emph{Discussion}.  
The extra term \(\theta_1h_1^{2}\) captures curvature; it vanishes faster than linearly when \(h_1\to0\), at which point fME converges to dME.

%-------------------------------------------------------------------
% \item \textbf{Numerical evaluation at} \((x_1,x_2,h_1)=(1,2,0.5)\)

% \[
% \begin{aligned}
%   \text{dME}_1(1,2) &= 2(1)+2(2)=6,\\[0.2em]
%   \text{fME}_1(1,2;0.5) &= 2(1)(0.5)+(0.5)^2+2(2)(0.5)=1+0.25+2=3.25.
% \end{aligned}
% \]

% \emph{Why different?}  
% The derivative \(6\) is the \emph{instantaneous} slope.  
% The finite step of \(0.5\) exposes curvature; the quadratic term makes the
% actual change only \(3.25\).  
% Hence, using dME for a sizeable perturbation would greatly
% over-predict the effect.

%-------------------------------------------------------------------
\item \textbf{Numerical evaluation at} \((x_1,x_2,h_1)=(1,2,1)\)

\[
\begin{aligned}
  \text{dME}_1(1,2) &= 2(1)+2(2)=6,\\[0.2em]
  \text{fME}_1(1,2;1) &= 2 \cdot 1 \cdot 1 + 1^2 + 2 \cdot 2 \cdot 1 =2+1+4=7.
\end{aligned}
\]

\emph{Why different?}  
The derivative \(6\) is the \emph{instantaneous} slope.  
The finite step of \(1\) exposes curvature; the quadratic term makes the actual change go up to \(7\).  
Hence, using dME for a sizeable perturbation would misestimate (over- or under-predict) the effect.

%-------------------------------------------------------------------
\item \textbf{Non-Linearity Measure (NLM)}  
(\(R^2\) of the linear secant along the path)

\textit{Hints.}
\begin{itemize}\setlength\itemsep{0.2em}
  \item Compute $\text{SSR}=\sum(\hat f - g)^2$ and
        $\text{SST}=\sum(\hat f -\bar f)^2$ from the table.
  \item Then evaluate $\mathrm{NLM}=1-\mathrm{SSR}/\mathrm{SST}$.
  \item Comment: Is the resulting NLM close enough to $1$ to accept the
        secant as a faithful local explanation?
\end{itemize}

\paragraph{Step 1: Path points.}  
For \(T=10\) equidistant \(t_i\in[0,1]\):
\[
  t_i=\frac{i}{9+1}, \quad
  x_1^{(i)} = 1 + t_i h_1 = 1+0.5\,t_i, \quad
  x_2^{(i)} = 2.
\]

\paragraph{Step 2: Model and secant values.}
\[
  f_i = \hat f(x_1^{(i)},2), 
  \quad
  g_i = \hat f(1,2) + t_i\cdot \text{fME}_1(1,2;0.5),
  \quad
  \hat f(1,2)=7,\;
  \text{fME}=3.25.
\]

\paragraph{Step 3: Compute \(R^2\).}
\[
  \text{NLM}=1-\frac{\sum_{i=1}^{9}(f_i-g_i)^2}%
                     {\sum_{i=1}^{9}(f_i-\bar f)^2},
  \quad
  \bar f=\frac1{9}\sum f_i.
\]

Numerically  
\[
  \boxed{\text{NLM}\;\approx\;0.9983}.
\]

\emph{Interpretation}.  
Along the half-unit move in \(x_1\), the quadratic surface is
\emph{almost} linear (NLM very close to 1).  
If we doubled \(h_1\), the numerator would grow faster than the denominator,
lowering NLM and signalling stronger curvature.

\end{enumerate}

}
%------------------------------------------------------------
