\textbf{Solution Quiz:}\\\noindent
\medskip

\begin{enumerate}
	\item What is the motivating idea behind Shapley values? 
	\begin{itemize}
		\item[$\Rightarrow$] Fairly divide the total achievable payout $v(P)$ among the players in a cooperative setting according to each player’s individual contribution.
	\end{itemize}
	\item How is this idea realized / implemented in IML?
	\begin{itemize}
		\item[$\Rightarrow$]
        game = prediction through the model for a single observation, \\
        players = features, \\
        Shapley value = individual payout per player (feature), so individual contribution of a feature to the predicted value, \\
        Value function / Payout function = PD function for this coalition evaluated at this observation
		\item[$\Rightarrow$] Calculate marginal contribution of a feature by adding the feature to every possible coalition and correctly average the resulting values.
	\end{itemize}
	\item What is a practical problem of Shapley values and how is it solved?
	\begin{itemize}
		\item[$\Rightarrow$] Computation for high dimensional feature spaces: Even for moderately large feature sets $P$, the power set $2^P$ is very large, so many marginal predictions are necessary.
		\item[$\Rightarrow$] Solution: Sampling. The more coalitions are sampled, the more exact, but also the more computationally expensive the computation will be.
	\end{itemize}
    \item Are Shapley values in ML concerned with the prediction or the prediction target? How do they relate to the model and how to the DGP?
	\begin{itemize}
		\item[$\Rightarrow$] Shapley values are concerned with the prediction $\yh = \fh(\xv)$, not the true target $y$.
		\item[$\Rightarrow$] This means they quantify contributions and distribute them fairly w.r.t. the model, but they never look at the underlying true target, hence they are only concerned with explaining the model, but not the underlying DGP.
	\end{itemize}
    \item When not using any approximations, but computing the exact Shapley value, which definition is more useful in practice?
	\begin{itemize}
		\item[$\Rightarrow$] Order definition contains $|P|!$ many terms, whereas set definition contains only $2^P$ many terms, hence the set definition is faster to compute.
        In the set definition, all terms in the order definition that are equal are already summed up and counted.
	\end{itemize}
\end{enumerate}
