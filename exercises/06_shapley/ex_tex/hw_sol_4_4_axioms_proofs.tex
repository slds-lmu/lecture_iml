\loesung{
\label{ex_sol:Shapley_axioms_proofs}

\newtheorem{theorem}{Theorem}
% \newtheorem{proof}{Proof}
\newtheorem{proofsketch}{Proof (Step-by-Step)}



\textbf{Setup.} 
Let $P = \{1,\dots,p\}$ be the set of $p := |P|$ players, $v$ the value function and $\phi_j$ the Shapley value of player $j$.

Recall the two equivalent definitions of Shapley values:
\paragraph{Permutation-Based Definition of Shapley Values.}
    Let $\mathrm{Pred}_\pi(j)$ be the set of players that appear before player $j$ in the permutation $\pi$.  
    Then for any $j$, the Shapley value of player $j$ is
    \[
    \phi_j 
    \;=\;
    \frac{1}{p!}\;
    \sum_{\pi \in \mathfrak{S}_P}
    \Bigl( v\bigl( \mathrm{Pred}_\pi(j) \cup \{\,j\} \bigr)
    \;-\;
    v\bigl( \mathrm{Pred}_\pi(j) \bigr) \Bigr),
    \]
    where $\mathfrak{S}_P$ is the set of all $p!$ permutations of $P$.  
    In words, we look at the ``marginal contribution'' of $j$ each time it arrives in a permutation (depending on whichever players arrived before it), and then average over all permutations.

\paragraph{Set-Based Definition of Shapley Values.}
    For any $j$, we consider all subsets $S$ of $P$ which do not contain $j$, then the Shapley value of player $j$ is
    \[
    \phi_j 
    \;=\;
    \sum_{S \subseteq P \setminus \{j\}}
    \frac{|S|!\,\bigl(p - |S| - 1\bigr)!}{p!}
    \,
    \Bigl( v(S \cup \{j\}) - v(S) \Bigr).
    \]
    
In essence, both definitions are almost the same, just that those terms from the first definition that yield the same subsets $S$ (although from different permutations), are summed up and counted in the second definition, so that the second definition has less terms in the sum.



\begin{enumerate}

    \item \label{ex_sol:Shapley_axioms_proofs_a} \textbf{Proof of the first 3 Axioms:}
    The Dummy and Additivity properties are relatively easy to proof.
    
    \begin{theorem}[Null Player (Dummy)]
    Let $\{\phi_j\}_{j \in P}$ be the Shapley values induced by $v$.
    For any player $j \in P$  it holds:
    $$
    \text{If } v(\Scupj) = v(S) \text{ for all } \SsubPnoj, \text{ then } \phi_j = 0.
    $$
    \end{theorem}

    \begin{proof}

        Assume $p \geq 1$, $ j \in P$ and $v(\Scupj) = v(S)$ for all subsets $\SsubPnoj$. We need to prove that $\phi_j = 0$.
        
        This follows directly from either of the two definitions: The assumption states that the difference $v(\Scupj) - v(S)$ inside the sum is 0, for every $\SsubPnoj$, in other words for every summand.
        Therefore, also the whole sum is 0, so $\phi_j = 0$.
        
    \end{proof}

    \begin{theorem}[Additivity]
    Let $v, v_1, v_2 : 2^P \to \R$ be value functions with $v = v_1 + v_2$, which means that for every subset $\SsubP$ we have $v(S) = v_1(S) + v_2(S)$.
    Let $\{\phi_{j, v}\}_{j \in P}$ be the Shapley values induced by $v$ and $\{\phi_{j, v_1}\}_{j \in P}$ and $\{\phi_{j, v_2}\}_{j \in P}$ those for $v_1$ and $v_2$ respectively.
    Then for any player $j \in P$ it holds:
    $$
    \phi_{j, v} = \phi_{j, v_1+v_2} = \phi_{j, v_1} + \phi_{j, v_2}.
    $$
    \end{theorem}

    \begin{proof}

        Assume $p \geq 1$, $ j \in P$ and that $v_1, v_2$ are two value functions. We need to prove that $\phi_{j, v_1+v_2} = \phi_{j, v_1} + \phi_{j, v_2}$.
        
        For the marginal contributions of player $j$ on any subset $\SsubP$ we have: $v(\Scupj) - v(S) = v_1(\Scupj) - v_1(S) + v_2(\Scupj) - v_2(S)$.
        Plugging this into either of the two definitions yields the result:
        \begin{align*}
            &\phi_{j, v} \,=\, \phi_{j, v_1+v_2}
            \,=\, \sum_{S \subseteq P \setminus \{j\}} \frac{|S|!\,\bigl(p - |S| - 1\bigr)!}{p!} \, \Bigl( v(S \cup \{j\}) - v(S) \Bigr)  \\
            &\,=\, \sum_{S \subseteq P \setminus \{j, k\}} \frac{|S|!\,\bigl(p - |S| - 1\bigr)!}{p!} \, \Bigl( v_1(\Scupj) - v_1(S) + v_2(\Scupj) - v_2(S) \Bigr) \\
            &\,=\, \sum_{S \subseteq P \setminus \{j, k\}} \frac{|S|!\,\bigl(p - |S| - 1\bigr)!}{p!} \, \Bigl( v_1(\Scupj) - v_1(S) \Bigr) 
            \,+\, \sum_{S \subseteq P \setminus \{j, k\}} \frac{|S|!\,\bigl(p - |S| - 1\bigr)!}{p!} \, \Bigl( v_2(\Scupj) - v_2(S) \Bigr) \\
            &\,=\, \phi_{j, v_1} + \phi_{j, v_2}.
        \end{align*}
        
    \end{proof}

    The symmetry axiom requires a little more calculation.

    \begin{theorem}[Symmetry]
    Let $\{\phi_j\}_{j \in P}$ be the Shapley values induced by $v$.
    For any two players $j,k \in P$  it holds:
    $$
    \text{If } v(\Scupj) = v(\Scupk) \text{ for all } \SsubPnojk, \text{ then } \phi_j = \phi_k.
    $$
    \end{theorem}

    \begin{proof}

        Assume $p \geq 2$ and $ j,k \in P$ and $v(\Scupj) = v(\Scupk)$ for all subsets $\SsubPnojk$. We need to prove that $\phi_j = \phi_k$.
        
        We will use the set-based definition of Shapley values, and will show the claim by straightforward calculation and rearranging the sum.
        We first split the sum, which is over all sets $\SsubPnoj$, into two parts, depending on whether player $k$ is in $S$ or not:
        \begin{align*}
            \phi_j
            & \,=\, \sum_{S \subseteq P \setminus \{j\}} \frac{|S|!\,\bigl(p - |S| - 1\bigr)!}{p!} \, \Bigl( v(S \cup \{j\}) - v(S) \Bigr) \\
            & \,=\, \sum_{S \subseteq P \setminus \{j, k\}} \frac{|S|!\,\bigl(p - |S| - 1\bigr)!}{p!} \, \Bigl( v(S \cup \{j\}) - v(S) \Bigr) 
            \,+\, \sum_{\substack{S \subseteq P \setminus \{j\}, \\ k \in S}} \frac{|S|!\,\bigl(p - |S| - 1\bigr)!}{p!} \, \Bigl( v(S \cup \{j\}) - v(S) \Bigr).
        \end{align*}
        Now, in the first part of the sum we can directly use our assumption:
        $$
        \sum_{S \subseteq P \setminus \{j, k\}} \frac{|S|!\,\bigl(p - |S| - 1\bigr)!}{p!} \, \Bigl( v(S \cup \{j\}) - v(S) \Bigr)
        \,=\, \sum_{S \subseteq P \setminus \{j, k\}} \frac{|S|!\,\bigl(p - |S| - 1\bigr)!}{p!} \, \Bigl( v(S \cup \{k\}) - v(S) \Bigr)
        $$

        For the second part, first note that both sums have the same number of summands, since there is a one-to-one correspondence (a bijection) between the subsets of $P \setminus \{j, k\}$ and those subsets of $P \setminus \{j\}$ that contain $k$, by either adding $k$ to a given subset $\SsubPnojk$ or not.
        We can use this bijection to define an ``index transformation'' for the second sum: We replace the set $S$, which we know must contain $k$, with $S = \Tilde{S} \cup \{k\}$, in other words we define $\Tilde{S} := S \setminus \{k\}$.
        We can then use our assumption on the set $\Tilde{S}$ as well.
        This yields:
        \begin{align*}
            & \sum_{\substack{S \subseteq P \setminus \{j\}, \\ k \in S}} \frac{|S|!\,\bigl(p - |S| - 1\bigr)!}{p!} \, \Bigl( v(S \cup \{j\}) - v(S) \Bigr)  \\
            \,=\, & \sum_{\Tilde{S} \subseteq P \setminus \{j, k\}} \frac{(|\Tilde{S}| + 1)!\,\bigl(p - (|\Tilde{S}| + 1) - 1\bigr)!}{p!} \, \Bigl( v(\Tilde{S} \cup \{j, k\}) - v(\Tilde{S} \cup \{k\}) \Bigr) \\
            \,=\, & \sum_{\Tilde{S} \subseteq P \setminus \{j, k\}} \frac{(|\Tilde{S}| + 1)!\,\bigl(p - |\Tilde{S}| - 2\bigr)!}{p!} \, \Bigl( v(\Tilde{S} \cup \{j, k\}) - v(\Tilde{S} \cup \{j\}) \Bigr) \\
            \,=\, & \sum_{\substack{S \subseteq P \setminus \{k\}, \\ j \in S}} \frac{|S|!\,\bigl(p - |S| - 1\bigr)!}{p!} \, \Bigl( v(S \cup \{k\}) - v(S) \Bigr).
        \end{align*}
        In the last step, we used this transformation backwards, so we add / remove $j$ instead of $k$ this time, in other words $S = \Tilde{S} \cup \{j\}$ and $\Tilde{S} := S \setminus \{j\}$. This is possible since $\Tilde{S}$ contains neither $j$ nor $k$.
        
        We now use the first step from the beginning backwards and arrive at the desired result:
        \begin{align*}
            \phi_j
            & \,=\, \sum_{S \subseteq P \setminus \{j, k\}} \frac{|S|!\,\bigl(p - |S| - 1\bigr)!}{p!} \, \Bigl( v(S \cup \{k\}) - v(S) \Bigr) 
            \,+\, \sum_{\substack{S \subseteq P \setminus \{k\}, \\ j \in S}} \frac{|S|!\,\bigl(p - |S| - 1\bigr)!}{p!} \, \Bigl( v(S \cup \{k\}) - v(S) \Bigr) \\
            & \,=\, \sum_{S \subseteq P \setminus \{k\}} \frac{|S|!\,\bigl(p - |S| - 1\bigr)!}{p!} \, \Bigl( v(S \cup \{k\}) - v(S) \Bigr) 
            \,=\, \phi_k.
        \end{align*}
        
    \end{proof}

    
    
    \item \textbf{Bonus: Proof of the Efficiency Axiom}
    
    Efficiency requires a little more effort than the others:

    \begin{theorem}[Efficiency]
    Let $\{\phi_j\}_{j \in P}$ be the Shapley values induced by $v$.
    Then
    $$
    \sum_{j=1}^p \phi_j \;=\; v(P).
    $$
    \end{theorem}

    The proof idea is as follows: Because we sum up the Shapley values over all players, the values $v(S)$ of each coalition not equal to $P$, so $S \subsetneq P$, appear with both minus and plus signs that exactly cancel each other out.
    Apart from that, the term $v(P)$ occurs for every player, and one only needs to calculate that its weights sum up to 1.
    
    We will use the order-based definition of Shapley values here, since with it the proof is simpler.

    \begin{proof}
        
        We first plug in the definition and swap the two sums:
        \begin{gather*}
            \sum_{j=1}^p \phi_j 
            \;=\;
            \sum_{j=1}^p \frac{1}{p!} \sum_{\pi \in \mathfrak{S}_P} \Bigl(
                v\bigl( \mathrm{Pred}_\pi(j) \cup \{j\} \bigr)
                - v\bigl( \mathrm{Pred}_\pi(j) \bigr)
            \Bigr) \\
            = \frac{1}{p!} \sum_{\pi \in \mathfrak{S}_P} \sum_{j=1}^p \Bigl(
                v\bigl( \mathrm{Pred}_\pi(j) \cup \{j\} \bigr)
                - v\bigl( \mathrm{Pred}_\pi(j) \bigr)
            \Bigr).
        \end{gather*}
        Now the inner sum corresponds to a single permutation, and sums over all players in this permutation.
        
        Fix a particular permutation $\pi$. 
        We consider the order of the players given by this permutation:~$\bigl(\pi(1), \pi(2), \dots, \pi(p)\bigr)$.
        Each marginal contribution inside the inner sum corresponds to one step in this ordering.
        We reorder the inner sum according to this ordering given by the permutation:
        \begin{multline*}
            \sum_{j=1}^p \Bigl( v\bigl( \mathrm{Pred}_\pi(j) \cup \{j\} \bigr)
                - v\bigl( \mathrm{Pred}_\pi(j) \bigr) \Bigr)
            = \sum_{i=1}^p \Bigl( v\bigl( \{ \pi(1), \dots \pi(i) \} \bigr)
                - v\bigl( \{ \pi(1), \dots \pi(i-1) \} \bigr) \Bigr) \\
            = \underbrace{v \bigl( \{ \pi(1) \} \bigr) - v(\varnothing)}_{i=1 \text{ or } j = \pi(1)}
                \;+\; \underbrace{v \bigl( \{ \pi(1), \pi(2) \} \bigr) - v \bigl( \{ \pi(1) \} \bigr)}_{i=2 \text{ or } j = \pi(2)}
                \;+\; \underbrace{v \bigl( \{ \pi(1), \pi(2), \pi(3) \} \bigr) - v \bigl( \{ \pi(1), \pi(2) \} \bigr)}_{i=3 \text{ or } j = \pi(3)} \\
                \;+\; \dots 
                \;+\; \underbrace{v \bigl( \{ \pi(1), \ldots, \pi(p) \} \bigr) - v \bigl( \{ \pi(1), \ldots, \pi(p - 1) \} \bigr)}_{i=p \text{ or } j = \pi(p)} 
            = v \bigl( \pi(1), \ldots, \pi(p) \} \bigr) - v(\varnothing)
            = v (P).
        \end{multline*}

        In other words, the inner sum is a telescope sum, where all the terms except the first and the last one cancel.
        Plugging this into the outer sum yields the desired result:
        \begin{gather*}
            \sum_{j=1}^p \phi_j 
            \;=\; \frac{1}{p!} \sum_{\pi \in \mathfrak{S}_P} \left( \sum_{j=1}^p \Bigl(
                v\bigl( \mathrm{Pred}_\pi(j) \cup \{j\} \bigr)
                - v\bigl( \mathrm{Pred}_\pi(j) \bigr)
            \Bigr) \right) \\
            \;=\; \frac{1}{p!} \sum_{\pi \in \mathfrak{S}_P} v(P)
            \;=\; \frac{1}{p!} \cdot \bigl(p! \cdot v(P)\bigr)
            \;=\; v(P) .
        \end{gather*}
        
    \end{proof}

    When using the set-based definition, one can show that the term $v(P)$ appears exactly once per player (hence $p$ times in total) for the coalition $S = P \setminus \{j\}$ with the weight $\frac{|P - 1|! \, \bigl(|P| - |P - 1| - 1\bigr)!}{|P|!} = \frac{1}{p}$ each, and then one just has to show that for all the other terms, their number of appearance together with their weights cause them to cancel out.
    See also \href{https://math.stackexchange.com/questions/2747088/shapley-value-is-efficient}{here}.

    
    
    \item \textbf{Bonus: Proof of Linearity}

    This is as straightforward as the additivity:

    \begin{theorem}[Linearity]
    Let $\alpha, \beta \in \R$ and $v, v_1, v_2 : 2^P \to \R$ be value functions with $v = \alpha v_1 + \beta v_2$, which means that for every subset $\SsubP$ we have $v(S) = \alpha \cdot v_1(S) + \beta \cdot v_2(S)$.
    Let $\{\phi_{j, v}\}_{j \in P}$ be the Shapley values induced by $v$ and $\{\phi_{j, v_1}\}_{j \in P}$ and $\{\phi_{j, v_2}\}_{j \in P}$ those for $v_1$ and $v_2$ respectively.
    Then for any player $j \in P$ it holds:
    $$
    \phi_{j, v} = \phi_{j, \alpha v_1 + \beta v_2} = \alpha \phi_{j, v_1} + \beta \phi_{j, v_2}.
    $$
    \end{theorem}

    \begin{proof}

        Since additivity was already proven in part~\ref{ex_sol:Shapley_axioms_proofs_a}, we only have to prove homogeneity, that means that $\phi_{j, \alpha v_1} = \alpha \phi_{j, v_1}$ for any $\alpha \in \R$.

        So let $p \geq 1$, $ j \in P$, $\alpha \in \R$ and $v_1$ be any value function.
        Denote $v = \alpha v_1$.        
        For the marginal contributions of player $j$ on any subset $\SsubP$ we have: $v(\Scupj) - v(S) = \alpha v_1(\Scupj) - \alpha v_1(S) = \alpha \Bigl( v_1(\Scupj) - v_1(S) \Bigr)$.
        As for the additivity, plugging this into either of the two definitions yields the results:
        \begin{align*}
            \phi_{j, v}
            & \,=\, \sum_{S \subseteq P \setminus \{j\}} \frac{|S|!\,\bigl(p - |S| - 1\bigr)!}{p!} \, \Bigl( v(S \cup \{j\}) - v(S) \Bigr) 
            \,=\, \sum_{S \subseteq P \setminus \{j\}} \frac{|S|!\,\bigl(p - |S| - 1\bigr)!}{p!} \, \alpha \Bigl( v_1(S \cup \{j\}) - v_1(S) \Bigr) \\
            & \,=\, \alpha \sum_{S \subseteq P \setminus \{j\}} \frac{|S|!\,\bigl(p - |S| - 1\bigr)!}{p!} \, \Bigl( v_1(S \cup \{j\}) - v_1(S) \Bigr) 
            \,=\, \alpha \phi_{j, v_1}.
        \end{align*}
        
    \end{proof}

\end{enumerate}

}
