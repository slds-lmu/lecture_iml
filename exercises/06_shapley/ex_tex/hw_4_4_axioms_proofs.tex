\aufgabe{Axioms of Shapley Values}{

In this exercise, we want to mathematically prove that the Shapley values in cooperative game theory, as defined in the lecture, fulfill the four axioms discussed in the lecture.
Throughout this exercise, we assume that the order and the set definition of Shapley values yield the same result, so you can always choose which definition to use.

\begin{enumerate}

    \item Prove that the symmetry, dummy player, and additivity axioms of Shapley values hold for any value function and any set $P$ of players.
    These axioms are rather straightforward to prove, and this can be done using either of the definitions
    
    \item \textbf{Bonus:} Prove the efficiency axiom for Shapley values.
    
    \textit{Hint:} This is easier when using the order definition. Plug the definition into the sum inside the axiom, then swap the sums and argue why almost all terms of the inner sum now vanish.
    
    \item \textbf{Bonus:} Similarly to the additivity axiom, prove the following homogeinity axiom:

    If $\alpha \in \R$ is a real number and $v$ the payout function of a game, and we define $\tilde{v} := \alpha v$ as another payout function, then the Shapley value also scales accordingly, that is for all features $j \in P$ we have:
    $$\phi_{j,\tilde{v}} = \phi_{j,\alpha v} = \alpha \phi_{j,v}.$$

    \textit{Note:}
    This axiom together with the additivity axiom show that the Shapley value is linear in the payout function, which means linearity of the Shapley value.
    
\end{enumerate}

}