\aufgabe{LIME Quiz}{
\begin{enumerate}
  \item Which of the following statement(s) about general local explanations is/are correct? 
          \begin{enumerate}
            \item A single ICE curve is a local explanation method. \textcolor{blue}{Correct}
            \item Robust local explanation methods should return similar explanations for 
            similar observations. \textcolor{blue}{Correct}
            \item In ordinary Gower's distance, all features receive different weights. \textcolor{blue}{Not correct, all receive a weight of 1. (or depending on range of each feature)}
          \end{enumerate}
  \item Which of the following statement(s) about local surrogate models is/are correct?  
            \begin{enumerate}
                \item Surrogate models produced by LIME should have the same prediction as the model to be explained for the whole training dataset. \textcolor{blue}{Not correct, they should be faithful in the neighborhood of the point of interest, the closer a point is to the point of interest, the closer the prediction of the local surrogate model should be to the original prediction.}
                \item The choice of the sampling process and the definition of locality are important hyperparameters of LIME that have a large impact on the behavior of the method. \textcolor{blue}{Correct}
                \item LIME does not require any adaptions to be applicable to deep learning models for image data. \textcolor{blue}{Not correct, adaption to distance function is necessary} 
                \item LIME requires the surrogate model to use all available features - a selection of features is not allowed. \textcolor{blue}{Not correct, L0-regularized/LASSO model possible}
                \item If the kernel width for the exponential kernel is set to infinity, all observations receive a proximity measure/weight of $1$ independent of their distance to $\xv$. \textcolor{blue}{Correct}
          \end{enumerate}
  
\end{enumerate}
}
