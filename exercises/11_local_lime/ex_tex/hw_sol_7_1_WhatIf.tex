\loesung{Counterfactuals - WhatIf}{
\begin{enumerate}[a)]
\item Implementation of WhatIf: 

\lstinputlisting[firstline=1,lastline=30]{"rsrc/helper_functions_whatif.R"}

Example: \href{https://raw.githubusercontent.com/slds-lmu/lecture_iml/master/exercises/06_lime_ce/rsrc/datasets/wheat_seeds.csv}{wheat\_seeds.csv}
\lstinputlisting{"rsrc/data_whatif.R"}
\input{"rsrc/x_interest.txt"}

\lstinputlisting[firstline=12,lastline=12]{"rsrc/run_whatif.R"}

\input{"rsrc/cf.txt"}

\item Counterfactuals generated with WhatIf are valid and proximal, since they reflect the closest training datapoint 
with the desired/different prediction. 
The counterfactuals are also plausible since by definition they adhere to the data manifold.
The counterfactuals are not sparse and might propose changes to many features - this is 
a clear disadvantage of this method. 

\item Evaluation 
\lstinputlisting[firstline=32,lastline=59]{"rsrc/helper_functions_whatif.R"}

Example:
\lstinputlisting[firstline=17,lastline=17]{"rsrc/run_whatif.R"}
Please note that this method only evaluates if \textit{single} feature changes still 
lead to the desired prediction but not multiple at once.

\end{enumerate}
}