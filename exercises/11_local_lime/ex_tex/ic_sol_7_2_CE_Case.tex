\aufgabe{Counterfactuals: Example}{
\begin{enumerate}[a)]
  \item Which of these counterfactuals would you like to present to the customer? \\
  Different strategies:
  \begin{itemize} 
      \item Actionability: no changes of sex, decrease in age or enormous jumps
      in salary (1, 2, 7, 8)
      \item Plausibility: no decrease in salary or savings. (7, 9, 6)
      \item Sparsity: not many feature changes (7, 9)
  \end{itemize}
  User preferences: 
  \begin{itemize}
      \item Do not change credit amount. (3)
      \item Get credit now, no increase in age (4, 5, 8)
      \item ...
  \end{itemize}
  $\leadsto$ If we would consider all aspect, we would not present any counterfactual to the customer.
  \item Which of these counterfactuals would you rather hide from the customer? \\
  The ones changing a sensitive attribute like sex (1, 2, 9) because they could object against the model.
  \item Which of these counterfactuals would you like to share with data scientist responsible for the model?  \\
  The ones indicating that individuals/subpopulations are discriminated against (changing sex, 1, 2, 9).
  The ones that are implausible like a decrease in salary or savings (7, 9, 6).
  Getting the credit for 2 might not be due to changes in sex but due to reducing credit amount. 
  A sanity check, if sex really needs to switch is therefore a good idea (homeworksheet 6/exercise 1).
\end{enumerate}
}
