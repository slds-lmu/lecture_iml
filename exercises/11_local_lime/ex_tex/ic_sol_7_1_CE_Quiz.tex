\aufgabe{Counterfactuals Quiz}{
Which of the following statement(s) about counterfactual explanations is/are correct? 
        \begin{enumerate}
        \item In case of only two continuous features, we could directly read counterfactuals from a prediction surface plot.
        \textcolor{blue}{Correct.
        Note that this still holds if we additionally only have categorical features, e.g. for a one additional binary feature, we get two surface plots (one for each category) and could find one or more counterfactuals for each category and compare them with another / think about how to weight the change in a categorical feature with the distance for each counterfactual.
        For more categories, or for more categorical features, we simply get more plots, in the end one plot for each possible combination of category values (which also means this still gets unfeasible to do by hand if there are too many categorical features or too many categories for a single feature).}
        \item Counterfactual explanations are not suitable for people without machine learning knowledge. %, because reasoning by "What if..." questions is not natural for human beings.
        \textcolor{blue}{Not correct, but actually the opposite: Counterfactual explanations are among the most suitable explanations for lay people, for two reasons:
        \begin{itemize}
            \item Reasoning of humans often happens in the form of counterfactuals / alternatives / "What if..." questions, that is, reasoning in binary alternatives, which is more simple and therefore preferred.
            \item Focus on single data points: easier to interpret, more relevant to specific use case, and easier to understand.
        \end{itemize}
        }
        \item Making use of domain-knowledge encoded in causal graphs, could help to receive more realistic counterfactuals.
        \textcolor{blue}{Correct. One would know which features may need to be modified to be able to change the target value.}
        \end{enumerate}
}
