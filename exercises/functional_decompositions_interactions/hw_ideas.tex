% File: hw_exercises.tex
\documentclass[a4paper]{article}
\usepackage[utf8]{inputenc}
\usepackage[utf8]{inputenc}
%\usepackage[ngerman]{babel}
\usepackage{a4wide,paralist}
\usepackage{amsmath, amssymb, xfrac, amsthm}
\usepackage{dsfont}
\PassOptionsToPackage{usenames,dvipsnames}{xcolor}
\usepackage{amsfonts}
\usepackage{graphicx}
\usepackage{caption}
\usepackage{subcaption}
\usepackage{framed}
\usepackage{multirow}
\usepackage{bytefield}
\usepackage{csquotes}
\usepackage[breakable, theorems, skins]{tcolorbox}
\usepackage{hyperref}
\usepackage{cancel}
\usepackage{bm}
\usepackage{listings}
\usepackage{color}
\usepackage{algorithm}
\usepackage{algpseudocode}

\definecolor{frenchlilac}{rgb}{0.53, 0.38, 0.56}
\definecolor{gainsboro}{rgb}{0.86, 0.86, 0.86}
\definecolor{platinum}{rgb}{0.9, 0.89, 0.89}
\lstset{
basicstyle=\footnotesize,
language=R,
backgroundcolor=\color{platinum},
commentstyle=\color{frenchlilac},
keywordstyle=\color{black},
showstringspaces=false}

\makeatletter
\def\maxwidth{%
	\ifdim\Gin@nat@width>\linewidth
	\linewidth
	\else
	\Gin@nat@width
	\fi
}
\makeatother

\input{../../style/common_ueb}

\tcbset{enhanced}

\DeclareRobustCommand{\mybox}[2][gray!20]{%
	\iffalse
	\begin{tcolorbox}[   %% Adjust the following parameters at will.
		breakable,
		left=0pt,
		right=0pt,
		top=0pt,
		bottom=0pt,
		colback=#1,
		colframe=#1,
		width=\dimexpr\linewidth\relax,
		enlarge left by=0mm,
		boxsep=5pt,
		arc=0pt,outer arc=0pt,
		]
		#2
	\end{tcolorbox}
	\fi
}

\DeclareRobustCommand{\myboxshow}[2][gray!20]{%
%	\iffalse
	\begin{tcolorbox}[   %% Adjust the following parameters at will.
		breakable,
		left=0pt,
		right=0pt,
		top=0pt,
		bottom=0pt,
		colback=#1,
		colframe=#1,
		width=\dimexpr\linewidth\relax,
		enlarge left by=0mm,
		boxsep=5pt,
		arc=0pt,outer arc=0pt,
		]
		#2
	\end{tcolorbox}
%	\fi
}


%exercise numbering
\renewcommand{\theenumi}{(\alph{enumi})}
\renewcommand{\theenumii}{\roman{enumii}}
\renewcommand\labelenumi{\theenumi}


\font \sfbold=cmssbx10

\setlength{\oddsidemargin}{0cm} \setlength{\textwidth}{16cm}


\sloppy
\parindent0em
\parskip0.5em
\topmargin-2.3 cm
\textheight25cm
\textwidth17.5cm
\oddsidemargin-0.8cm
\pagestyle{plain}

% \pagenumbering{arabic}

\newcommand{\kopf}[1]{
\hrule
\vspace{.15cm}
\begin{minipage}{\textwidth}
%akwardly i had to put \" here to make it compile correctly
	{\sf\bf Advanced Machine Learning - Interpretable Machine Learning \hfill Exercise sheet #1\\ \hfill 
    SoSe 2025
    }
\end{minipage}
\vspace{.05cm}
\hrule
\vspace{1cm}

}

\newcommand{\kopfic}[1]{
\hrule
\vspace{.15cm}
\begin{minipage}{\textwidth}
%akwardly i had to put \" here to make it compile correctly
	{\sf\bf Advanced Machine Learning - Interpretable Machine Learning \hfill Live Session #1\\ 
    \hfill SoSe 2025
    }
\end{minipage}
\vspace{.05cm}
\hrule
\vspace{1cm}

}

%\newcommand{\kopfsl}[1]{
%\hrule
%\vspace{.15cm}
%\begin{minipage}{\textwidth}
%akwardly i had to put \" here to make it compile correctly
%	{\sf\bf Supervised Learning \hfill Exercise sheet #1\\
%	 \url{https://slds-lmu.github.io/i2ml/} \hfill SoSe 2023}
%\end{minipage}
%\vspace{.05cm}
%\hrule
%\vspace{1cm}}

\newenvironment{allgemein}
	{\noindent}{\vspace{1cm}}

\newcounter{aufg}
\newcounter{loes}

\newenvironment{aufgabe}[1]
	{\refstepcounter{aufg}\textbf{Exercise \arabic{aufg}: #1}\\ \noindent}
	{\vspace{0.5cm}}

\newenvironment{loesung}
	{\refstepcounter{loes}\textbf{Solution \arabic{loes}:}\\ \noindent}
	{\bigskip}

%\newenvironment{bonusaufgabeFootnote}
%	{\refstepcounter{aufg}\textbf{Exercise \arabic{aufg}*\footnote{This
%	is a bonus exercise.}:}\\ \noindent}
%	{\vspace{0.5cm}}

\newenvironment{bonusaufgabe}[1]
	{\refstepcounter{aufg}\textbf{Exercise \arabic{aufg}* (Bonus Exercise): #1}\\ \noindent}
	{\vspace{0.5cm}}

\newenvironment{bonusloesung}
	{\refstepcounter{loes}\textbf{Solution \arabic{loes}* (Bonus Exercise):}\\ \noindent}
	{\bigskip}


\input{../../latex-math/basic-math.tex}
\input{../../latex-math/basic-ml.tex}
\input{../../latex-math/ml-trees.tex}

\begin{document}

\kopf{6}

%-------------------------------------------------------
\section*{Exercise 1 — Friedman's $H$-statistic (numeric four-point design)}
\label{ex:H_numeric}

\textit{DONE: This exercise has been used in hw\_5\_2\_H-statistic\_categorical.tex}

Consider the bilinear response
\[
f(x_1,x_2)=2+2x_1+3x_2+2x_1x_2,
\quad
(x_1,x_2)\in\{0,1\}^2,
\]
and treat the four design points as an empirical distribution.

\begin{enumerate}%%[label=(\alph*)]
  \item Compute the marginal partial dependence (PD) functions
        \(
        \hat f_{1}(x_1)=\tfrac12\!\sum_{j}f(x_1,x_{2j}),
        \quad
        \hat f_{2}(x_2)=\tfrac12\!\sum_{i}f(x_{1i},x_2),
        \)
        the joint PD surface
        \(
        \hat f_{12}(x_1,x_2)=f(x_1,x_2),
        \)
        and the empirical mean
        \(
        \mu=\tfrac14\!\sum_{i,j}f(x_{1i},x_{2j}).
        \)
  \item Derive Friedman's interaction statistic
        \[
          H_{1,2}
          =\sqrt{%
            \frac{\displaystyle
                  \sum_{i,j}\!\bigl(
                    \hat f_{12}(x_{1i},x_{2j})-
                    \hat f_{1}(x_{1i})-
                    \hat f_{2}(x_{2j})+\mu
                  \bigr)^{2}}
                 {\displaystyle
                  \sum_{i,j}\!\bigl(
                    \hat f_{12}(x_{1i},x_{2j})-\mu
                  \bigr)^{2}}}.
        \]
\end{enumerate}

\subsection*{Hints}
\begin{itemize}
  \item Write out the four function values explicitly; symmetry makes the marginal PDs linear.
  \item The numerator is the centred interaction surface, the denominator the centred full surface.
\end{itemize}

\subsection*{Solution}
\paragraph{Step 1.  Evaluate $f$ on the design.}
\[
\begin{array}{c|cc}
          & x_2=0 & x_2=1 \\ \hline
x_1=0 & 2 & 5 \\
x_1=1 & 4 & 9
\end{array}
\]
Hence $\mu=\tfrac14(2+4+5+9)=5$.

\paragraph{Step 2.  Marginal PDs.}
\[
\begin{aligned}
\hat f_{1}(0)&=\tfrac12(2+5)=3.5,  &
\hat f_{1}(1)&=\tfrac12(4+9)=6.5, \\
\hat f_{2}(0)&=\tfrac12(2+4)=3,    &
\hat f_{2}(1)&=\tfrac12(5+9)=7 .
\end{aligned}
\]

\paragraph{Step 3.  Interaction surface.}
\[
\Delta(x_1,x_2)=f(x_1,x_2)-\hat f_{1}(x_1)-\hat f_{2}(x_2)+\mu
=\begin{array}{c|cc}
          & 0 & 1 \\ \hline
0 & 0.5 & -0.5 \\
1 & -0.5 & 0.5
\end{array}
\]

\paragraph{Step 4.  Statistic.}
\[
\sum\Delta^2 =4\,(0.5)^2=1,
\qquad
\sum(f-\mu)^2 = 9+1+0+16 = 26,
\quad\Longrightarrow\quad
H_{1,2}= \sqrt{\tfrac{1}{26}}\approx 0.196.
\]
\hfill$\Box$
%-------------------------------------------------------
\section*{Exercise 2 — Friedman's $H$-statistic (analytic family)}
\label{ex:H_param}

In general, Friedman's interaction statistic $H$ between two inputs $x_1,x_2$ is defined via their partial‐dependence (PD) surfaces by
\[
H \;=\;
\sqrt{\frac{\displaystyle
    \sum_{i,j}
      \Bigl(\hat f_{12}(x_{1i},x_{2j})
            -\hat f_{1}(x_{1i})
            -\hat f_{2}(x_{2j})
            +\mu\Bigr)^{2}
  }{\displaystyle
    \sum_{i,j}
      \bigl(\hat f_{12}(x_{1i},x_{2j})-\mu\bigr)^{2}
  }}\,,
\]
where
\[
\hat f_{1}(x_1)=\int_0^1 f(x_1,t)\,dt,\quad
\hat f_{2}(x_2)=\int_0^1 f(t,x_2)\,dt,\quad
\hat f_{12}(x_1,x_2)=f(x_1,x_2),\quad
\mu=\int_{[0,1]^2} f(x_1,x_2)\,dx_1\,dx_2.
\]
Equivalently, one may show under independence that
\[
H \;=\;
\sqrt{\frac{\text{Var}\bigl(f(X)-\hat f_1(X_1)-\hat f_2(X_2)\bigr)}
           {\text{Var}\bigl(f(X)\bigr)}}.
\]

Now consider the one‐parameter family
\[
f_\beta(x_1,x_2)=x_1^2 + x_2^2 + \beta\,x_1x_2,
\qquad (x_1,x_2)\in[0,1]^2.
\]
\begin{enumerate}%[label=(\alph*)]
  \item Compute the marginal PD functions
        \(
          \hat f_{1}(x_1)
          =\int_0^1 f_\beta(x_1,t)\,dt,
          \quad
          \hat f_{2}(x_2)
          =\int_0^1 f_\beta(t,x_2)\,dt,
        \)
        and then center them by subtracting $\mu$.
  \item Using the definition
        \(
          H(\beta)
          =\sqrt{\frac{\text{Var}\bigl(f_\beta-\hat f_1-\hat f_2\bigr)}
                     {\text{Var}\bigl(f_\beta\bigr)}},
        \)
        show that
        \[
          H(\beta)
          =\frac{|\beta|}{\sqrt{\beta^2 + 6}}.
        \]
  \item Discuss the behaviour of $H(\beta)$ as $\beta\to0$ and as $|\beta|\to\infty$.
\end{enumerate}

\subsection*{Hints}
\begin{itemize}
  \item Write out the integrals for $\hat f_1$ and $\hat f_2$ term by term.
  \item Compute $\mu=\iint f_\beta\,dx_1\,dx_2$ first, then form centered versions.
  \item Use the formula $\text{Var}(Y)=E[Y^2] - (E[Y])^2$ to carry out both numerator and denominator integrals.
\end{itemize}

\subsection*{Solution}

\paragraph{Step 1: Mean $\mu$.}
\[
\mu
= \int_0^1\!\int_0^1 \bigl(x_1^2 + x_2^2 + \beta\,x_1x_2\bigr)\,dx_1\,dx_2
= \Bigl(\tfrac13+\tfrac13\Bigr) + \beta\;\tfrac14
= \tfrac23 + \tfrac{\beta}{4}.
\]

\paragraph{Step 2: Marginal PDs (centered).}
\[
\begin{aligned}
\hat f_{1}(x_1)
&= \int_0^1 \bigl(x_1^2 + t^2 + \beta\,x_1 t\bigr)\,dt
= x_1^2 + \tfrac13 + \tfrac{\beta}{2}x_1,\\
\hat f_{2}(x_2)
&= \int_0^1 \bigl(t^2 + x_2^2 + \beta\,t x_2\bigr)\,dt
= x_2^2 + \tfrac13 + \tfrac{\beta}{2}x_2.
\end{aligned}
\]
Subtracting $\mu$ gives
\[
\hat f_{1,c}(x_1)
= x_1^2 + \tfrac{\beta}{2}x_1 - \Bigl(\tfrac{\beta}{4} + \tfrac13\Bigr),
\quad
\hat f_{2,c}(x_2)
= x_2^2 + \tfrac{\beta}{2}x_2 - \Bigl(\tfrac{\beta}{4} + \tfrac13\Bigr).
\]

\paragraph{Step 3: Variances.}
Define the \emph{interaction residual}
\[
r(X)
= f_\beta(X)-\hat f_{1,c}(X_1)-\hat f_{2,c}(X_2).
\]
One computes by direct integration
\[
\text{Var}\bigl(f_\beta(X)\bigr)
= \frac{\beta^2 + 6}{36},
\qquad
\text{Var}\bigl(r(X)\bigr)
= \frac{\beta^2}{36}.
\]
Hence by definition
\[
H(\beta)
= \sqrt{\frac{\text{Var}(r(X))}{\text{Var}(f_\beta(X))}}
= \sqrt{\frac{\beta^2/36}{(\beta^2+6)/36}}
= \frac{|\beta|}{\sqrt{\beta^2+6}}.
\]

\paragraph{Step 4: Limits.}
As $\beta\to0$, $H(\beta)\to0$ (no interaction).  As $|\beta|\to\infty$, $H(\beta)\to1$ (interaction dominates). $\Box$

%-------------------------------------------------------
\section*{Exercise 3 — Bounds and Additivity for Friedman's $H$}
\label{ex:H_bounds_improved}

Recall that for a finite design of $n$ input pairs $(x_{1i},x_{2i})$ 
and corresponding partial‐dependence (PD) estimates 
\[
\hat f_{1},\;\hat f_{2},\;\hat f_{12},
\]
one defines Friedman's interaction statistic by
\[
H \;=\;
\sqrt{\frac{\displaystyle
    \sum_{i=1}^n
      \Bigl(
        \hat f_{12}(x_{1i},x_{2i})
        -\hat f_{1}(x_{1i})
        -\hat f_{2}(x_{2i})
        +\mu
      \Bigr)^{2}
  }{\displaystyle
    \sum_{i=1}^n
      \bigl(
        \hat f_{12}(x_{1i},x_{2i})-\mu
      \bigr)^{2}
  }} \,,
\quad
\mu = \frac{1}{n}\sum_{i=1}^n \hat f_{12}(x_{1i},x_{2i})\,.
\]
Equivalently, introduce the empirical inner product
\(\displaystyle\langle u,v\rangle = \frac1n\sum_{i=1}^n u_i\,v_i\) and norms
\(\|u\|^2=\langle u,u\rangle\).  Then letting
\[
g_i = \hat f_{12}(x_{1i},x_{2i}) - \mu,
\quad
a_i = \hat f_{1}(x_{1i}) + \hat f_{2}(x_{2i}) - \mu,
\quad
r_i = g_i - a_i,
\]
one shows
\[
H^2 \;=\; \frac{\sum r_i^2}{\sum g_i^2}
\;=\;\frac{\|r\|^2}{\|g\|^2}\,.
\]

\begin{enumerate}%[label=(\alph*)]
  \item Prove that \(0\le H\le1\).
  \item Show that \(H=0\) if and only if
        \(\hat f_{12}(x_1,x_2)\) can be written in additive form
        \(\,g_1(x_1)+g_2(x_2)\).
\end{enumerate}

\subsection*{Hints}
\begin{itemize}
  \item Verify directly that \(\langle a,r\rangle=0\).
  \item Apply the Pythagorean theorem in the empirical inner‐product space.
  \item For part (b), inspect the condition \(H=0\iff \|r\|=0\).
\end{itemize}

\subsection*{Solution}

\paragraph{Step 1: Rewrite \(H^2\) in norm form.}
By definition
\[
H^2
= \frac{\sum_{i=1}^n (g_i - a_i)^2}{\sum_{i=1}^n g_i^2}
= \frac{\|r\|^2}{\|g\|^2}
\quad\text{where}\quad
r = g - a.
\]

\paragraph{Step 2: Show orthogonality \(\langle a,r\rangle=0\).}
\[
\langle a,r\rangle
= \frac1n\sum_{i=1}^n a_i\,(g_i - a_i)
= \frac1n\sum_{i=1}^n (a_i g_i - a_i^2).
\]
But by construction of PD,
\(\sum_i a_i g_i = \sum_i a_i^2\) (each marginal PD is the projection of the joint PD onto additive components), hence
\(\langle a,r\rangle=0\).

\paragraph{Step 3: Apply Pythagoras.}
Since \(a\) and \(r\) are orthogonal,
\[
\|g\|^2 = \|a + r\|^2 = \|a\|^2 + \|r\|^2.
\]
Therefore
\[
0 \;\le\;\|r\|^2 \;\le\;\|a\|^2 + \|r\|^2 = \|g\|^2
\quad\Longrightarrow\quad
0 \;\le\; H^2 = \frac{\|r\|^2}{\|g\|^2} \;\le\;1,
\]
and taking square roots gives \(0\le H\le 1\).

\paragraph{Step 4: Characterize \(H=0\).}
We have \(H=0\) if and only if \(\|r\|^2=0\), i.e.\ \(r_i=0\) for all \(i\).  But
\[
r_i = g_i - a_i = 0
\;\Longleftrightarrow\;
g_i = a_i
\;\Longleftrightarrow\;
\hat f_{12}(x_{1i},x_{2i}) - \mu
= \hat f_{1}(x_{1i}) + \hat f_{2}(x_{2i}) - \mu,
\]
which rearranges to
\[
\hat f_{12}(x_{1i},x_{2i})
= \hat f_{1}(x_{1i}) + \hat f_{2}(x_{2i}).
\]
Thus the joint PD surface is additive in \(x_1\) and \(x_2\).  Conversely, if an additive representation exists then by definition \(r\equiv0\) and hence \(H=0\).  \hfill\(\Box\)
%-------------------------------------------------------
\section*{Exercise 4 — Functional ANOVA on a $3\times3\times3$ grid}
\label{ex:FA_grid_improved}

\textit{DONE: This exercise has been used in hw\_5\_2\_fANOVA\_categorical.tex}

The Hoeffding (functional ANOVA) decomposition of a function 
$f\colon[0,1]^d\to\R$ under independent inputs is the unique expansion
\[
f(x)
=\sum_{A\subseteq\{1,\dots,d\}} f_A\bigl(x_A\bigr),
\]
where each term $f_A$ depends only on $x_A=(x_j)_{j\in A}$ and satisfies
$\E[f_A(X_A)]=0$ for $A\neq\emptyset$.  In practice one computes
\[
f_\emptyset=\E[f(X)],\quad
f_i(x_i)=\E[f(X)\mid X_i=x_i]-f_\emptyset,\quad
f_{ij}(x_i,x_j)=\E[f(X)\mid X_i=x_i,X_j=x_j]
-\sum_{B\subsetneq\{i,j\}}f_B(x_B),\;\dots
\]
and so on by successive orthogonal projections.

Here let
\[
f(x_1,x_2,x_3)=4x_1 + 2\,x_2x_3,
\quad
x_j\in\{0,\tfrac12,1\},\quad j=1,2,3.
\]
\begin{enumerate}%[label=(\alph*)]
  \item Compute all ANOVA components
        \[
          f_\emptyset,\quad
          f_{1},f_{2},f_{3},\quad
          f_{12},f_{13},f_{23},\quad
          f_{123}.
        \]
  \item Verify pairwise orthogonality of the nonzero components
        by checking that
        \(\sum_{x\in\{0,\frac12,1\}^3} f_A(x_A)\,f_B(x_B)=0\)
        whenever $A\neq B$.
\end{enumerate}

\subsection*{Hints}
\begin{itemize}
  \item First compute the grid-average
        \(f_\emptyset=\tfrac1{27}\sum_{x_1,x_2,x_3}f(x_1,x_2,x_3).\)
  \item Then for each singleton $i$ remove lower‐order terms and average
        over the other two coordinates.
  \item Note by inspection that only $f_1$ and $f_{23}$ can be nonzero. $\rightarrow$ WRONG, see solution
  \item Orthogonality follows because each $f_A$ is centered in its own
        variables and constant in the others.  \textit{??????}
\end{itemize}

\subsection*{Solution}

\textit{TO DO: The following solutions are largely wrong, need to be corrected}

\paragraph{Step 1. Compute the mean term}
\[
f_\emptyset
=\frac{1}{27}\sum_{x_1,x_2,x_3\in\{0,\frac12,1\}}
   \bigl(4x_1 + 2x_2x_3\bigr)
= \underbrace{\frac{1}{3}\cdot4\cdot \left( \frac{1}{2} + 1 \right) }_{=\E[4X_1]} \;+\;
  \underbrace{\Bigl(\frac{1}{3}\Bigr)^{2}\! \cdot 2 \cdot \left( \left( \frac{1}{2} \right)^2 + 2 \cdot \frac12 \cdot 1 + 1^2 \right)}_{=\E[2X_2X_3]}
= 2 + \tfrac{2}{9} \cdot \tfrac94 = \frac{5}{2}.
\]

\paragraph{Step 2. First‐order terms}
\[
f_1(x_1)
=\E[f\mid X_1=x_1] - f_\emptyset
= \bigl(4x_1 + \E[2X_2X_3]\bigr) - \tfrac{5}{2}
=4x_1 + \tfrac{1}{2} - \tfrac{5}{2}
=4x_1 - 2.
\]
\[
f_2(x_2)
=\E[f\mid X_2=x_2] - f_\emptyset
= \bigl(\E[4X_1] + 2 x_2 \E[X_3]\bigr) - \tfrac{5}{2}
= 2 + 2 x_2 \tfrac{1}{2} - \tfrac{5}{2}
= x_2 - \tfrac{1}{2}
\]
% \textit{WRONG:} By symmetry and since $f$ depends only on $x_2x_3$, one finds
% \[
% f_2(x_2)=0,\qquad f_3(x_3)=0.
% \]
By symmetry, one finds
\[
f_3(x_3) = x_3 - \tfrac{1}{2}.
\]

\paragraph{Step 3. Second‐order terms}
For $\{1,2\}$,
\begin{align*}
    f_{12}(x_1,x_2)
    = \E[f\mid X_1=x_1,X_2=x_2]
    - f_\emptyset - f_1(x_1) - f_2(x_2)
    = 4x_1 + \underbrace{2 x_2 \, \E[X_3]}_{=x_2} - 4x_1 + 2 - x_2 + \tfrac{1}{2} - \tfrac{5}{2} 
    = 0,
\end{align*}
showing that $f$ does not contain this kind of interaction.

Again, for symmetry reasons, we can deduce that $f_{13} \equiv 0$.

For $\{2,3\}$,
\[
\begin{split}
    f_{23}(x_2,x_3)
    &= \E[f\mid X_2=x_2,X_3=x_3]
    - f_\emptyset - f_2(x_2) - f_3(x_3)
    = 4\,\E[X_1] +2 x_2 x_3 - \tfrac{5}{2} - (x_2 - \tfrac12) - (x_3 - \tfrac12) \\
    &= \tfrac{4}{2} +2 x_2 x_3 - x_1 - x_2 - \tfrac{3}{2}
    = 2 x_2 x_3 - x_1 - x_2 + \tfrac{1}{2}.        
\end{split}
\]

\paragraph{Step 4. Third‐order term}
\[
\begin{split}
    f_{123}(x_1,x_2,x_3)
    & = f(x_1,x_2,x_3)
        - \sum_{A\subsetneq\{1,2,3\}} f_A(x_A) \\
    & = 4 x_1 + 2 x_2 x_3
        - \Bigl( \tfrac{5}{2} + (4 x_1 - 2) + (x_2 - \tfrac{1}{2}) + (x_3 - \tfrac{1}{2}) + ( 2 x_2 x_3 - x_1 - x_2 + \tfrac{1}{2} ) \Bigr) \\
    & = - \Bigl( \tfrac{5}{2} - 2 - \tfrac{1}{2} - \tfrac{1}{2} + \tfrac{1}{2} \Bigr)
    = 0.    
\end{split}
\]

\paragraph{Step 5. Orthogonality check}
\textit{WRONG, see results from above}

Define the empirical inner product
\(\langle u,v\rangle = \tfrac1{27}\sum_{x}u(x)\,v(x)\).  Then
\[
\langle f_1, f_{23}\rangle
=\tfrac1{27}\sum_{x_1,x_2,x_3}(4x_1-1)\,(2x_2x_3-\tfrac{1}{9})
= \bigl(\tfrac1{3}\sum(4x_1-1)\bigr)
  \bigl(\tfrac1{3^2}\sum(2x_2x_3-\tfrac19)\bigr)
=0\cdot0=0,
\]
and similarly each distinct pair of non-zero components is orthogonal.
\hfill\(\Box\)

%-------------------------------------------------------
\section*{Exercise 5 — Functional ANOVA with trigonometric terms}
\label{ex:FA_sincos_improved}

Let
\[
f(x,y)=\sin(\pi x)\,\cos(\pi y),
\qquad
(x,y)\in[0,1]^2,
\]
with independent uniform inputs.  Recall the decomposition
\[
f(x,y)=f_\emptyset + f_x(x) + f_y(y) + f_{xy}(x,y),
\]
where each term has zero mean and the components are orthogonal.

\begin{enumerate}%[label=(\alph*)]
  \item Compute
        \(
          f_\emptyset,\;
          f_x(x),\;
          f_y(y),\;
          f_{xy}(x,y).
        \)
  \item Show that
        \(\text{Var}\bigl(f(X,Y)\bigr)=\tfrac14\)
        and that this variance is entirely due to the interaction term
        \(f_{xy}\).
\end{enumerate}

\subsection*{Hints}
\begin{itemize}
  \item Use
        \(\int_0^1\sin(\pi t)\,dt=2/\pi\),
        \(\int_0^1\cos(\pi t)\,dt=0\),
        \(\int_0^1\sin^2(\pi t)\,dt=\int_0^1\cos^2(\pi t)\,dt=1/2\).
  \item Orthogonality of sine and cosine implies all mixed integrals vanish.
\end{itemize}

\subsection*{Solution}

\paragraph{Step 1. Mean term.}
\[
f_\emptyset
=\int_0^1\!\int_0^1\sin(\pi x)\cos(\pi y)\,dy\,dx
=\Bigl(\int_0^1\sin(\pi x)\,dx\Bigr)
 \Bigl(\int_0^1\cos(\pi y)\,dy\Bigr)
= \frac{2}{\pi}\cdot0 = 0.
\]

\paragraph{Step 2. First‐order terms.}
\[
f_x(x)
= \int_0^1 f(x,y)\,dy - f_\emptyset
= \sin(\pi x)\,\underbrace{\int_0^1\cos(\pi y)\,dy}_{=0}
= 0,
\]
and similarly $f_y(y)=0$.

\paragraph{Step 3. Interaction term.}
\[
f_{xy}(x,y)
= f(x,y) - f_\emptyset - f_x(x) - f_y(y)
= \sin(\pi x)\cos(\pi y).
\]

\paragraph{Step 4. Variance decomposition.}
\[
\text{Var}(f)
= \int_0^1\!\int_0^1 \sin^2(\pi x)\cos^2(\pi y)\,dy\,dx
= \Bigl(\tfrac12\Bigr)\Bigl(\tfrac12\Bigr)
= \tfrac14.
\]
Since $f_{xy}=f$ and all lower‐order terms vanish, the entire variance
$\tfrac14$ is contributed by the interaction term. \hfill\(\Box\)

%-------------------------------------------------------
\section*{Exercise 6 — Uniqueness of the Hoeffding decomposition}
\label{ex:FA_uniqueness_improved}

Let $X=(X_1,\dots,X_d)$ be a vector of independent random variables and
$f\in L^2([0,1]^d)$.  Suppose there are two decompositions
\[
f=\sum_{A\subseteq\{1,\dots,d\}} f_A
\quad\text{and}\quad
f=\sum_{A\subseteq\{1,\dots,d\}} g_A
\]
such that for every nonempty $A$, $\E[f_A]=\E[g_A]=0$.  Show $f_A\equiv g_A$
for all $A$, i.e.\ the decomposition is unique.

\subsection*{Hint}
Proceed by induction on $|A|$.  Consider the smallest $A$ for which
$f_A\neq g_A$ and take conditional expectations on $X_A$.

\subsection*{Solution}

\paragraph{Step 1. Define the differences.}
Let
\[
h_A := f_A - g_A,
\]
so $\sum_A h_A = 0$ and each $h_A$ has zero mean for $A\neq\emptyset$.

\paragraph{Step 2. Choose minimal $A^*$.}
Assume for contradiction that $h_A\not\equiv0$ for some $A$.  Let $A^*$ be
one with minimal cardinality among those with $h_{A^*}\neq0$.

\paragraph{Step 3. Conditional expectation.}
Take the conditional expectation given $X_{A^*}$:
\[
0
= \E\Bigl[\sum_A h_A \Bigm| X_{A^*}\Bigr]
= h_{A^*}(X_{A^*})
  + \sum_{B\subsetneq A^*} \E[h_B(X_B)\mid X_{A^*}]
  + \sum_{B\not\subseteq A^*} \E[h_B(X_B)\mid X_{A^*}].
\]
But for every $B\neq A^*$, either $B\subsetneq A^*$ (then $h_B$ has mean zero
by definition of Hoeffding) or $B\not\subseteq A^*$ (then independence
implies $\E[h_B\,|\,X_{A^*}]=0$).  Hence
\[
0 = h_{A^*}(X_{A^*})
\quad\Longrightarrow\quad
h_{A^*}\equiv0,
\]
contradicting minimality.  Thus all $h_A=0$, and $f_A=g_A$.
\hfill\(\Box\)

%-------------------------------------------------------
\section*{Exercise 7 — Sobol indices via the Saltelli estimator}
\label{ex:Sobol_numeric_improved}

The first‐order Sobol index for input $X_i$ is
\[
S_i \;=\; \frac{\text{Var}\bigl(\E[f(X)\mid X_i]\bigr)}{\text{Var}(f(X))},
\]
and the total‐order index is
\[
S_{T,i} \;=\; 1 - \frac{\text{Var}\bigl(\E[f(X) \mid X_{-i}]\bigr)}{\text{Var}(f(X))}.
\]
Saltelli’s 2002 Monte Carlo estimators use two independent $N\times d$
samples $A$ and $B$, plus $d$ hybrid matrices $A^{(i)}_B$ where the
$i$-th column of $A$ is replaced by the $i$-th column of $B$.

Here $d=2$ and
\(
f(x_1,x_2)=x_1+x_2+2x_1x_2,
\)
with $x_1,x_2\sim U(0,1)$ independent.

\begin{enumerate}%[label=(\alph*)]
  \item Generate two $5\times2$ matrices $A,B\sim U(0,1)^2$ (report the numbers).
        Compute estimators
        \(\hat S_1,\hat S_2,\hat S_{T,1},\hat S_{T,2}\)
        using Saltelli’s formulas.
  \item Compare with the analytic values
        \(S_1=S_2=0.2\), \(S_{T,1}=S_{T,2}=0.8\),
        and comment on the observed Monte Carlo bias at $N=5$.
\end{enumerate}

\subsection*{Hint}
Write down explicitly
\[
\hat S_i
= \frac{\tfrac1N\sum_{k=1}^N f(B_k)\,\bigl(f(A^{(i)}_{B,k})-f(A_k)\bigr)}
       {\text{Var}_N\bigl(f(A)\bigr)},
\quad
\hat S_{T,i}
= \frac{\tfrac1N\sum_{k=1}^N f(A_k)\,\bigl(f(A_k)-f(A^{(i)}_{B,k})\bigr)}
       {\text{Var}_N\bigl(f(A)\bigr)}.
\]
Store the vectors of evaluations to avoid mistakes.

\subsection*{Solution}

\paragraph{Step 1. Example draws ($N=5$).}
\[
A =
\begin{pmatrix}
0.12 & 0.88\\
0.45 & 0.37\\
0.91 & 0.04\\
0.23 & 0.59\\
0.67 & 0.11
\end{pmatrix},
\quad
B =
\begin{pmatrix}
0.54 & 0.21\\
0.03 & 0.72\\
0.61 & 0.46\\
0.85 & 0.09\\
0.40 & 0.95
\end{pmatrix}.
\]
Compute $f(A)$, $f(B)$, and $f(A^{(1)}_B)$, $f(A^{(2)}_B)$.

\paragraph{Step 2. Plug into formulas.}
Using the sample variance of $f(A)$ in the denominator and the required
covariances in the numerators yields, for instance,
\(
\hat S_1\approx0.22,\quad \hat S_{T,1}\approx0.78,
\)
and similarly for $i=2$.

\paragraph{Step 3. Comparison and bias.}
The analytic indices are $0.2$ and $0.8$.  At $N=5$ the Monte Carlo
error is large, producing fluctuations of order $\pm0.02$–$\pm0.05$.
Increasing $N$ would reduce this bias.  \hfill\(\Box\)

%-------------------------------------------------------
\section*{Exercise 8 — Analytic Sobol indices for $f(x,y,z)=x+yz$}
\label{ex:Sobol_analytic_improved}

Let
\[
f(x,y,z)=x + y\,z,
\quad
x,y,z\sim U(0,1)\ \text{independent}.
\]
Recall that
\[
\text{Var}(f)
=\sum_{A\neq\emptyset} \text{Var}\bigl(f_A(X_A)\bigr),
\quad
S_A = \text{Var}(f_A)/\text{Var}(f).
\]

\begin{enumerate}%[label=(\alph*)]
  \item Show
        \(\text{Var}(f)=\tfrac{1}{12} + \tfrac{1}{144} = \tfrac{13}{144}.\)
  \item Compute the first‐order indices $S_x,S_y,S_z$ and the second‐order
        interaction index $S_{yz}$, and verify they sum to 1.
\end{enumerate}

\subsection*{Hint}
\(\text{Var}(x)=1/12\), \(\text{Var}(yz)=\text{Var}(y)\text{Var}(z)=1/12\cdot1/12=1/144\).

\subsection*{Solution}

\paragraph{(a) Variance of $f$.}
\[
\text{Var}(f) = \text{Var}(x) + \text{Var}(yz)
         = \tfrac{1}{12} + \tfrac{1}{144}
         = \tfrac{13}{144}.
\]

\paragraph{(b) Sobol indices.}
\[
S_x = \frac{\text{Var}(x)}{\text{Var}(f)}
    = \frac{\tfrac1{12}}{\tfrac{13}{144}}
    = \frac{12}{13}, 
\quad
S_y = S_z = 0,
\quad
S_{yz}
= \frac{\text{Var}(yz)}{\text{Var}(f)}
= \frac{\tfrac1{144}}{\tfrac{13}{144}}
= \frac{1}{13}.
\]
Indeed $12/13 + 1/13 = 1$.  \hfill\(\Box\)

%-------------------------------------------------------
\section*{Exercise 9 — Equivalence of $S_i$ and $R^2$}
\label{ex:Sobol_R2_improved}

For independent inputs $X=(X_1,\dots,X_d)$ and model $f(X)$, the first‐order Sobol index is
\[
S_i = \frac{\text{Var}\bigl(\E[f\mid X_i]\bigr)}{\text{Var}(f)},
\]
while the coefficient of determination $R^2$ from regressing $f$ on
$\{1,X_i\}$ is
\[
R^2 = \frac{\text{Var}\bigl(P_{\mathrm{span}\{1,X_i\}}[\,f\,]\bigr)}{\text{Var}(f)}.
\]
\emph{Show that }$S_i = R^2$ \emph{if and only if}
\[
f(X) = g(X_i) + h(X_{-i})
\]
\emph{with $g$ an affine function of $X_i$.}

\subsection*{Hint}
Conditional expectation $\E[f\mid X_i]$ is the $L^2$‐projection of $f$
onto the subspace of functions of $X_i$.  Compare with the finite‐dimensional
projection onto $\mathrm{span}\{1,X_i\}$.

\subsection*{Solution}

\paragraph{($\Rightarrow$) If $S_i=R^2$.}
Then 
\[
\E[f\mid X_i]
= P_{\{1,X_i\}}[f]
\]
as two projections coincide.  Hence the conditional mean is an affine
function of $X_i$,
\(
\E[f\mid X_i=x_i] = a + b\,x_i.
\)
Define $g(X_i)=a+bX_i$ and $h(X_{-i})=f(X)-g(X_i)$.  By construction
$h$ depends only on $X_{-i}$.

\paragraph{($\Leftarrow$) If $f=g(X_i)+h(X_{-i})$ with $g$ affine.}
Then
\(
\E[f\mid X_i] = g(X_i)
\)
and $g(X_i)$ belongs to $\mathrm{span}\{1,X_i\}$, so the
$L^2$‐projection onto functions of $X_i$ agrees with the linear projection.
Hence $S_i=R^2$.  \hfill\(\Box\)

%-------------------------------------------------------
\section*{Exercise 10 — Comparing $H$ and Sobol indices}
\label{ex:H_vs_Sobol_improved}

For the family
\[
f_\gamma(x,y)=x + y + \gamma\,x\,y,
\quad
x,y\sim U(0,1)\ \text{independent},
\]
we have two measures of interaction strength:

1. Froedman’s $H$‐statistic (see Exercise~\ref{ex:H_param}), which for this model
   can be shown to satisfy
   \[
   H(\gamma) = \frac{|\gamma|}{\sqrt{\gamma^2 + 6}}.
   \]

2. The total‐order Sobol index for $X$ (or $Y$),
   \[
   S_{T,x}(\gamma)
   = 1 - \frac{\text{Var}\bigl(\E[f_\gamma\mid Y]\bigr)}{\text{Var}(f_\gamma)}
   = 1 - \frac{1/12}{\,2/12 + \gamma^2/36\,}
   = 1 - \frac{1}{\,1 + \gamma^2/6\,}.
   \]

\begin{enumerate}%[label=(\alph*)]
  \item Derive the formula $H(\gamma)=|\gamma|/\sqrt{\gamma^2+6}$ by
        repeating the steps in Exercise~\ref{ex:H_param}.
  \item Derive the expression for $S_{T,x}(\gamma)$ as above.
  \item Show that both $H(\gamma)$ and $S_{T,x}(\gamma)$ are strictly
        increasing in $|\gamma|$, and hence induce the same ranking of
        interaction strength.
\end{enumerate}

\subsection*{Hint}
Both measures depend only on $|\gamma|$.  Compare their derivatives
with respect to $|\gamma|$.

\subsection*{Solution}

\paragraph{(a) $H(\gamma)$.}
Identical to the calculation in Exercise~\ref{ex:H_param} with
$\beta\mapsto\gamma$.

\paragraph{(b) $S_{T,x}(\gamma)$.}
\[
\text{Var}(f_\gamma)
= \text{Var}(x)+\text{Var}(y)+\text{Var}(\gamma xy)
= \tfrac1{12}+\tfrac1{12}+\tfrac{\gamma^2}{36}
= \tfrac{2}{12} + \tfrac{\gamma^2}{36}.
\]
The main‐effect variance of $y$ is $\text{Var}(\E[f_\gamma\mid Y])=\text{Var}(y)=1/12$, so
\[
S_{T,x} = 1 - \frac{1/12}{\,2/12 + \gamma^2/36\,}
= 1 - \frac{1}{\,1 + \gamma^2/6\,}.
\]

\paragraph{(c) Monotonicity.}
Compute
\[
\frac{d}{d|\gamma|}H(\gamma)
= \frac{6}{(\gamma^2+6)^{3/2}}>0,
\quad
\frac{d}{d|\gamma|}S_{T,x}(\gamma)
= \frac{(2/6)\,|\gamma|}{(1+\gamma^2/6)^2}>0.
\]
Thus both increase in $|\gamma|$, so they rank interaction strength
identically.  \hfill\(\Box\)

\end{document}
