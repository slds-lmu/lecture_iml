\textbf{Solution Quiz:}\\\noindent
\medskip

\begin{enumerate}
  \item What is the problem of PDP when interactions between features are present? How about extrapolation?
        \begin{itemize}
        	\item[$\Rightarrow$] When interactions are apparent, the effects may in average cancel each other out, i.e. PDP shows no effect.
        	\item[$\Rightarrow$] Extrapolation can cause issues in regions with few observations or if features are correlated, i.e. resulting data point may be unrealistic or very unlikely.
        \end{itemize}
        \item How do PDPs and ICE curves correspond with each other?
        \begin{itemize}
        	\item[$\Rightarrow$] The value of the PDP at a point $x_j$, corresponds to the point-wise average of the values of the ICE curves at this point.
        \end{itemize}
        \item Which problem do we need to keep in mind when using centered ICE/PDP for categorical features? 
        \begin{itemize}
        	\item[$\Rightarrow$] If we center the ICE/PDPs for categorical features, the expected changes always refer to a selected reference category. 
        \end{itemize}
        \item M-Plots handle correlated data well and do not suffer from extrapolation. Which disadvantage does this method have?
        \begin{itemize}
        	\item[$\Rightarrow$] M-plots suffer from omitted variable bias.
        \end{itemize}
        \item Name the advantages of ALE over PDP.
        \begin{itemize}
        	\item[$\Rightarrow$] Computationally faster (measurable when they are based on the same grid); less to no extrapolation.
        \end{itemize}
    	\item Can you think of a situation in which ALE equals PDP?
    	\begin{itemize}
    		\item[$\Rightarrow$] If features are uncorrelated, ALE plots are equal to PDPs. 
    	\end{itemize}
        \item How does the interpretation between M-Plots and ALE differ?
        \begin{itemize}
        	\item[$\Rightarrow$] In the M-Plot one can not infer, if the effect is due to the feature of interest or due to correlated features. ALE only shows the effect of the feature of interest.
        \end{itemize}


\item You fitted a model that should predict the value of a property depending on 
    the number of rooms and square meters. 
    You want to compute feature effects using the following methods: 
    PDP, M-plots and (uncentered) ALE plots. 
    Which of the following strategies reflect which method? \\
    The feature effect for a 30 m$^2$ corresponds to... 
\begin{enumerate}[a)]
  \item ... what the model predicts on average for flats that also have around 30 m$^2$, for example, 28 m$^2$ to 32 m$^2$. $\Rightarrow$ \textbf{M-plot}
  \item ... how the model's predictions change on average when flats with 28 m$^2$ to 32 m$^2$ have 32 m$^2$ vs. 28 m$^2$. $\Rightarrow$ \textbf{uncentered ALE}
  \item ... what the model predicts on average if all properties in the dataset have 30 m$^2$. $\Rightarrow$ \textbf{PDP}
\end{enumerate}

\end{enumerate}
