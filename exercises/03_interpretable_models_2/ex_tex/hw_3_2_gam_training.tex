\aufgabe{Interactions in GAMs}{

This exercise is about training and interpreting GAMs and about their predictive performance, when interactions are present in the data.
For this, we will train a linear model and a GAM, both with and without interactions, and compare their performance as well as their interpretability.

In this exercise, we use a small data set contained in the file "\textit{gam\_interactions.csv}" on the Moodle webpage.
It consists of observations of 2 features $x_1$ and $x_2$ and one prediction target $y$.
Your task is to fit different models on this data.

\begin{enumerate}[a)]
    \item
    Fit a simple linear model as well as a normal GAM (that is, a GAM with smooth functions each depending only on single features) on the data.
    How do the two models perform?
    What can you conclude about the models and about the data from interpreting them?

    \textit{Hint:} As discussed in the lecture, one can interpret a GAM by plotting the single one-dimensional components, which each depend only on a single feature.

    \item
    Now fit a linear model additionally containing a simple interaction term.
    How does it perform compared to the two models above?
    
    What can you conclude about the data?
    In particular, given your observations from all three models, do you consider any of them to model the data very well?

    \item
    Again, fit a GAM to the data, but which also includes an interaction term.
    You can do this by, for example, adding the interaction term from the linear model as another function to the GAM, or by treating this linear interaction term as an additional third feature and adding a smooth function of this third feature to the GAM.

    How does this model now perform?
    Compare to the three other models.
    How could you interpret this complex model, if at all?
\end{enumerate}

\textit{Hint:} Useful software packages for this exercise:

\begin{itemize}
    \item
    When using Python, you can use the package \texttt{pyGAM} together with \texttt{numpy} and \texttt{matplotlib}. To train linear models, you can use the section on linear models from scikit-learn.

    \item 
    When working with R, the package \texttt{mgcv} contains a \href{https://www.rdocumentation.org/packages/stats/versions/3.6.2/topics/glm}{gam} function.
\end{itemize}

}