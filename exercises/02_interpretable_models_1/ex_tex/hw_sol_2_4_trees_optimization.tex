

\loesung{

The implementations of the solution in R or in python can be found on Moodle in the files "\textit{CART\_sol.R}" and "\textit{CART\_sol.py}".
As described in the exercise, the main idea is to not compute the sums for the two potential child nodes in every step.
In the following, we present two ways to work around this.

The first one is to compute the cumulative sums for the target vector, that is, compute a vector which at position $k$ contains the entry $\sum_{i=1}^k y_i$ (or $y_i^2$ respectively), and then for each split point find the corresponding index of the feature of interest, and look up the respective sums in the precomputed vector.
This solution is implemented as ``Solution 1'' in the code.

The other, more direct way of implementation is to maintain two sums, one for the potential left child of the current split point and one for the potential right child.
These two sums are then updated in every step, so that the left sum always increases and the right sum always decreases.
This method is implemented as ``Solution 2'' in the code.

Both these solutions in principle achieve the desired result, namely that the whole algorithm of finding the optimal split point has a computational complexity of $O(dn)$ instead of $O(d n^2)$.
Nevertheless, the two implementations differ in another point, which may be practically even more relevant.
Namely, the ``Solution 1'' relies on vectorized operations, acting on the whole vector of observations or potential split points in parallel. These vectorized operations in theory require an effort of order $O(n)$ each (linear in the length of the vector), because they perform one operation for every element in the vector, but they are much faster in practice due to parallelization.

The ``Solution 2'' does not use such operations at all and therefore strictly fulfills the theoretical runtime requirement, whereas ``Solution 1'' strictly speaking does not.
Nevertheless, at least for the implementation in R, ``Solution 1'' actually runs faster in practice.

The solution also contains the calculation of a deeper tree reflecting the structure of the DGP used.

NOTE: This exercise also serves a preparation for the FAST algorithm introduced in chapter 3.

}