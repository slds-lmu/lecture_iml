\loesung{
%
%\begin{table}[ht]
%	\centering
%	\begin{tabular}{lcc|c}
%		\hline
%		 & Hepatitis A & no Hepatitis A & $\Sigma$\\
%		\hline
%		Salsa eaten & $218 (a)$ & $45 (b)$ & $263$ \\
%		Salsa not eaten & $21 (c)$ & $85 (d)$ & $106$\\
%		\hline
%		$\Sigma$ & $239$ & $130$ & $369$\\
%		\hline
%	\end{tabular}
%\end{table}

\begin{table}[ht]
	\centering
	\begin{tabular}{rrrrrr}
		\hline
		& WINTER & SPRING & SUMMER & FALL & $\Sigma$ \\ 
		\hline
		$y$=0 & 174.00 & 111.00 & 98.00 & 128.00 & 511.00 \\ 
		$y$=1 & 7.00 & 73.00 & 90.00 & 50.00 & 220.00 \\ 
		$\Sigma$ & 181.00 & 184.00 & 188.00 & 178.00 & 731.00 \\ 
		\hline
	\end{tabular}
\end{table}

\begin{enumerate}[a)]
	\item Odds for ``high number of bike rentals'' vs. ``low to medium number of bike rentals'' in winter: $$\text{odds} = \frac{P(y=1\, | \,\texttt{season}=\text{WINTER})}{P(y=0\, | \,\texttt{season}=\text{WINTER})} = \frac{7}{174} = 0.04 $$
	\textbf{Interpretation:} In winter the occurrence of \texttt{cnt} $> 5531$ ($y=1$) is $0.04$ times as likely as \texttt{cnt} $\leq 5531$ ($y=0$), the odds are $1:25$, which means $y=0$ is 25 times as likely than $y=1$.
    
	\item Odds Ratio of spring vs. winter: 
	\begin{align*}
		\text{odds ratio} 
		&= \frac{P(y=1\, | \,\texttt{season}=\text{SPRING})\,/\,P(y=0\, | \,\texttt{season}=\text{SPRING})}{P(y=1\, | \,\texttt{season}=\text{WINTER})\,/\,P(y=0\, | \,\texttt{season}=\text{WINTER})} \\
		&= \frac{73/111}{7/174} = 16.35
	\end{align*}

    For summer we get
    \[
    \text{odds ratio} 
    = \frac{P(y=1\, | \,\texttt{season}=\text{SUMMER})\,/\,P(y=0\, | \,\texttt{season}=\text{SUMMER})}{P(y=1\, | \,\texttt{season}=\text{WINTER})\,/\,P(y=0\, | \,\texttt{season}=\text{WINTER})}
    = \frac{90/98}{7/174} = 22.83,
    \]

    and for fall:
    \[
    \text{odds ratio} 
    = \frac{P(y=1\, | \,\texttt{season}=\text{FALL})\,/\,P(y=0\, | \,\texttt{season}=\text{FALL})}{P(y=1\, | \,\texttt{season}=\text{WINTER})\,/\,P(y=0\, | \,\texttt{season}=\text{WINTER})}
    = \frac{50/128}{7/174} = 9.71.
    \]
    
	\textbf{Interpretation:} The chances (the odds) of having "high bike rentals" are $16.35$ times higher in season SPRING compared to the reference category (WINTER).
    As in winter the odds of $y=0$ are $25:1$, this means in spring they are roughly $(25:16):1$, which means roughly $5:3$.
    Similarly, in summer the odds are $22.83$ times higher that in winter, which means that in summer they are close to $1:1$ (since in winter they were $1:25$), so the chances in summer are roughly 50-50.

    \newpage
    
	\item Table:
	\begin{table}[!ht]
		\centering
		\begin{tabular}{rrrr}
			\hline
			& Estimate & Std. Error & Pr($>$$|$z$|$) \\ 
			\hline
			(Intercept) & -3.2131 & 0.3854 & 0.0000 \\ 
			seasonSPRING & 2.7941 & 0.4138 & 0.0000 \\ 
			seasonSUMMER & 3.1280 & 0.4121 & 0.0000 \\ 
			seasonFALL & 2.2731 & 0.4199 & 0.0000 \\ 
			\hline
		\end{tabular}
	\end{table}

	The intercept gives the odds for ``high number of bike rentals'' vs. ``low to medium number of bike rentals'' for the default category, in winter: exp$(-3.2131) = 0.04$. Interpretation as in a).
	
	Regarding the estimate of seasonSPRING (and analogous for all the other seasons): $\text{odds ratio (when season changes from winter to spring)} = \text{exp}(2.7941) =  16.35$. Interpretation as in b).

    \item

    \begin{enumerate}[(i)]
    
        \item\textbf{Offset.}
    $
    \delta=
    (-0.0627)(62.79)
    +(-0.0925)(12.76)
    +(0.0166)(365.00)
    =\boxed{0.94}.
    $
    
    \item\textbf{Probabilities.}
    
    %Effective intercept: \(\beta_0+\delta=-7.58\).
    
    \[
    \beta_0+\delta=-7.58
    \qquad
    \eta(x_1)=-7.58 + 0.29\,x_1,
    \qquad
    p(x_1)=\sigma\!\bigl(\eta(x_1)\bigr).
    \]
    
    \begin{center}
    \begin{tabular}{cccc}
    \toprule
    \(x_1\) (\si{\celsius}) & \(\eta(x_1)\) & \(p(x_1)\) \\
    \midrule
    10 &  -4.677 & 0.009 \\
    15 &  -3.227 & 0.038 \\
    20 &  -1.777 & 0.144 \\
    25 &  -0.327 & 0.419 \\
    30 &   1.123 & 0.755 \\
    35 &   2.573 & 0.929 \\
    \bottomrule
    \end{tabular}
    \end{center}
    \item\textbf{Marginal‐effect derivation.}
    
    \textbf{Goal:} Compute the instantaneous change in the predicted probability  
    when temperature \(x_{1}\) increases, holding all other features constant:
    \[
    \frac{\partial p}{\partial x_{1}}
       \;=\;
       \frac{\partial}{\partial x_{1}}
       \sigma\!\bigl(\eta\bigr),
       \qquad
       \eta=\beta_{0}+\delta+\beta_{1}x_{1},
    \]
    where \(p=\sigma(\eta)=\bigl(1+e^{-\eta}\bigr)^{-1}\) and  
    \(\delta\) collects the fixed contributions of the remaining covariates.
    
    \begin{enumerate}
      \item \textbf{Apply the chain rule}
      \[
        \frac{\partial p}{\partial x_{1}}
         \;=\;
         \frac{\partial \sigma(\eta)}{\partial \eta}\,
         \frac{\partial \eta}{\partial x_{1}} .
      \]
    
      \item \textbf{Derivative of the sigmoid}
      
      We begin with the definition of the sigmoid function:
    \[
    \sigma(\eta) = \frac{1}{1 + e^{-\eta}}.
    \]
    
    We aim to compute its derivative with respect to \( \eta \):
    \[
    \frac{d}{d\eta} \left( \frac{1}{1 + e^{-\eta}} \right).
    \]
    
    Rewriting using the power rule:
    \[
    \sigma(\eta) = (1 + e^{-\eta})^{-1},
    \]
    so by the chain rule:
    \[
    \frac{d\sigma}{d\eta}
      = -1 \cdot (1 + e^{-\eta})^{-2} \cdot \frac{d}{d\eta}(1 + e^{-\eta})
      = \frac{e^{-\eta}}{(1 + e^{-\eta})^2}.
    \]
    
    To express this in terms of \( \sigma(\eta) \), note:
    \[
    \sigma(\eta) = \frac{1}{1 + e^{-\eta}},
    \quad\text{and}\quad
    1 - \sigma(\eta) = \frac{e^{-\eta}}{1 + e^{-\eta}}.
    \]
    
    Hence:
    \[
    \sigma(\eta)\,(1 - \sigma(\eta)) = \frac{1}{1 + e^{-\eta}} \cdot \frac{e^{-\eta}}{1 + e^{-\eta}} = \frac{e^{-\eta}}{(1 + e^{-\eta})^2}.
    \]
    
    So the derivative simplifies to:
    \[
    \frac{\partial p}{\partial \eta}
      = \frac{d\sigma}{d\eta}
      = \sigma(\eta)\,(1 - \sigma(\eta))
      = p\,(1 - p).
    \]
    
      \item \textbf{Derivative of the linear predictor}
      \[
        \frac{\partial \eta}{\partial x_{1}}
          =\beta_{1},
        \qquad
        \text{because } \eta=\beta_{0}+\delta+\beta_{1}x_{1}.
      \]
    
      \item \textbf{Combine the two results}
      \[
        \boxed{\displaystyle
          \frac{\partial p}{\partial x_{1}}
            = p\,(1-p)\,\beta_{1}} .
      \]
    
      \item \textbf{Interpretation}  
      The factor \(p(1-p)\) is maximal at \(p=0.5\) and vanishes as \(p\to0\) or \(p\to1\),
      illustrating that the marginal effect of \(x_{1}\) is largest when the model is
      most uncertain and negligible in the extreme–probability regions.  The coefficient
      \(\beta_{1}\) scales this intrinsic sensitivity, so the overall effect is
      \emph{context‐dependent}: it varies with the current value of the linear predictor
      through \(p\).
    \end{enumerate}
    
    \noindent%
    The scaling term \(p(1-p)\le 0.25\) peaks at \(p=0.5\), explaining
    why the same coefficient \(\beta_{1}=0.29\) produces the largest
    probability change near the mid-range and almost none in the tails.
    
    % \[
    % \frac{dp}{dx_1}
    % =\frac{dp}{d\eta}\,\frac{d\eta}{dx_1}
    % \;\;=\;
    % \sigma(\eta)\bigl(1-\sigma(\eta)\bigr)\,\beta_1
    % \;=\; p(1-p)\beta_1 .
    % \]
    
    % Steps:
    
    % 1. \(p=\sigma(\eta)\), \(\dfrac{d\sigma}{d\eta}=\sigma(1-\sigma)\).
    % 2. \(\eta=\beta_0+\delta+\beta_1x_1 \;\Longrightarrow\; d\eta/dx_1=\beta_1\).
    
    \item\textbf{Numeric marginal effects:}
    $
    dp/dx_1 = p(1-p)\beta_1 = 0.29\,p(1-p).
    $
    
    $\Rightarrow$ Largest effect occurs near \(x_1\approx30^\circ\text{C}\) where \(p\approx0.5\).
    
    \begin{center}
    \begin{tabular}{ccc}
    \toprule
    \(x_1\) & \(p\) & \(dp/dx_1\) \\
    \midrule
    15 °C & 0.038 & 0.011 \\
    30 °C & 0.755 & \textbf{0.054} \\
    35 °C & 0.929 & 0.019 \\
    \bottomrule
    \end{tabular}
    \end{center}
    
    
    \item\textbf{Classification at threshold 0.5.}
    
    % Predicted “high-rental” for temperatures with \(p\ge0.5\):
    % \(\boxed{30^\circ\text{C},\,35^\circ\text{C}}\).
    
    %\bigskip
    %\noindent
    %\textit{R verification}
    
    % \begin{verbatim}
    % beta0 <- -8.52; beta1 <- 0.29
    % delta <- -0.06266594*62.79 - 0.09247547*12.76 + 0.01659578*365
    % T <- c(10,15,20,25,30,35)
    % eta <- beta0 + delta + beta1*T
    % p   <- plogis(eta)
    % dp  <- p*(1-p)*beta1
    % data.frame(T, eta=round(eta,3), p=round(p,3), dp=round(dp,3))
    % \end{verbatim}
    
	\item Table:	
	\begin{table}[ht]
		\centering
		\begin{tabular}{rrrr}
			\hline
			& Estimate & Std. Error & Pr($>$$|$z$|$) \\ 
			\hline
			(Intercept) & -8.5176 & 1.2066 & 0.0000 \\ 
			seasonSPRING & 1.7427 & 0.5977 & 0.0035 \\ 
			seasonSUMMER & -0.8566 & 0.7660 & 0.2635 \\ 
			seasonFALL & -0.6417 & 0.5543 & 0.2470 \\ 
			temp & 0.2902 & 0.0391 & 0.0000 \\ 
			hum & -0.0627 & 0.0124 & 0.0000 \\ 
			windspeed & -0.0925 & 0.0305 & 0.0024 \\ 
			days\_since\_2011 & 0.0166 & 0.0014 & 0.0000 \\ 
			\hline
		\end{tabular}
	\end{table}
	
	If all features are considered in the model, the $\beta$-value for the intercept is higher in absolute terms, but the odds changes to exp($-8.5176) = 0.0002$, i.e. the probability of ``high number of bike rentals'' is even less in winter when considering the full model compared to the one only containing feature \texttt{season}. Also, the higher chance of having "high bike rentals" in season SPRING compared to WINTER declined to exp($1.7427)=5.71$ (vs. $16.35$ in the smaller model).
\end{enumerate}
}
