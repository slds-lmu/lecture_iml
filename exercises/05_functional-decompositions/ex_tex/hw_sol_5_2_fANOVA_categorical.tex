\loesung{
\begin{enumerate}[a)]
\item
\textbf{Step 1. Compute the constant term.}
\[
g_\emptyset
= \E [ \fh(X_1, X_2, X_3) ]
= \E [4 X_1 + 2 X_2 X_3 - 4]
= 4 \ \E [X_1] + 2 \ \E [X_2] \ \E [X_3] - 4
= - 4 ,
\]
because the variables are all independent and centered (i.e., $\E [X_i] = 0$ for all $i$.).

Alternatively, using all data points:
\begin{align*}
    g_\emptyset
    = \ & \frac{1}{27} \sum_{x_1,x_2,x_3 \in \{-1,0,1\}}
       \bigl( 4x_1 + 2x_2x_3 - 4 \bigr)
    = \frac{1}{27} \Bigl( \underbrace{9 \cdot 4 \cdot (-1 + 0 + 1)}_{\text{All possible values of }x_1} \\
      & + \underbrace{3 \cdot 2 \cdot \bigl( -1 \cdot (-1) + (-1) \cdot 0 + (-1) \cdot 1 + 0 \cdot (-1) + 0 \cdot 0 + 0 \cdot 1 + 1 \cdot (-1) + 1 \cdot 0 + 1 \cdot 1 \bigr) - 27 \cdot 4}_{\text{All possible combinations of values for }x_2 \text{ and } x_3} \Bigr) \\
    = \ & \frac{1}{27} \Bigl( 0 + 0 - 27 \cdot 4 \Bigr) = -4,
\end{align*}
because each value of $x_1$ occurs 9 times and each combination of values of $x_2$ and $x_3$ occurs 3 times.

Analogously, for all the PD-functions (which are marginal expectations), we can use the fact that the variables are independent and centered.

\textbf{Step 2. First‐order terms.}
\[
g_1(x_1)
= \E[ \fh(x_1, X_2, X_3) ] - g_\emptyset
= \E[ 4 x_1 + 2 X_2 X_3 - 4 ] - g_\emptyset
= 4 x_1 + 2 \ \underbrace{\E[ X_2 ]}_{=0} \ \underbrace{\E[ X_3 ]}_{=0} - 4 - (-4)
=4x_1.
\]
\[
g_2(x_2)
= \E[ \fh(X_1, x_2, X_3) ] - g_\emptyset
= \E[ 4 X_1 + 2 x_2 X_3 - 4 ] - g_\emptyset
= 4 \underbrace{\E[ X_1 ]}_{=0} + 2 x_2 \underbrace{\E[ X_3 ]}_{=0} - 4 - (-4)
= 0.
\]
Because $\fh$ is symmetric in $x_2$ and $x_3$, we get analogously
\[
g_3(x_3)=0.
\]

\textbf{Step 3. Second‐order terms.}
\[
g_{1,2}(x_1, x_2)
= \E[ \fh(x_1, x_2, X_3) ] - g_1(x_1) - g_2(x_2) - g_\emptyset
= \E[ 4 x_1 + 2 x_2 X_3 - 4 ] - 4 x_1 + 4
= 4 x_1 + 2 x_2 \underbrace{\E[ X_3 ]}_{=0} - 4 - 4 x_1 +4
= 0.
\]
Again using that $\fh$ is symmetric in $x_2$ and $x_3$, we receive exactly the same result for $S=\{1,3\}$, so $g_{13}(x_1, x_3) = 0$.

For $\{2,3\}$, we get:
\[
g_{2,3}(x_2,x_3)
= \E[ \fh(X_1, x_2, x_3) ] - g_2(x_2) - g_3(x_3) - g_\emptyset
= \E[ 4 X_1 + 2 x_2 x_3 - 4 ] + 4
= 4 \underbrace{\E[ X_1 ]}_{=0} + 2 x_2 x_3 - 4 +4
= 2 x_2 x_3.
\]

\textbf{Step 4. Third‐order term.}
\[
g_{1,2,3}(x_1,x_2,x_3)
= \fh(x_1,x_2,x_3)
- \sum_{V\subsetneq\{1,2,3\}} g_V(x_V)
= \underbrace{4 x_1 + 2x_2x_3 - 4}_{=\fh(x_1,x_2,x_3)}
- \Bigl( \underbrace{- 4}_{=g_\emptyset} + \underbrace{4x_1}_{=g_1(x_1)} + \underbrace{2x_2x_3}_{=g_{2,3}(x_2,x_3)} \Bigr)
=0.
\]
So, all in all, we only have three nonzero components in the fANOVA, which are exactly equal to the terms in the given equation that we would expect:
\[
    g_\emptyset = -4, \quad
    g_1 = 4 x_1, \quad
    g_{2,3} = 2 x_2 x_3.
\]

\item 
\textbf{Orthogonality check.}
As indicated by the hint, we only need to consider pairs of two nonzero components, since otherwise the terms are trivially orthogonal.
This is because for any $U, V$ with one component being 0, w.l.o.g. $g_U(\Xv_U) = 0$, we have:
\[
\E_\Xv \left[ g_U(\Xv_U) \, g_V(\Xv_V) \right]
= \E_\Xv \left[ 0 \cdot \, g_V(\Xv_V) \right]
= \E_\Xv \left[ 0 \right] = 0.
\]
For the remaining components, we have three possibilities we need to check:
\begin{align*}
    & \text{For } (V,U) = (\emptyset, \{1\}): 
    \, \E_\Xv \left[ g_\emptyset \, g_1(X_1) \right]
    = \E_\Xv \left[ -4 \cdot 4 X_1 \right]
    = -16 \E [X_1] = 0, \\
    & \text{For } (V,U) = (\emptyset, \{2,3\}): 
    \, \E_\Xv \left[ g_\emptyset \, g_{2,3}(\Xv_{2,3}) \right]
    = \E_\Xv \left[ -4 \cdot 2 X_2 X_3 \right]
    = -8 \ \E [X_2] \ \E [X_3] = 0, \\
    & \text{For } (V,U) = (\{1\}, \{2,3\}): 
    \, \E_\Xv \left[ g_1(X_1) \, g_{2,3}(\Xv_{2,3}) \right]
    = \E_\Xv \left[ 4 X_1 \cdot 2 X_2 X_3 \right]
    = 8 \ \E [X_1] \ \E [X_2] \ \E [X_3] = 0,
\end{align*}
again using independence and centering. This shows the desired orthogonality.

\end{enumerate}
One can also see in this exercise, that
\begin{itemize}
    \item the symmetry of $\fh$ in $x_2$ and $x_3$ translates to the fANOVA components.
    In general, the whole fANOVA decomposition inherits the same symmetries as the original function.
    \item after computing the expected value $g_\emptyset$ in the first step, one could center the whole function $\fh^c (x_1,x_2,x_3) = \fh(x_1,x_2,x_3) - g_\emptyset$ and then use this centered function $\fh^c$ throughout.
    In this case, we would not need to subtract $g_\emptyset$ in each step anymore.
\end{itemize}

}