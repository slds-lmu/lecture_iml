\aufgabe{Friedman's H-statistic for categorical features}{ \label{ex:h-statistic_categorical}

As in exercise \ref{ex:fANOVA_categorical}, we consider categorical features, and want to calculate the H-statistic for such a case.
Calculating the H-statistic for categorical features or targets basically works the same as for the fANOVA in exercise \ref{ex:fANOVA_categorical}.

We are given two independent categorical random variables, which are Bernoulli distributed:
$$
(X_1,X_2) \in \{0,1\}^2, \text{ i.e. } \P(X_1 = 0) = \P(X_1 = 1) = \P(X_2 = 0) = \P(X_2 = 1) = \frac12,
$$
which means the two-dimensional feature space only contains four points in total.
We consider the model
\[
\fh(x_1,x_2) = 2 + 2 x_1 + 3 x_2 + 2 x_1 x_2.
\]
Your task is to compute the H-statistic $H_{1,2}$ for measuring the interaction in this model.

\begin{enumerate}[a)]
    
    \item Compute all partial dependence (PD) functions:
    \begin{align*}
        \fh_{\emptyset,PD} = \mu
        & = \E[f(X_1, X_2)], \\
        \fh_{1,PD}(x_1 )
        & = \E_{X_2}[\fh(x_1, X_2)] = \tfrac12 \sum_{j=1}^2 f(x_1, x_2^{(j)}), \\
        \fh_{2, PD}(x_2)
        & = \E_{X_1}[\fh(X_1, x_2)].
    \end{align*}
        % \(
        % \fh_{1,PD}(x_1 )
        % = \E_{X_2}[\fh(x_1, X_2)] = \tfrac12 \sum_{j} f(x_1, x_2^{(j)}),
        % \)
        % \(
        % \fh_{2, PD}(x_2) = \E_{X_1}[\fh(X_1, x_2)] = \tfrac12\!\sum_{i}f(x_{1i},x_2),
        % \)
        % the joint PD surface
        % \(
        % \fh_{12,PD} (x_1,x_2) = f(x_1,x_2),
        % \)
        % and the empirical mean
        % \(
        % \fh_{\emptyset,PD} = \mu = \E[f(X_1, X_2)].
        % \)
    (The last PD function is of course $\fh_{12,PD} (x_1,x_2) = f(x_1,x_2)$.)
    
    When computing the PD-functions, you can simply write out the four function values explicitly.
    
    \item Compute the interaction statistic $H_{1,2}$ using either one of the formulas
    \[
    \begin{aligned}
        H_{1,2}
        & = \sqrt{
            \frac{\displaystyle
                \sum_{i,j = 1}^2 \bigl(
                \hat f^c_{12,PD}(x_1^{(i)},x_2^{(j)})-
                \hat f^c_{1,PD}(x_1^{(i)})-
                \hat f^c_{2,PD}(x_2^{(j)})
                \bigr)^2
            }{\displaystyle
                \sum_{i,j = 1}^2 \bigl(
                \hat f^c(x_1^{(i)},x_2^{(j)})
                \bigr)^2
            }
        } \\
        & = \sqrt{
            \frac{\displaystyle
                \sum_{i,j = 1}^2 \bigl(
                \hat f_{12,PD}(x_1^{(i)},x_2^{(j)})-
                \hat f_{1,PD}(x_1^{(i)})-
                \hat f_{2,PD}(x_2^{(j)})+\mu
                \bigr)^2
            }{\displaystyle
                \sum_{i,j = 1}^2 \bigl(
                \hat f(x_1^{(i)},x_2^{(j)})-\mu
                \bigr)^2
            }
        } \\
        & = \sqrt{
            \frac{\var \Bigl[
                \hat f_{12,PD}(X_1, X_2)-
                \hat f_{1,PD}(X_1)-
                \hat f_{2,PD}(X_2)
                \Bigr]
            }{\var \Bigl[
                \hat f( X_1, X_2 )
                \Bigr]
            }
        } \\.  
    \end{aligned}
    \]

    \item \textbf{Bonus:} Prove that the three formulas given above for calculating the H-statistic are equivalent in general.
    
\end{enumerate}

}