\textbf{Solution Quiz:}\\\noindent
\medskip

\begin{enumerate}

    \item 
    % With respect to what property of a function or model do functional decompositions decompose it?
    Functional decompositions are decompositions with respect to what property of a function?
    \begin{itemize}
        \item[$\Rightarrow$] interaction structure
    \end{itemize}
    \item 
    What are functional decompositions useful for?
    \begin{itemize}
        \item[$\Rightarrow$] High interpretability value, w.r.t. interactions / interaction structure, can be sparsified, evtl. enabling more efficient computation, sensitivity analysis
    \end{itemize}
    \item 
    Why is the general definition of functional decompositions insufficient for practical use? How can these problems be alleviated?
    % Alternative: Why are additional constraints besides the general definition necessary to arrive at a useful functional decomposition?
    \begin{itemize}
        \item[$\Rightarrow$] See standard fANOVA: Remove lower-order terms from higher-order terms, and try to include as much as possible / reasonable in the lower-order terms
    \end{itemize}
    \item 
    True or false: When given a function explicitly through a formula and applying a functional decomposition algorithm on it, one always recovers the original components of the formula as components in the resulting decomposition.
    \begin{itemize}
        \item[$\Rightarrow$] Wrong, see examples lecture
    \end{itemize}
    \item 
    How do PDPs and general multidimensional PD-functions relate to the components in classical fANOVA?
    \begin{itemize}
        \item[$\Rightarrow$] 
    \end{itemize}
    \item 
    In what order are the components in the classical fANOVA computed?
    \begin{itemize}
        \item[$\Rightarrow$] 
    \end{itemize}
    \item 
    How are Sobol indices defined, and what property of the classical fANOVA decomposition does one need for them?
    \begin{itemize}
        \item[$\Rightarrow$] 
    \end{itemize}
    \item
    You have trained a random forest classifier. Can you easily and directly compute the functional decomposition of the classifier and how would you do it? Or why can you not?
    \begin{itemize}
        \item[$\Rightarrow$] 
        Trivial / straightforward decomposition of trees also works for tree ensembles like e.g. a random forest \\
        Problem: This naive decomposition is not very insightful and usually does not capture main effects => Still need one of the model-agnostic methods to compute a functional decomposition
    \end{itemize}
    \item 
    How is the interaction type of a node inside a decision tree defined?
    \begin{itemize}
        \item[$\Rightarrow$] 
    \end{itemize}
    \item 
    Why are problems of computing and using the classical fANOVA in practice?
    \begin{itemize}
        \item[$\Rightarrow$] computationally expensive (as for any functional decomposition), independence assumption, high computational burden of PD-functions (can be alleviated via calculating all PD-functions in an hierarchical manner from the highest-order to the lowest-order, using dynamical programming
    \end{itemize}
    % \item
    % Why are problems of computing and using functional decompositions in practice, and how can one deal with them?
    % \begin{itemize}
    %     \item[$\Rightarrow$]
        % Too many features => expensive & complicated to understand => sparse decomposition or only focus on specific interactions of interest, or models like EBMs
        % Only for tabular data, extracting tabular data from raw data is difficult to address
        % Need to choose one method, all of which have individual disadvantages
    % \end{itemize}
    % \item 
    % What do the vanishing condition and the orthogonality condition intuitively mean?
    % \begin{itemize}
    %     \item[$\Rightarrow$] 
    % \end{itemize}
    \item 
    Which other IML methods discussed in this lecture so far have used or are related to the concept of functional decomposition?
    \begin{itemize}
        \item[$\Rightarrow$] LMs, GLMs, GAMs, EBMs, RPFs, PDPs, ALE, Shapley, feature importance (Verify for each one, why!)
    \end{itemize}
    \item 
    Given a model $\fh$, what is the advantage of Friedman's H-statistic for some interaction compared to computing the functional decomposition for $\fh$?
    \begin{itemize}
        \item[$\Rightarrow$] Computes only the single component necessary, and may also give some (although bad) information in case the features are not independent.
    \end{itemize}


\end{enumerate}
