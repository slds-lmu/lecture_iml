\aufgabe{Implementation of standard fANOVA using PD-functions}{

    Recall exercises 1 e) and f) on exercise sheet 4, which were about implementing ICE curves and PDPs.
    It implemented the following functions:
    \begin{itemize}
          \item \texttt{ice(model, X, j, grid)} - returns an $n \times g$ matrix of ICE predictions (one curve per row, one grid value per column);
          \item \texttt{pdp(model, X, j, grid)} - returns the length-\(g\) vector of PDP values by averaging the ICE matrix column-wise;
          \item \texttt{c\_ice(model, X, j, grid, ref = 1)} - again, returns an $n \times g$ matrix of ICE predictions, but centers every ICE curve by subtracting its value at the grid point at the reference index \texttt{ref};
          \item \texttt{c\_pdp(model, X, j, grid, ref = 1)} - analogous to \texttt{pdp}, this function averages the centered ICE curves to obtain a centered PDP which is zero at the reference point.
    \end{itemize}
    Here, \texttt{model} is a given fitted model (as an object), \texttt{X} is the data set of size $n \times p$, $j$ is the index or the name of the feature of interest, and \texttt{grid} is a vector of grid values.

    Now, we want to build on that exercise to implement the standard fANOVA algorithm.
    In case you did not finish that other exercise back then, the files \textit{XX.py} and \textit{XX.R} on Moodle contain an implementation of these functionalities, so you can also use those instead of your own solution for this exercise here.

    \begin{enumerate}[a)]
    
        \item
        Extend the \texttt{pdp} function to a new function \texttt{pd\_func}, which can compute an arbitrary PD-function of arbitrary dimension.
        This means that the parameters change as follows:
        \begin{itemize}
            \item Instead of a single feature index \texttt{j}, we know consider an ordered list of indices \texttt{S}, which must be a subset of $\{1, \dots, p\}$, but can also be empty (for the intercept case),
            \item the \texttt{grid} is an $\texttt{length(S)} \times g$ matrix, in which the $j$-th row contains all grid values for the $j$-th feature in $S$,
            \item the output of \texttt{pd\_func(model, X, S, grid)} is a $\underbrace{g \times g \times \dots \times g}_{|S| \text{ many dimensions}}$ matrix / tensor, containing the values of the PD-function at all grid points.
        \end{itemize}
        You can either change your existing \texttt{ice} function accordingly and generalize it to multiple dimensions, in which case the \texttt{pd\_func} can again simply call the \texttt{ice} function and average over all the ($|S|$-dimensional) ICE functions.
        Or you can directly implement the model calls and the averaging inside the function \texttt{pd\_func}.

        \item 
        Implement centered PD-functions, as they are used in this chapter, in a function \texttt{pd\_func\_centered}.
        That is, subtract the average of the PD-function over the given grid.

        \item
        Implement a function \texttt{fANOVA} that computes the standard fANOVA algorithm, using the function \texttt{pd\_func} from above.
        There are two possible ideas for the data structure to store the fANOVA components, which is also the data structure of the output of this function:
        \begin{enumerate}
            \item As a \textbf{dictionary}:
            A dictionary in python (or a ??? in R) works similar to an array, but with arbitrarily named indices, called the \texttt{keys} of the dictionary.
            Here, the idea would be to have tuples of integers as keys (i.e., the component for $S = \{1,3\}$ would be stored under the key \texttt{(1,3)}, and for each key to save a tensor of shape $\underbrace{g \times g \times \dots \times g}_{|S| \text{ many}}$.
            \item As a big tensor using \textbf{one-hot encoding}:
            Here, we use binary vectors to encode the sets of features.
            A subset $S$
        \end{enumerate}

        \item
        Implement a function calculating the H-statistic

        \item \textbf{Bonus:}
        \begin{itemize}
            \item Extend your functions such that if the \texttt{grid} parameter is not given, they can construct a grid from the data (e.g., via quantiles or equidistantly) automatically.
            \item Extend your functions to allow for a different number of grid values for different features.
        \end{itemize}
        
            
    \end{enumerate}
    
}