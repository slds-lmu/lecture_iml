\loesung{
\begin{enumerate}

\item \textbf{M Plot:}
The M plot for \(x_1\) is defined as:
$
\hat{f}_{1, M}(x_1) = \mathbb{E}_{x_2 \mid x_1} \left[ \hat{f}(x_1, x_2) \mid x_1 \right].
$

\textbf{Derivation:}
Substituting \(\hat{f}(x_1, x_2) = -x_1 + 2x_2\):
$$\hat{f}_{1, M}(x_1) = \mathbb{E}_{x_2 \mid x_1} \left[ -x_1 + 2x_2 \mid x_1 \right] \overset{(-x_1 \text{ const.})}{=} -x_1 + 2 \cdot \mathbb{E}[x_2 \mid x_1].$$
    
    % Since \(-x_1\) is constant, it simplifies to:
    % $$
    % \hat{f}_{1, M}(x_1) = -x_1 + 2 \cdot \mathbb{E}[x_2 \mid x_1].
    % $$
     
If \(\mu_1 = \mu_2 = 0\) (as in the given exercise), the conditional expectation reduces to: 
$$\mathbb{E}[x_2 \mid x_1] = \rho \frac{\sigma_2}{\sigma_1} x_1 \overset{(\sigma_1 = \sigma_2)}{=} \rho x_1.
$$

    Using this property of multivariate normal distributions, \(\mathbb{E}[x_2 \mid x_1] = \rho x_1\) and $\rho \approx 1$, we have:
    \[
    \hat{f}_{1, M}(x_1) = -x_1 + 2 \cdot (\rho x_1) = x_1(-1 + 2\rho) \approx x_1(-1 + 2) =  x_1.
    \]

\item \textbf{PD Plot:}
The PD plot for \(x_1\) is defined as:
$$
\hat{f}_{1, \text{pdp}}(x_1) = \mathbb{E}_{x_2} \left[ \hat{f}(x_1, x_2) \right].
$$
\begin{enumerate}
    \item \textbf{Derivation:}
    Substituting \(\hat{f}(x_1, x_2) = -x_1 + 2x_2\): 
    $$\hat{f}_{1, \text{pdp}}(x_1) = \mathbb{E}_{x_2} \left[ -x_1 + 2x_2 \right] \overset{(-x_1 \text{ const.})}{=}  -x_1 + 2 \cdot \mathbb{E}[x_2].
    $$
    
    The marginal expectation of \(x_2\) is \(\mathbb{E}[x_2] = 0\), so:
    $$\hat{f}_{1, \text{pdp}}(x_1) = -x_1.$$

    \item \textbf{Comparison:}
    The PD plot and M plot differ because the M plot incorporates the dependency between \(x_1\) and \(x_2\) (via \(\rho\)) and integrates over the conditional distribution, while the PD plot integrates over the marginal distribution and thereby considers \(x_1\) and \(x_2\) as if they were independent (which is ok if we are interested in the model's behavior).
\end{enumerate}

\item \textbf{ALE Plot:}
The ALE plot for \(x_1\) is defined as:
\[
\text{ALE}_1(x_1) = \int_{z_{\min}}^{x_1} \mathbb{E}_{x_2 \mid x_1 = z} \left[ \frac{\partial \hat{f}(z, x_2)}{\partial x_1} \right] dz.
\]
\begin{enumerate}
    \item \textbf{Uncentered ALE Plot:}
    Compute the partial derivative of \(\hat{f}(x_1, x_2)\) with respect to \(x_1\):
    \[
    \frac{\partial \hat{f}(x_1, x_2)}{\partial x_1} = -1.
    \]
    Substituting into the ALE definition:
    \[
    \text{ALE}_1(x_1) = \int_{z_{\min}}^{x_1} \mathbb{E}_{x_2 \mid x_1 = z}[-1] \, dz = \int_{z_{\min}}^{x_1} -1 \, dz.
    \]
    Solve the integral:
    \[
    \text{ALE}_1(x_1) = -(x_1 - z_{\min}).
    \]

    \item \textbf{Centered ALE Plot:}
    To center the ALE plot, compute the mean of the uncentered ALE values:
    \[
    \mathbb{E}_{x_1}[\text{ALE}_1(x_1)] = \mathbb{E}_{x_1}[-(x_1 - z_{\min})] = -( \mathbb{E}[x_1] - z_{\min}) = z_{\min},
    \]
    since \(\mathbb{E}[x_1] = 0\). Therefore, the centered ALE plot is:
    \[
    \text{ALE}_1^{\text{centered}}(x_1) = \text{ALE}_1(x_1) - z_{\min} = -(x_1 - z_{\min}) - z_{\min} = -x_1.
    \]

    \item \textbf{Comparison:}
    The ALE plot accounts for the conditional dependence of \(x_2\) on \(x_1\), unlike the PD plot.
    The centered ALE plot and uncentered ALE plot differ only by a constant.
\end{enumerate}

\end{enumerate}

}