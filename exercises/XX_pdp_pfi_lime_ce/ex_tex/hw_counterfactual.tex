\aufgabe{Counterfactual Explanations: Credit Scoring Use Case}{
A binary classifier predicts whether a person receives a loan (\texttt{credit = yes/no}) based on
\texttt{sex, age, salary, savings, credit\_amount}.  
On the first row, you are given one customer's record (denied credit) and several generated counterfactuals.

\begin{center}
\begin{tabular}{lllllll}
\toprule
ID & sex & age & salary & savings & credit\_amount & credit \\
\midrule
\bfseries
1 & \bfseries m & \bfseries 50 & \bfseries 1300 & \bfseries 500 & \bfseries 5000 & \bfseries no \\
\hline
2 & f & 50 & 1300 & 500 & 5000 & yes \\
3 & m & 50 & 1300 & 500 & 2600 & yes \\
4 & m & 50 & 1300 & 250 & 5000 & yes \\
5 & m & 50 & 1400 & 250 & 5000 & yes \\
6 & m & 51 & 1390 & 700 & 5000 & yes \\
\bottomrule
\end{tabular}
\end{center}

\subsection*{Tasks}
For each counterfactual (IDs 2--6), decide whether to \textbf{show} or \textbf{hide} it from the customer.  
For each decision, explain using: 

\begin{itemize}
   \item \emph{Actionability and practical feasibility}: Is the change practically feasible?
   \item \emph{Sparsity}: How many features changed?
   \item \emph{Model logic}: Does the counterfactual make sense, or does it reveal a model flaw?
\end{itemize}

For counterfactuals that reveal model flaws or bias issues (identified under Model logic), also explain what type of problem they indicate and whether they should be reported to the data science team for investigation.
}

