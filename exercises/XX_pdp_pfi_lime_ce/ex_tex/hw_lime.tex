\aufgabe{Local Interpretable Model-agnostic Explanations (LIME)}{
In Local Interpretable Model-agnostic Explanation (LIME), the surrogate model is trained on observations $z$ that are sampled around the point of interest $x$ and weighted by their similarity (to the point of interest) according to the exponential kernel. 
We have drawn samples $z$ from a Gaussian distribution centered at $x$, i.e., for each feature $z_j$ we sample $z_j \sim \mathcal{N}(x_j, \sigma^2), j = 1,\dots,p$
for a chosen standard deviation $\sigma$.

The following figures show the prediction surface plot of a three-class classification problem with two features. The white dot corresponds to $x$, while the black dots display $z$. Observations $z$ were sampled with different $\sigma$ for each plot: $\{0.01, 0.05, 0.10, 0.20\}$. Answer the following:
\begin{enumerate}
  \item Identify which panel corresponds to each $\sigma$ value.
  \item For each of the four $\sigma$ configurations, comment on their appropriateness for LIME.
\end{enumerate}
\begin{figure}[h!]
  \centering
  % Composite image with four parameter settings
  \includegraphics[width=0.9\textwidth]{fig-man/lime_4_params.png}
\label{fig:lime_gaussian_sampling}
\end{figure}
}

