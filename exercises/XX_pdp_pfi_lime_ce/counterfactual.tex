
\documentclass[11pt,a4paper]{article}
\usepackage{amsmath, amssymb, booktabs, geometry}
\geometry{margin=2.5cm}
\setlength{\parskip}{0.7em}
\setlength{\parindent}{0pt}

\title{Counterfactual Explanations: Credit Scoring Use Case}
\date{}

\begin{document}
\maketitle

\section*{Exercise: Counterfactual Explanations in Credit Scoring}

A binary classifier predicts whether a person receives a loan (\texttt{credit = yes/no}) based on
\texttt{sex, age, salary, savings, credit\_amount}.  
On the first row, you are given one customer's record (denied credit) and several generated counterfactuals.

\begin{center}
\begin{tabular}{lllllll}
\toprule
ID & sex & age & salary & savings & credit\_amount & credit \\
\midrule
1 & m & 50 & 1300 & 500 & 5000 & no \\
2 & f & 50 & 1300 & 500 & 5000 & yes \\
3 & f & 45 & 1300 & 500 & 500 & yes \\
4 & m & 50 & 1300 & 500 & 2600 & yes \\
5 & m & 52 & 1400 & 500 & 5000 & yes \\
6 & m & 50 & 1400 & 250 & 5000 & yes \\
7 & m & 49 & 1390 & 700 & 5000 & yes \\
8 & m & 51 & 1800 & 500 & 5100 & yes \\
9 & f & 50 & 1301 & 570 & 5019 & yes \\
\bottomrule
\end{tabular}
\end{center}

\subsection*{Tasks}
\begin{enumerate}
\item Which counterfactuals would you show to the \textbf{customer}?  
   Explain using:  
   \emph{Actionability} (can the customer change it?),  
   \emph{Plausibility} (is it realistic?), and  
   \emph{Sparsity} (fewest features changed).

\item Which counterfactuals would you \textbf{hide} from the customer and why?

\item Which counterfactuals would you report to the \textbf{data science team} as potential bias indicators?

\item (Rashomon effect) Several different well-performing models produce different counterfactuals for this same denied applicant.
\begin{enumerate}
   \item In one sentence, what does this tell you about trusting a single counterfactual?
   \item If two changes (lower credit amount, higher salary) appear in most models, how would you present them to the customer so the advice is simple and robust?
\end{enumerate}

\end{enumerate}

\newpage

\section*{Solutions}
\begin{enumerate}
   \item \textbf{Counterfactuals to show (customer focus).} We prefer explanations that change \emph{actionable, financially meaningful} variables with minimal edits:
   \begin{itemize}
      \item \textbf{(4)} m, 50, 1300, 500, credit\_amount = 2600 $\rightarrow$ yes. \emph{Only one feature changed} (requested amount). Highly sparse, clearly actionable: apply with a lower loan request.
      \item \textbf{(7)} m, 49, salary = 1390, savings = 700, credit\_amount = 5000 $\rightarrow$ yes. Improves both liquid reserves and income (both attainable over time). Two changes, still relatively sparse; all modifications plausible and aligned with reduced risk.
   \end{itemize}
   We \emph{exclude} (6) even though accepted because it \textbf{lowers savings} (250 from 500), which is counter-intuitive recourse.

   \item \textbf{Counterfactuals to hide (non-actionable / unethical / implausible).}
   \begin{itemize}
      \item \textbf{(2), (9)} (sex flipped) and \textbf{(3)} (sex + large credit reduction): involve a sensitive, immutable characteristic (sex). Should never be presented as recourse.
      \item Minor age shifts (50 $\rightarrow$ 49 or 52) as decisive (rows 5,7) are \emph{not actionable}; age is time-dependent and slow, not an intervention.
      \item \textbf{(6)} reduces savings while approving credit: implausible direction; discouraging financial stability.
      \item \textbf{(8)} raises credit amount (5100) while granting approval; requesting more money to become eligible is contradictory.
   \end{itemize}

   \item \textbf{Counterfactuals for data science team (bias / modeling diagnostics).}
   \begin{itemize}
      \item \textbf{Sex changes (2,3,9)}: may indicate \emph{gender bias} or proxy leakage if sex alone flips prediction.
      \item \textbf{Age variants (5,7)}: sensitivity to small age increments suggests potential over-fitting to age or data imbalance.
      \item \textbf{(6)} decreasing savings: model may not enforce monotonic relationship between financial strength and approval.
      \item \textbf{(8)} higher requested amount flipping to yes: contradicts typical risk logic; investigate feature interactions or data artifact.
   \end{itemize}

   \item \textbf{Rashomon effect.}
      \begin{enumerate}
         \item Different well-performing models yielding distinct counterfactuals shows that \emph{a single counterfactual is not uniquely trustworthy}; explanations depend on modeling assumptions.
         \item If most models agree on: (i) lower credit amount and (ii) higher salary, present a \emph{robust recourse summary}: ``Reduce requested amount (e.g., 2600 instead of 5000) \textbf{or} increase income (target \~1400+) to improve approval likelihood.'' Emphasize these as stable levers across models, avoiding model-specific quirks.
      \end{enumerate}
\end{enumerate}
\end{document}
