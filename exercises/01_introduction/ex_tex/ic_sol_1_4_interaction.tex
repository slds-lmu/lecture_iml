\loesung{

Problem: The function $	f(\xv) = 2 x_1 + 3 x_2 - x_1 |x_2|$ is not differentiable for $x_2 = 0$. Hence, different cases need to be considered:
\begin{center}
	Case 1: $x_2 > 0$ \, ; \quad
	Case 2: $x_2 < 0$ \, ; \quad
	Case 3: $x_2 = 0$
\end{center}

Case 1: $x_2 > 0$
\begin{align*}
	\left( \frac{\partial^2 f(\xv)}{\partial x_1 \partial x_2} \right)^2 
	=  \left( \frac{\partial^2}{\partial x_1 \partial x_2} \, \left(2 x_1 + 3 x_2 - x_1 x_2 \right)\right)^2 
	=  \left( \frac{\partial }{\partial x_2}\, \left(2 - x_2\right) \right)^2 
	=  \left(-1\right) ^2  =  1 > 0
\end{align*}

Case 2: $x_2 < 0$
\begin{align*}
	\left( \frac{\partial^2 f(\xv)}{\partial x_1 \partial x_2} \right)^2 
	=  \left( \frac{\partial^2}{\partial x_1 \partial x_2} \, \left(2 x_1 + 3 x_2 - x_1 (-x_2) \right)\right)^2 
	=  \left( \frac{\partial }{\partial x_2}\, \left(2 + x_2\right) \right)^2 
	= 1^2 = 1 > 0
\end{align*}

Case 3: $x_2 = 0$ \\
Not considered, as analysis of interactions via definition requires the consideration of intervals. The examination of single points does not make sense.


$\Rightarrow$ $x_1$ and $x_2$ interact with each other.

}
