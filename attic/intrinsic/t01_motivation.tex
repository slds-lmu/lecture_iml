\documentclass[11pt,compress,t,notes=noshow, aspectratio=169, xcolor=table]{beamer}

\usepackage{../../style/lmu-lecture}
% Defines macros and environments
\usepackage{bbm}
% basic latex stuff
\newcommand{\pkg}[1]{{\fontseries{b}\selectfont #1}} %fontstyle for R packages
\newcommand{\lz}{\vspace{0.5cm}} %vertical space
\newcommand{\dlz}{\vspace{1cm}} %double vertical space
\newcommand{\oneliner}[1] % Oneliner for important statements
{\begin{block}{}\begin{center}\begin{Large}#1\end{Large}\end{center}\end{block}}


%new environments
\newenvironment{vbframe}  %frame with breaks and verbatim
{
 \begin{frame}[containsverbatim,allowframebreaks]
}
{
\end{frame}
}

\newenvironment{vframe}  %frame with verbatim without breaks (to avoid numbering one slided frames)
{
 \begin{frame}[containsverbatim]
}
{
\end{frame}
}

\newenvironment{blocki}[1]   % itemize block
{
 \begin{block}{#1}\begin{itemize}
}
{
\end{itemize}\end{block}
}

\newenvironment{fragileframe}[2]{  %fragile frame with framebreaks
\begin{frame}[allowframebreaks, fragile, environment = fragileframe]
\frametitle{#1}
#2}
{\end{frame}}


\newcommand{\myframe}[2]{  %short for frame with framebreaks
\begin{frame}[allowframebreaks]
\frametitle{#1}
#2
\end{frame}}

\newcommand{\remark}[1]{
  \textbf{Remark:} #1
}


\newenvironment{deleteframe}
{
\begingroup
\usebackgroundtemplate{\includegraphics[width=\paperwidth,height=\paperheight]{../style/color/red.png}}
 \begin{frame}
}
{
\end{frame}
\endgroup
}
\newenvironment{simplifyframe}
{
\begingroup
\usebackgroundtemplate{\includegraphics[width=\paperwidth,height=\paperheight]{../style/color/yellow.png}}
 \begin{frame}
}
{
\end{frame}
\endgroup
}\newenvironment{draftframe}
{
\begingroup
\usebackgroundtemplate{\includegraphics[width=\paperwidth,height=\paperheight]{../style/color/green.jpg}}
 \begin{frame}
}
{
\end{frame}
\endgroup
}
% https://tex.stackexchange.com/a/261480: textcolor that works in mathmode
\makeatletter
\renewcommand*{\@textcolor}[3]{%
  \protect\leavevmode
  \begingroup
    \color#1{#2}#3%
  \endgroup
}
\makeatother


\providecommand{\tightlist}{%
  \setlength{\itemsep}{0pt}\setlength{\parskip}{0pt}}

%\setbeamerfont{footnote}{size=\tiny}
\usepackage[hang,flushmargin]{footmisc}
\renewcommand*{\footnotelayout}{\tiny}
\renewcommand*{\thefootnote}{} %\fnsymbol{footnote}

% https://tex.stackexchange.com/questions/30720/footnote-without-a-marker
% \makeatletter
% \def\blfootnote{\gdef\@thefnmark{}\@footnotetext}
% \makeatother

% https://tex.stackexchange.com/questions/357717/beamer-allowframebreaks-option-and-vertical-spacing-when-using-lists-itemize
% \setbeamertemplate{frametitle continuation}{%
%     (\insertcontinuationcount)%
%     \ifnum\insertcontinuationcount>1%
%     \vspace*{\topsep}%
%     \else%
%     %
%     \fi%
% }


%\title{iML: Ante-hoc Methods for Neural Networks}
%\subtitle{Motivation}

\title{Interpretable Machine Learning}
\date{}
\begin{document}
%	\maketitle
	\graphicspath{ {./figure/} }

 
\newcommand{\titlefigure}{figure/bild2}
\newcommand{\learninggoals}{
\item Interpretability by sparsity
\item Regularisation for interpretability
\item Sequential feature selection}

\lecturechapter{Ante-hoc Methods for Neural Networks}
\lecture{Interpretable Machine Learning}


\begin{frame}{Motivation}
    \begin{itemize}
        \item Post-hoc methods do not always give you the entire picture
        \item Post-hoc methods are not always accurate
        \begin{itemize}
            \item An explanation that is 10\% inaccurate leads to lack of trust in the ML model
            \item Hard to measure the accuracy of post-hoc methods
        \end{itemize}
        \item Wherever possible use models that are interpretable-by-design
    \end{itemize}
    \begin{figure}
        \centering
        \includegraphics[scale=.4]{bild1}
    \end{figure}
\end{frame}

\begin{frame}{Simpler Models}
    \begin{columns}
    \begin{column}{0.5\textwidth}
    \begin{itemize}
        \item Models that have an understandable decision-making process
        \item Models that have a smaller set of parameters or weights
        \begin{itemize}
            \item Examples: Linear models, GAMs
        \end{itemize}
        \item Models that have human-understandable decision structure
        \begin{itemize}
            \item Examples: decision trees, random forests
        \end{itemize}
        \item Models that have sparsity or only a few set of parameters or features that matter
        \begin{itemize}
            \item Example: 1\% of a large feature space, 1-hot encodings in language tasks
        \end{itemize}
    \end{itemize}
    \end{column}
    \begin{column}{0.5\textwidth}
    \begin{figure}
        \centering
        \includegraphics[scale=.7]{bild2}
    \end{figure}
    \end{column}
    \end{columns}
\end{frame}

\begin{frame}[c]{Interpretable by Design Models - Sparse Models}
\begin{itemize}
    \item Models that have explicitly enforce sparsity
    \begin{itemize}
        \item through regularisation
        \item through feature selection
    \end{itemize}
    \bigskip
    \item Sparsity through regularisation
    \begin{itemize}
        \item E.g. L0, L1 regularisation
    \end{itemize}
    \bigskip
    \item Sparsity through feature selection
    \begin{itemize}
        \item select a subset of impacting features for the prediction task
    \end{itemize}
\end{itemize}
    
\end{frame}

\begin{frame}{Regularisation in Neural Networks}
\begin{itemize}
    \item L0 norm is the number of non-zero parameters — setting weights to 0
    \item L1 sparsity — sum of the weights should be small
\end{itemize}

$$
$$
\centerline{\includegraphics[width=0.5\linewidth]{bild3}}

\end{frame}

\begin{frame}{L1 Regularisation}
    \begin{itemize}
        \item Optimising using L0 regularisation is hard
        \item L1 regularisation in neural networks can be achieved by gradient-based optimisation
        \item Degree of regularisation is a user-controllable parameter
    \end{itemize}
   
   \begin{align*}
       \mathcal{\hat{L}}(W) &= \alpha\|W\|_1 + \mathcal{L}(W)\\
       \nabla_W\mathcal{\hat{L}}(W) &= \alpha sign(W) + \nabla_W\mathcal{L}(W)
   \end{align*}
 
    %\begin{figure}
     %   \centering
      %  \includegraphics[scale=.4]{bild4}
    %\end{figure}
\end{frame}

\begin{frame}{Feature Selection}
    “Select a smaller features space which can efficiently describe the input data while reducing effects from noise or
irrelevant variables and still provide good prediction results”
\bigskip
    \begin{itemize}
        \item Wrapper methods - Treat the model as a blackbox
        \item Filter methods
        \item Embedded methods
        \item Other methods
        \bigskip
        \item Smaller feature space: subset of features, an embedded hyperspace
    \end{itemize}
\end{frame}

\begin{frame}{Sequential Feature Selection}
\begin{itemize}
    \item Number of feature subsets is $2^N$
    \item How do we reduce the computational complexity of checking each subset ?
    \begin{itemize}
        \item Sequentially choose the most promising feature at each iteration
    \end{itemize}
    \bigskip
    \item Selection Set S= \{\}, All features N= $\{f_1,f_2, \dots , F_n\}$
    \item In each iteration
    \begin{itemize}
        \item compute utility of f- train a model with S $\cup\{f\}$ and measure validation perf.
        \item terminate loop if no improvement of utility and return S
        \item choose f in N/S that has max utility and add f to S
    \end{itemize}
\end{itemize}
    
\end{frame}
\begin{frame}{Feature Selection}
\begin{itemize}
    \item What are the short comings of sequential feature selection ?
    \begin{itemize}
        \item Greedy might not be optimal
        \item Global feature selection method
    \end{itemize}
    \bigskip
    \item How do we improve the greedy solution ?
     \begin{itemize}
        \item Allow for backtracking, branch-and-bound
        \item Use genetic algorithms GA, swarm optimisation
    \end{itemize}
    \bigskip
    \item How do we choose a local feature selection method ?
    \begin{itemize}
        \item instance-wise feature selection methods
    \end{itemize}
\end{itemize}
    
\end{frame}

\endlecture
\end{document}