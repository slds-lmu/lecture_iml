\usepackage[]{graphicx}
\usepackage[]{color}
% maxwidth is the original width if it is less than linewidth
% otherwise use linewidth (to make sure the graphics do not exceed the margin)
\makeatletter
\def\maxwidth{ %
\ifdim\Gin@nat@width>\linewidth
\linewidth
\else
\Gin@nat@width
\fi
}
\makeatother
%\usepackage[fontsize=10.5pt]{scrextend}
\definecolor{ggred}{rgb}{0.973, 0.463, 0.427}
\definecolor{ggblue}{rgb}{0, 0.749, 0.769}
\definecolor{fgcolor}{rgb}{0.345, 0.345, 0.345}
\newcommand{\hlnum}[1]{\textcolor[rgb]{0.686,0.059,0.569}{#1}}%
\newcommand{\hlstr}[1]{\textcolor[rgb]{0.192,0.494,0.8}{#1}}%
\newcommand{\hlcom}[1]{\textcolor[rgb]{0.678,0.584,0.686}{\textit{#1}}}%
\newcommand{\hlopt}[1]{\textcolor[rgb]{0,0,0}{#1}}%
\newcommand{\hlstd}[1]{\textcolor[rgb]{0.345,0.345,0.345}{#1}}%
\newcommand{\hlkwa}[1]{\textcolor[rgb]{0.161,0.373,0.58}{\textbf{#1}}}%
\newcommand{\hlkwb}[1]{\textcolor[rgb]{0.69,0.353,0.396}{#1}}%
\newcommand{\hlkwc}[1]{\textcolor[rgb]{0.333,0.667,0.333}{#1}}%
\newcommand{\hlkwd}[1]{\textcolor[rgb]{0.737,0.353,0.396}{\textbf{#1}}}%
\newcommand{\predvar}{Var\left[\hat{f}(\xv)\right]}
\let\hlipl\hlkwb




\newcommand{\Sm}{S_m}%{Pre(\tau,j)} % S(\tau,j)
\newcommand{\Smj}{\Sm \cup \{j\}}
\newcommand{\minusSmj}{- \{\Sm \cup \{j\} \} }
\newcommand{\Stau}{S_{j}^\tau}%{Pre(\tau,j)} % S(\tau,j)
\newcommand{\Stauj}{\Stau \cup \{j\}}
\newcommand{\minusStauj}{- \{\Stau \cup \{j\} \} }

\newcommand{\xjp}{\xv_{+j}^{(m)}}
\newcommand{\xjm}{\xv_{-j}^{(m)}}
%\newcommand{\xjp}{\xv_{\Smj}, \mathbf{z}^{(m)}_{\minusSmj}}
%\newcommand{\xjm}{\xv_{\Sm}, \mathbf{z}^{(m)}_{- \Sm}}




\usepackage{pdfpages}
\usepackage{framed}
\makeatletter
\newenvironment{kframe}{%
\def\at@end@of@kframe{}%
\ifinner\ifhmode%
\def\at@end@of@kframe{\end{minipage}}%
\begin{minipage}{\columnwidth}%
\fi\fi%
\def\FrameCommand##1{\hskip\@totalleftmargin \hskip-\fboxsep
\colorbox{shadecolor}{##1}\hskip-\fboxsep
% There is no \\@totalrightmargin, so:
\hskip-\linewidth \hskip-\@totalleftmargin \hskip\columnwidth}%
\MakeFramed {\advance\hsize-\width
\@totalleftmargin\z@ \linewidth\hsize
\@setminipage}}%
{\par\unskip\endMakeFramed%
\at@end@of@kframe}
\makeatother

\definecolor{shadecolor}{rgb}{.97, .97, .97}
\definecolor{messagecolor}{rgb}{0, 0, 0}
\definecolor{warningcolor}{rgb}{1, 0, 1}
\definecolor{errorcolor}{rgb}{1, 0, 0}
\newenvironment{knitrout}{}{} % an empty environment to be redefined in TeX

\usepackage{alltt}
\newcommand{\SweaveOpts}[1]{}  % do not interfere with LaTeX
\newcommand{\SweaveInput}[1]{} % because they are not real TeX commands
\newcommand{\Sexpr}[1]{}       % will only be parsed by R

\usepackage[english]{babel}
\usepackage[utf8]{inputenc}

\usepackage[export]{adjustbox}
\usepackage{dsfont}
\usepackage{verbatim}
\usepackage{amsmath}
\usepackage{amsfonts}
\usepackage{bm}
\usepackage{csquotes}
\usepackage{multirow}
\usepackage{longtable}
\usepackage{booktabs}
\usepackage{enumerate}
\usepackage[absolute,overlay]{textpos}
\usepackage{psfrag}
\usepackage{algorithm}
\usepackage{algpseudocode}
\usepackage{eqnarray}
\usepackage{arydshln}
\usepackage{tabularx}
\usepackage{placeins}
\usepackage{tikz}
\usepackage{setspace}
\usepackage{colortbl}
\usepackage{mathtools}
\usepackage{wrapfig}
\usepackage{bm}
%\usepackage[backend=biber]{biblatex}

\usetikzlibrary{tikzmark, shapes,arrows,automata,positioning,calc,chains,trees,  shadows, decorations.pathreplacing}
\tikzset{
%Define standard arrow tip
>=stealth',
%Define style for boxes
punkt/.style={
rectangle,
rounded corners,
draw=black, very thick,
text width=6.5em,
minimum height=2em,
text centered},
% Define arrow style
pil/.style={
->,
thick,
shorten <=2pt,
shorten >=2pt,}
}

\usepackage{subfig}

\usepackage{bbm}
%\newcommand\hmmax{0}
%\newcommand\bmmax{0}
% basic latex stuff
\newcommand{\pkg}[1]{{\fontseries{b}\selectfont #1}} %fontstyle for R packages
\newcommand{\lz}{\vspace{0.5cm}} %vertical space
\newcommand{\dlz}{\vspace{1cm}} %double vertical space
\newcommand{\oneliner}[1] % Oneliner for important statements
{\begin{block}{}\begin{center}\begin{Large}#1\end{Large}\end{center}\end{block}}

% Latexmath Notation
\input{../../latex-math/basic-math}
\input{../../latex-math/basic-ml}
\input{../../latex-math/ml-ensembles.tex}
\input{../../latex-math/ml-interpretable.tex}

% \perp now defined in latex-math
% \newcommand{\indep}{\perp}
% latex-math defines \ind (lower case I) identically
% \newcommand{\Ind}{\mathcal{I}}

%new environments
\newenvironment{vbframe}  %frame with breaks and verbatim
{
\begin{frame}%[containsverbatim,allowframebreaks]
}
{
\end{frame}
}

% \newenvironment{vframe}  %frame with verbatim without breaks (to avoid numbering one slided frames)
% {
%  \begin{frame}[containsverbatim]
% }
% {
% \end{frame}
% }

\newenvironment{blocki}[1]   % itemize block
{
 \begin{block}{#1}\begin{itemize}
}
{
\end{itemize}\end{block}
}

\newenvironment{fragileframe}[2]{  %fragile frame with framebreaks
\begin{frame}[allowframebreaks, fragile, environment = fragileframe]
\frametitle{#1}
#2}
{\end{frame}}


\newcommand{\myframe}[2]{  %short for frame with framebreaks
\begin{frame}[allowframebreaks]
\frametitle{#1}
#2
\end{frame}}

\newcommand{\remark}[1]{
  \textbf{Remark:} #1
}


\newenvironment{deleteframe}
{
\begingroup
\usebackgroundtemplate{\includegraphics[width=\paperwidth,height=\paperheight]{../style/color/red.png}}
 \begin{frame}
}
{
\end{frame}
\endgroup
}
\newenvironment{simplifyframe}
{
\begingroup
\usebackgroundtemplate{\includegraphics[width=\paperwidth,height=\paperheight]{../style/color/yellow.png}}
 \begin{frame}
}
{
\end{frame}
\endgroup
}\newenvironment{draftframe}
{
\begingroup
\usebackgroundtemplate{\includegraphics[width=\paperwidth,height=\paperheight]{../style/color/green.jpg}}
 \begin{frame}
}
{
\end{frame}
\endgroup
}
% https://tex.stackexchange.com/a/261480: textcolor that works in mathmode
\makeatletter
\renewcommand*{\@textcolor}[3]{%
  \protect\leavevmode
  \begingroup
    \color#1{#2}#3%
  \endgroup
}
\makeatother

% \makeatletter
% %\newcommand\notsotiny{\@setfontsize\notsotiny\@vipt\@viipt}
% \newcommand\notsotiny{\@setfontsize\notsotiny{6.31415}{7.1828}}
% \makeatother

\providecommand{\tightlist}{%
  \setlength{\itemsep}{0pt}\setlength{\parskip}{0pt}}

%\setbeamerfont{footnote}{size=\tiny}
\usepackage[hang,flushmargin]{footmisc}
\renewcommand*{\footnotelayout}{\tiny}
\renewcommand*{\thefootnote}{} %\fnsymbol{footnote}

% https://tex.stackexchange.com/questions/638616/beamer-frametitle-always-uppercase-for-text-but-not-for-href-or-url
\usepackage{textcase}

% https://stackoverflow.com/questions/377301/is-there-a-latex-command-to-make-text-all-lower-caps
%\usepackage[overload]{textcase}
%\setbeamertemplate{frametitle}{\MakeTextUppercase{\insertframetitle}}
%\setbeamertemplate{frametitle}{\expandafter\uppercase\expandafter\insertframetitle}

\setbeamertemplate{frametitle}{
 %\begin{columns}[T, onlytextwidth]
 %     \begin{column}{\paperwidth}
      \MakeTextUppercase{\insertframetitle}
 %     \end{column}
 %   \end{columns}
}
\newcommand{\citebutton}[2]{%
\NoCaseChange{\resizebox{!}{9pt}{\protect\beamergotobutton{\href{#2}{#1}}}}%
}

%\setbeamertemplate{frametitle}{\expandafter\uppercase\expandafter\insertframetitle}
%\newcommand{\citebutton}[2]{%
%\href{#2}{\footnotesize\color{black!60}[#1]}
%\resizebox{!}{9pt}{\beamergotobutton{\lowercase{\href{#2}}{#1}}}% \MakeLowercase
%}
%\newcommand{\lit}[2]{\href{#2}{\footnotesize\color{black!60}[#1]}}

\newcommand{\lit}[2]{\href{#2}{\footnotesize\color{black!60}[#1]}}

\let\code=\texttt
\let\proglang=\textsf

\setkeys{Gin}{width=0.9\textwidth}

% https://tex.stackexchange.com/questions/30720/footnote-without-a-marker
% \makeatletter
% \def\blfootnote{\gdef\@thefnmark{}\@footnotetext}
% \makeatother

% https://tex.stackexchange.com/questions/357717/beamer-allowframebreaks-option-and-vertical-spacing-when-using-lists-itemize
% \setbeamertemplate{frametitle continuation}{%
%     (\insertcontinuationcount)%
%     \ifnum\insertcontinuationcount>1%
%     \vspace*{\topsep}%
%     \else%
%     %
%     \fi%
% }
